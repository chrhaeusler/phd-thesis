Backup of quotes for general discussion (or SRM study).  Studies that average
data across study participants may draw

\begin{itemize}

\item ``provide only an approximate view of any individual's brain organization,
    potentially obscuring meaningful individual differences in cortical
        organization'' \citep{laumann2015functional},

\item may just ``capture the common denominator of each individual cognitive
    circuit and lose a large amount of information''

\item ``obscure(s) patterns of brain organization specific to each individual''
    \citep{laumann2015functional}.

\end{itemize}


\todo[inline]{Phrasing here is still similar to phrasing in general intro}

\todo[inline]{mih: auditory stimulus for visually impaired is pretty much a
footnote given the current state of the thesis}

Human brain mapping studies have traditionally averaged \ac{fmri} data across
participants.
%
However, data need to be assessed on the level of individual persons in order to
advance the field towards a clinical application.
% functional localizer
A promising tool to perform this advancement are/is functional localizers
[because localizers aim to characterize the location size, and shape of
functional areas on the level of individual subject].
% contra localizers
However, traditional localizer paradigms employ selectively sampled, tightly
controlled stimuli, rely heavily on a participant's compliance, and can usually
map just one domain of brain functions
% naturalistic stimuli could replace
Localizer paradigms based on naturalistic stimuli could provide higher
ecological as well as external validity, higher data quality due to increased
compliance, and potentially map a variety of brain functions [ranging from
low-level perception (e.g., luminance) to high-level cognition (e.g., social
cognition)] simultaneously.
% visually impaired
Lastly, an exclusively auditory stimulus like an audiobook or audio drama would
also be appropriate for visually impaired persons[, e.g., suffering from
nystagmus or lack of eyesight].

% PPA as proof of concept
Focussing on the \ac{ppa}, a ``classic'' higher-visual area,
\citep{epstein1998ppa}, the goal of this thesis was to explore whether a movie
and the movie's audio-description could, in principle, substitute a traditional
localizer paradigm.
% paragraph on open science
An additional goal of this dissertation was to perform all studies under the
principles of open, shared, and transparent science.
%
In order to enable independent researchers to validate current results and use
written code to replicate current findings in prospective studies, all created
data, code, analysis steps, and results are published as version-controlled
DataLad \citep[\href{www.datalad.org}{datalad.org};][]{halchenko2021datalad}
datasets.


\section{Recapitulation of work packages}

\todo[inline]{maybe, very short summary of parts here; momentarily it's a draft}

\todo[inline]{mih: keep this part short;
%
don't go into details of large aspects per-study, but aggregate across
studies; better show how things are connected, i.e.
%
what comes from open-science for your work in general, using studies as examples, but not focused on each study separately
%
same for progress re localization: how is there potential for doing better
than block-design localizers, what can be achieved, under which conditions?
}

\todo[inline]{In line with general discussion (and main messages of SRM study):
datalad as technical foundation for open science, annotation of naturalistic
stimuli, application on localization: a) PPA study, b) SRM study}

%
First, we extended the studyforrest dataset (study 1).
%
Second and similarly to traditional localizer paradigms, we modeled hemodynamic
activity based on annotated stimulus features embedded in the movie ``Forrest
Gump'' and its audio-description, and created \ac{glm} $t$-contrasts in order to
localize the \ac{ppa} (study 2).
%
Third, we estimated results of the localizer by projecting data through a
\ac{cfs} (study 3).

\todo[inline]{maybe, give an overview of how the remaining part of the thesis is
structured; at the moment: each study as such \& study in light of open science,
general discussion about open science across studies}


\subsection{Speech anno}


\subsubsection{Goal of speech anno}

% what we did in 1 sentence
In study 1 \citep{haeusler2021speechanno}, we created and validated an
annotation of speech occurring in the movie and its audio-description pursuing
two goals.
% aim #1: groundwork for PPA study
The first aim was to build the groundwork that enabled us to conduct study 2.
% aim #2: extend studyforrest
The second aim was to create an exhaustive annotation of speech that
substantially exceeds the groundwork necessary to conduct study 2 in order to
extend the studyforrest dataset as a public resource for independent research.


\subsubsection{Discussion of speech anno}

% validation analysis
We validated the annotation's quality in study 1 and performed a canonical
\ac{glm} analysis by contrasting regressors correlating with speech-related
events to a regressor correlating with events without speech.
% results
As hypothesized, results revealed statistically significant increased
hemodynamic activity in a bilateral cortical network known to be involved in the
perception of speech \citep[e.g.,][]{friederici2011brain, wilson2008beyond}.
% conclusion
These results encouraged us to a) use the annotation as the groundwork for study
2, b) publish the annotation as an extension of the studyforrest project.


\subsubsection{Open science in context of speech anno}

\todo[inline]{how "personal" am I supposed to get?}

\todo[inline]{mih: such remarks could be one of the most interesting parts of a
thesis. Maybe it is best to condense them all into a single section, but
interspersed like this could work too}

\paragraph{Intro}

% additional effort 1
Pursuing the goal of creating a publication-worthy dataset led to additional
work that goes far beyond the work that was necessary to build the \ac{glm} in
study 2.
% additional effort 2
The published annotation provides, among others, time-stamps of phonemes, words
and sentences of all speakers, a grammatical tagging, and an annotation of
syntactic dependencies and semantics.


\paragraph{The self-flagellation retrospectively}
%
Great care was taken during the initial creation and subsequent iterative
corrections in order to provide accurate information to the scientific
community.
%
However, over the course of generating the dataset it became apparent that there
is no such thing as a ``perfect'' annotation:
%
As in human language in general, an annotation of speech will always contain
ambiguities.
%
Additionally, there is a trade-off that needed to be balanced between a) doing
the ``mere minimum'' and putting time and effort in creating additional
information that might not be fruitful, and b) providing a sound/substantial
groundwork for potential use-cases that needed to be anticipated.
%
Any further processing step might be based on a decision that might not match
the requirement of a specific use case.
%
For example, an annotation of semantics might be based on a current state-of-the
art language model that might be superseded by future language models.
%
Therefore, the published annotation does not only comprise the final outcome but
also the raw data and documented code that can automatically be rerun step by
step to reproduce the final outcome of both the annotation and its validation
analysis, all freely accessible in a version-controlled dataset.


\paragraph{Conclusion}
%
In summary, the annotation provides extensive information about the time course
of stimulus features, and therefore a headstart to independent researchers that
wish to ``model hemodynamic brain responses that correlate with a variety of
aspects of spoken language ranging from a speaker's identity, to phonetics,
grammar, syntax, and semantics'' \citep{haeusler2021speechanno} under more
real-life like conditions.
%
Consequently, the outcome of study 1 contributes to the studyforrest project as
a resource for the scientific community by further widening the ``annotation
bottleneck'' \citep{aliko2020naturalistic} of two naturalistic stimuli.


\subsection{PPA paper}

\subsubsection{Recapitulation}

\paragraph{Goal of PPA paper}
% study in one sentence
The goal of study 2 \citep{haeusler2022processing} was to explore whether an
audio-visual and an exclusively auditory naturalistic stimulus could be used in
order to localize the \ac{ppa} as it was previously identified in the same set
of participants by a traditional block-design functional localizer that employed
static pictures \citep{sengupta2016extension}.


\paragraph{Method \& results of PPA paper}
% AV operationalization
For the model-based mass-univariate statistical analysis (i.e.\ac{glm}) of the
movie's data, we operationalized the perception of visual spatial information
based on an annotation of movie cuts and depicted locations
\citep{haeusler2016cutanno}.
% AD operationalization
For the \ac{glm} of the audio-description's data, we extended the annotation of
speech \citep{haeusler2021speechanno} by further annotating nouns that the
narrator uses to describe the movie's absent visual content.

% group results: AV \& AD
On a group-average level, findings demonstrate that increased activation in the
\ac{ppa} generalizes to the perception of spatial information embedded in the
audio-visual movie and the audio-description.
% individual AD
On an individual level, semantic spatial information occurring in the
audio-description is correlated with significant activity in the anterior part
of the \ac{ppa} bilaterally in nine individuals and unilaterally in one
individual.


\paragraph{Discussion \& conclusion of PPA paper}

% conclusion 1
Results add evidence \citep[cf.][]{bartels2004mapping} that a functionally
defined region, such as the \ac{ppa}, can be localized using a model-driven
analysis that is based on a naturalistic stimulus' annotated temporal structure.
% conclusion 2
Further, results suggest that a purely auditory naturalistic stimulus like an
audio-description could potentially substitute a visual localizer as a
diagnostic procedure to assess brain functions in visually impaired individuals
[phrasing pretty similar to \citep{haeusler2022processing}].


\subsubsection{Future studies: PPA}

\todo[inline]{mih: worth stating again somehere in the discussion that the
studyforrest dataset is not for this (diverse), but it is a small dataset, which
limits the generality}

%
Two aspect revealed by our analyses invite further investigations on the
properties of the \ac{ppa}.
%
First, the responses correlating with an auditory stimulation are spatially
restricted to the anterior \ac{ppa}, and
%
Second, we observed higher intersubject variability of responses of the \ac{ppa}
to a naturalistic auditory stimulation compared to the visual stimulation during
the localizer and movie paradigm.


\paragraph{Just anterior PPA during auditory stimulation}

% our interpretation
Previous studies in the field of visual perception suggest that the \ac{ppa} can
be divided into functionally subregions that might process different stimulus
features.

\todo[inline]{paraphrase stuff from paper here}

%
Hence, we attributed the revealed pattern to different features inherent in the
visual stimuli compared to features inherent in the naturalistic auditory
stimulus [phrasing pretty similar to \citep{haeusler2022processing}].
%
However, our interpretation of the observed pattern ``can only be preliminary,
because the auditory stimulation dataset differs in key acquisition properties
(field-strength, resolution) from the datasets of the movie and visual localizer
representing a confound of undetermined impact'' [still pretty similar to
phrasing in \citep{haeusler2022processing}].
% conclusion
Future studies could employ controlled stimuli, maybe accompanied by a task, to
investigate in detail whether the observed differential activations during
visual and auditory stimulation are replicable.


\paragraph{Interindividual variability in response to auditory stimulation}

% statement
On an individual-level, we observed higher intersubject variability of responses
of the \ac{ppa} to a naturalistic auditory stimulation compared to the
audio-visual movie and visual localizer paradigms \citep[cf. Table 3
in][]{sengupta2016extension}.
% not necessarily noise
However, the divergent pattern from the group mean in four of fourteen
individuals should not necessarily be interpreted as measurement errors,
artefacts, or ``random noise'' \todo{choose just one term} but could also be
attributed to individual differences in responses to the task free, auditory
paradigm.
%
Our naturalistic auditory paradigm differs from block-design localizer paradigms
not just in the exclusively auditory stimulation but also in the accidental,
event-like presentation of spatial information, and the absence of a task which
leaves study participant naive to the investigated cognitive process.


\paragraph{Possibly correlated factors}

\todo[inline]{rephrase in a "readable" (but still preliminary) way}

``consistent with previous reports showing significant differences between
topographies estimated by static and dynamic localizers, especially in superior
temporal and frontal cortices (Fox et al., 2009; Pitcher et al., 2011)''
\citep{jiahui2022cross}.

%
The revealed pattern could correlated with [influenced by] situational factors
like the experimental design (stimulus type, no task), (transient) state of a
participant (e.g., alertness or engagement),  our simply our ``adventurous''
modeling approach[?].
%
\todo{well...}
%
More stable factors might be individual differences in cognitive tendencies or
cognitive abilities like susceptibility / predisposition [?] to attend to, [or]
recognize [or process] auditory spatial information.
% Conclusion
Future studies could employ both controlled and naturalistic stimuli to
investigate whether our results that revealed higher intersubject variability in
response to auditory spatial information are a) replicable across different
experiments and paradigms, and b) reliable within subjects.


\paragraph{Brain \& behavior: intro to "fingerprints"}

\todo[inline]{following is pretty speculative 'cause reliability of differences
is premise for "fingerprints"; hence, just a draft; does it make sense to
discuss it?}

% kanai
In case the pattern is stable within individual subjects / is ``highly
consistent across different sessions [or experiments], then they are
characteristics of the individuals and may reflect differences in their brain
function'' \citep{kanai2011structural} [on structural diffences].
%
``Individual differences in topology (i.e. location, size, shape of functional
areas) and the activity within functional areas can also be considered to be
interesting cases of inter-individual variability to understand the neural basis
of human cognition and behavior, brain-phenotype relationships'', and ``present
useful phenotypes or biomarkers \citep{glasser2016multi,
vanhorn2008individual}''
%
\todo{paraphrase}

\paragraph{Brain \& behavior: example studies}

\todo[inline]{shorten heavily or drop altogether}

%
For example, \citet{kong2019spatial} suggested based on resting-state functional
connectivity measures ``that individual-specific network topography (i.e.,
location and spatial arrangement) might serve as a fingerprint of human behavior
that can predict behavioral phenotypes across cognition, personality, and
emotion'' \citep{kong2019spatial} [with modest accuary, comparable to previous
reports predicting phenotypes based on connectivity strength].

%
\citep{bijsterbosch2018relationship}'s ``results indicate that spatial variation
in the topography of functional regions across individuals is strongly
associated with behaviour'' \citep{bijsterbosch2018relationship}.
%
\citet{bijsterbosch2018relationship} found ``that the spatial arrangement of
functional regions is strongly predictive of non-imaging measures of behavior
and lifestyle'' [however shape \& exact location of brain regions interacted
strongly with  modeling of brain connectivity].
%
\citet{bijsterbosch2018relationship} found ``that individual differences in the
size, shape and exact position of the brain regions [as identified by
resting-state functional connectivity measures] was strongly linked to
individual differences in behavioral tests and questionnaires [including
intelligence, life satisfaction, drug use and aggression problems]''
\citep{bijsterbosch2018relationship}.

%
``The variations in spatial topographical features captured a more direct and
unique representation of subject variability than temporal correlations between
regions defined by group parcellation approaches (coupling).
%
Hence, the cross-subject information represented in commonly adopted
'connectivity fingerprints' could largely reflect spatial variability in the
location of functional regions across individuals, rather than variability in
coupling strength (at least for methods that directly map group-level
parcellations onto individual data)'' \citep{bijsterbosch2018relationship}.

\todo[inline]{drop following paragraph; just here to better understand paragraph
above}
%
``Depending on the employed spatial alignment algorithm and the amount of
removed spatial intersubject variability, the degree to which spatial
information may influence FC estimates possibly varies considerably across
studies.
%
In recent years, significant efforts have gone into the methods that more
accurately estimate the spatial location of functional parcels in individual
subjects [Chong et al., 2017; Glasser et al., 2016; Gordon et al., 2016; Hacker
et al., 2013; Harrison et al., 2015; Varoquaux et al., 2011; Wang et al., 2015],
and into advanced hyperalignment approaches [Chen et al., 2015; Guntupalli et
al., 2016; Guntupalli and Haxby, 2017]'' \citep{bijsterbosch2018relationship}.


\subsubsection{Conclusion on future PPA studies}

\todo[inline]{draft; clean \& rephrase!}
%
In summary, our results invite further studies that investigate the properties
of the parahippocampal area in response to
%
a) in response to an auditory naturalistic stimulation (with or without a
simultaneous task to attend to spatial information, or with subsequent memory
task),
%
b) in response to a controlled paradigm (with or without any task).
%
The respective paradigms could elaborate whether interindividual variability of
responses in the parahippocampal area is related to cognitive processes like
``auditory scene perception'' [is that a valid term?], and to degree auditory
spatial information is utilized for, e.g., spatial orientation, way finding or
[planing, remembering, executing] navigation.


\subsubsection{Future studies: other scene-selective areas}

\todo[inline]{imo, RSC and OPA do not need to be discussed in general
discussion; could be shortly discussed in "open science in general" (->
opportunity costs)}


\subsubsection{Future studies: other functional areas (studyforrest dataset)}

\todo[inline]{imo, a side note; not necessarily needed to be discussed; hence,
draft}

The visual localizer performed by \citet{sengupta2016extension} employed images
from six categories (houses, landscapes, faces, bodies without heads, small
objects, and scrambled images).
%
As a results, the corresponding dataset provides subject-specific \acp{roi}
masks for higher visual areas besides the \ac{ppa}:
%
the fusiform face area (FFA) \citep{kanwisher1997ffa} and the occipital face
area (OFA) \citep{pitcher2011occipitalfacearea},
%
the extrastriate body area (EBA) \citep{downing2001bodyarea},
%
and the lateral occipital complex (LOC) \citep{malach1995loc}.
%
Future studies (e.g., a master's thesis or part of a PhD project) could adjust
our extension of the annotation of speech created in study 2 and the
corresponding analysis pipeline in order to explore hemodynamic responses
correlating with auditory information related to faces, body parts or small
objects.


\subsubsection{Open science in context of PPA paper}

% goal PPA study
Under the perspective of an open science project, the goal of study 2 was to use
three \ac{fmri} datasets \citep{hanke2014audiomovie, hanke2016simultaneous,
sengupta2016extension}, two stimulus annotations \citep{haeusler2021speechanno,
haeusler2016cutanno}, as well as previously published results
\citep{sengupta2016extension} for a new research question.


% skipped work
On the one hand, the availability of the \ac{fmri} data and the subject-specific
\acp{roi} enabled us to shift our focus from acquiring the raw data to
subsequent stages of a research project.
% example of skipped work
For example, the annotations that were created in a "general purpose state"
could be extended immediately to match the needs of study 2, followed by writing
the scripts that preprocessed and statistically analyzed the data.
% additional work
On the other hand, pursuing the goal of an open science project lead to
additional work.
% example of additional work
For example, code needed to be in a state worthy to be published and documented
for other readers, analyses pipelines needed to be in a state to be automatized,
every processing step documented, saved, protocolled, and shared to allow
reproducibility of results and facilitate replicability of finding.

\todo[inline]{shorten results/findings}

% results
Our results have been published in a peer-reviewed, open-access journal
%
``offer evidence that a model-driven GLM analysis based on annotations can be
applied to a naturalistic paradigm to localize concise functional areas and
networks correlating with specific perceptual processes''
\citep{haeusler2022processing},
%
``demonstrate that increased activation in the PPA during the perception of
static pictures generalizes to the perception of spatial information embedded in
a movie and an exclusively auditory stimulus \citep{haeusler2022processing}, and
%
``provide further evidence that the PPA can be divided into functional
subregions that coactivate during the perception of visual scenes''
\citep{haeusler2022processing}

% conclusion
In summary, we re-used existing data as a foundation for a new investigation in
order to generate novel findings encourage further studies, and illustrate the
benefits of publicly and freely available datasets.


\subsection{SRM study}

\todo[inline]{following parts are an old draft}

\subsubsection{Transition from study 2 to study 3}
%
Despite exploratory approach in study 2 (a.k.a. shitty modeling of subjectively
assessed events), results suggest:
%
the response to spatial information must be somewhere in within the response
time series and is detectable.
%
Hence, we might be able to use the response patterns measured during the
presentation of the audio-description in order to generate a \ac{cfs} [needs to
be defined above in overview] in study 3, and align an ``unknown'' test
participant to that \ac{cfs}.

% align left-out subject
Based on our findings in study 2 \citep{haeusler2022processing}, we assumed that
the event structure in both naturalistic stimuli would correlate, among others,
with brain responses that are similar to those correlating with the event
structure in a dedicated functional localizer.

% summary of study 2
Results of study 2 suggest that a naturalistic stimulus might provide an
engaging, task-free paradigm to localize brain functions in individual subjects.


\subsubsection{Goal of SRM study}
% the problem
Considering practical and monetary constraints in a clinical context, a paradigm
lasting 90 to 120 minutes is inappropriate for even an extensive individual
diagnostic procedure.

% goal 1: new procedure
The first goal was to assess a procedure to estimate results of a dedicated
localizer \citep{sengupta2016extension} based on data acquired during
naturalistic stimulation.
%
Following leave-one-subject out cross-validation, we estimated (i.e. predicted)
the results of the visual localizer experiment ($Z$-values of voxels within a
\ac{roi}) of a left-out test participant based on localizer results of a
reference group.

% goal 2: partial alignment
The second goal was to assessed the relationship between length of
naturalistic stimulation used to align the test participant to the fixed
\ac{cfs} and the estimation performance.
%
Lastly, the estimation performance of our new procedure based on \ac{fa} was
compared an estimation performance based on \ac{aa}.


\paragraph{Hypotheses}
%
We hypothesized that increased quantity of data used to calculate the
transformation matrices of the left-out subjects for a \ac{fa} would to increase
prediction performance.
%
Further, we hypothesized that \ac{fa} would eventually perform
``better'' than an estimation based on \ac{aa}.


\subsubsection{Discussion of SRM study}

\todo[inline]{...to be written}


\subsubsection{Future studies on SRM / functional alignment}

\todo[inline]{...to be written}


\subsubsection{Open science in context of SRM study}

\todo[inline]{...to be written}




\section{Open Science}

\todo[inline]{check: anything about open science of each paper missing here?}

\todo[inline]{Following is preliminary phrasing; part has way more captions than
necessary}

\todo[inline]{most importantly:
%
which points make sense to be brought up (and need to be phrased nicely)?
%
which points are missing?
%
which points should be left out?}

\todo[inline]{less on the stept (a.k.a. I did this and that); one paragraph
should be enough}

\todo[inline]{more: the "conceptual story": it's complicated; challenging data;
therefore, proper handling: not just consumed but also published contribution,
i.e. the management aspect, so it gets trust}

\subsection{Intro}
%
Following the practices of open and reproducible science was not mandatory for
submitting the thesis but required additional work and time.
%
``Open and reproducible research means you need to guarantee the accuracy of the
methods used and to explicitly describe and document all stages of the
scientific process to ensure its transparency and traceability''.
%
Standards to follow are not yet fully established and corresponding software
tools are still emerging.
%
Since the best practices are not yet part of a graduate or PhD curriculum,
learning about the principles and standards and applying the corresponding
procedures and necessary tools was based on self-initiative and self-learning.
%
Over the course of the present project, affected stages were
% data-related
a) collection, description and storage of data,
% processing-related
c) processing and analyzing data via code, and
% publishing shit
d) publication of data, materials and results.


\subsection{Data collection; data analysis; publishing}
%
In the context of open data, data are not collected for mere internal purposes
but re-use by third parties needs to be considered.
%
Hence, dataset creators need to anticipate which people might use the data for
which purpose, collect the data according to best practices, convert data into a
standardized format (considering, e.g., naming conventions, folder structure,
separating raw from analyzed data), and create metadata.

% data analysis & automatization
The data, materials and code need to be documented more rigorously, and coverage
of procedures need to exceed the coverage given in method sections of regular
articles.
%
Every change made to data, materials or code, and command line invocations need
to be tracked via a version control system \citep[e.g.,][]{halchenko2021datalad}
to allow other researchers to inspect a study's full history.
%
In order to allow reproducibility from input data, changes made to the data, to
computation and visualization of results, every processing step needs to be
designed and tested to be run reliably and automatically.

% publication: findable, accessible, interoperable, reusable
In the stage of preparing a publication, a researcher needs to facilitate
discovery by humans and web bots (e.g., via extensive description and
machine-readable metadata), ensure long-term curation [e.g., maintenance],
availability [e.g., accessibility], and choose an appropriate data host.
% legal issues
Finally, a researcher need to resolve legal issues that raise during due to the
publication of data, materials and code (e.g. statement of agreement / consent,
anonymization, intellectual property rights, use license).


\subsection{Cons from perspective of creators}



% jeez! it's annoying!
In general, creating open data, materials and code requires a considerable about
of time and effort with little incentives and little, immediate rewards.
%
Publishing data (or merely using open data) is associated with the risk that
other working groups, possibly having more funding and ``people and brain
power'' at disposal, are using the same data for a similar research question at
the same time.
%
Hence, there is the concern that someone else might ``claim priority, usually
through publishing, to a research idea or result you yourself have been working
on'' \citep{laine2017afraid}.
%
On the one hand, being second in the ``competitive race in the sciences'' leads
to diminished opportunities to publish own results because high impact journals
favour novel findings.
%
The risk, and therefore stress, is aggravated in case of researchers that are
early in their scientific career \citep[cf.][]{toribio2021early}, created and
curate the published dataset, pre-registered studies based on open data, or have
to stick to inflexible project plans.
%
On the other hand one advantage of publishing a dataset with an assigned DOI is
that it might get re-used and cited in another study.
%
In case of a dataset creator's fear of being superseded / outrun, ``concerns can
be alleviated by delaying the sharing or using a data-sharing repository with an
embargo period'' \citep{nichols2017best}.


\subsection{Pros from perspective of creators}

% better organized, better documented
An immediate benefit of following the principles of a reproducible study is that
a researcher is forced to work more organized and document every step from a
study's start to finish more rigorously.
%
First, documenting every step and justifying every step by weighting pros and
cons of alternative [mutual exclusive] procedural paths leads to better
understanding, avoids tricking oneself into performing unnecessary statistical
tests, HARKing hypothesizing after the results are known; Kerr, 1998,
HARKing...), and therefore supports following general good scientific practices.
%
Similarly, the extra time and effort spend on inspecting data and testing code
leads to higher confidence in one's own work and the reliability of results.
%
Second, a researcher who records all changes to data and code from the start to
the final results can restore a particular state of data and code and trace,
identify and correct errors more easily [similar: Klein, 2018. A practical guide
for transparency in psychological science].
%
Third, tracking and documenting every step increases reliability of results and
can be seen as a "lab protocol" containing information / templates for writing
scientific articles.
%
Last, automating recurring tasks gives yourself the possibility of reusing
certain data, code, documents, etc. in the future.


\subsection{Pros from perspective of consumers}

\todo[inline]{following needs revision but general train of thought should be
clear}

%
From the perspective of reader, open access journals provides low-cost access to
information.
% readers get a better understanding
A ``transparent and complete reporting of all facets of a study, allowing a
critical reader to evaluate the work and fully understand its strengths and
limitations'' \citep{nichols2017best}.
%
It helps readers to take a look at the methods in more detail it is conveyed in
a often limited method section of a regular study.
%
``This also facilitates subsequent research efforts by other investigators, who
can exactly follow (or carefully manipulate) each aspect of a study''
\citep{nichols2017best}.
%
Open materials facilitate tracking (and understanding!) the process (esp.
analyzes) in detail (pipelines are often far easier to understand by reading the
code step-by-step than just reading the method section).
%
Interesting to students how can not just read a method section but also take a
look at the code and follow step by step every command.
%
Fully transparent studies that also include the input data can serve as an
``education playground'' by enabling (undergraduate) student to trace the
progress of real world project, to learn coding and data analysis.
%
Open data provide groups that enjoy minor funding low-threshold access to
datasets.
%
While open data helps researchers to shift time and resources from data
collection to subsequent stages of a study, share code helps researchers adjust
and extent the analysis pipeline as part of an exploratory data analysis.


\subsection{Cons from perspective of consumers}

% avoid "the bigger, the better" and "garbage in, garbage out"
An mostly not explicitly stated issue in the context of open science is that
standards (quality, formats, parameters) and open sciences practices (e.g.,
documenting) might vary across scientific field, or within scientific fields
depending on a working group's knowledge and rigor.
% check the data
Dataset consumers need to assume that everything that is not explicitly stated
in the description of a dataset has not been considered and done by a dataset's
creator.
% laugh with many, don't trust any
Even if the data a from a renowned source, researchers should consider
themselves to be obliged to test and validate a dataset's quality according to
their standards and specific use cases.


\subsubsection{Tell me about opportunity costs without saying opportunity costs}

% problem of preprocessed data
Moreover, choices made during data collection and preprocessing -- despite being
state-of-the-art at the time of being published -- might not be optimal for
every use case or made obsolete by more advanced methods.
% tell me about opportunity costs without saying opportunity costs
Hence, researchers need to weigh the costs and benefits of one path (e.g.,
preprocessing the same data differently than the preprocessing performed as part
of an open dataset) relative to an alternative path (e.g., using the
preprocessed data and performing new analyses), and choose the path with the
greater net return.



\subsection{Short personal assessment}

\todo[inline]{This whole (sub)section might be too personal; nobody (me
included) cares about the my life's journey}

\todo[inline]{Following are some ideas to connect the dry text above to
experiences during the present thesis (that actually made me come up with the
text above)}

% intro
The additional time and effort feel like a burden and do not contribute to the
immediate benefits of the PhD project, how-fucking-ever:


\subsubsection{I used open-source software packages}

Often forgotten, but open-source packages were prerequisite.


\subsubsection{I used existing data}

\todo[inline]{Why was is good for you that the data were already existing?}

\todo[inline]{imo, irrelevant in my case: participants' consent, anonymization,
most legal issues}

\todo[inline]{Caveat of data re-use: know your data! I could write that I pulled
my hair out while searching for inconsistencies in the timings, and found that
the audio track of the audio-description is essentially unsystematically
shifted; similar cases stay probably undiscovered in non-open datasets; vice
versa, a point for less error-prone open science}

\todo[inline]{Covid-Pandemic has impressively shown that collecting data based
on human subjects" can go dormant for almost 1.5 years; plan of the project was
adjusted accordingly; and I am so lucky that I could fall back on open data}

\todo[inline]{it is/was somewhere between "stupid" and "brave" to mess with
someone like Haxby's group; also, cf. "my precious" of Susanne and Lisa}

\todo[inline]{Opportunity costs: analyzes in \citet{sengupta2016extension} were
performed and \acp{roi} published in voxel-space (i.e., not surface-space)}


\subsubsection{Opportunity costs: missing ROIS of RSC \& OPA}

\todo[inline]{does it make sense to talk about "missing" \acp{roi} of RSC and
OPA at all?}

\todo[inline]{a short version could simply state "it would be fucking awesome to
perform a follow-up study that investigates \ac{rsc}, \ac{opa}; most text is
simply a template to be summarized}

%
``Apart from the PPA, results show significantly increased activity in the
ventral precuneus and posterior cingulate region (referred to as ``retrosplenial
complex'', RSC) of the medial parietal cortex, and in the superior lateral
occipital cortex (referred to as ``occipital place area'', OPA) for both
naturalistic stimuli.
% RSC intro
Like the PPA, the RSC and OPA have repeatedly shown increased hemodynamic
activity in studies investigating visual spatial perception and navigation
\citep{chrastil2018heterogeneity, bettencourt2013role, dilks2013occipital,
epstein2019scene}'' \citep{haeusler2022processing}.

%
In the progress of conducting study 2, ``we assumed that results (at least of
the audio-visual stimulus) could also yield significant clusters in the
retrosplenial complex (RSC) and superior lateral occipital cortex (i.e.
“occipital place area”, OPA) but did not explicitly hypothesize that fact''.
%
``We (and Sengupta) chose the PPA among three possible candidates
because it was the first area to be discovered as a visual ``scene-selective''
region, is the most reliably activated region across studies that investigate
visual scene perception'' [response to reviewer \#2].

%
``Whereas the PPA is assumed to be more involved in landmark recognition by
processing basal perceptual features that constitute a scene, the RSC (i.e.
ventral precuneus and posterior cingulate region) exhibits stronger responses
when the scenes are familiar to the participants suggesting the RSC might be
more concerned with localizing, i.e. orienting, the observer in space [e.g.
Epstein \& Vass, 2016]'' [response to reviewer \#2].
% medial parietal cortex: anterior-posterior gradient
``Similarly to the parahippocampal cortex \citep{aminoff2013role}, the medial
parietal cortex exhibits a posterior-anterior gradient from being more involved
in perceptual processes to being more involved in memory related processes
\citep{chrastil2018heterogeneity, hassabis2009construction, silson2019posterior,
steel2021network}'' \citep{haeusler2022processing}.

%
''We believe that a detailed discussion of the RSC Activation in RSC and OPA was
out of scope of the current study, results are an incentive for further
studies''.
% future studies
``Future, complementary studies using specifically designed paradigms could
investigate where in the posterior-anterior axis of the parahippocampal and
medial parietal cortex auditory semantic information is correlated with
increased hemodynamic activity:
% we hypothesize
we hypothesize that the auditory perception of spatial information (compared to
non-spatial information) is correlating with clusters in the middle of possibly
overlapping clusters correlating with visual perception (peak activity more
posterior) and scene construction from memory (peak activity more anterior)''
\citep{haeusler2022processing}.

%
In the context of open science, this was kind of an limitation / aftereffect of
coverage of mask for functional areas \citet{sengupta2016extension}, and the
non-algorithmic procedure of \citet{sengupta2016extension} based on subjective
decision that is hard to replicate \& to amply to the other functional areas
\citep[cf. algorithmic procedure in, e.g.,][]{julian2012algorithmic}.
%
In summary: everything that is not 100\% automatized is not 100\% reproducible,
sadly because automatization takes a long time for some stuff or is not possible
for some stuff.


\paragraph{Code: documenting, version-controlled, automatize pipeline}

\todo[inline]{Retrospectively, I cannot remember one clear case which made me
glad having documented, version-controlled, and automatized stuff; maybe,
materials are a little better organized or my code was a little better to grasp
after not having taken a look at it for a longer time...}

%
This can get super fucking annoying in case of automatized creation of (complex)
figures from numeric results without final editing of the figure by a human.
Jeez! Fuck me sideways!



\paragraph{Publishing stuff}

\todo[inline]{choosing the data host was easy because filtered based on DataLad
support (and Michael knows everything anyway)}

\todo[inline]{mention "open paper"? speech anno paper is on github;
ppa paper is not a public github repository (yet)}

\todo[inline]{My super interesting papers might get cited more often because,
yeah, open access that gets, on average, cited more often [REFERENCE?]}
%
``greater potential impact of a work when it may be cited not just for its
scientific findings but also when its data is reused in other works''
\citep{nichols2017best}.
%
But this is no immediate benefit, and gambling (for the 20\% of PhD's that stay
in science).


\paragraph{Conclusion of my life's story}

% no incentive
``Current incentives do not justify spending large amounts of time preparing
data for sharing, as institutional promotion panels or grant reviewers currently
do not adequately reward such efforts'' \citep{nichols2017best}.

Grateful, that open source neuro-software as well as the data were available,
which allowed me to go far beyond what would have been possible without open
data [kind of not true because I used "in-house data"].
%
Lastly, I feel more confident about my work now compared to the dilettantish
procedures experienced (and followed) in, uhm, another lab in Magdeburg.



\subsection{Conclusions on open science}


\todo[inline]{following could be heavily summarized as a conclusion or phrased
like an (summarizing) personal opinion}

%
Open sciences makes researchers accountable to collect, document, process and
store data and materials according to best practices.
%
Published data and analysis pipelines allow external persons to check the data
and analyses for undiscovered errors, and replicate the results step by step.
%
Varying parameters and running different statistics on the data allows
inspection of robustness of results.

%
``As scientists, we are supposed to be objective arbiters of evidence and
theory, but we are not infallible and must be ready to accept criticism and
revise our claims when errors are discovered'' \citep{nichols2017best}.
%
``However, we need to develop a culture of constructive criticism, which
recognizes that errors are an inevitable part of scientific progress and
protects individual researchers from inappropriately harsh consequences when
honest mistakes are discovered'' \citep{nichols2017best}.

%
``We see no better way to advance understanding on a contested finding than to
have as many researchers as possible puzzling over the data at hand''
\citep{nichols2017best}

%
Benefits are increased robustness and reliability of science when all steps are
openly documented and data are openly available.
%
Multiple datasets can be combined to perform unanticipated use cases, and
extensively and openly documented results of multiple studies facilitate
performing meta-analyses to strengthen the claims of individual studies.
%
Thus, open transparent science is the way to make knowledge and technologies
widely accessible, and increase reproducibility of study results and
replicability of scientific findings while increasing trust of the public into
scientific process and its results.
%
Open science promises increased efficiency (time and financial expense) of
making scientific progress, make advance, and promotes innovation.


\subsection{Call to action: create an incentive or imperative to do it}

\paragraph{Questionable, immediate benefits}
%
``The weight that OS currently has for researchers’ career advancement is rather
small, despite'' \citep{toribio2021early}.
%
At the same time, they might feel that investing additional effort in making
research open (i.e., transparent and reproducible) is unrewarded [Nicholas et
al., 2017], such as conducting replication studies that might not be considered
for publication in high-impact journals'' \citep{toribio2021early}.
%
``As long as scientists are being evaluated on traditional journal metrics,
there are few incentives from a career perspective to fully commit to OS''
\citep{toribio2021early}.


\paragraph{Gambling on long-term benefits}

\todo[inline]{80\% of PhD students leave science anyway}
%
``Thus, while conducting OS may initially require more work in the short term,
it can greatly benefit one’s career in the long term'' \citep{toribio2021early}.
%
``OA publications have been found to receive more citations than paywalled
publications [Piwowar et al., 2018] and can therefore aid ECRs’ career
advancement'' \citep{toribio2021early}.
%
``Similarly, open data can be highly beneficial to promote new collaborations
and increase the number of citations and the confidence that the field has in
the findings [Popkin, 2019]'' \citep{toribio2021early}.
%
Currently, playing be the old rules is the ``smarter way'' than gambling getting
cited when data or code get re-used.


\paragraph{Guidelines}
%
``OS still requires the establishment of clear guidelines for transparency and
openness of research at the international level.
%
Examples for guidelines for OA publishing [Nosek et al., 2015; Schiltz, 2018],
and collaborations [Gold et al., 2019] are already existing, and their use
should be promoted by governments and funding agencies, as well as integrated in
the training of ECRs by academic institutions.
%
Organizations and/or regulators in charge of overviewing the open scholarly
system need to be established [cf. Nicholas et al., 2019, 2020]''
\citep{toribio2021early}.


\paragraph{Education}

\todo[inline]{Breed open science enthusiast in (under)graduate curriculi}

%
``It is necessity to promote further training on the benefits and risks of OS
practices [cf. Schönbrodt, 2019].
%
Promoting training would not only increase the knowledge of ECRs about specific
OS practices but also could foster its implementation among ECRs.
%
It would be highly beneficial to introduce these training schemes in the
curriculum of undergraduates programs.
%
Courses should cover the benefits and risks of OS practices, together with a
guideline on how to implement them [cf. Farnham et al., 2017]''
\citep{toribio2021early}.


\paragraph{Carrots...}

\todo[inline]{Start with incentives to foster self-learning and applying the
principles "voluntarily"}

%
More incentives to conduct open science project needs to be established.
%
``Extra efforts may not be valued appropriately by a scientific community who
assesses research based on journal impact metrics and number of publications
[Moher et al., 2018]'' \citep{toribio2021early}.
%
``Individual incentives for researchers should be introduced through, for
example, professional recognition or the allocation of extra funding [Kidwell et
al., 2016; Fecher et al., 2015; Ali-Khan et al., 2018].


\paragraph{...and sticks}

\todo[inline]{later, convince the remaining insurgents by force}
%
``Funding agencies already require publication of findings in OA
schemes and data-sharing plans [Neylon, 2017]'' \citep{toribio2021early}.
%
``Compulsory requirements from funders, which might only lead researchers to
show minimal compliance [Neylon, 2017]'' \citep{toribio2021early}.


\section{Conclusion: Naturalistic stimuli as functional localizer}

\todo[inline]{literature says it is impossible (but Alexander Hut!). But: how
"prohibitive" is it? How naive was it to do it? What did take the most time?}

\todo[inline]{yes, you can analyze data from naturalistic stimuli based on
stimulus annotation despite people saying it is not possible}

Summary of PPA paper:
%#13 natural stimulation
``In summary, natural stimuli like movies \citep{eickhoff2020towards,
hasson2008neurocinematics, sonkusare2019naturalistic} or narratives
\citep{hamilton2018revolution, honey2012not, lerner2011topographic,
silbert2014coupled, wilson2008beyond} can be used as a continuous, complex,
immersive, task-free paradigm that more closely resembles our natural dynamic
environment than traditional experimental paradigms.
% method
We took advantage of three fMRI acquisitions and two stimulus annotations that
are part of the open-data resource
\href{http://www.studyforrest.org}{studyforrest.org} to operationalize the
perception of spatial information embedded in an audio-visual movie and an
auditory narrative, and compare current results to a previous report of a
conventional, block-design localizer.
% results
The present study offers evidence that a model-driven GLM analysis based on
annotations can be applied to a naturalistic paradigm to localize concise
functional areas and networks correlating with specific perceptual processes --
an analysis approach that can be facilitated by the neuroscout.org platform
\citep{delavega2021neuroscout}.
% interpretation
More specifically, our results demonstrate that increased activation in the PPA
during the perception of static pictures generalizes to the perception of
spatial information embedded in a movie and an exclusively auditory stimulus.
% interpretation: aPPA vs. pPPA
Our results provide further evidence that the PPA can be divided into functional
subregions that coactivate during the perception of visual scenes.
% interpretation
Finally, the presented evidence on the in-principle suitability of a naturally
engaging, purely auditory paradigm for localizing the PPA may offer a path to
the development of diagnostic procedures more suitable for individuals with
visual impairments or conditions like nystagmus''
\citep{haeusler2022processing}.


\subsection{Localizer is "ground truth"?}

We estimate the $Z$-map of a visual localizer which is the fucking established /
traditional method to identify the \ac{ppa}.
%
% localizer data
\citet{sengupta2016extension} successfully delineated the left-hemispheric
\ac{ppa} in 12 of 14 subjects and right-hemispheric \ac{ppa} in 14 of 14
subjects based on localizer data.

% usually, visual PPA works pretty well
The visual \ac{ppa} can be reliably localized using a localizer, which is why we
considered it to be the "gold standard" or benchmark.
%
However, ``the PPA definition may depend on the type of experiment, task, and
stimuli used'' \citep{weiner2018defining}.

%
The primary movie contrast in \citet{haeusler2022processing} ``yielded bilateral
clusters in five participants, a unilateral right cluster in six participants
(of which one participant yielded a unilateral cluster in the visual localizer),
and a unilateral left cluster in one participant. We find bilateral clusters for
participant sub-20, whereas the block-design localizer yielded only one cluster
in the right hemisphere'' \citep{haeusler2022processing};
%
i.e. 5 + 1  left-hemispheric, 5 + 6 right-hemispheric.
%
%
The primary audio-description contrast in \citet{haeusler2022processing}
``yielded bilateral clusters in nine participants that are within or overlapping
with the block-design localizer results. In participant sub-04, two bilateral
clusters are apparent, whereas block-design localizer, and movie stimulus
yielded only one cluster in the right hemisphere.  For another participant
(sub-09) the analysis yielded one cluster in the left-hemispheric PPA''
\citep{haeusler2022processing};
i.e. 9 + 1 left-hemispheric, 9 right-hemispheric.
%
Hence, we should have sampled response vector(s) that carry the spatial response
information (if you know what I mean).


\subsection{Caveats of naturalistic stimuli}

\todo[inline]{imo, you cannot simply transfer the approach to analyse controlled
paradigms to naturalistic stimuli; a.k.a. it's a pain; plus, reliability}

\todo[inline]{reliability of an auditory naturalistic stimulus as a localizer
for an area traditionally localized via visual block-paradigm?}

\todo[inline]{it's not about length (hihi) but more about what is happening
(here: spatial events)}

\todo[inline]{the modeling in \citet{haeusler2022processing}? Or does an
auditory naturalistic stimulus give too much room to participants to do not give
a shit? a.k.a. freely listening is bad? In hindsight: it was a very stupid idea
to do it}

\todo[inline]{make distinction between "dynamic localizer", i.e. blocks of video
snippets vs. event-related modeling of "face-events" or "landscape-events"}

%
Challenging data data analysis, but create \& share annotation.
%
Hence, data analysis pipelines should be implemented in common and
well-documented packages and code, and custom code shared along the paper.

%
Just an approximation of real life.
%
Setting is still the scanner.
%
Passive watching \& listening.
%
Executive functions?


\section{Vision: Atlas}

% examples of probabilistic atlasses: \citet{rosenke2021probabilistic}:
% Cortical atlases have been developed, which allow localization of visual areas
% ``in new subjects by leveraging ROI data from an independent set of typical
% participants: Frost and Goebel 2012;
% ventral temporal cortex (VTC) category selectivity: Julian et al. 2012,
% Zhen et al. 2017, Weiner et al. 2018; visual field maps: Benson et al. 2012,
% Benson and Winawer 2018; Wang et al. 2015''.

\todo[inline]{check discussion of \citet{jiahui2020predicting} as last resource}


``Identifying all of the currently known topographic regions of the human visual
system requires multiple scanning sessions'' \citep{wang2015probabilistic}.
%
``Given the expense and availability of fMRI, this is not always practical''
\citep{wang2015probabilistic}.
%
``For example, time-limitations and subject-fatigue both potentially limit the
time researchers may be able to spend with patients suffering from neurological
or neuropsychological disorders, or with implanted subdural or deep electrodes
(e.g., ECoG)'' \citep{wang2015probabilistic}.
%
The database ``may prove especially useful for predicting functional patterns in
case no localizer data are available, saving scanning time and expenses''
\citep{rosenke2021probabilistic}

%
``An atlas can be used under conditions in which collecting the data to define
maps in individual subjects is impractical or not feasible''
\citep{wang2015probabilistic}.


%
Naturalistic stimuli ``engage in parallel multiple neural systems for vision,
audition, language, person perception, social cognition, and other functions''
\citep{jiahui2020predicting} and offer higher generalizability [and provide
higher validity?] of transformations matrices.

%
A naturalistic stimulus like a move or audio-description could be used to align
a test subject to a \ac{cfs} created from data of a normative reference group.
%
The reference group requires a ``database of data for movies and a range of
functional localizers in a normative group of subjects''
\citep{jiahui2020predicting}.

%
Once a valid alignment is established, known functional properties of the
(normative) reference can then be projected into the respective individual voxel
space (s. Fig. 1 in \citep{nishimoto2016lining}) by mapping a variety $Z$-maps
created from a variety of $t$-contrast from a normative reference group onto an
individual subjects and thus potentially substitute a variety of localizers.

%
``A new subject's functional topographies could be estimated based only on that
subject's movie data and other subjects' localizer data from the normative
database that could be projected into that subject's cortical anatomy using
hyperalignment transformation matrices derived from movie data and could replace
tedious functional localizers with an engaging movie''
\citep{jiahui2020predicting}.

%
``Functional topographies could be mapped from a database containing a wide
range of perceptual and cognitive functions to new subjects based only on fMRI
data collected while watching an engaging, naturalistic stimulus and other
subjects' localizer data from a normative sample'' \citep{jiahui2020predicting}.

% from Jiahu
``From a single movie dataset multiple functional topographies can be estimated
\citep{guntupalli2016model}, whereas different localizers are typically required
to map different functional topographies, making a thorough mapping of selective
topographies time-consuming and inefficient'' \citep{jiahui2020predicting}.

%
That reference would enable an qualitative and quantitative description of an
individual's brain function with respect to such a norm, and consequently
progress the field towards neuroimaging studies of individual differences that
more closely resemble their psychological counterparts.

\paragraph{"Non-compliant" population}

\todo[inline]{quickly check if it makes sense to cite any of the following
papers at all (might be just random papers with clinical population)}

%
Patient populations, such as patients who
%
are blind
%
[Mahon et al. 2009; Bedny et al. 2011; Striem-Amit, Dakwar, et al.  2012b; van
den Hurk et al. 2017] \citep{rosenke2021probabilistic},
%
[Amedi et al., 2007; He et al., 2013; Mahon et al., 2009; Wolbers et al., 2011]
\citep{weiner2018defining}, or
%
have a brain lesion
%
[Schiltz and Rossion 2006; Steeves et al. 2006; Sorger et al. 2007; Barton 2008;
Gilaie-Dotan et al.  2009; de Heering and Rossion 2015]
\citep{rosenke2021probabilistic},
%
individuals with visual agnosia/prosopagnosia
%
[Schiltz and Rossion 2006; Steeves et al. 2006; Sorger et al. 2007; Barton 2008;
Gilaie-Dotan et al. 2009; Susilo et al. 2015] \citep{rosenke2021probabilistic}.


\paragraph{"Deviant" population}

\todo[inline]{Quantify the deviation from the norm vs. predict deviant pattern?}

\todo[inline]{I abandoned the idea to come up with language area (asymmetry);
the topic is clinically more relevant, but problem in case of prediction (esp.
using ROI): most interesting is atypical language lateralization, and there is
usually no lateralization in naturalistic stimuli (but operationalization is
different from localizer paradigms), assumption of (strict) lateralization
probably wrong anyway; templates from papers using \ac{fmri} to localize
language areas are outsourced to separate file}

%
In case the reference is big, it could also allow to investigate subgroups by
providing data populations (pediatric, elderly, psychiatric etc.).

%
``The ability to non-invasively and automatically delineate cortical areas in
living subjects may have clinical implications, for example by providing
neurosurgeons with detailed, individualized maps of the brains on which they
operate'' \citep{glasser2016multi}.

%
A full feature film might substitute traditional localizer paradigms dedicated
localizer by mapping a variety of brain functions beyond category-specific
visual areas.

%
Therefore, a new independent dataset should be collected that employs
naturalistic stimulation.
%
Additional measures of a variety of localizers would enable comparison of
results from analyzing the naturalistic stimuli and results from localizer
contrasts.

%
For example \ac{fmri} could be used as an noninvasive alternative to map
language areas and potentially assess lateralization (or hemispheric asymmetry)
of functional brain topography related to language (sub)functions, in order to
guide pre- and perioperative assessment of neurosurgery, e.g., in case of
epilepsy.

%
``To be useful for clinical diagnostics and prognostics, fMRI data must be
interpretable on the level of the individual case \citep{dubois2016building}.
%
In the clinical context, fMRI plays an important role for planning surgery in
patients with tumors or epilepsies, as it aids the understanding of which parts
of the brain need to be spared in order to preserve sensory, motor or cognitive
abilities'' \citep{wegrzyn2018thought}.



\paragraph{Future studies}

\todo[inline]{a.k.a. how far can you push it (using other algos; ROIs)}

\todo[inline]{this can be very short; condense following to gist of the studies}

\todo[inline]{maybe, cite studies that did research on other brain functions
using naturalistic stimuli; the awesome response vector space should be sampled
by a movie or narrative (language, memory, emotions etc.)o}

\todo[inline]{e.g. \citep{richardson2018development}; just check some reviews}

%
audio-description lacks visual stimulation; interesting in case "executive
functions" are supposed to be sampled/predicted;
%
however \citet{haxby2011common} removed \acp{tr} (+ following \acp{tr} from the
movie that contained, e.g. faces).

``Results show that the computational principles underlying this common model
have broad general validity for representational spaces in occipital, temporal,
parietal, and frontal cortices'' \citep{guntupalli2016model}.
%
So, what is the limit of "brain functions" that we can estimate reliably?
%
Retinotopy, language, executive functions (from low-level perception to higher
cognition, whereas the latter might not be sampled sufficiently by a movie).

%
Asymptotic ``performance curve'' might be different for another brain region
(temporal receptive fields?); retinotopic mapping vs. ``higher'' cognition  vs.
executive functions (prefrontal cortex)?

% Zhen 2017
``Scene-selective regions showed larger interindividual variability [after
nonlinear volume-based alignment] than the face-selective regions in both
spatial topography and functional selectivity'' \citet{zhen2017quantifying}.


% category-specific areas
Similar to nonlinear volume-based alignment, similarity across person is higher
after surface-based alignment ``for retinotopically defined regions, with
character-selective regions showing the lowest consistency for both alignments,
closely followed by mFus- and IOG-faces'' \citep{rosenke2021probabilistic}.

%
``We localized 13 widely studied functional areas and found a large variability
in the degree to which functional areas respect macro-anatomical boundaries
across the cortex'' \citep{frost2012measuring}.
%
``The percent gain in overlap [after surface-based alignment] differed greatly
across the different functional regions throughout the cortex''
\citep{frost2012measuring}.
%
``There is a strong structural-functional correspondence in some areas whilst in
others the spatial location of the functional area varies greatly across
subjects within a cortical area'' \citep{frost2012measuring}.
%
``There is a surprising amount of variability in that not all functional areas
are tightly bound to anatomical landmarks'' \citep{frost2012measuring}.
%
``Surface-based alignment is able to separate functional regions which are more
prone to blur together if individual anatomical curvature patterns are not
accounted for'' \citep{frost2012measuring}.

%
``Some areas, such as the frontal eye fields (FEF) are strongly bound to a
macro-anatomical location.
%
``Both the sensory and motor hand areas (bank of the central sulcus) were much
better aligned after surface-based alignment'' \citep{frost2012measuring}.
%
``Language areas were found to vary greatly across subjects whilst a high degree
of overlap was observed in sensory and motor areas'' \citep{frost2012measuring}.

%
``The area which shows the most gain in overlap of these regions is V5 / hMT+
with 70.9\% gain in the left hemisphere and 55.6\% gain in the right
hemisphere'' \citep{frost2012measuring}.
%
``Area LOC also showed increased overlap after CBA with a 62.7\% gain in the
left hemisphere and 38.4\% on the right'' \citep{frost2012measuring}.
%
Finally PPA exhibit more gain in the right hemisphere with 27.7\% gain, than on
the left with 17.6\%'' \citep{frost2012measuring}.
%
``FFA varies in its location along the length of the fusiform gyrus even though
the gyri themselves are well aligned across subjects''
%
The FFA did not exhibit the same strong structural-functional correspondence and
saw more modest increases in overlap after macro-anatomical alignment with
44.1\% and 12\% gain for the left and right hemispheres''
\citep{frost2012measuring}.


\section{General conclusion (= naturalistic stimuli in general?)}
%
Naturalistic stimuli are not a panacea but traditional paradigms and
naturalistic paradigms should be used in tandem / reciprocally to generate new
hypotheses and progress our understanding of the brain.

%
In summary, naturalistic stimuli ``impose a meaningful timecourse across
subjects while still allowing for individual variation in brain activity and
behavioral responses, and lend themselves to a broader set of analyses than
either pure rest or pure event-related task designs'' \citep{finn2017can}.
%
``Naturalistic paradigms do not aim to replace the classic, controlled
neuroimaging paradigms (Sonkusare et al., 2019). Due to their complexity and
current limitations in understanding the statistical properties of different
features in naturalistic conditions, naturalistic stimuli are not optimal for
model development [see, e.g., Rust and Movshon, 2005]. Controlled experiments
are still needed for hypothesis testing and developing models, while
naturalistic stimuli are best employed to test models in ecologically valid
settings and to expand them to situations where context matters
more'' \citep{saarimaki2021naturalistic}.
