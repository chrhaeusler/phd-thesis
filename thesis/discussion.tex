\todo[inline]{most importantly: is anything missing that needs to be discussed?}

\todo[inline]{delete subsection(s); delete pagebreaks}

Human brain mapping studies have traditionally averaged \ac{fmri} data across
participants.
%
In order to advance the field towards a clinical application, data need to be
assessed on the level of individual persons.
% functional localizer
Functional localizers are an established method to describe the topography (i.e.
the location, size, and shape) of functional areas on the level of individuals
are functional localizers.
% contra localizers
However, traditional localizer paradigms employ selectively sampled, tightly
controlled stimuli, rely heavily on a participant's compliance, and can usually
map only one domain of brain functions.
% movies & narratives
Naturalistic stimuli like movies and auditory narratives \citep[cf.][for
reviews]{jaaskelainen2021movies, jaaskelainen2020neural} offer a time-locked
event structure that samples a variety of brain functions ranging from low-level
perception (e.g., luminance) to high-level cognition (e.g., social cognition).
%
A localizer based on naturalistic stimuli could map a variety of brain functions
and would offer higher external validity because naturalistic stimuli more
closely resemble how we perceive the real world outside of the laboratory.

% goal of thesis
Consequently, the purpose of this thesis was---while adhering to the principles
of open, transparent, and reproducible science---to investigate whether a movie
and the movie's audio-description may, in principle, replace a traditional
localizer paradigm.
% PPA as proof of concept
As a proof of concept, we focused on the \ac{ppa}, a ``classic'' higher visual
area.
%
The \ac{ppa} exhibits increased hemodynamic activity when participants view
photos of landscapes, buildings or landmarks, compared to, for instance, photos
of faces or tools \citep[e.g.,][for reviews]{epstein2014neural,
aminoff2013role}.
%
Moreover, results of \citet{aziz2008modulation}, who compared hemodynamic
activity levels in the \ac{ppa} correlated with different categories presented
in spoken sentences, revealed that semantic scene-related information also
modulates the \ac{ppa}'s activity level.

\todo[inline]{drop the second way}
%
We assessed the potential of both the movie and the audio-description to replace
a visual localizer two different ways.
% direct modeling
First, we modified the procedure for the analysis of data from a standard
localizer paradigm to data from the two naturalistic paradigms:
%
hemodynamic responses correlating with the temporal structure of annotated
stimulus features \citep[cf.][]{haeusler2016cutanno, haeusler2021speechanno}
were modeled in order to create \ac{glm} $t$-contrasts that aimed to localize
the \ac{ppa}.
% estimation
Second, we applied functional alignment procedure as a novel method in order to
estimate results from the visual localizer \citep[cf.][]{sengupta2016extension},
movie and audio-description \citep[cf.][]{haeusler2022processing} in one
participant from results of participants in a reference group.



\pagebreak


\section{Open Science}

\todo[inline]{it doesn't matter where I place the part on "open science": it's
an excursion \& doesn't fit in anywhere perfectly}

%
An overarching objective of this dissertation was to satisfy the standards of
open, shared, accessible, and transparent science \citep[cf.][]{watson2015will,
fecher2014open} as well as the standards of a reproducible and replicable
research project.
%
This goal included to a) using open data and open-source software, and b)
publishing data, materials, code, and results openly available.


\subsection{Using open data and open-source software}

%
The first subgoal was to use open data, open materials, and open-source
software.
%
The current thesis capitalized on publicly available
%
\ac{fmri} data \citep{hanke2014audiomovie, hanke2016simultaneous,
sengupta2016extension},
%
subject-specific \acp{roi} \citep{sengupta2016extension} and
%
stimulus annotations \citep{haeusler2016cutanno}
%
that are part of the studyforrest project
(\href{www.studyforrest.org}{\url{studyforrest.org}}).
%
All analyses are implemented in open-source software packages to prevent the
creation of a ``artificial paywall'' for running the analyses again on the
initially openly accessible data.\todo{FSL is not OS?}.
%
We benefited from established free software like
%
Python and
%
FSL \citep[\href{https://www.fmrib.ox.ac.uk/fsl}{FMRIB's Software
Library;}][]{smith2004fsl} that have been developed and debugged since years by
an collaborative effort,
%
but also from scientific software packages like
%
DataLad
\citep[\href{www.datalad.org}{\url{datalad.org}};][]{halchenko2021datalad} or
%
BrainIAK
\citep[\href{https://brainiak.org}{\url{brainiak.org}};][]{kumar2020brainiak,
kumar2020brainiaktutorial}
%
that emerged recently.

%
Preexisting software packages, data, and results from previous analyses enabled
us shift time and resources from software development and data collection to
subsequent stages of the project.
%
The already existing data were immensely valuable, when the project plan had to
be refocused owing to the COVID-19 epidemic that severely collecting \ac{fmri}
data from new subjects.
%
Moreover, it was very advantageous to have already been involved in the data
collection and being rather familiar with the data's location, accessibility,
storage format and already performed preprocessing step.

% issue 1: data quality
However, an issue of open data that is all too frequently overlooked is the fact
it does not exempt the ``data consumer'' from the responsibility of carefully
checking data's underlying experimental paradigm (e.g., stimuli and code) and
the data themselves for potential errors.
%
It is too tempting to ``just push the data through an analysis pipeline'' as
soon as possible without carefully assessing the quality of the data first.
% check the data
Consumers of datasets must presume that anything not specified clearly in the
description of a dataset has not been taken into account.
% standards may differ
Moreover, standards (quality, formats, parameters) and open sciences practices
(e.g., documenting) may differ across scientific field, or even within a
scientific field depending on a working group's expertise and rigor.
% laugh with many, don't trust any
Even if the data are provided by a reputable source, researchers that consider
using third-party data should also consider themselves to be obliged to test and
validate a dataset's quality as if collected by themselves, and in accordance
with their standards and particular use case.
%
In this regard, open data---despite being collected to the best of knowledge and
belief---could be compared to open-source software:
% super fancy literal quote
Data will contain noise, errors, or artifacts as software will contain bugs but
``given enough eyeballs, all bugs are shallow'' \citep[][p.
30]{raymond1999cathedral}.


% issue 2: decisions were made
Another issue of open data is that the decisions made during data collection,
preprocessing, and further analyses might influence or even limit subsequently
performed analyses.
%
For instance, we and \citet{sengupta2016extension} selected the \ac{ppa} among
possible candidates of scene-selective area because the \ac{ppa} was the first
scene-selective area to be discovered and is the most reliably activated region
across studies that investigate visual scene perception.
%
Like the \ac{ppa}, the \ac{rsc} and \ac{opa} have repeatedly shown increased
hemodynamic activity in studies investigating visual spatial perception and
navigation \citep{chrastil2018heterogeneity, bettencourt2013role,
dilks2013occipital, epstein2019scene}.
%
Although we did not explicitly hypothesize it, we assumed that at least the
analysis of the movie might yield significant clusters in the medial parietal
and lateral occipital cortex that might correspond to the \ac{rsc} and \ac{opa}
respectively.
%
Indeed, results revealed significantly increased activity in the medial parietal
and lateral occipital cortex, and are an incentive for further studies.
% \citep[cf. algorithmic procedure in, e.g.,][]{julian2012algorithmic}.
However, in case results from \citep{sengupta2016extension}, one would have to
replicate the non-automatized procedure of \citep{sengupta2016extension} in
order to create the corresponding masks.
%
This example shows how decisions made during data collection, preprocessing or
preceding investigations---despite being state-of-the-art at the time of being
published---are affecting subsequent studies.
% pro & contra: opportunity costs
Hence, when considering using open data, researchers need to weigh the costs and
benefits of one option (such as using preprocessed data as provided) relative to
an alternative option (such as preprocessing raw data differently than
provided), then choose the option that will yield the highest net return.
%
In summary, any previous step that required human intervention or has not been
fully automated influences the degree to which data or materials can be
replicated, updated, or extended.




\subsection{Publishing data, materials, code, and results openly available}

\todo[inline]{speech anno paper \& ppa paper are on github and public}

\todo[inline]{thesis is on github but not public}

\todo[inline]{state that and the URLs here?}
%
%The annotation comprises, e.g., time-stamps of phonemes, words and sentences of
%all speakers, a grammatical tagging, and an annotation of syntactic
%dependencies and semantics.

%
The second subgoal was to publish data, materials, and results openly available.
%
Data and custom code created for this dissertation are version-controlled, i.e.
modifications to the data and code were logged and documented, in order to
promote transparency.
%
In order to promote reproducibility, processing steps ranging from downloading
input data to plotting figures are implemented in scripts that can be rerun from
the command line.
%
The speech annotation has been made publicly available
\citep{haeusler2021speechanno} and its content goes beyond what was required to
conduct the analyses in \citet{haeusler2022processing}.
%
The annotation serves as an extension of the studyforrest project and widens the
``annotation bottleneck'' \citep[][p.  16]{aliko2020naturalistic} of two
naturalistic stimuli, and might give independent researchers a head start in
modeling hemodynamic brain responses that are correlated with features of spoken
language.
%
In \citet{haeusler2022processing}, we used open \ac{fmri} data to look into a
research subject that was not anticipated when the data were first made
available.
%
Results of \citet{haeusler2022processing} indicate that increased hemodynamic
activity in the \ac{ppa} generalizes from blocks of pictures to spatial
information embedded in a movie and an auditory narrative, and demonstrate the
advantages of sharing and reusing data.

% contra
From a negative perspective, creating data, materials, and code to be published
requires a considerable about of time and effort.
%
To encourage third parties to reuse the data, dataset creators must anticipate
potential use cases, collect the data with appropriate extent and rigor, convert
the data into a standardized format (taking into account, for example, naming
conventions and folder structure).
% automation
Analyses pipelines need to be designed and tested in a way that they can
reliably and replicate every stage stage of a dataset.
% publication: findable, accessible, interoperable, reusable
Additionally, dataset creators must take into account legal matters (such as
intellectual property rights, use licenses, statements of agreement, and
anonymization of participant data), facilitate discovery by humans and web bots
(e.g., by including detailed descriptions and machine-readable metadata), and
guarantee long-term curation and accessibility an appropriate data host.

%
From a positive perspective, a researcher gains immediate benefits from creating
a dataset that is intended to be published.
%
Meticulously recording each step and commenting the advantages and disadvantages
of alternate procedural options results in deeper understanding of the
scientific area, its practices, and methods.
%
The version-control of every step reduces the likelihood of a look-ahead bias.
%
Tracking and extensively documenting each stage of the data and code from the
beginning to the final results can also be thought of a lab protocol that
comprises structured information for writing the corresponding scientific
article.
%
Hence, creating a dataset supposed to be published encourages precise work
habits and good scientific practices in general.


\subsection{Interim summary}

% Open access publications might receive more citations than paywalled
% publications [\citep{piwowar2018state}], open data might get cited, and
% promote new collaborations [\citep{popkin2019data}].

% incentives like ``professional recognition or the allocation of extra funding
% [Kidwell et al., 2016; Fecher et al., 2015; Ali-Khan et al., 2018];
% Funding agencies already require publication of findings in OA schemes and
% data-sharing plans [Neylon, 2017]'' \citep{toribio2021early}.

%
In summary, pursuing the objective of conducting an open and reproducible
research project was not a requirement for submitting the thesis and required
considerably additional work and time.
%
Since open science practices are not yet covered in graduate or PhD curricula,
learning about principles and standards, as well as putting them into practice,
relied on self-initiative and self-learning.
%
Steadily emerging standards and principles and steadily developing software
packages to apply said standards impeded putting theoretical knowledge into
practice.
%
The time and effort needed, in my opinion, greatly exceed any short-term
advantages:
%
The lengthy procedures are not justified by ``gambling'' on being cited in the
event that released data are reused or by merely pursuing the ``higher purpose''
of addressing the replication crisis.
%
Particularly, designing and testing fully automated analysis pipelines or
scripts to plot complex figures without any manual finishing for the perceived
sole purpose of reproducibility is out of proportion to the immediate benefit
of, e.g., easier bug tracking or simply ``higher confidence in one's own work''.
%
On the contrary, PhD students that pursue a career in science are faced with the
concern of exposing themselves to critique owing to a maximum of voluntary
transparency and possibly (and blamelessly?) overlooked errors.
%
Another concern is the potential of being ``scooped'', i.e. the risk that
another working group is using the same data for a similar research question at
the same time and eventually claiming priority to the research idea and its
findings \citep[cf.][]{laine2017afraid}.
%
This risk is aggravated in case of early-career scientists that created and
maintain a public dataset, pre-registered studies based on a public dataset, or
have to adhere to inflexible project plans.
%
Hence, undergraduate programs should teach risks, benefits and best practices of
open science and provide practical training in related software packages;
%
postgraduate programs should create incentives to conduct open science projects.
% Instead of compulsory requirements from funders, which might only lead
% researchers to show minimal compliance [Neylon, 2017].
%
After all, open sciences is a suitable tool to
%
a) hold researchers responsible for collecting, storing, documenting,
processing, and publishing data and materials in accordance with best practices,
%
b) increase the reproducibility of results and reliability of findings,
%
c) make knowledge and technologies widely accessible,
%
d) therefor increase the efficiency of the scientific progress and promote
innovation, and
%
e) ultimately increase the public's trust in the scientific process and its
findings.

\todo[inline]{just bold statements at the end, without any references but
probably okay}





\pagebreak

\section{Naturalistic stimuli for functional localization?}

\todo[inline]{if excursion about open science makes the switch back to
naturalistic stimuli confusing, start more generally by repeating some stuff
from the general discussion's intro or \citet{bartels2004mapping} as hook)}

\todo[inline]{mih: worth stating again somewhere in the discussion that the
studyforrest dataset is not for this (diverse), but it is a small dataset, which
limits the generality; coh: I don't know where this point could be mentioned}

%
The \ac{ppa} is traditionally identified by contrasting hemodynamic responses to
blocks of pictures of scenes or landscapes to, e.g., blocks of pictures of tools
or faces.
%
Although the exact outline of the \ac{ppa} varies depending on the type of
stimuli, task, and contrast (as well as statistical threshold), the traditional
localizer approach can reliably delineate the \ac{ppa} bilaterally in a large
proportion of subjects \citep{zhen2017quantifying}.
%
\citet{sengupta2016extension}, for instance, were able to identify the
left-hemispheric \ac{ppa} in 12 of 14 subjects and right-hemispheric \ac{ppa} in
14 of 14 subjects based on localizer data.


\subsection{Functional localization via modeling hemodynamic responses}

\todo[inline]{Two separate parts have been merged; structure might be improvable
(redundancies and stuff)}


\subsubsection{Recapitulation of current thesis}

% study in one sentence
In \citet{haeusler2022processing}, we explored whether the \ac{ppa}, which had
previously been identified in the same group of participants by a conventional
block-design functional localizer using static pictures
\citep{sengupta2016extension}, could be localized using an audio-visual
naturalistic stimulus and an exclusively auditory naturalistic stimulus.

\paragraph{Method}

% stress similarity to localizer approach
We adapted the approach of traditional localizers and modeled hemodynamic
responses to events in two naturalistic stimuli to create $t$-contrasts that
aimed to localize the \ac{ppa}.
% AV operationalization
For the model-based mass-univariate statistical analysis (i.e. \ac{glm}) of the
movie, we capitalized on an annotation of movie cuts and depicted locations
\citep{haeusler2016cutanno}.
% AD operationalization
For the \ac{glm} of the audio-description, we extended the annotation of speech
that we created and validated in \citep{haeusler2021speechanno} by further
annotating nouns that the audio-description's narrator uses to describe lacking
visual content.


\paragraph{Results}

% group results: generally
On a group-average level, results demonstrate a model-driven \ac{glm} analysis
based on a naturalistic stimulus' annotation may replicate findings of studies
that employed traditional paradigms, and add evidence
\citep[cf.][]{bartels2004mapping} that functional specialization of cortical
areas is maintained during naturalistic stimulation.
% group results: PPA specifically
Further, results demonstrate that increased activation in the \ac{ppa} during
the perception of static images generalizes to the perception of spatial
information embedded in an audio-visual movie and exclusively auditory
naturalistic stimulus \citep{haeusler2022processing}.
% individual results
On an individual level, the analysis of the movie yielded bilateral clusters of
increased hemodynamic activity in the \ac{ppa} of five participants and a
unilateral cluster in seven participants, whereas the analysis of the
audio-description revealed bilateral clusters in nine participants and one
unilateral cluster in one participant.
% conclusion
Results indicate that a movie and an exclusively auditory naturalistic stimulus
could potentially replace a visual localizer to assess brain functions in
individuals.
%
Due to the multi-dimensionality and richness of naturalistic stimuli, one
naturalistic paradigm might, in theory, replace a number of localizers while
simultaneously offering a more ecologically valid stimulation.
%
However, the multi-dimensionality comes at the expense of a challenging
model-driven data analysis that impedes the development of a ``multi-functional
localizer''.


\subsubsection{...and the from current thesis derived general discussion}


\paragraph{Origin and assumptions of GLM}

\todo[inline]{check the references and improve phrasing}

%
The currently dominant \ac{glm}, which has its roots in \ac{pet} research, is
tailored to analyze data of parametric\todo{Def?} experimental designs that
manipulate isolated experimental variables of interest.
%
The \ac{glm} requires the researcher specify which stimulus features are
presumably correlated with the brain process under investigation.
%
Then, the researcher needs to model a hypothesized hemodynamic time course that
is fit to the data in order to predict the observed hemodynamic activity and
contrast parameter estimates \citep{friston1998event}.
%
Depending on the investigated perceptual or cognitive process, the continuity
and complexity of naturalistic stimuli stress physiological assumptions like
%
a) the consistency of responses across events \citep[the rationale behind
\textit{trial-averaging};][]{dale1997selective},
%
b) the linearity of the hemodynamic responses [Cohen (1997). Parametric Analysis
of fMRI; Boynton (1996). Linear systems analysis; Dale (1999). Optimal
experimental design for event-related fmri]
%
c) and cognitive subtraction [Friston (1996)].


\paragraph{Annotations}
%
Further, naturalistic stimuli stress statistical assumptions like the absence of
collinearity among variables.
%
Detailed annotations can alleviate the lack of experimental control over
stimulus features by enabling modeling and statistically controlling potentially
confounding variables \citep[e.g.,][]{deniz2019representation}.
%
Low-level visual features like brightness\todo{or is it luminance???} and
low-level auditory features like root-mean square power (i.e., volume) can be
automatically extracted on a low temporal scale (e.g., per movie frame; cf.
\href{https://github.com/psychoinformatics-de/studyforrest-data-confoundsannotation
}{\url{github.com/psychoinformatics-de/studyforrest-data-confoundsannotation}}
for low-level annotations of the studyforrest dataset).
%
Recent advances in machine learning allow the creation tools \citep[e.g., the
toolbox ``pliers'' that is implemented in the
\href{https://neuroscout.org/}{\url{neuroscout.org}}
platform;][]{mcnamara2017developing, delavega2022neuroscout} that can
automatically extract higher-level features like semantics or clearly defined
object categories.
%
Such tools can replace time-consuming manual annotations or provide a
provisional scaffold that reduces manual labor for a growing number of stimulus
features.
%
However, ambiguous variables, variables that fluctuate on longer time scales, or
subject-related variables like the level of engagement or felt emotions
\citep[cf., e.g.,][]{lettieri2019emotionotopy, saarimaki2021naturalistic} still
defy an automatic annotation.
%
Hence, merely assessing the confound structure of a naturalistic stimulus as an
possible candidate for a prospective model-driven analysis might become
time-consuming.
%
Depending on the covariance of variables, a naturalistic stimulus---if not all
naturalistic stimuli---might eventually turn out be inappropriate for testing a
specific experimental hypothesis given current statistical methods.
%
Moreover, a large amount of annotated features might lead to annotations that
are ``hard to use'' \citep[][p. 2]{richard2019fast} or ``high-dimensional,
cumbersome models'' \citep[][p. 2]{richard2019fast} that eventually lack
statistical power due to insufficient data samples.


\paragraph{Lack of task: ``Demons to some, angels to others'' --- Pinhead}

\todo[inline]{maybe, add that auditory stimulation is ``less crowded'' /
``less rich'', more ``linear'' than the content of movie frames but level of
engagement is harder to assess}

%
The individual-level results of \citet{haeusler2022processing} also indicate
that a lack of task can be viewed as a dubious advantage.
%
On the one hand, the absence of a task reduces the need for instructions,
training, and generally relieves the participants of a burden.
%
On the other hand, ``freely viewing'' a movie or ``freely listening'' to an
auditory narrative requires attending to the paradigm, which some participants
may find difficult to sustain over longer periods of time.
%
On a group-level, the occasional disengagement of single subjects may have a
negligible effect on group average results, especially when statistics are
calculated over long stimulus intervals.
%
On an individual level, a fluctuating level of engagement is difficult to
measure and statistically control, negatively impacts reliability across
stimulus segments, and may ultimately obscure effects of interest.
%
The level of engagement, however, might also be an reflection of a personal
preference towards the presented stimulus or of a personal trait.
%
In summary, researcher always need to consider carefully advantages and
disadvantages of naturalistic stimuli and carefully assess
%
Therefor, a researcher needs to consider these advantages and disadvantages of
naturalistic stimuli in general and evaluate a naturalistic stimulus among
different candidates.


\subsubsection{Interim summary}

\todo[inline]{Still begging the question: WTF does the PPA actually do? What is
the construct?}

%
Given the current challenges and limitations, it is probably not surprising
that, even 20 years after the group-level findings of
\citet{bartels2004mapping}, no functional localizer exists that is based on a
movie or an auditory narrative.
%
Today, the ecologically ``most valid'' localizers are so called \textit{dynamic
localizers} \citep{pitcher2011differential, fox2009defining} that employ blocks
of short videos (each video lasting $\approx$\unit[2-3]{s}) of scenes, faces
etc., and thus lend themselves conveniently to traditional modeling procedures.
%
However, even naturalistic stimuli, despite being labeled as ``naturalistic''
and ubiquitously labeled as ``ecologically more valid'', are not in a strict
sense ``naturalistic'', but rather an approximation of real-life that presented
in the laboratory on a screen and/or via headphones.
% entertain, inform, or advertise.
Similar to traditionally used stimuli that have been carefully designed by
researchers to probe specific brain processes, most naturalistic stimuli used in
neuroscience have been carefully designed by professional media creators to
appeal to their target audience.
%
Film directors intentionally manipulate the viewers' attentional focus and
mental states using a variety of techniques like camera-movement, composition,
movie editing, voice-overs \citep{brown2012cinematography,
dancyger2011film-technique, katz1991film, mercado2011filmmakers} that, when used
correctly, largely occur unnoticed.
%
For example, participants asked to spot movie cuts miss between 10\% and 50\% of
them depending on the type of cut \citep{smith2008edit}.
%
These techniques not only result in reduced individual variation and thus
reliably synchronized spatial-temporal responses across subjects in a large part
of the cortex \citep{hasson2008neurocinematics}, but also in ``naturalistic
stimulus statistics'' (i.e. a medium-specific confound structure) that differ
from ``real-world statistics''.


\paragraph{Recommendations}

%
In summary, naturalistic stimuli might be considered to provide a higher
ecological validity, but at the expense of a challenging data analysis and
interpretability of results.
%
Despite the growing popularity of naturalistic stimuli, a researcher must
carefully weigh the advantages and disadvantages of naturalistic stimuli in
general.
%
To determine whether a naturalistic stimulus is an appropriate candidate for a
given study, researcher should become acquainted with media techniques and how
they are applied to craft professional media products.
%
In particular, a naturalistic stimulus' confound structure should be quantified
before launching a full study.
%
The covariance of variables of interest and confound variables must be
evaluated, with some, if not most, naturalistic stimuli ultimately deemed
inappropriate to investigate some aspect of perception and cognition under
current modeling approaches.
%
In comparison to traditional paradigms, naturalistic stimuli are designed to be
naturally engaging on their own.
%
However, the lack of a task allows for more individual variation, which is
difficult to record or self-report and thus difficult to statistically
control which is problematic when individual variation is viewed as noise
%
Therefore, \ac{fmri} scanning session should be accompanied by physiological
recordings (e.g., skin conductance response, heart rate) and ideally by
\ac{eeg} or eye-tracking, and combined with follow-up self-reports (e.g.,
regarding alertness, engagement, an audio track's audibility within the noisy
scanner).



\subsection{Functional localization via estimation from reference group}

\todo[inline]{write when SRM part is finished}

\todo[inline]{explicitly (re)state aims}

%
Results from \citet{haeusler2022processing} suggest that the movie as well as
the audio-description sample hemodynamic responses that correlate with the
occurrence of spatial information.
%
Therefor, we did the following ...

% goal 1: new procedure
We estimated results of a dedicated localizer \citep{sengupta2016extension} via
functional alignment from results of a reference group.
% the problem
Considering practical and monetary constraints in a clinical context, a paradigm
lasting 90 to 120 minutes is inappropriate for even an extensive individual
diagnostic procedure, we also assessed the relationship between length of
naturalistic stimulation used to align the test participant to the fixed
\ac{cfs} and the estimation performance.



\subsubsection{Interim summary and future studies}

\todo[inline]{This topic is only touched in the SRM study's discussion}

\todo[inline]{probably, it is better to discuss it here more than in SRM part}

\todo[inline]{Following are some "templates" outlining the general idea}

% examples of probabilistic atlasses: \citet{rosenke2021probabilistic}:
% Cortical atlases have been developed, which allow localization of visual areas
% ``in new subjects by leveraging ROI data from an independent set of typical
% participants: Frost and Goebel 2012;
% ventral temporal cortex (VTC) category selectivity: Julian et al. 2012,
% Zhen et al. 2017, Weiner et al. 2018; visual field maps: Benson et al. 2012,
% Benson and Winawer 2018; Wang et al. 2015''.


\paragraph{Problem space}

%
``Identifying all of the currently known topographic regions of the human visual
system requires multiple scanning sessions'' \citep{wang2015probabilistic}.
%
``Given the expense and availability of fMRI, this is not always practical''
\citep{wang2015probabilistic}.
%
``Our approach has the potential to estimate an unlimited variety of functional
topographies at the individual level based on the responses to a single
naturalistic, dynamic stimulus'' \citep{jiahui2020predicting}.


\paragraph{Database}

%
``A normative database of participants who were scanned during movie viewing and
during functional localizers for different perceptual and cognitive functions
would serve as a reference'' \citep{jiahui2020predicting}.
%
``There are numerous other functional localizers in other perceptual and
cognitive domains, such as simple visual motion, biological motion, tonotopy,
voice perception, music perception, language, calculations, working memory, and
theory of mind'' \citep{jiahui2020predicting}.

%
``A database from a normative group could allow researchers to estimate new
subjects' functional topographies by collecting only a movie-viewing data set
and then deriving the individualized topographies with the normative database''
\citep{jiahui2020predicting}.
%
The database ``may prove especially useful for predicting functional patterns in
case no localizer data are available, saving scanning time and expenses''
\citep{rosenke2021probabilistic}.


\paragraph{Calibration}
%
A naturalistic stimulus like a move or audio-description could be used to align
a test subject to a \ac{cfs} created from data of a normative reference group.
%
Naturalistic stimuli ``engage in parallel multiple neural systems for vision,
audition, language, person perception, social cognition, and other functions''
\citep{jiahui2020predicting} and offer higher generalizability [and provide
higher validity?] of transformations matrices.

%
''Some high-level cognitive processes, such as calculation, working memory, and
logical reasoning, may be less well sampled by movie viewing, and further work
is necessary to test whether hyperalignment based on movie-viewing data can be
used to estimate topographies for these other domains of high-level cognition.
'' \citep{jiahui2020predicting}.


\paragraph{Application: estimation}

%
Once a valid alignment is established, known functional properties of the
(normative) reference can then be projected into the respective individual voxel
space by mapping a variety $Z$-maps created from a variety of $t$-contrast from
a normative reference group onto an individual subjects and thus potentially
substitute a variety of localizers.
%
``A new subject's functional topographies could be estimated based only on that
subject's movie data and other subjects' localizer data from the normative
database that could be projected into that subject's cortical anatomy using
hyperalignment transformation matrices derived from movie data and could replace
tedious functional localizers with an engaging movie''
\citep{jiahui2020predicting}.

%
``Investigators would need to scan their participants only during movie viewing
and a wide range of idiosyncratic functional topographies could then be
estimated individually based on localizer data projected from the brains in the
normative sample into the new participants' cortical anatomies''
\citep{jiahui2020predicting}.
%
``Functional topographies could be mapped from a database containing a wide
range of perceptual and cognitive functions to new subjects based only on fMRI
data collected while watching an engaging, naturalistic stimulus and other
subjects' localizer data from a normative sample'' \citep{jiahui2020predicting}.

%
''Because naturalistic movies include people, human actions, conversations,
social interactions, background music etc., we predict that hyperalignment
transformation matrices based on these movies also will work for localizers of
functional topographies for audition, language, and social cognition''
\citep{jiahui2020predicting}.
%
``From a single movie dataset multiple functional topographies can be estimated
\citep{guntupalli2016model}, whereas different localizers are typically required
to map different functional topographies, making a thorough mapping of selective
topographies time-consuming and inefficient'' \citep{jiahui2020predicting}.


\paragraph{Application: deviation (a.k.a. clinical application)}

\todo[inline]{cf. general introduction (on "individual neuroscience")}

\todo[inline]{cf. SRM discussion: quantify deviation from a norm vs.  predict
deviant pattern?}


\todo[inline]{\citet{silva2018challenges, szaflarski2017practice}}

%
That reference would enable an qualitative and quantitative description of an
individual's brain function with respect to such a norm, and consequently
progress the field towards neuroimaging studies of individual differences that
more closely resemble their psychological counterparts.


``In the clinical context, fMRI plays an important role for planning surgery in
patients with tumors or epilepsies, as it aids the understanding of which parts
of the brain need to be spared in order to preserve sensory, motor or cognitive
abilities'' \citep{wegrzyn2018thought}.


\paragraph{Language lateralization}

\todo[inline]{I abandoned the idea to come up with language area asymmetry (at
least in the SRM study); the problem in case of prediction is that most
interesting are cases of atypical language lateralization; problem: the models
don't model individual component (except, maybe,
\citet{feilong2022individualized}) and ROIs and searchlight spheres are too
small}

\todo[inline]{templates from papers using \ac{fmri} to localize language areas
and discuss atypical language organization are outsourced to separate file}

%
For example \ac{fmri} could be used as an noninvasive alternative to map
language areas and potentially assess lateralization (or hemispheric asymmetry)
of functional brain topography related to language (sub)functions, in order to
guide pre- and perioperative assessment of neurosurgery, e.g., in case of
epilepsy.



\section{Conclusion}

\todo[inline]{Similarly to the summary / abstract, the general conclusion
belongs to the parts of the thesis written at the very end; here are some rough
ideas}

\subsection{Naturalistic stimuli}

%
Naturalistic stimuli ``impose a meaningful timecourse across subjects while
still allowing for individual variation in brain activity and behavioral
responses, and lend themselves to a broader set of analyses than either pure
rest or pure event-related task designs'' \citep[][p. 142]{finn2017can}.

%
In case, minimization of intersubject-variability and reliability in all
subjects is the main goal, stick to classic localizers in a practical setting;
in a research setting, adapt traditional methods and develop new methods to
analyze data from naturalistic stimuli.
%
Naturalistic stimuli might provide a higher ecological validity.
%
However, this is at the expense of experimental control, aggravated in case of
subject-related variables that are harder to record and therefor statistically
to control than stimulus features that are time-locked and replicable.
%
The present thesis's results suggest this is at the expense of increased
within-subject variability in some subjects.
%
In case reliability is the key and findings are supposed to represent general
mechanism common to all humans, this is shitty.
%
In case maximization of within-subject and intersubject-variability, this might
be desirable.

%
Naturalistic stimuli are not a panacea but traditional paradigms and
naturalistic paradigms should be used in tandem / reciprocally to generate new
hypotheses and progress our understanding of the brain.

%
``Caveats can be avoided by utilization of artificial and naturalistic stimulus
paradigms in parallel (e.g., by using a continuum of stimuli from artificial to
naturalistic across experiments in the same subjects) and by modeling both
factors of interest (e.g., self-reports of emotional experiences) and potential
nuisance factors (e.g., shot sizes in the movie) and taking them into account in
the analyses'' \citep{jaaskelainen2021movies}.
%
``Naturalistic paradigms do not aim to replace the classic, controlled
neuroimaging paradigms \citep{sonkusare2019naturalistic}.
%
Due to their complexity and current limitations in understanding the statistical
properties of different features in naturalistic conditions, naturalistic
stimuli are not optimal for model development [see, e.g., Rust and Movshon,
2005].
%
Controlled experiments are still needed for hypothesis testing and developing
models, while naturalistic stimuli are best employed to test models in
ecologically valid settings and to expand them to situations where context
matters more'' \citep[][p. 19]{saarimaki2021naturalistic}.

%
``Combining more traditional controlled experimental designs, such as ToM and
other localizer tasks, with use of movies and narratives represents another
highly promising research direction, as does expanding naturalistic stimulation
from movie clips and narratives to virtual reality and computer-game/simulated
environments'' \citep{jaaskelainen2021movies}.


%
``Modeling the stimulus and task properties is a central challenge with
naturalistic paradigms and one that has great potential if solved [Simony and
Chang, 2020].
%
Compared to controlled paradigms, naturalistic stimuli allow investigating
multiple functional components simultaneously, and studying individual
variation'' \citep{saarimaki2021naturalistic}.

%
``Developments in neuroimaging and in complementary behavior- and data-analysis
methods hold keys to advancing rapidly to even more robust use of naturalistic
stimuli'' \citep{jaaskelainen2021movies}.


%
``As is the case with any scientific field, the introduction of new
methods requires independent validation and the reproduction of findings from
previous studies before embarking upon higher aspiration and practical
applications.
%
Exploratory studies and ``methodological developments are still needed to
account for the complex interaction dynamics, such as feedback loops, between
different component processes [e.g., Pessoa, 2018]''
\citep{saarimaki2021naturalistic}.


\paragraph{Example: auditory PPA is in anterior visual PPA}

\todo[inline]{This does not fit into the general discussion well anymore
anywhere; maybe, shift it to SRM study}

\todo[inline]{If it is given as an example here; shorten it heavily}

%
However, the usability of a purely auditory paradigm to localize the \ac{ppa} as
it is defined by a visual paradigm might be limited by our findings that suggest
that increased hemodynamic activity during auditory stimulation is spatially
restricted to the anterior \ac{ppa}.
%
Our results provide further evidence to studies in the field of
visual perception that suggest that the \ac{ppa} can be divided into
functionally subregions that are coactivate during the perception of visual
scenes but process different stimulus features
\citep{aminoff2007parahippocampal, baldassano2013differential}.
%
On the other hand, results encourage future studies to investigate response in
the parahippocampal region to auditory stimuli under more controlled conditions.
%
In case the controlled paradigm reveals less variation in reliability across
participants, future studies employing naturalistic stimuli could then
investigate factors that might have influenced reliability of individual results
in \citet{haeusler2022processing}:
%
a) stimulus-related factors like the number available events that can be modeled
and contrasted per stimulus segment, or the general confound structure,
%
b) modeling-related factors like the assumed shape of the hemodynamic response
%
c) subject-specific factors like alertness or engagement over the course of the
experiment.


\subsection{Sharing is caring: Advancing the field through collaborative effort}

%
Challenging data data analysis, but create \& share (data from naturalistic
stimuli \& ) annotations.

%
Naturalistic stimuli are so rich that dataset decay is low, everybody can get
a slice of the cake and do his/her part to advance the field.

%
We re-used existing data as a foundation for a new investigation in order to
generate novel findings encourage further studies, and illustrate the benefits
of publicly and freely available datasets.

%
``There lacks a movie database with robustly validated measures, especially with
test--retest reliability.
%
This is perhaps a key requirement for valid and reproducible research, which the
field should strive to establish.
%
``No doubt, such data sharing initiatives will provide a further impetus to the
development of novel analytical methods and insights into brain processing of
complex information in naturalistic paradigms''
\citep{sonkusare2019naturalistic}.

%
Since all that shit is more difficult than one might think (it's not ``just
present the movie in the scanner'') and naturalistic stimuli are so rich and
versatile, a researcher should consider making corresponding data, additional
measures public eventually and collect the data according to best practices, and
to the best knowledge and belief.
%
Especially, creating and sharing stimulus annotations to ease the``annotation
bottleneck'' \citep[][p. 16]{aliko2020naturalistic} of naturalistic stimuli is
probably the biggest prerequisite to test physiological assumptions, tackle
methodological issues of model-driven analyses, and foster further developments
[cf.  Nunez-Elizalde (2019), Dupre la Tour (2022) for recent examples] in order
to draw valid inferences from naturalistic stimuli and advance the field as a
whole.


\pagebreak


\section{Some backups}


\subsection{Group-level vs individual-level analyses}

Backup of quotes for general discussion (or SRM study).  Studies that average
data across study participants may draw

\begin{itemize}

\item ``provide only an approximate view of any individual's brain organization,
    potentially obscuring meaningful individual differences in cortical
        organization'' \citep{laumann2015functional},

\item may just ``capture the common denominator of each individual cognitive
    circuit and lose a large amount of information''

\item ``obscure(s) patterns of brain organization specific to each individual''
    \citep{laumann2015functional}.

\end{itemize}


\subsection{Dynamic localizers}
%
``Previous reports showing significant differences between topographies
estimated by static and dynamic localizers, especially in superior temporal and
frontal cortices [Fox et al., 2009; Pitcher et al., 2011]''
\citep{jiahui2022cross}.
%
``For all categories, the dynamic localizer elicited stronger and broader
category-selective activations than the static localizer''
\citep{jiahui2022cross}.
%
``For example, for the face-selective topographies, the dynamic localizer
activated more areas than the static localizer (e.g., in superior temporal and
frontal cortices)'' \citep{jiahui2022cross}.
%
``The searchlight analysis showed that the dynamic localizer had higher
reliabilities across the cortex, especially in regions that were selectively
responsive to the target category'' \citep{jiahui2022cross}.
%
``The low correlations were not because the prediction method failed but
reflected the difference in the topographies activated by different types of
localizers.'' \citep{jiahui2022cross}.


\subsection{Brain \& behavior: "fingerprints"}


%
Maybe might be stable individual differences in cognitive tendencies or
cognitive abilities like susceptibility / predisposition [?] to attend to, [or]
recognize [or process] auditory spatial information.
% kanai
In case the pattern is stable within individual subjects / is ``highly
consistent across different sessions [or experiments], then they are
characteristics of the individuals and may reflect differences in their brain
function'' \citep{kanai2011structural} [on structural diffences].
%
``Individual differences in topology (i.e. location, size, shape of functional
areas) and the activity within functional areas can also be considered to be
interesting cases of inter-individual variability to understand the neural basis
of human cognition and behavior, brain-phenotype relationships'', and ``present
useful phenotypes or biomarkers \citep{glasser2016multi,
vanhorn2008individual}''


\subsubsection{Brain \& behavior: example studies}

\todo[inline]{cf. also \citep{gordon2017individual, gordon2017precision}}
%
For example, \citet{kong2019spatial} suggested based on resting-state functional
connectivity measures ``that individual-specific network topography (i.e.,
location and spatial arrangement) might serve as a fingerprint of human behavior
that can predict behavioral phenotypes across cognition, personality, and
emotion'' \citep{kong2019spatial} [with modest accuary, comparable to previous
reports predicting phenotypes based on connectivity strength].

%
\citep{bijsterbosch2018relationship}'s ``results indicate that spatial variation
in the topography of functional regions across individuals is strongly
associated with behaviour'' \citep{bijsterbosch2018relationship}.
%
\citet{bijsterbosch2018relationship} found ``that the spatial arrangement of
functional regions is strongly predictive of non-imaging measures of behavior
and lifestyle'' [however shape \& exact location of brain regions interacted
strongly with  modeling of brain connectivity].
%
\citet{bijsterbosch2018relationship} found ``that individual differences in the
size, shape and exact position of the brain regions [as identified by
resting-state functional connectivity measures] was strongly linked to
individual differences in behavioral tests and questionnaires [including
intelligence, life satisfaction, drug use and aggression problems]''
\citep{bijsterbosch2018relationship}.

%
``The variations in spatial topographical features captured a more direct and
unique representation of subject variability than temporal correlations between
regions defined by group parcellation approaches (coupling).
%
Hence, the cross-subject information represented in commonly adopted
'connectivity fingerprints' could largely reflect spatial variability in the
location of functional regions across individuals, rather than variability in
coupling strength (at least for methods that directly map group-level
parcellations onto individual data)'' \citep{bijsterbosch2018relationship}.

%
``Depending on the employed spatial alignment algorithm and the amount of
removed spatial intersubject variability, the degree to which spatial
information may influence FC estimates possibly varies considerably across
studies.
%
In recent years, significant efforts have gone into the methods that more
accurately estimate the spatial location of functional parcels in individual
subjects [Chong et al., 2017; Glasser et al., 2016; Gordon et al., 2016; Hacker
et al., 2013; Harrison et al., 2015; Varoquaux et al., 2011; Wang et al., 2015],
and into advanced hyperalignment approaches [Chen et al., 2015; Guntupalli et
al., 2016; Guntupalli and Haxby, 2017]'' \citep{bijsterbosch2018relationship}.
