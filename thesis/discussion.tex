\section{Conclusions}



\subsection{Study 1}
%
From OHBM-Abstract: vgl. long abstract in doc-file
%
The present results are evidence that a functionally de ned region, such as the
PPA, can be localized using a complex natural stimulus, by exploiting its
inherent temporal structure with respect to a particular hypothesized cognitive
or perceptual function.
%
While natural stimulation is not a universal solution, it can represent one
approach to overcome the threat of lacking generalizability associated with
small and simpli ed stimulus sets.
%
Using this approach we replicate a previous nding of PPA modulation by verbal
stimuli with descriptions of spatial con gurations.
%
This adds to the evidence that the PPA is not exclusively involved in visual
processing, but a bilateral anterior portion also exhibits activation increases
correlated with the semantic processing of verbal statements of spatial
references in a rich natural auditory stimulus.

%
From OHBM-Poster:


\subsection{Study 2}



\subsection{Study 3}

N=14;



\subsection{Future research}

\todo[inline]{connectivity based hyperalignment?}


\paragraph{Evaluate (short) movie with respect to localization of other brain
functions}
%
Lastly, the new short movie stimulus will be assessed with respect to its
diagnostic potential for other functionally defined regions in the brain.
%
Again, data from the studyforrest project will be used to perform assessments
analog to WP1e for a range of additional individually determined regions based
on established localizer paradigms (Sengupta, et al., 2016).
%
These regions include: the \ac{ffa}, \ac{ppa}, and \ac{eba}  which are
associated with face perception \citep{kanwisher1997ffa,
pitcher2011occipitalfacearea}, scene perception \citep{epstein1998ppa}, and the
perception of human bodies \citep{downing2001bodyarea}, respectively.
%
The actual selection will depend on the specific properties of the chosen short
movie stimulus.
%
2) The functional alignment on the full-length movie from WP2b will be
recomputed, but this time based on functional connectivity vectors, as opposed
to BOLD response time series.
%
This will be performed using an implementation developed by Falko Kaule as part
of his PhD project.
%
Analog connectivity vectors will then by estimated from the short movie scan of
the left-out subject and will be used for direct hyperalignment.
%
Differences in the transformations between the two approaches will be quantified
in terms of overall similarity, but also in terms of within-subject differences
regarding the weighting of individual brain areas across all common space
components.
%
This comparison is possible, because in both cases the subject’s voxel space is
identical, and the orthonormal transformation into the respective common space
preserves all variance.
%
It should be noted that the number of voxels that can be considered
simultaneously for functional BOLD response time series alignment is limited by
the number of timepoints in the calibration scan (about 300-400 voxels for a
15min scan with a 2s TR, corresponding to a local cortical neighborhood of about
1cm in diameter for a standard resolution).
%
This limitation does not exist in
this form for a functional alignment that is based on connectivity vectors.
%
The
length of these connectivity vectors is determined by the number of reference
(or seed) regions in the brain.
%
The results of this work package will
determine whether it is necessary to include dedicated reference scans for short
calibration stimulation sequences in a future collection of normative data.


\subsection{Vision}
%
Moreover, beyond the scope of this project the targeted homogenization of
acquisition procedures will further the goal of large scale data collection for
the purpose of producing a normative reference of brain function as measured by
fMRI.

% one stimulus (segment) in every study similar to "anatomic scan"
The acquisition will be integrated into an existing protocol and an ongoing
acquisition effort performed by the LIN group that is designed to diagnose the
lateralization of speech processing in the brain, using a standard task-based
localizer paradigm (Fernandez et al., 2001).
%
This integrative approach will yield comparative within-subject BOLD fMRI
datasets for the task-free movie stimulation and a dedicated established
localizer paradigm.
%
Taken together, the results of all work packages detail what performance and
versatility can be expected from an approach to localization that is based on
task-free natural stimulation fMRI.
%
Moreover, it comprehensively informs a future study about scan type, and scan
time requirements for the collection of a normative reference.
%
The availability of such a reference would enable quantitative and qualitative
description of an individual's brain function with respect to such a norm, and
consequently progress the field towards neuroimaging studies of individual
differences that more closely resemble their psychological counterparts
\citep{dubois2016building}.

\paragraph{Extend dataset with standard speech localizer}
%
In order to be able to evaluate the prediction performance of brain areas
associated with speech processing, the studyforrest dataset has to be extended
with BOLD fMRI scans for an established localizer paradigm.
%
At the same time, additional fMRI data acquisitions for the full-length movie
might have to be performed in order to compensate for original studyforrest
participants that may no longer be able or willing to volunteer for additional
measurements (N expected to be less than 10).
%
This work will be performed in
collaboration with the Non-Invasive Brain Imaging Unit at the LIN.



