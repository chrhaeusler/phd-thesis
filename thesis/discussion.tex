\todo[inline]{Intro is still pretty similar to phrasing in introduction}

Human brain mapping studies have traditionally averaged \ac{fmri} data across
participants.
%
However, data need to be assessed on the level of individual persons in order to
advance the field towards a clinical application.
% functional localizer
A promising tool to perform this advancement are functional localizers [because
localizers aim to characterize the location size, and shape of functional areas
on the level of individual subject].
% contra localizers
However, traditional localizer paradigms employ selectively sampled,
tightly-controlled stimuli, rely heavily on a participant's compliance, and can
usually map just one domain of brain functions
% naturalistic stimuli could replace
Localizer paradigms based on naturalistic stimuli could provide higher
ecological as well as external validity, higher data quality due to increased
compliance, and potentially map a variety of brain functions [ranging from
low-level perception (e.g., luminance) to high-level cognition (e.g., social
cognition)] simultaneously.
% visually impaired
Lastly, an exclusively auditory stimulus like an audiobook or audio drama would
also be appropriate for visually impaired persons[, e.g., suffering from
nystagmus or lack of eyesight].
% PPA as proof of concept
This thesis focused on the \ac{ppa}, a classic higher-visual area,
\citep{epstein1998ppa}, in order to explore whether a movie and the movie's
audio-description could, in principle, substitute a traditional localizer
paradigm.

\todo[inline]{maybe, very short summary of parts here}

%
First, we extended the studyforrest dataset (study 1).
%
Second and similarly to traditional localizer paradigms, we modeled hemodynamic
activity based on annotated stimulus features embedded in the movie ``Forrest
Gump'' and its audio-description, and created \ac{glm} $t$-contrasts in order to
localize the \ac{ppa} (study 2).
%
Third, we estimated results of the localizer by projecting data through a
\ac{cfs} (study 3).


\paragraph{Open science}

% intro
The self-imposed, overarching goal of this dissertation was to perform all
studies under the principles of open, shared, and transparent science.
% mention Datalad
To meet the requirements of a reproducible research project and facilitate
replication of findings in prospective studies all created data, written code
and results are published as version-controlled DataLad
\citep[\href{www.datalad.org}{datalad.org};][]{halchenko2021datalad} datasets.
% datalad rerun
Consequently, independent researchers can re-run all performed analyses and
validate the results.


\section{Recapitulation of work packages}

\todo[inline]{maybe, give an overview of how the remaining part of the thesis is
structured: study as such, study in light of open science, general assessment
of open science, conclusion on naturalistic stimuli as localizer, conclusion on
naturalistic stimuli in general}


\subsection{Speech anno}


\subsubsection{Goal of speech anno}

% what we did in 1 sentence
In study 1 \citep{haeusler2021speechanno}, we created and validated an
annotation of speech occurring in the movie and its audio-description pursuing
two goals:
% aim #1: groundwork for PPA study
The first aim was to built the groundwork that enabled us to conduct study 2.
% aim #2: extend studyforrest
The second aim was to create an exhaustive annotation of speech that
substantially exceeds the groundwork necessary to conduct study 2 in order to
extend the studyforrest dataset as a public resource for independent research.


\subsubsection{Discussion of speech anno paper}

% validation analysis
We validated the annotation's quality in study 1 and performed a canonical
\ac{glm} analysis by contrasting regressors correlating with speech-related
events to a regressor correlating with events without speech.
% results
As hypothesized, results revealed statistically significant increased
hemodynamic activity in a bilateral cortical network known to be involved in the
perception of speech \citep[e.g.,][]{friederici2011brain, wilson2008beyond}.
% conclusion
These results encouraged us to a) use the annotation as the groundwork for study
2, b) publish the annotation as an extension of the studyforrest project.


\subsubsection{Speech anno paper in light of open science}

\paragraph{Personal assessment}

\todo[inline]{how "personal" am I supposed to get?}

%
\todo[inline]{I could write here or in the summary of "open science stuff"
across all studies that I pulled my hair out while searching for inconsistencies
in the timings, and found that the audio track of the audio-description is
essentially unsystematically shifted}

%
Additional work not necessary to conduct study 2 was put into the creation
process of the annotation in order create a publication-worthy dataset:
%
The annotation provides, e.g., information about the two stimuli's time-courses
time-stamps of phonemes, words and sentences of all speakers, a grammatical
tagging, and an annotation of syntactic dependencies and semantics.

%
Great care was taken to provide sound information to the scientific community.
%
Realization that there is no "perfect" annotation that fits all uses cases.
%
An annotation of speech will always contain ambiguities and noise.
%
Additionally, it turned out to be impossible to anticipate potential use-cases.
%
Everything that is not super raw and has been submitted to further processing
steps is subject to decisions that might not be the right choice for some
analyses.
%
Subjective judgment to find the balance between
%
a) do the mere minimum (raw data) or
%
b) do additional work that might not be ``perfect''.
%
For example, semantics: might be based of models that are "outdated" a couple of
years later, or do not necessarily meet the requirements of specific use cases
in terms of semantic analyses.


\subsubsection{Conclusion}

\todo[inline]{s. general intro on "difficult model-based analyses"}

%
This extension is standardized and available as online repositories comprising
raw data, code, and results.
%
The published annotation widens the ``annotation bottleneck''
\citep{aliko2020naturalistic} and potentially lays the foundation for
independent research projects.
%
``The current publication enables researchers to model hemodynamic brain
responses that correlate with a variety of aspects of spoken language ranging
from a speaker's identity, to phonetics, grammar, syntax, and semantics.
%
All annotations can be used in any study focusing on aspects of real-life
cognition by serving as additional confound measures describing the temporal
structure and feature space of the stimuli'' \citep{haeusler2021speechanno}.



\subsection{PPA paper}

\subsubsection{Recapitulation}

\paragraph{Goal}
% study in one sentence
The goal of study 2 investigate whether an audio-visual and an exclusively
auditory naturalistic stimulus could be used in order to localize the \ac{ppa}
as previously identified in the same set of participants by means of a
traditional block-design functional localizer using static pictures
\citep{sengupta2016extension}.


\paragraph{Method}
% AV operationalization
For the model-based mass-univariate statistical (i.e. \ac{glm}) analysis of the
movie's data, we operationalized the perception of visual spatial information
based on an annotation of movie cuts and depicted locations.
% AD operationalization
For the \ac{glm} analysis of the audio-description's data, we extended the
annotation of speech \citep{haeusler2021speechanno} by further annotating nouns
that the narrator uses to describe the movie's absent visual content.


\paragraph{Results \& Discussion}
% group results
On a group-average level, findings demonstrate that increased activation in the
\ac{ppa} generalizes to the perception of spatial information embedded in the
audio-visual movie and its audio-description.
% individual AD
``On an individual level, we find significant bilateral activity correlating
with semantic spatial information occurring in the audio-description in the
anterior \ac{ppa} of nine individuals and unilateral activity in one
individual'' \citep{haeusler2022processing}.

% conclusion 1
Results add evidence \citep[cf.][]{bartels2004mapping} that a functionally
defined region, such as the \ac{ppa}, can be localized using a model-driven
\ac{glm} analysis that is based on a naturalistic stimulus' annotated temporal
structure.
% conclusion 2
Results also suggest that a purely auditory paradigm like an audio-description
could, in principle, substitute a visual localizer as a diagnostic procedure to
assess brain functions in visually impaired individuals
\citep{haeusler2022processing}.


\paragraph{Transition to further studies}
%
Our results that suggest that a) the responses correlating with an auditory
stimulation are spatially restricted to the anterior \ac{ppa}, and b) higher
intersubject variability of responses to an naturalistic auditory stimulation
invite further studies to investigate the properties of an ``auditory
\ac{ppa}''.


\subsubsection{Future studies on auditory PPA}

\todo[inline]{check that part in the paper again}

%
``Previous studies in the field of visual perception provide evidence that the
PPA can be divided into functionally subregions that might process different
stimulus features'' \citep{haeusler2022processing}.
%
``This raises the question if the revealed correlation patterns can be attributed
to different features inherent in the visual stimuli compared to a purely
auditory stimulus'' \citep{haeusler2022processing}.
%
However, ``due to the nature of the datasets investigated here, such an
attribution can only be preliminary, because the auditory stimulation dataset
also differs in key acquisition properties (field-strength, resolution) from the
comparison datasets, representing a confound of undetermined impact''
\citep{haeusler2022processing}.

\todo[inline]{write conclusion: future studies using dedicated stimuli + tasks}



\subsubsection{Future studies on auditory PPA: inter-individual variability}

%
Compared to the visual \ac{ppa}, we observe more variability of the auditory
\ac{ppa} across subjects.

%
Our study ``invites future work on the origin of the variability and its
relation to individual differences in behavioral performance''
\citep{zhen2015quantifying}.


\paragraph{Intro on differences (a.k.a. not noise)}

``We find significant bilateral activity correlating with semantic spatial
information occurring in the audio-description in the anterior \ac{ppa} of nine
individuals and unilateral activity in one individual''.
%
The divergent pattern from the group mean in four of fourteen individuals should
not necessarily interpreted as noise or measurement error/artefacts but could be
attributed a) to our naturalistic, task-free stimulation (``environmental
effect'') or b) differences in persistent traits or transient states between
study individual study participants.


\paragraph{Possible reasons}

%
Intraindividual sources of variability in functional activation might be the
differential engagement [or differences in cognitive strategy,
style, and tendencies, cognitive capacities during recognition, processing
blabla]
%
Persistent traits might be shaped by development and/or genetics.

%
``Possible factors to be explored are situational factors (e.g. the influence of
experimental design, laboratory environment, stimulus type, state of the
subject), cognitive factors (such as other differences in cognitive style or
executive functioning ability), physiological factors (such as differences in
white matter connectivity, resting state metabolism, or recent caffeine
consumption), personality factors, or genetic differences.
%
Some of these factors may be related to specific tasks while other factors may
lead to individual deviations that cut across a variety of tasks.
%
A full understanding of the relationship between functional brain activity and
task performance will depend on systematic investigations of these fundamental
differences'' \citep{vanhorn2008individual}.


``These differences [in the structure of grey matter and white matter tracts]
are typically viewed as measurement errors or as uninteresting peculiarities of
individuals, and are therefore discarded.
%
However, if they are highly consistent across different tests, then they are
characteristics of the individuals and may ultimately reflect differences in
their brain function'' \citep{kanai2011structural}.



\paragraph{If differences are reliable...}

%
``An important consideration when evaluating the pattern of activity from a
single individual is how stable that pattern of activity is over time''
\citep{vanhorn2008individual}.
%
Influence of task (kind of a classic ``test effect''): is the shit dependent on
task vs. non-tasks, i.e. reliable in task-free paradigms but different and
reliable patterns during task-based paradigm.


\paragraph{...they could be used as "fingerprints"}

% Zhen
``Interindividual variability of functional regions in both functional and
spatial features should reflect brain function to some extent.
%
Thus, interindividual variability is a rich and important source of information
for revealing the neural basis of human cognition and behavior''
\citep{zhen2015quantifying}.

%
``Only by understanding the generic characteristic of topographic organization
can we start to appreciate idiosyncrasies and their relationships to
socio-demographic, cognitive or affective profiles''
\citep{eickhoff2018imaging}.

%
``Topological outliers can also be considered to be interesting cases of
inter-individual variability to understand brain-phenotype relationships [Zilles
(2013). Individual variability is not noise]'' \citep{eickhoff2018imaging}.

%
``Variations in these patterns are not likely to be entirely random noise but
present useful phenotypes or biomarkers for a further examination of individual
differences in relation to such variables'' \citep{vanhorn2008individual}.

``Individual differences and topology (i.e. location, size, shape of functional
areas) and the activity within functional areas should facilitate the dissection
of how each property is related to behavior and genetic underpinnings, for
example, in learning disabilities or those with distinctive cognitive traits''
\citep{glasser2016multi}.


\paragraph{Example studies}

%
For example, \citet{kong2019spatial} suggested ``that individual-specific network
topography (i.e., location and spatial arrangement) [based on resting-state
functional connectivity measures] might serve as a fingerprint of human
behavior that can predict behavioral phenotypes across cognition, personality,
and emotion [with modest accuary, comparable to previous reports predicting
phenotypes based on connectivity strength]'' \citep{kong2019spatial}.

%
\citet{bijsterbosch2018relationship} found ``that the shape and exact location
of brain regions interact strongly with the modeling of brain connectivity, and
present evidence that the spatial arrangement of functional regions is strongly
predictive of non-imaging measures of behavior and lifestyle''.
%
\citet{bijsterbosch2018relationship} ``found that individual
differences in/of the size, shape and exact position of the brain regions (as
identified by resting-state functional connectivity measures) was strongly
related to individual differences in/of behavioral tests and questionnaires of
healthy individuals [behavioural factors including intelligence, life
satisfaction, drug use and aggression problems].
%
Some psychiatric and neurological diseases may be associated with changes in
size and position of the different regions in the brain.
%
In future, the findings of this study could be applied to individuals affected
by such conditions, to see if the location of a region could be used as a
diagnostic indicator'' \citep{bijsterbosch2018relationship}.


\citep{bijsterbosch2018relationship}'s ``results indicate that spatial variation
in the topography of functional regions across individuals is strongly
associated with behaviour.
%
The fact that these variations in spatial topographical features capture a more
direct and unique representation of subject variability than temporal
correlations between regions defined by group parcellation approaches
(coupling), was unexpected.
%
The implication of this finding is that the cross-subject information
represented in commonly adopted ‘connectivity fingerprints’ largely reflects
spatial variability in the location of functional regions across individuals,
rather than variability in coupling strength (at least for methods that directly
map group-level parcellations onto individual data).
%
It is possible that the degree to which spatial information may influence FC
estimates varies considerably across studies, depending on the spatial alignment
algorithm that was used, and the amount of subject spatial variability this has
removed.
%
It is encouraging that significant efforts have recently gone into the methods
for more accurately estimating the spatial location of functional parcels in
individual subjects in recent years [Chong et al., 2017; Glasser et al., 2016;
Gordon et al., 2016; Hacker et al., 2013; Harrison et al., 2015; Varoquaux et
al., 2011; Wang et al., 2015], and into advanced hyperalignment approaches [Chen
et al., 2015; Guntupalli et al., 2016; Guntupalli and Haxby, 2017]''
\citep{bijsterbosch2018relationship}.



\paragraph{Applied to PPA}

\todo[inline]{cf. master's thesis on lesion studies of PPA; what does the PPA do}

%
Individual differences in brain activity in the \ac{ppa} during naturalistic
auditory stimulation might be correlated with
%
a) individual differences in alertness / fatigue in general but also
%
b) susceptibility (predisposition) to incidentally occurring auditory spatial
information embedded in the audio-description specifically.

Predisposition to ``select'' or processing spatial relevant information
differently, susceptibility to attend

%
The latter might be intertwined with difference in attentional foci, perceived
salience of spatially relevant stimulus features and their further encoding into
and retrieval from memory.



%
``The movie stimulus shares the stimulation in the visual domain with classical
localizer stimuli, while featuring real-life-like visual complexity and
naturalistic auditory stimulation.
%
The audio-description maintains the naturalistic nature of the movie stimulus,
but limited to the auditory domain'' \citep{haeusler2022processing}.
%
Do we find more variability due to naturalistic stimulation ("more random
sampling"), incidentally presented information, task-free (participants naive to
the process being investigated) or our (adventurous) modeling approach of
spatial information or due to auditory spatial information per se.


\paragraph{Applied to PPA: naturalistic stimuli}
%
Therefore, future studies using naturalistic stimuli could do...
%
...and compare that to dedicated task-based paradigms.



\paragraph{Applied to PPA: dedicated stimuli}

%
Results invite prospective studies employing dedicated stimuli.
%
Further studies using dedicated auditory localizer should shed light on the
inter-individual differences of persons exhibiting an auditory \ac{ppa}.
%
Future studies using dedicated task-based paradigms could do measure alertness,
memory, utilization of spatial information for (subsequent visual) place
recognition, spatial orientation, way finding, navigation...

%
For example, processing oral way description and using it for planing,
remembering, executing navigation.


\paragraph{Conclusion on further PPA studies}
%
Future studies need to elaborate on the significance of the variability of
functional regions and utilize this variability to explore the neural mechanisms
of specific behavioral or cognitive processes [Van Horn (2008). Individual
variability]'' \citep{zhen2015quantifying}.
%
``Future studies will be needed to elaborate on the mechanism and functional
significance of the interindividual variability of the SSRs and to utilize the
variability in understanding the neural mechanisms of scene perception''
\citep{zhen2017quantifying}.



\subsubsection{Future studies: other scene-selective areas}

\todo[inline]{cf. reviewer response of PPA study}

We focused on the \ac{ppa} as a proof-of-concept [we have the masks from
Sengupta]..
%
We (and Sengupta) chose the PPA among three possible candidates (cf. answer to
point 2) because it was the first area to be discovered as a visual
“scene-selective” region, is the most reliably activated region across studies
that investigate visual scene perception, and the most robust to confounds.

%#12 RSC + LOC
``Apart from the PPA, results show significantly increased activity in the
ventral precuneus and posterior cingulate region (referred to as ``retrosplenial
complex'', RSC) of the medial parietal cortex, and in the superior lateral
occipital cortex (referred to as ``occipital place area'', OPA) for both
naturalistic stimuli.
% RSC intro
Like the PPA, the RSC and OPA have repeatedly shown increased hemodynamic
activity in studies investigating visual spatial perception and navigation
\citep{chrastil2018heterogeneity, bettencourt2013role, dilks2013occipital,
epstein2019scene}.
% inference
Thus, our model-driven approach to operationalize spatial perception based on
stimulus annotations reveals increased hemodynamic responses in a network that
is implicated in visual spatial perception and cognition.
% medial parietal cortex: anterior-posterior gradient
Similarly to the parahippocampal cortex \citep{aminoff2013role}, the medial
parietal cortex exhibits a posterior-anterior gradient from being more involved
in perceptual processes to being more involved in memory related processes
\citep{chrastil2018heterogeneity, hassabis2009construction, silson2019posterior,
steel2021network}.
% future studies
Future, complementary studies using specifically designed paradigms could
investigate where in the posterior-anterior axis of the parahippocampal and
medial parietal cortex auditory semantic information is correlated with
increased hemodynamic activity:
% we hypothesize
we hypothesize that the auditory perception of spatial information (compared to
non-spatial information) is correlating with clusters in the middle of possibly
overlapping clusters correlating with visual perception (peak activity more
posterior) and scene construction from memory (peak activity more anterior)''
\citep{haeusler2022processing}.



\subsubsection{Future studies: other functional areas}

\todo[inline]{cf. discussion of SRM part}





\paragraph{Studyforrest dataset}

%
However, the localizer performed by \citep{sengupta2016extension} contrasts
that aimed at localizing FFA, OPA, etc.
%
Other mask of functional areas in studyforrest dataset (e.g. for master or
bachelor thesis, or as part of a PhD project).
%
``Future studies that aim to use a movie to localize visual areas in individual
participants should extensively annotate the content of frames (e.g., using the
open-source solution ``Pliers''\citep{mcnamara2017developing} for feature
extraction from a visual naturalistic stimulus)''
\citep{haeusler2022processing}.


\paragraph{New dataset}

\todo[inline]{Clinical relevance localization of language areas; problem in case
of prediction (esp. using ROI): atypical language lateralization \& usually no
lateralization in naturalistic stimuli (but operationalization is different from
localizer paradigms), assumption of (strict) lateralization probably wrong
anyway}


%
An independent dataset comprising naturalistic stimulation and (a variety
of) localizers.
%
A full feature film might substitute traditional localizer paradigms dedicated
localizer by mapping a variety of brain functions beyond category-specific
visual areas.

\ac{fmri} could be used as an noninvasive alternative to map language areas and
potentially assess lateralization (or hemispheric asymmetry) of functional brain
topography related to language (sub)functions, in order to guide pre- and
perioperative assessment of neurosurgery, e.g., in case of epilepsy.



\subsubsection{PPA paper in light of open science}





% goal PPA study
Under the perspective of an open science project, one goal of study 2 was to use
already existing data of the studyforrest project and extend the published
annotation for the specific needs of a new research question.
%
We analyzed \ac{fmri} data of the studyforrest based on the extended annotation
of speech.
%
Thus, we were able to interpret the data under a new perspective that was not
foreseen during the original generation of the studyforrest project.
%
Our results suggesting that the PPA is fucking awesome have been published in a
peer-reviewed, open-access journal \citep{haeusler2022processing}.

%
Consequently, we successfully reused existing data as a foundation for a new
investigation in order to generate novel findings that illustrate the benefits
of publicly and freely available dataset like the studyforrest project.


\paragraph{personal assessment}

%
Opportunity costs [look up definition]: One example of the current thesis is the
analyses performed by \citet{sengupta2016extension} that were performed (and
published) in voxel-space (not surface-space) and resulting masks were
restricted to one scene-selective area (\ac{ppa}; no \ac{rsc}, \ac{opa} =
transverse occipital sulcus).


\subsubsection{Conclusion}
%
We extended the annotation for our specific use case:
%
model hemodynamic responses were modeled based on the event structure as
similarly performed in traditional block-design localizer paradigms.
%
Despite exploratory approach (a.k.a. shitty modeling of subjectively assessed
events), results suggest:
%
the response to spatial information must be somewhere in there and is
detectable.
%
Results suggest that we might be able to use the response patterns
measured during the presentation of the audio-description in order to generate a
\ac{cfs} [needs to be defined above in overview] in study 3, and
align an ``unknown'' test participant to that \ac{cfs}.


\subsection{SRM study}

% align left-out subject
Based on our findings in study 2 \citep{haeusler2022processing}, we assumed that
the event structure in both naturalistic stimuli would correlate, among others,
with brain responses that are similar to those correlating with the event
structure in a dedicated functional localizer.

\subsubsection{goal of SRM study}

From intro:
\paragraph{Problem}
% summary of study 2
Results of study 2 suggest that a naturalistic stimulus might provide an
engaging, task-free paradigm to localize brain functions in individual subjects.
%
However considering practical and monetary constraints in a clinical context, a
paradigm lasting 90 to 120 minutes is inappropriate for even an extensive
individual diagnostic procedure.


\paragraph{Solution}
%
Therefore, the first goal of study 3 was to assess a procedure to estimate
results of a dedicated localizer \citep{sengupta2016extension} based on data
acquired during naturalistic stimulation.
%
Following leave-one-subject out cross-validation, we estimated (i.e. predicted)
the results of the visual localizer experiment ($Z$-values of voxels within a
\ac{roi}) of a left-out test participant based on localizer results of a
reference group.\todo{phrasing!}
% partial alignment
The second goal of study 3 was to assessed the relationship between length of
naturalistic stimulation used to align the test participant to the fixed
\ac{cfs} and the estimation performance.
%
Lastly, the estimation performance of our new procedure based on \ac{fa} was
compared an estimation performance based on \ac{aa}.


\paragraph{Hypotheses}
%
We hypothesized that increased quantity of data used to calculate the
transformation matrices of the left-out subjects for a \ac{fa} would to increase
prediction performance.
%
Further, we hypothesized that \ac{fa} would eventually perform
``better'' than an estimation based on \ac{aa}.


\subsubsection{discussion of SRM study}

\todo[inline]{phrase more clearly: we = event-related; localizers = block}


\subsubsection{Future studies on functional alignment}


\subsubsection{SRM study in light of open science}


\paragraph{personal assessment}



\section{Open Science: summary, personal assessment, conclusion}


\subsection{Contra}

\paragraph{Not required, not established}
%
Following the guidelines was not mandatory for submitting the thesis but
constituted voluntary, additional careful work.
%
The additional benefits that go along this additional work are not immediately
visible to researchers/authors.
%
Further, the standards to follow are still emerging and [probably] not yet
standardized [?].
%
Guidelines are not [yet] part of a curriculum, neither in a bachelor's/master's
program nor PhD program and therefore self-imposed and learned based on
self-initiative.


\paragraph{Additional workload}
%
Take extra care to make everything really, really good 'cause other people will
rely on your shit.
%
You have to make your data and script extra pretty so other people can benefit
from your work but you do not benefit yourself from doing more than is necessary
for your own project.
%
Thus it is an added ``burden'' during PhD thesis.

%
People who do not give a shit about it and don't do it, do ``worse'' research
but are faster/``better''.


\paragraph{Science is competition; data leeching}
%
Other working groups (more money, brain/people power) can be faster and can use
your data for similar findings.


\subsection{Trustful data?}

\todo[inline]{check the data you created or others will do it; might get
"embarrassing"}

%
Dataset will never be perfect and will contain noise, or even artefacts.
%
Data description need to be as exhaustive as possible.
%
Re-users need to check the data according to their use case!
%
The unknown unknowns of dataset creators?
%
Quality standards vary from field to field and even within subfields of
neuroscience (a.k.a. ``interessiert uns ja nicht'').
%
Everything that is not explicitly stated needs to be assumed to not be
considered.
%
Which is consequently leading to opportunity costs:


\subsection{Opportunity costs: raw data}

Wikipedia: ``opportunity cost of a particular activity is the value or benefit
given up by engaging in that activity, relative to engaging in an alternative
activity.
%
More simply, it means if you chose one activity (for example, an investment) you
are giving up the opportunity to do a different option.
%
The optimal activity is the one that, net of its opportunity cost, provides the
greater return compared to any other activities, net of their opportunity
costs''.

%
You save time and money (especially if you do not collect the data yourself) but
you have to trust that the data are ``good'' and start at some point to reach
goals that you would not accomplished if you would have done all from scratch
(i.e. preprocessing).
%
On the one hand, not spending time on collecting data is speeding up the process
of getting to analyze the data, and go beyond what would have been possible if
you would have collected the data yourself.
%
On the other hand, there is an hard to judge ``correct'' balance between
trusting the work of others that collected the raw data, and checking the data
and familiarize yourself with the data.
%
Researchers need to handle that two-sided sword and balance trusting the
creators of the dataset and checking/validating everything themselves.

%
This essentially boils down to researchers that want to publish data to highly
take care that the dataset gets really, really good, and test and validate the
data substantially and rigorously.
%
Contingencies need to be taken care of and use-cases considered/anticipated that
go beyond the originally anticipated use-cases (``a.k.a. interessiert uns ja
nicht'').

%
The same applies not just for the collection of raw data but also any
preprocessing steps, (that might lead to different results based on parameters,
operation system, software version etc.) or any kind of results.
%
Standardizing the analyses pipelines helps.
%
Nevertheless, there is no "one size fits all" uses-cases.
%
Hence, being able to re-run analyses with custom parameters for specific
use-cases helps
%
In summary, it is mandatory to create awareness, and teach established and
emerging standards to students.


\subsection{Pro}

\todo[inline]{think big regarding reproducibility}

\todo[inline]{Why was is good for science that the data were already existing?}

\todo[inline]{Why was is good for you that the data were already existing?}

From introduction:
% reproducibility crisis
Over the last decade, there has been a growing awareness that results of
scientific publications are not reproducible or general scientific findings are
not replicable letting some authors speak of a ``reproducibility crisis'' or
``replication crisis'' in the sciences \citep{baker2016reproducibility,
plesser2018reproducibility, stupple2019reproducibility, nosek2022replicability}.
% reproducibility: definition
``A study is reproducible if all of the code and data used to generate the
numbers and figures in the paper are available and exactly produce the published
results'' \citep{leek2017most}.
% replicability: definition
A study is replicable if the same analysis of an equivalent experiment's data
leads to consistent results \citep{dubois2016building, leek2017most}.


\section{Conclusion: Naturalistic stimuli as functional localizer}

Summary of PPA paper:
%#13 natural stimulation
``In summary, natural stimuli like movies \citep{eickhoff2020towards,
hasson2008neurocinematics, sonkusare2019naturalistic} or narratives
\citep{hamilton2018revolution, honey2012not, lerner2011topographic,
silbert2014coupled, wilson2008beyond} can be used as a continuous, complex,
immersive, task-free paradigm that more closely resembles our natural dynamic
environment than traditional experimental paradigms.
% method
We took advantage of three fMRI acquisitions and two stimulus annotations that
are part of the open-data resource
\href{http://www.studyforrest.org}{studyforrest.org} to operationalize the
perception of spatial information embedded in an audio-visual movie and an
auditory narrative, and compare current results to a previous report of a
conventional, block-design localizer.
% results
The present study offers evidence that a model-driven GLM analysis based on
annotations can be applied to a naturalistic paradigm to localize concise
functional areas and networks correlating with specific perceptual processes --
an analysis approach that can be facilitated by the neuroscout.org platform
\citep{delavega2021neuroscout}.
% interpretation
More specifically, our results demonstrate that increased activation in the PPA
during the perception of static pictures generalizes to the perception of
spatial information embedded in a movie and an exclusively auditory stimulus.
% interpretation: aPPA vs. pPPA
Our results provide further evidence that the PPA can be divided into functional
subregions that coactivate during the perception of visual scenes.
% interpretation
Finally, the presented evidence on the in-principle suitability of a naturally
engaging, purely auditory paradigm for localizing the PPA may offer a path to
the development of diagnostic procedures more suitable for individuals with
visual impairments or conditions like nystagmus''
\citep{haeusler2022processing}.


\subsubsection{Clinical application}

%
``The ability to non-invasively and automatically delineate cortical areas in
living subjects may have clinical implications, for example by providing
neurosurgeons with detailed, individualized maps of the brains on which they
operate'' \citep{glasser2016multi}.


\subsubsection{Caveats of naturalistic stimuli}
%
Challenging data data analysis, but create \& share annotation.
%
Hence, data analysis pipelines should be implemented in common and
well-documented packages and code, and custom code shared along the paper.

%
Just an approximation of real life.
%
Setting is still the scanner.
%
Passive watching \& listening.
%
Executive functions?


\section{Conclusion: naturalistic stimuli in general}
%
Naturalistic stimuli are not a panacea but traditional paradigms and
naturalistic paradigms should be used in tandem / reciprocally to generate new
hypotheses and progress our understanding of the brain.


%
In summary, naturalistic stimuli ``impose a meaningful timecourse across
subjects while still allowing for individual variation in brain activity and
behavioral responses, and lend themselves to a broader set of analyses than
either pure rest or pure event-related task designs'' \citep{finn2017can}.
%
``Naturalistic paradigms do not aim to replace the classic, controlled
neuroimaging paradigms (Sonkusare et al., 2019). Due to their complexity and
current limitations in understanding the statistical properties of different
features in naturalistic conditions, naturalistic stimuli are not optimal for
model development [see, e.g., Rust and Movshon, 2005]. Controlled experiments
are still needed for hypothesis testing and developing models, while
naturalistic stimuli are best employed to test models in ecologically valid
settings and to expand them to situations where context matters
more'' \citep{saarimaki2021naturalistic}.
