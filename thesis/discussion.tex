\section{General recapitulation of the whole project \& its aims}

\todo[inline]{supposed to be a short intro}

\todo[inline]{cf. general introduction: overview of (non-specific) aims}


\section{Open, transparent, and reproducible science}

\todo[inline]{I guess the best place in the general discussion is here}

\todo[inline]{cf. general introduction}

\todo[inline]{you might give a high-level summary of all studies here already,
and thus do not need to do it before summarizing results of individual studies}


\subsection{Data re-use for new investigation \& new results}

\todo[inline]{mih: du hast mit dem PPA paper eine Publikation in SciData, weil
du separate und unabhängig veröffentlichte Daten, durch zusätzliche, von der
erstellte Werte, ganz neu betrachten und interpretieren konntest}

\todo[inline]{Du zeigst damit, dass Idee studyforrest erfolgreich sein kann}


\subsection{Reproducibility \& sharing}

\todo[inline]{mih: Datenveröffentlichungen nochmal hervorheben; Code repos mit
ausführbaren Analysen hervorheben; du hast die von dir gesetzten Ziele auch
erreicht}


\subsection{Pro \& Contra open, reproducible research}

\todo[inline]{Du bist daher in der Lage in persönliches Fazit zu ziehen: was hat
es der Wissenschaft gebracht, dass diese Daten "schon da waren"; Was hat es dir
gebracht? Was war schwerer als wenn du die Daten selbst erhoben hättest? Was
wäre dann aber nicht möglich gewesen.}


\paragraph{Contra}

\todo[inline]{my project is using existing data; new investigation; new results}
%
It takes fucking time with just minor immediate contribution to the workflow.
%
That is especially the cause because it is not standard yet (established tools
of trade?); it is not taught in study programs (but needs to be standard).
Thus it is an added ``burden'' during PhD thesis.
%
People who do not give a shit about it and don't do it, do ``worse'' research
but are faster/``better''.

%
Opportunity costs. It is not just about the raw data but also about
preprocessing, templates, results of the localizer. For other persons, its also
my speech annotation; we had no mask for RSC and transverse occipital sulcus.
%
Hence, do everything possible to make it really, really good 'cause other people
will rely on your shit.

%
Sengupta published results of analyses that were compared in voxel space.
%
``Opportunity costs'': I could have run the analyses/contrasts on the surface
but that needs time.
%
Is also a ``problem'' of data sharing: you save time and money (especially if
you do not collect the data yourself) but you have to trust that the data are
``good'' and start at some point to reach goals that you would not accomplished
if you would have done all from scratch (i.e. preprocessing).


%
Vice versa: Can you trust published data? Balance of checking everything vs.
taking things for granted.
%
The unknown unknowns of dataset creators? What did they fail to consider?
Standards may vary depending on the use case. Assume that everything that is not
explicitly stated in a paper along the dataset was not done?


\paragraph{Pro}

\todo[inline]{think big}



\section{Recapitulation of work packages}

\todo[inline]{cf. general introduction; objectives/aims per study}


\subsection{Speech anno}

\todo[inline]{cf. paper \& general introduction}


\subsection{PPA paper}

\todo[inline]{cf. study \& general introduction}

\todo[inline]{annotations are hard but you can do it}

\todo[inline]{traditional GLM ``works''}

%
``Future studies that aim to use a movie to localize visual areas in individual
participants should extensively annotate the content of frames (e.g., using the
open-source solution ``Pliers''\citep{mcnamara2017developing} for feature
extraction from a visual naturalistic stimulus)''
\citep{haeusler2022processing}.

%
``Movies also simulate better the statistics of natural viewing and listening
and may provide more ecologically valid maps'' \citep{jiahui2020predicting}.

%
Our modeling approach is pretty similar because adopted from classical
paradigms.
%
Modeling was shitty (especially in AV);
%
still results suggest: the response to spatial information must be somewhere in
there.
%
However, I did not test quality/reliability of individual results depending on
quantity of runs to assess naturalistic stimulus as potential replacement;


\subsubsection{Inter-individual differences / variability}


``Interindividual variability of functional regions in both functional and
spatial features should reflect brain function to some extent, and thus is a
rich and important source of information for revealing the neural basis of human
cognition and behavior [Kanai and Rees, 2011; Zilles and Amunts, 2013]''
\citep{zhen2015quantifying}.

%
``Divergent patterns of brain organization from the most common pattern (that
is, changes in the spatial arrangement of cortical regions) can be observed in
approximately 5–10\% of the healthy population [16,19], and care should
therefore be taken to avoid the undue influence of such outliers''
\citep{eickhoff2018imaging}.
%
``Topological outliers, if they do not result from artefacts, can also be
considered to be interesting cases of inter-individual variability to understand
brain-phenotype relationships [98]'' \citep{eickhoff2018imaging}.
%
``Recent studies have suggested that the topography (location and size)
of individual-specific brain parcellations is predictive of individual
differences in demographics, cognition, emotion and personality [3,5,99]''
\citep{eickhoff2018imaging}.
%
``Only by understanding the generic characteristic of topographic organization
can we start to appreciate idiosyncrasies and their relationships to
socio-demographic, cognitive or affective profiles''
\citep{eickhoff2018imaging}.

% PPA
``The spatial and selectivity variability of the SSRs may likely underlie
individual differences in navigation, especially when scene perception is being
utilized'' \citep{zhen2017quantifying}.

%
For example, processing oral way description and using it for planing,
remembering, executing navigation.

%
``Future studies will be needed to elaborate on the mechanism and functional
significance of the interindividual variability of the SSRs and to utilize the
variability in understanding the neural mechanisms of scene perception [Kanai
and Rees, 2011; Van Horn et al., 2008]'' \citep{zhen2017quantifying}.

%
Future studies need to elaborate on the significance of the variability of
functional regions and utilize this variability to explore the neural mechanisms
of specific behavioral or cognitive processes [Van Horn et al., 2008]''
\citep{zhen2015quantifying}.
%
Our study ``invites future work on the origin of the variability and its
relation to individual differences in behavioral performance''
\citep{zhen2015quantifying}.

%
Hence, further studies using dedicated auditory localizer should shed light on
the inter-individual differences of persons exhibiting a auditory \ac{ppa}.



\subsection{SRM study}

\todo[inline]{cf. general introduction \& and study 4}




\section{Vision: Functional atlas}

\todo[inline]{probably emphasis on general vision here in general discussion}

\todo[inline]{most text of this topic is in SRM discussion at the moment}

\todo[inline]{check discussion of \citet{jiahui2020predicting} as last resource}

\subsection{Previously: anatomical alignment}

%
Previous studies used anatomical alignment in order to map data from
probabilistic functional atlases onto individual subjects in order to predict
the location of functional areas in new/unknown subjects.

%
Anatomical alignment relies on data of an anatomical scan that is essentially
always performed as an additional scan during studies of brain functions using
fMRI.


\subsubsection{Now: naturalistic stimuli for functional alignment}

%
Our procedure relies on data of an additional functional scan session.
%
While the functional alignment can also be applied to fMRI data from
stimulation paradigms with simplified stimuli, the transformations for
functional alignment have greatly diminished general validity
\citep{haxby2011common}, presumably because such experiments sample a sparser
range of brain states \citep{guntupalli2016model}.
%
We need an additional scan session of functional data but naturalistic stimuli
``engage in parallel multiple neural systems for vision, audition, language,
person perception, social cognition, and other functions''
\citep{jiahui2020predicting}.


%
``SRM will improve sensitivity for detecting a cognitive process of interest in
the test data if the training stimuli or trials strongly and variably engage
that process in a way that is reliable across participants''
\citep{cohen2017computational}.
%
``The algorithm also can be applied to simpler, controlled experimental data,
but our previous results showed that the sampling of response vectors from these
experiments is impoverished and produces a model representational space that
does not generalize well to new stimuli in other experiments (Haxby et al.
2011)'' \citep{guntupalli2016model}.

%
``Naturalistic stimuli may better sample the full range of responses to faces
and other stimuli that contribute to face-selective topographies''
\citep{jiahui2020predicting}.

%
``Hyperalignment of data using a set of stimuli that is less diverse than the
movie is effective, but the resultant common space has validity that is limited
to a small subspace of the representational space in VT cortex''
\citep{haxby2011common}.

%
``Estimating the parameters to transform high-dimensional spaces from individual
brains into a common high-dimensional space requires a rich set of data that
samples a wide variety of cortical patterns in order to generalize to novel
stimuli or tasks.
%
For response hyperalignment, a rich variety of stimuli or conditions are
necessary to sample the response vector space.
%
For connectivity hyperalignment, the sampling of connectivity vector space is
defined by the selection of connectivity targets, but the richness and
reliability of connectivity estimates depends on the variety of brain states
over which connectivity is estimated'' \citep{haxby2020hyperalignment}.


%
Naturalistic stimuli promise an ``increased validity of derived transformation
for functional alignment by sampling a more diverse set of mental states  ''
[project proposal].
%
Compared to paradigms with simplified stimuli, naturalistic stimuli sample a
broad range of brain states \citep{guntupalli2016model, haxby2011common} that
reflect (confound) statistics of the natural environment, promising an increased
validity of transformations of functional alignment and enable investigation
of the acquired data for a variety of research questions to research
questions/domains/paradigms (e.g.  visual or auditory perception, spatial
cognition; emotion; music, speech or social perception).


\subsubsection{Now: naturalistic stimuli can substitute multiple localizers}
%
``From a single movie dataset multiple functional topographies can be estimated
\citep{guntupalli2016model}, whereas different localizers are typically required
to map different functional topographies, making a thorough mapping of selective
topographies time-consuming and inefficient'' \citep{jiahui2020predicting}.
%
``Consequently, movies have the potential to estimate selective topographies in
all of these domains'' \citep{jiahui2020predicting}.
%
``findings lay the foundation for developing an efficient tool for mapping
functional topographies from a database containing a wide range of perceptual
and cognitive functions to new subjects based only on fMRI data collected while
watching an engaging, naturalistic stimulus and other subjects' localizer data
from a normative sample'' \citep{jiahui2020predicting}.
%
``Such a tool would require a database of data for movies and a range of
functional localizers in a normative group of subjects''
\citep{jiahui2020predicting}.
%
``A new subject's functional topographies could be estimated based only on that
subject's movie data and other subjects' localizer data from the normative
database that could be projected into that subject's cortical anatomy using
hyperalignment transformation matrices derived from movie data and could replace
tedious functional localizers with an engaging movie''
\citep{jiahui2020predicting}.

%
The reference would enable an qualitative and quantitative description of an
individual's brain function with respect to such a norm, and consequently
progress the field towards neuroimaging studies of individual differences that
more closely resemble their psychological counterparts.

%
Once a valid alignment is established, known functional properties of the
(normative) reference can then be projected into the respective individual voxel
space (s. Fig. 1 in \citep{nishimoto2016lining}).
%
A naturalistic stimulus like a move or audio-description could be used to map a
variety $z$-maps created from a variety of $t$-contrast from a normative
reference group onto an individual subjects and thus potentially substitute a
variety of localizers.


\subsubsection{Additional (clinical) benefits of naturalistic stimuli}

``Compared with functional localizers, naturalistic stimuli provide several
advantages such as stronger and widespread brain activation, greater engagement,
and increased subject compliance'' \citep{jiahui2020predicting}.
%
``Movies are more engaging and result in better compliance
\citep{vanderwal2015inscapes}.
%
Movie viewing can also be used in subject populations, such as children
\citep{richardson2018development} or patients, that may have trouble maintaining
attention during repetitions of a tedious localizer task''
\citep{jiahui2020predicting}.

%
On the one hand, an engaging naturalistic stimulus before the main experiment
would have the benefit of putting a study participant at ease and letting the
subject accommodate to the scanner environment).
%
On the other hand, an engaging naturalistic stimulus after the main experiment
would not suffer less from fatigue than a one localizer or even a localizer
battery demanding voluntary attention and handling a repetitive task.


\subsubsection{Partial alignment would lower scanning time}
%
A \textit{calibration scan} could be used to align a subject to a fixed \ac{cfs}
based on extensive scans and analyses of other subjects' data serving as a
reference / functional atlas.

%
Similar to the standard procedure of a anatomical scan, additional 15 minutes
functional scanning using an engaging naturalistic stimuli  could provide
sufficient data to perform a functional alignment to a functional atlas.


\subsection{Clinical Application}

\todo[inline]{cf. general introduction}

\paragraph{template from \citet{wegrzyn2018thought}}

``Functional localization has proven to be of direct practical use [Bunzl
(2010). An Exchange about Localism. In: Foundational Issues in Human Brain
Mapping, p. 49–54; Szaflarski (2017). Practice guideline summary: Use of fMRI in
the presurgical evaluation of patients with epilepsy. Neurology].
%
In the clinical context, fMRI plays an important role for planning surgery in
patients with tumors or epilepsies, as it aids the understanding of which parts
of the brain need to be spared in order to preserve sensory, motor or cognitive
abilities [Stippich (2017). Introduction to Presurgical Functional MRI. In:
Clinical Functional MRI, p. 1–7.]'' \citep{wegrzyn2018thought}.

``To be useful for clinical diagnostics and prognostics, fMRI data must be
interpretable on the level of the individual case \citep{dubois2016building}.
%
Because in group studies idiosyncratic activity patterns can be obscured by
averaging, the precise mapping of brain function in a single person has become a
vanguard of fMRI research [\citet{laumann2015functional, huth2016natural,
gordon2017precision}]'' \citep{wegrzyn2018thought}.


\paragraph{subgroup matching/classification}
%
Imagine you scan a new, unknown subject for just 15 minutes more, and you
additionally get results from a whole variety of other paradigms mapped onto
that brain: results from localizers of low-level perceptual processes, but also
higher-level cognitive processes like language, memory, emotions and so on.
%
If you have different \acp{cfs} for different subgroups, you could investigate
which alignment onto which subgroup's \ac{cfs} results in less
error letting you classify to which subgroup your new subject might belong to.


\subsection{Caveats of naturalistic stimuli}

%
Challenging data data analysis, but create \& share annotation.
%
Hence, data analysis pipelines should be implemented in common and
well-documented packages and code, and custom code shared along the paper.

%
Just an approximation of real life.
%
Setting is still the scanner.
%
Passive watching \& listening. Hence, executive functions?


\section{Conclusion (on naturalistic stimuli)}
%
In summary, naturalistic stimuli ``impose a meaningful timecourse across
subjects while still allowing for individual variation in brain activity and
behavioral responses, and lend themselves to a broader set of analyses than
either pure rest or pure event-related task designs'' \citep{finn2017can}.
%
``Naturalistic paradigms do not aim to replace the classic, controlled
neuroimaging paradigms (Sonkusare et al., 2019). Due to their complexity and
current limitations in understanding the statistical properties of different
features in naturalistic conditions, naturalistic stimuli are not optimal for
model development [see, e.g., Rust and Movshon, 2005]. Controlled experiments
are still needed for hypothesis testing and developing models, while
naturalistic stimuli are best employed to test models in ecologically valid
settings and to expand them to situations where context matters
more'' \citep{saarimaki2021naturalistic}.
