\section{General recapitulation of the whole project \& its aims}

\todo[inline]{supposed to be a short intro}

\todo[inline]{cf. general introduction: overview of (non-specific) aims}


\section{Open, transparent, and reproducible science}

\todo[inline]{I guess the best place in the general discussion is here}

\todo[inline]{cf. general introduction}

\todo[inline]{you might give a high-level summary of all studies here already,
and thus do not need to do it before summarizing results of individual studies}


\subsection{Data re-use for new investigation \& new results}

\todo[inline]{mih: du hast mit dem PPA paper eine Publikation in SciData, weil
du separate und unabhängig veröffentlichte Daten, durch zusätzliche, von der
erstellte Werte, ganz neu betrachten und interpretieren konntest}

\todo[inline]{Du zeigst damit, dass Idee studyforrest erfolgreich sein kann}


\subsection{Reproducibility \& sharing}

\todo[inline]{mih: Datenveröffentlichungen nochmal hervorheben; Code repos mit
ausführbaren Analysen hervorheben; du hast die von dir gesetzten Ziele auch
erreicht}


\subsection{Pro \& Contra open, reproducible research}

\todo[inline]{Du bist daher in der Lage in persönliches Fazit zu ziehen: was hat
es der Wissenschaft gebracht, dass diese Daten "schon da waren"; Was hat es dir
gebracht? Was war schwerer als wenn du die Daten selbst erhoben hättest? Was
wäre dann aber nicht möglich gewesen.}


\paragraph{Contra}

\todo[inline]{my project is using existing data; new investigation; new results}
%
It takes fucking time with just minor immediate contribution to the workflow.
%
That is especially the cause because it is not standard yet (established tools
of trade?); it is not taught in study programs (but needs to be standard).
Thus it is an added ``burden'' during PhD thesis.
%
People who do not give a shit about it and don't do it, do ``worse'' research
but are faster/``better''.

%
Opportunity costs. It is not just about the raw data but also about
preprocessing, templates, results of the localizer. For other persons, its also
my speech annotation; we had no mask for RSC and transverse occipital sulcus.
%
Hence, do everything possible to make it really, really good 'cause other people
will rely on your shit.

%
Vice versa: Can you trust published data? Balance of checking everything vs.
taking things for granted.
%
The unknown unknowns of dataset creators? What did they fail to consider?
Standards may vary depending on the use case. Assume that everything that is not
explicitly stated in a paper along the dataset was not done?


\paragraph{Pro}

\todo[inline]{think big}



\section{Recapitulation of work packages}

\todo[inline]{cf. general introduction; objectives/aims per study}


\subsection{Speech anno}

\todo[inline]{cf. paper \& general introduction}


\subsection{PPA paper}

\todo[inline]{cf. study \& general introduction}

\todo[inline]{annotations are hard but you can do it}

\todo[inline]{traditional GLM ``works''}

%
``Future studies that aim to use a movie to localize visual areas in individual
participants should extensively annotate the content of frames (e.g., using the
open-source solution ``Pliers''\citep{mcnamara2017developing} for feature
extraction from a visual naturalistic stimulus)''
\citep{haeusler2022processing}.

%
Our modeling approach is pretty similar because adopted from classical
paradigms.
%
Modeling was shitty (especially in AV);
%
still results suggest: the response to spatial information must be somewhere in
there.
%
However, I did not test quality/reliability of individual results depending on
quantity of runs to assess naturalistic stimulus as potential replacement;


\subsection{SRM study}

\todo[inline]{cf. general introduction \& and study4}




\section{Vision: Functional atlas}

\todo[inline]{probably emphasis on general vision here in general discussion}

\todo[inline]{most text of this topic is in SRM discussion at the moment}


\subsection{Functional atlas: Naturalistic stimuli as multiple localizer
(vision)}

\todo[inline]{if an functional atlas would exist, a movie could substitute a
variety of localizers}

Project proposal: ``Moreover, naturalistic stimuli sample a broader range of
brain states paradigms with simplified stimuli \citep{guntupalli2016model,
haxby2011common} promising an increased validity of transformations of
functional alignment and increased generalizability to [research
questions/domains/paradigms].
%
Naturalistic stimuli promise an ``increased validity of derived transformation
for functional alignment by sampling a more diverse set of mental states that
reflect (confound) statistics of the natural environment, and enable
investigation of the acquired data for a variety of research questions (e.g.
visual or auditory perception, spatial cognition; emotion; music, speech or
social perception)'' [project proposal].

``Results suggests that the validity for a model of a specific subspace may be
enhanced by designing a stimulus paradigm that samples the brain states in that
subspace more extensively'' \citep{haxby2011common}.


\subsubsection{Template from \citet{jiahui2020predicting}}

\todo[inline]{check discussion of \citet{jiahui2020predicting} as last resource}

\paragraph{general idead}

``findings lay the foundation for developing an efficient tool for mapping
functional topographies for a wide range of perceptual and cognitive functions
in new subjects based only on fMRI data collected while watching an engaging,
naturalistic stimulus and other subjects' localizer data from a normative
sample'' \citep{jiahui2020predicting}.

%
``These results lay a foundation for building a computational tool with a
database that could allow others to map multiple functional topographies in new
subjects using only data collected during movie viewing''
\citep{jiahui2020predicting}.

%
``Such a tool would require a database of data for movies and a range of
functional localizers in a normative group of subjects''
\citep{jiahui2020predicting}.

%
``A new subject's functional topographies could be estimated based only on that
subject's movie data and other subjects' localizer data from the normative
database that could be projected into that subject's cortical anatomy using
hyperalignment transformation matrices derived from movie data''
\citep{jiahui2020predicting}.

%
``Such a resource would be more efficient and replace tedious functional
localizers with an engaging movie and could enable mapping of multiple
functional topographies with data from a single fMRI using a naturalistic
stimulus'' \citep{jiahui2020predicting}.

\paragraph{why movies are so fucking awesome}

%
``Functional localizers are inefficient because they only estimate one or a few
topographies for each localizer'' \citep{jiahui2020predicting}.

%
``Movies, by contrast, engage in parallel multiple neural systems for vision,
audition, language, person perception, social cognition, and other functions''
\citep{jiahui2020predicting}.

%
``From a single movie dataset multiple functional topographies can be estimated
\citep{guntupalli2016model}, whereas different localizers are typically required
to map different functional topographies, making a thorough mapping of selective
topographies time-consuming and inefficient'' \citep{jiahui2020predicting}.

%
``Consequently, movies have the potential to estimate selective topographies in
all of these domains'' \citep{jiahui2020predicting}.


``Compared with functional localizers, naturalistic stimuli provide several
advantages such as stronger and widespread brain activation, greater engagement,
and increased subject compliance'' \citep{jiahui2020predicting}.

%
``Naturalistic stimuli may better sample the full range of responses to faces
and other stimuli that contribute to face-selective topographies''
\citep{jiahui2020predicting}.

%
``Movies also simulate better the statistics of natural viewing and listening
and may provide more ecologically valid maps'' \citep{jiahui2020predicting}.

%
``Movies are more engaging and result in better compliance
\citep{vanderwal2015inscapes}.
%
Movie viewing can also be used in subject populations, such as children
\citep{richardson2018development} or patients, that may have trouble maintaining
attention during repetitions of a tedious localizer task''
\citep{jiahui2020predicting}.



\subsubsection{Functional alignment works better with naturalistic stimuli}

% Cohen 2017
``SRM will improve sensitivity for detecting a cognitive process of interest in
the test data if the training stimuli or trials strongly and variably engage
that process in a way that is reliable across participants''
\citep{cohen2017computational}.

%
``The algorithm also can be applied to simpler, controlled experimental data,
but our previous results showed that the sampling of response vectors from these
experiments is impoverished and produces a model representational space that
does not generalize well to new stimuli in other experiments (Haxby et al.
2011)'' \citep{guntupalli2016model}.

%
``Hyperalignment of data using a set of stimuli that is less diverse than the
movie is effective, but the resultant common space has validity that is limited
to a small subspace of the representational space in VT cortex''
\citep{haxby2011common}.

%
``Estimating the parameters to transform high-dimensional spaces from individual
brains into a common high-dimensional space requires a rich set of data that
samples a wide variety of cortical patterns in order to generalize to novel
stimuli or tasks.
%
For response hyperalignment, a rich variety of stimuli or conditions are
necessary to sample the response vector space.
%
For connectivity hyperalignment, the sampling of connectivity vector space is
defined by the selection of connectivity targets, but the richness and
reliability of connectivity estimates depends on the variety of brain states
over which connectivity is estimated'' \citep{haxby2020hyperalignment}.


%
``Analogously, the introduction of dynamic videos of faces and control
categories to localize face-selective topographies provides more reliable maps
and better estimate the extent of face-selective regions than do localizers with
still image stimuli [Fox et al., 2009; Pitcher et al., 2011]''
\citep{jiahui2020predicting}.




\subsubsection{Partial alignment would lower scanning time}
%
A \textit{calibration scan} could be used to align a subject to a fixed \ac{cfs}
based on extensive scans and analyses of other subjects' data serving as a
reference / functional atlas.

``While the functional alignment can also be applied to fMRI data from
stimulation paradigms with simplified stimuli, the transformations for
functional alignment have greatly diminished general validity
\citep{haxby2011common}, presumably because such experiments sample a sparser
range of brain states \citep{guntupalli2016model}'' [project proposal].

%
Once a valid alignment is established, known functional properties of the
(normative) reference can then be projected into the respective individual voxel
space (s. Fig. 1 in \citep{nishimoto2016lining}).
%
The reference would enable an qualitative and quantitative description of an
individual's brain function with respect to such a norm, and consequently
progress the field towards neuroimaging studies of individual differences that
more closely resemble their psychological counterparts.

%
``Single participant studies can also provide valuable impulses for the use of
fMRI as a clinical tool.
%
This includes the possibility to assess how stable results are within a single
participant, and how much data should be collected to provide a reliable
description of the individual’s functional brain organization
[\citet{laumann2015functional, gordon2017precision}] \citep{wegrzyn2018thought}.


\subsection{Clinical Application}

\todo[inline]{cf. general introduction}

\paragraph{template from \citet{wegrzyn2018thought}}

``Although there is no one-to-one mapping between brain region and cognitive
process [Cacioppo (2007). Pscyhophyisological Science. In: Handbook of
psychophysiology, p. 3–23.], functional localization has proven to be of direct
practical use [Bunzl (2010). An Exchange about Localism. In: Foundational Issues
in Human Brain Mapping, p. 49–54; Szaflarski (2017). Practice guideline summary:
Use of fMRI in the presurgical evaluation of patients with epilepsy. Neurology].
%
In the clinical context, fMRI plays an important role for planning surgery in
patients with tumors or epilepsies, as it aids the understanding of which parts
of the brain need to be spared in order to preserve sensory, motor or cognitive
abilities [Stippich (2017). Introduction to Presurgical Functional MRI. In:
Clinical Functional MRI, p. 1–7.]'' \citep{wegrzyn2018thought}.

``To be useful for clinical diagnostics and prognostics, fMRI data must be
interpretable on the level of the individual case \citep{dubois2016building}.
%
Because in group studies idiosyncratic activity patterns can be obscured by
averaging, the precise mapping of brain function in a single person has become a
vanguard of fMRI research [\citet{laumann2015functional, huth2016natural,
gordon2017precision}].
%
These studies are important to deepen our understanding of how the brain works,
because the functional organization of brains becomes more heterogeneous on a
finer anatomical scale [\citet{laumann2015functional, poldrack2017precision}].

%
Also, when looking at increasingly smaller ‘regions of the mind’, such as the
neural correlates of specific words instead of language in general, averaging on
the group level can obscure the fine spatial information which allows to
differentiate these contents in the individual brain [\citet{huth2016natural}]''
\citep{wegrzyn2018thought}.


\paragraph{subgroup matching/classification}
%
Imagine you scan a new, unknown subject for just 15 minutes more, and you
additionally get results from a whole variety of other paradigms mapped onto
that brain: results from localizers of low-level perceptual processes, but also
higher-level cognitive processes like language, memory, emotions and so on.
%
If you have different \acp{cfs} for different subgroups, you could investigate
which alignment onto which subgroup's \ac{cfs} results in less
error letting you classify to which subgroup your new subject might belong to.


\subsection{Caveats of naturalistic stimuli}

%
Challenging data data analysis, but create \& share annotation.
%
Hence, data analysis pipelines should be implemented in common and
well-documented packages and code, and custom code shared along the paper.

%
Just an approximation of real life.
%
Setting is still the scanner.
%
Passive watching \& listening. Hence, executive functions?


\section{Conclusion (on naturalistic stimuli)}
%
In summary, naturalistic stimuli ``impose a meaningful timecourse across
subjects while still allowing for individual variation in brain activity and
behavioral responses, and lend themselves to a broader set of analyses than
either pure rest or pure event-related task designs'' \citep{finn2017can}.
%
``Naturalistic paradigms do not aim to replace the classic, controlled
neuroimaging paradigms (Sonkusare et al., 2019). Due to their complexity and
current limitations in understanding the statistical properties of different
features in naturalistic conditions, naturalistic stimuli are not optimal for
model development [see, e.g., Rust and Movshon, 2005]. Controlled experiments
are still needed for hypothesis testing and developing models, while
naturalistic stimuli are best employed to test models in ecologically valid
settings and to expand them to situations where context matters
more'' \citep{saarimaki2021naturalistic}.
