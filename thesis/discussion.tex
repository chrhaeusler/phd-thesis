\section{Study 1}
%
From OHBM-Abstract: vgl. long abstract in doc-file
%
The present results are evidence that a functionally de ned region, such as the
PPA, can be localized using a complex natural stimulus, by exploiting its
inherent temporal structure with respect to a particular hypothesized cognitive
or perceptual function.
%
While natural stimulation is not a universal solution, it can represent one
approach to overcome the threat of lacking generalizability associated with
small and simpli ed stimulus sets.
%
Using this approach we replicate a previous nding of PPA modulation by verbal
stimuli with descriptions of spatial con gurations.
%
This adds to the evidence that the PPA is not exclusively involved in visual
processing, but a bilateral anterior portion also exhibits activation increases
correlated with the semantic processing of verbal statements of spatial
references in a rich natural auditory stimulus.


\section{Study 2}



\section{Study 3}

N=14;


\section{naturalistic stimuli}

\paragraph{pro}

\todo[inline]{repeat the usual stuff}

\todo[inline]{annotations are hard but you can do it}

\todo[inline]{traditional GLM ``works''}


\paragraph{contra}
%
just an approximation;
%
setting is still the scanner;
%
passive watching, listening: executive functions?
%
challenging data data analysis;
%
solutions -> share annotations,
%
implement data analysis pipelines in common and well-documented packages and code


\paragraph{interim summary}

``Naturalistic paradigms do not aim to replace the classic, controlled
neuroimaging paradigms (Sonkusare et al., 2019). Due to their complexity and
current limitations in understanding the statistical properties of different
features in naturalistic conditions, naturalistic stimuli are not optimal for
model development (see, e.g., Rust and Movshon, 2005). Controlled experiments
are still needed for hypothesis testing and developing models, while
naturalistic stimuli are best employed to test models in ecologically valid
settings and to expand them to situations where context matters
more'' \citep{saarimaki2021naturalistic}.

\todo[inline]{my project is using existing data; new investigation; new results}


\section{Open, transparent, and reproducible science}

\todo[inline]{mih: In der General Diskussion würde ich unbedingt den Aspekt
Reproducibility wieder aufgreifen, mit einem eigenen Abschnitt, damit es auch in
die Gliederung kommt mit folgenden Aspekten: Datenveröffentlichungen nochmal
hervorheben; Code repos mit ausführbaren Analysen hervorheben; du hast du von
dir gesetzten Ziele auch erreicht; du hast mit dem PPA paper eine Publikation in
SciData, weil du separate und unabhängig veröffentlichte Daten, durch
zusätzliche, von der erstellte Werte, ganz neu betrachten und interpretieren
konntest; Du zeigst damit, dass die Idee studyforrest erfolgreich sein kann; Du
bist daher in der Lage in persönliches Fazit zu ziehen: was hat es der
Wissenschaft gebracht, dass diese Daten "schon da waren". Was hat es dir
gebracht? Was war schwerer als wenn du die Daten selbst erhoben hättest? Was
wäre dann aber nicht möglich gewesen.}

\todo[inline]{mih: Sowas in das finale der Arbeit zu schreiben, hebt alles
nochmal auf eine Metaebene und könnte für Reviewer nochmal aus einer ganz
anderen Richtung interessant sein}


%
The non-standard procedure that needs to become standard (in study curriculum)

\subsection{input = open science \& studyforrest}

\subsection{analyses, code, output}



\section{Future research}

\paragraph{Other functional areas}

Again, data from the studyforrest project will be used to perform assessments
analog to WP1e for a range of additional individually determined regions based
on established localizer paradigms (Sengupta, et al., 2016).
%
These regions include: the \ac{ffa}, \ac{ppa}, and \ac{eba}  which are
associated with face perception \citep{kanwisher1997ffa,
pitcher2011occipitalfacearea}, scene perception \citep{epstein1998ppa}, and the
perception of human bodies \citep{downing2001bodyarea}, respectively.

``First, our brains directly process exogenous information about the external
environment by transducing physical phenomena (e.g., changes in energy,
molecular concentrations, etc.) into sensory perceptions that allow us to
generate and maintain a sense of what is happening around us (1, 2). Mental
representations that are directly driven by the external world are likely to be
highly similar across individuals who share the same sensory experience. Second,
our brains also process endogenous information that reflects our current
internal homeostatic states, past experiences, and future goals (3). The
integration of exogenous and endogenous informa- tion allows us to meaningfully
interpret our surroundings, prioritize information that is relevant to our
goals, and develop action plans (4). Given the same input information,
individuals may have unique interpretations, feelings, and plans, often leading
endogenous rep- resentations to be idiosyncratic across individuals''
\citep{chang2021endogenous}.

``Human brains have much in common with one another. Similarities exist not only
at the anatomical level, but also in terms of functional organization. Given the
same stimulus—an expanding ring, for example—regions of the brain that process
sensory (visual) stimuli will respond in a highly predictable and similar manner
across differ- ent individuals. This predictability is not limited to sensory
systems: shared activity across people has also been observed in higher-order
brain regions (for example, the default mode network6, or DMN) dur- ing the
processing of semantically complex real-life stimuli such as movies and
stories7–13. Notably, shared responses in these high-order areas seem to be
associated with narrative content and not with the physical form used to convey
it11,14,15. It is unknown, at any level of the cortical hierarchy, to what
extent the similarity of human brains during shared perception is recapitulated
during shared recollection. This prospect is made especially challenging when
recall is spontane- ous and spoken, and the selection of details is left up to
the remem- berer (rather than the experimenter), as is often the case in real
life'' \citep{chen2017shared}.


\paragraph{connectivity based hyperalignment?}
%
It should be noted that the number of voxels that can be considered
simultaneously for functional BOLD response time series alignment is limited by
the number of timepoints in the calibration scan (about 300-400 voxels for a
15min scan with a 2s TR, corresponding to a local cortical neighborhood of about
1cm in diameter for a standard resolution).
%
This limitation does not exist in this form for a functional alignment that is
based on connectivity vectors.
%
The length of these connectivity vectors is determined by the number of
reference (or seed) regions in the brain.


\section{Vision}
%
Moreover, beyond the scope of this project the targeted homogenization of
acquisition procedures will further the goal of large scale data collection for
the purpose of producing a normative reference of brain function as measured by
fMRI.
%
The availability of such a reference would enable quantitative and qualitative
description of an individual's brain function with respect to such a norm, and
consequently progress the field towards neuroimaging studies of individual
differences that more closely resemble their psychological counterparts
\citep{dubois2016building}.
