\section{General recapitulation of the whole project \& its aims}

\todo[inline]{cf. general introduction}


\section{Recapitulation of work packages}

\todo[inline]{cf. general introduction; objectives/aims per study}


\subsection{Speech anno}

lore ipsum...


\subsection{PPA paper}
%
``Future studies that aim to use a movie to localize visual areas in individual
participants should extensively annotate the content of frames (e.g., using the
open-source solution ``Pliers''\citep{mcnamara2017developing} for feature
extraction from a visual naturalistic stimulus)''
\citep{haeusler2022processing}.
%
However, I did not test quality/reliability of individual results depending on
quantity of runs to assess naturalistic stimulus as potential replacement;

%
Our modeling approach is pretty similar because adopted from classical
paradigms.

%
Modeling was shitty (especially in AV);
%
still results suggest: the response to spatial information must be somewhere in
there


\subsection{SRM study}

lore ipsum...


\section{Naturalistic stimuli}

\subsection{Pro naturalistic stimuli}

\todo[inline]{annotations are hard but you can do it}

\todo[inline]{traditional GLM ``works''}

%
``Compared with functional localizers, naturalistic stimuli provide several
advantages such as stronger and widespread brain activation, greater engagement,
and increased subject compliance'' \citep{jiahui2020predicting}.

\subsection{Contra naturalistic stimuli}

%
Just an approximation.
%
Setting is still the scanner.
%
Passive watching \& listening. Hence, executive functions?
%
Challenging data data analysis, but create \& share annotation.
%
Implement data analysis pipelines in common and well-documented packages and
code, and share it along the paper.


\section{Open, transparent, and reproducible science}

\todo[inline]{cf. general introduction}

\subsection{Data re-use for new investigation \& new results}


\todo[inline]{mih: du hast mit dem PPA paper eine Publikation in SciData, weil
du separate und unabhängig veröffentlichte Daten, durch zusätzliche, von der
erstellte Werte, ganz neu betrachten und interpretieren konntest}

\todo[inline]{Du zeigst damit, dass die Idee studyforrest erfolgreich sein kann}



\subsection{Reproducibility \& sharing}

\todo[inline]{mih: Datenveröffentlichungen nochmal hervorheben; Code repos mit
ausführbaren Analysen hervorheben; du hast die von dir gesetzten Ziele auch
erreicht}


\subsubsection{Pro}

\todo[inline]{Du bist daher in der Lage in persönliches Fazit zu ziehen: was hat
es der Wissenschaft gebracht, dass diese Daten "schon da waren"; Was hat es dir
gebracht? Was war schwerer als wenn du die Daten selbst erhoben hättest? Was
wäre dann aber nicht möglich gewesen.}


\subsubsection{Contra}

\todo[inline]{my project is using existing data; new investigation; new results}

%
It takes fuckin time with just minor immediate contrubtion to the workflow.
%
That is especially the cause because it is not standard yet (established tools
of trade?); it is not taught in study programmes (but needs to be standard).
Thus it is an added ``burden'' during PhD thesis.
%
People who do not give a shit about it and don't do it, do ``worse'' research
but are faster/``better''.

%
Opportunity costs. It is not just about the raw data but also about
preprocessing, templates, results of the localizer. For other persons, its also
my speech annotation.
%
Hence, do everything possible to make it really, really good 'cause other people
will rely on your shit.

%
Vice versa: Can you trust published data? Balance of checking everything vs.
taking things for granted.
%
The unknown unknowns of dataset creators? What did they fail to consider?
Standards may vary depending on the use case. Assume that everything that is not
explicitly stated in a paper along the dataset was not done?


\section{Conclusion}
%
In summary, naturalistic stimuli ``impose a meaningful timecourse across
subjects while still allowing for individual variation in brain activity and
behavioral responses, and lend themselves to a broader set of analyses than
either pure rest or pure event-related task designs'' \citep{finn2017can}.
%
``Naturalistic paradigms do not aim to replace the classic, controlled
neuroimaging paradigms (Sonkusare et al., 2019). Due to their complexity and
current limitations in understanding the statistical properties of different
features in naturalistic conditions, naturalistic stimuli are not optimal for
model development (see, e.g., Rust and Movshon, 2005). Controlled experiments
are still needed for hypothesis testing and developing models, while
naturalistic stimuli are best employed to test models in ecologically valid
settings and to expand them to situations where context matters
more'' \citep{saarimaki2021naturalistic}.
