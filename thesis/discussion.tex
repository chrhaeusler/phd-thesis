\section{General recapitulation of the whole project \& its aims}

\todo[inline]{following paragraph is from intro: shorten it!}

%
Traditional studies averaged data across participants.
%
However, fMRI need to be analyzed on the level of an individual person in order
to advance brain mapping towards an clinical application.
% functional localizer
Functional localizer paradigms aim to characterize the location size, and shape
of functional areas on the level of individual subject.
%
Therefore, they are a promising tool to advance brain mapping towards a clinical
application.
% contra localizers
However, traditional localizer paradigms employ selectively sampled,
tightly-controlled stimuli, rely heavily on a participant's compliance, and can
usually map just one domain of brain functions
% naturalistic stimuli could replace
Naturalistic stimuli have a couple of advantages over a paradigms employing
carefully chosen, tightly-controlled, simplified stimuli.
% fixed time-course
a) naturalistic stimuli offer a fixed time-course correlating with a variety of
different brain functions ranging from low-level perception (e.g., luminance) to
high-level cognition (e.g., social cognition);
% richness
b) the richness and multi-dimensionality of naturalistic stimuli might impose a
challenging data analysis put promises a higher ecological and external
validity;
% compliance
c) naturalistic stimuli are an easy-to-follow and immersive paradigm leading to
improved data quality.
% visually impaired
Lastly, an exclusively auditory stimulus like an audiobook or audio drama would
also be appropriate for visually impaired persons, e.g., suffering from
nystagmus or lack of eyesight.
% therefore
Therefore, this dissertation's aim was --- while following the principles of
open, transparent, and reproducible science --- to explore  a movie and the
movie's audio-description could, in principle, substitute a traditional
localizer paradigm.
%
The dissertation focussed on the \ac{ppa}, a fucking classic functional area of
higher-visual perception.

%
First (study 1), we extended the studyforrest dataset.
%
Second (study 2)  and similarly to traditional localizer paradigms, we modeled
hemodynamic activity based on annotated stimulus features embedded in the movie
``Forrest Gump'' and its audio-description, and created GLM $t$-contrasts in
order to locate the \ac{ppa}.
%
Third (study 3), we estimated results of the localizer by projecting the data
through a \ac{cfs}.


\section{Open, transparent, and reproducible science}

The overarching goal of this dissertation was to perform all three studies under
the principles of open, transparent, and reproducible research.
%
Hence, on a metalevel, this dissertation aims to meet both the requirements of
open, accessible, shared, and transparent science \citep{watson2015will,
fecher2014open} as well as the requirements of a reproducible and replicable
research project:
%
the dissertation follows guidelines and best practices for a) coding and
scientific computing \citep{wilson2014best}, b) procedures and data analyses
\citep{nichols2017best, poldrack2017scanning, poldrack2019establishment}, and c)
sharing code, created data, and results \citep{eglen2017toward, nichols2017best,
pernet2015improving}.


\subsection{Extend open science project}

The additional stimulus annotations that have been created over the course of
the dissertation are version-controlled and published in a standardized file
format \citep{haeusler2021speechanno}, and therefore contribute to the
studyforrest project as a resource for the scientific community.

\subsection{Data re-use for new investigation \& new results}
%
Two peer-reviewed papers published based on existing data that were extended and
thus analyzed and interpreted from a new perspective that was not forseen in the
original publication of the dataset.
%
We have shown that the idea of studyforrest can be successful.

%
My project is using existing data, performs a new investigation, generates new
results.

% \paragraph{input data}
% open data \citep{eglen2017toward}
First, my work is built on top of publicly and freely available \ac{fmri} data
that are part of the \textit{studyforrest} project
(\href{www.studyforrest.org}{\url{studyforrest.org}}).
%
The studyforrest project is an open science project that aims to provide a
versatile resource for investigating human brain function under quasi-natural
conditions.
% core stimuli
The core of this dataset are two-hour long \ac{bold} \ac{fmri} scans of
participants watching the movie Forrest Gump \citep{ForrestGumpMovie} and
listening to the movie's audio-description that was created for a visually
impaired audience by adding a narrator to the movie's audio track.
% used by 3rd parties
Since its first publication in 2014 \citep{hanke2014audiomovie}, the
studyforrest project has served as a resource of raw (and preprocessed) data for
international working groups to conduct and publish independent, peer-reviewed
research (s.
\href{www.studyforrest.org/publications.html}{\url{studyforrest.org/publications.html}}).


\subsection{Reproducibility}

% \paragraph{code, analyses, output} intro
Further, all code is shared to improve reproducibility of current results and to
facilitate replicability of findings on other datasets.
% automatization
Therefore, data analyses pipelines are designed in a way that enables automated
processing.
% FOSS
Analyses pipelines are not implemented in proprietary software but in freely
available and, if possible, open-source software.
% which tools to choose why?
Among potential software packages, we chose the tools that offer the most solid
documentation, and a broad basis of developers and maintainers to ensure
long-term support.
% my code
Custom code written by myself is written in open-source programming languages
(Python and Bash), is version-controlled, documented, and released publicly and
freely accessible.


\subsection{Data Sharing}

\todo[inline]{mih: die von dir in intro gesetzten Ziele sind auch erreicht}

We published an extension for the studyforrest dataset (annotation bottleneck).
%
This extension is standardized and available as online repositories comprising
raw data, code, and results.
%
All analyses can be rerun and validated.

% DataLad
All input data, custom code, analysis steps and output data are accessible in
standardized \textit{DataLad} (\href{www.datalad.org}{datalad.org}) datasets.
Since DataLad provides a free and open-source software solution that manages
provenience, distribution, and version-control of code and data
\citep{halchenko2021datalad}, all executed steps from downloading the input data
to visualizing the results can be rerun to check and validate the dissertation's
results.

% \paragraph{publications}
% open-access publishing
Last, because ``nature abhors a paywall'' \citep{dupre2020nature}, publications
describing generated data, reasoning of methodological choices, analysis steps,
and results are published in open-access journals.
% neurovault
Unthresholded statistical maps of all computed statistical $t$-contrasts are
additionally published at Neurovault
(\href{https://neurovault.org/}{neurovault.org}).

\subsection{Personal assessment of following open science}

Following these principles of open science came with a couple of disadvatages
and advantages.


\subsubsection{Contra}

\paragraph{Additional work load}

You need to do stuff that is not necessary or as just minor benefit for your
current project.

%
That is especially the cause because working with the additional aim of data
publishing is not standard yet.
%
Tools of trade, best practices are in the process of being established.
%
It is not taught in study programs (but needs to be standard).


\paragraph{Opportunity costs}
%
It is not just about the raw data but also about preprocessing, templates,
results of the localizer.
%
Sengupta's data were analyzed (and published) in voxel space.
%
I could have run the analyses/contrasts on the surface but that needs time (not
so importante anyway because \ac{fa} was performed in voxel-space, too.
%
In case of the PPA, we had no mask for RSC and transverse occipital sulcus (cf.
reviewer response.


\paragraph{Trustful data?}
%
Can you trust published data? Balance of checking everything vs.  taking things
for granted.
%
The unknown unknowns of dataset creators? What did they fail to consider?
Standards may vary depending on the use case. Assume that everything that is not
explicitly stated in a paper along the dataset was not done and needs to be
checked?

%
Is also a ``problem'' of data sharing: you save time and money (especially if
you do not collect the data yourself) but you have to trust that the data are
``good'' and start at some point to reach goals that you would not accomplished
if you would have done all from scratch (i.e. preprocessing).

%
Hence, do everything possible to make it really, really good 'cause other people
will rely on your shit.

%
Still, re-users need to check the data according to their use case! It will
never be perfect and there is no "one size fits all".

%
For other persons, it's also my speech annotation


\paragraph{An additional burden}
%
You have to make your data and script extra pretty so other people can benefit
from your work but you do not benefit yourself from doing more than is necessary
for your own project.
%
Thus it is an added ``burden'' during PhD thesis.
%
People who do not give a shit about it and don't do it, do ``worse'' research
but are faster/``better''.
%
Other working groups can be faster and can use your data for similar findings


\subsubsection{Pro}
%
Why was is good for science that the data were already existing?
%
Why was is good for you that the data were already existing?

%
What was more diffcult compared to when data were acquired by myself?
%
But what would not have been possible if the data would have needed to be
acquired on my own?

From introduction:
% reproducibility crisis
Over the last decade, there has been a growing awareness that results of
scientific publications are not reproducible or general scientific findings are
not replicable letting some authors speak of a ``reproducibility crisis'' or
``replication crisis'' in the sciences \citep{baker2016reproducibility,
plesser2018reproducibility, stupple2019reproducibility, nosek2022replicability}.
% reproducibility: definition
``A study is reproducible if all of the code and data used to generate the
numbers and figures in the paper are available and exactly produce the published
results'' \citep{leek2017most}.
% replicability: definition
A study is replicable if the same analysis of an equivalent experiment's data
leads to consistent results \citep{dubois2016building, leek2017most}.

\todo[inline]{think big regarding reproducibility}


\section{Recapitulation of work packages}


\subsection{Speech anno}

Annotations are hard but you can do it (cf. "bottleneck").


\subsection{PPA paper}
%
Similarly to previous studies/classical localizer, we also model hemodynamic
responses based on the event structure.

%
But ``movies also simulate better the statistics of natural viewing and
listening and may provide more ecologically valid maps''
\citep{jiahui2020predicting}.

\subsubsection{Traditional GLM ``works''}
%
Despite ad hoc / exploratory approach (a.k.a. shitty modeling of subjectively
assessed events), results suggest: the response to spatial information must be
somewhere in there and is detectable.
%
In the PPA study,  I did not test the quality/reliability of individual results
depending on quantity of runs to assess naturalistic stimulus as potential
replacement.


\subsubsection{Future studies on PPA}

\todo[inline]{cf. reviewer response of PPA study}
%
We need dedicated studies; question for standard experiment is: is it due to
auditory information or due to being an naturalistic stimulus and it's "more
random sampling" and modeling of spatial information.

%
Naturalistic stimuli are not a panacea but traditional paradigms and
naturalistic paradigms should be used in tandem / reciprocially to generate new
hypotheses and progress our understanding of the brain.


\subsubsection{Future studies: other functional areas}
%
A full feature film might substitute traditional localizer paradigms dedicated
localizer by mapping a variety of brain functions beyond category-specific
visual areas.
%
Next step: other functional areas in studyforrest dataset [cf. discussion of SRM
part]; an independent dataset comprising naturalistic stimulation and (a variety
of) localizers.
%
``Future studies that aim to use a movie to localize visual areas in individual
participants should extensively annotate the content of frames (e.g., using the
open-source solution ``Pliers''\citep{mcnamara2017developing} for feature
extraction from a visual naturalistic stimulus)''
\citep{haeusler2022processing}.


\subsubsection{Future studies: inter-individual differences / variability}
%
Compared to the visual \ac{ppa}, we observe more variability of the auditory
\ac{ppa} across subjects.
%
Possible reasons are (shitty) modelling per se, individual alertness/fatigue,
attentional focus, predisposition to select/process spatial relevant
information.
%
``Divergent patterns of brain organization from the most common pattern (that
is, changes in the spatial arrangement of cortical regions) can be observed in
approximately 5–10\% of the healthy population [Glasser, 2016, A multi-modal
parcellation; Gordon, 2017, Individual variability], and care should
therefore be taken to avoid the undue influence of such outliers''
\citep{eickhoff2018imaging}.
%
``Topological outliers, if they do not result from artefacts, can also be
considered to be interesting cases of inter-individual variability to understand
brain-phenotype relationships [98]'' \citep{eickhoff2018imaging}.
%
``Recent studies have suggested that the topography (location and size)
of individual-specific brain parcellations is predictive of individual
differences in demographics, cognition, emotion and personality [3,5,99]''
\citep{eickhoff2018imaging}.
%
``Only by understanding the generic characteristic of topographic organization
can we start to appreciate idiosyncrasies and their relationships to
socio-demographic, cognitive or affective profiles''
\citep{eickhoff2018imaging}.

%
``Interindividual variability of functional regions in both functional and
spatial features should reflect brain function to some extent, and thus is a
rich and important source of information for revealing the neural basis of human
cognition and behavior [Kanai and Rees, 2011; Zilles and Amunts, 2013]''
\citep{zhen2015quantifying}.
%
Future studies need to elaborate on the significance of the variability of
functional regions and utilize this variability to explore the neural mechanisms
of specific behavioral or cognitive processes [Van Horn et al., 2008]''
\citep{zhen2015quantifying}.
%
Our study ``invites future work on the origin of the variability and its
relation to individual differences in behavioral performance''
\citep{zhen2015quantifying}.

% PPA
``The spatial and selectivity variability of the SSRs may likely underlie
individual differences in navigation, especially when scene perception is being
utilized'' \citep{zhen2017quantifying}.
%
For example, processing oral way description and using it for planing,
remembering, executing navigation.
%
``Future studies will be needed to elaborate on the mechanism and functional
significance of the interindividual variability of the SSRs and to utilize the
variability in understanding the neural mechanisms of scene perception [Kanai
and Rees, 2011; Van Horn et al., 2008]'' \citep{zhen2017quantifying}.
%
Hence, further studies using dedicated auditory localizer should shed light on
the inter-individual differences of persons exhibiting an auditory \ac{ppa}.




\subsection{SRM study}

% align left-out subject
Based on our findings in study 2 \citep{haeusler2022processing}, we assumed that
the event struture in both naturalistic stimuli would correlate, among others,
with brain responses that are similar to those correlating with the event
structure in a dedicated functional localizer.



\subsection{Naturalistic stimuli as functional localizer?}


\subsubsection{Caveats of naturalistic stimuli}

%
Challenging data data analysis, but create \& share annotation.
%
Hence, data analysis pipelines should be implemented in common and
well-documented packages and code, and custom code shared along the paper.

%
Just an approximation of real life.
%
Setting is still the scanner.
%
Passive watching \& listening. Hence, executive functions?



\subsubsection{Pro = summary of the whole project/thesis}
%
Summary of the whole thesis.

%
Why naturalistic stimuli are so fucking awesome.


\section{Conclusion (on naturalistic stimuli)}
%
In summary, naturalistic stimuli ``impose a meaningful timecourse across
subjects while still allowing for individual variation in brain activity and
behavioral responses, and lend themselves to a broader set of analyses than
either pure rest or pure event-related task designs'' \citep{finn2017can}.
%
``Naturalistic paradigms do not aim to replace the classic, controlled
neuroimaging paradigms (Sonkusare et al., 2019). Due to their complexity and
current limitations in understanding the statistical properties of different
features in naturalistic conditions, naturalistic stimuli are not optimal for
model development [see, e.g., Rust and Movshon, 2005]. Controlled experiments
are still needed for hypothesis testing and developing models, while
naturalistic stimuli are best employed to test models in ecologically valid
settings and to expand them to situations where context matters
more'' \citep{saarimaki2021naturalistic}.
