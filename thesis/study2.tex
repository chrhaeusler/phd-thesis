
\section{Recapitulate aims here}

From TeaP talk:
%
What is still kind of unclear is if these results generalize from static
pictures to results from a more ecologically valid stimuli like a movie.
%
What is still unclear is if these results generalize to the auditory domain.  To
frame it as a question: Does increased hemodynamic activity in the PPA correlate
with auditory spatial information?
%
The goals of your study were the following: We asked ourselves:
%
1) How can we operationalize the perception of spatial information that is
embedded in our naturalistic stimuli?
%
2) Further, we wanted to compare our results to a classic visual localizer
experiment that used blocks of pictures.
%
And lastly, we wanted to explore if an auditory narrative could substitute a
visual experiment to do individual diagnostics
%
Which means: can we localize the PPA in individual persons using a more engaging
\& entertaining, but also exclusively auditory stimulus
%
To operationalize the perception of spatial information, we looked at time
points in the stimuli that should correlate with the perception of spatial
information.
%
And we looked for time points correlating with other perceptual processes that
we needed to build general linear model contrasts.
%
For the movie, we used the movie cuts that basically re-orient the observer in
space.
%
In the context of naturalistic stimuli - not just data-driven but also
model-driven analyses can be applied to data from naturalistic stimuli.
%
- but: to model event-related hemodynamic activity you need to know the temporal
structure of the stimulus
%
- and because naturalistic stimuli are so versatile and trigger so many
perceptual and cognitive processes you can re-use already existing data to
answer new research questions.

\section{How is this ground work for Study 4}
%
results suggest that you can ``localize a auditory PPA'' but not a the ``visual
PPA'' using an auditory narrative; It's similar but different; SRM study will
quantify the prediction performance
