This part of the dissertation has been published:

\bigbreak

\noindent
%
Häusler, C. O.,
%
\& Hanke, M.
%
(2021).
%
A studyforrest extension, an annotation of spoken language in the German dubbed
movie ``Forrest Gump'' and its audio-description.
%
F1000Research, 10(54).
%
doi: \href{https://doi.org/10.12688/f1000research.27621.1}
{\url{10.12688/f1000research.27621.1}}.

\begin{chapterabstract}
%
% up to 300 words intro
Here we present an annotation of speech in the audio-visual movie
``Forrest Gump'' and its audio-description for a visually impaired audience,
as an addition to a large public functional brain imaging dataset
(\href{www.studyforrest.org}{\url{studyforrest.org}}).
% anno content 1
The annotation provides information about the exact timing of each of the more
than 2500 spoken sentences, 16,000 words (including 202 non-speech
vocalizations), 66,000 phonemes, and their corresponding speaker.
% anno content 2
Additionally, for every word, we provide lemmatization, a simple
part-of-speech-tagging (15 grammatical categories), a detailed part-of-speech
tagging (43 grammatical categories), syntactic dependencies, and a semantic
analysis based on word embedding which represents each word in a
300-dimensional semantic space.
% validation
To validate the dataset's quality, we build a model of hemodynamic brain
activity based on information drawn from the annotation.
% why bother
Results suggest that the annotation's content and quality enable independent
researchers to create models of brain activity correlating with a
variety of linguistic aspects under conditions of near-real-life complexity.
%
\end{chapterabstract}
