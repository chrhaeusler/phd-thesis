%% LaTex template for a monographic dissertation
%% at the Faculty of Medicine at Heinrich Heine University Düsseldorf

%% author: Christian Olaf Häusler
%% date: January 2022
%% license:
%% Creative Commons Attribution 4.0 International Public License
%% https://creativecommons.org/licenses/by/4.0/

%% This template was created to meet the requirements as provided by:
%% ENG-PhD_Ordnung_vom_05.03.2018.pdf
%% DE-PhD_Ordnung_vom_05.03.2018.pdf
%% 2020-06-19_GZ-Guidelines_for_preparation_of_Dissertations_-_Monograph.pdf
%% 2021-04-01_GZ-Leitfaden_zur_Erstellung_von_klassischen_Dissertationen.pdf


%% requirements that did not fit anywhere else:


%% Copyright information: For all illustrations and tables that you have not
%% created yourself, but have taken over or modified from other sources, a
%% source reference, analogous to the citation of text sources, must be given
%% in the caption. The source reference must be included in the reference
%% list/bibliography.

%% Please note, especially in view of the fact that your dissertation will be
%% published online, that the reproduction of illustrations and tables is subject
%% to copyright. [...] you should always ask the responsible publisher for
%% permission. This also applies to illustrations from publications in which
%% you are listed as the author, as the reproduction rights are usually
%% transferred to the publisher. However, most publishers allow the re-use of
%% parts of your own publications. In the case of journals, this is usually
%% stipulated in a contract. Therefore, please refer to the websites of the
%% respective publishers for information on author rights and licenses. The
%% granting of a licence by a publisher may also be subject to fees.

% font size 11-12pt
\documentclass[english,12pt]{report}
\usepackage[utf8]{inputenc}

% use ä, ü, ...
\usepackage[T1]{fontenc}

% language settings
\usepackage[USenglish]{babel}

% use a common font like Times New Roman or Arial:
% keep LaTex' standard ("Computer Modern")

% margins of the pages;
% left: 30-35mm (due to binding)
% right, top, bottom: 20-25mm
\usepackage[
    a4paper,
    left=30mm,
    right=20mm,
    top=20mm,
    bottom=20mm
]
{geometry}


% manage header, footers, and page numbers
\usepackage{fancyhdr}

% add to dos as comments
\usepackage{todonotes}

% manage spacing between lines
\usepackage{setspace}

% format the chapters' header
\usepackage{titlesec}
\titleformat{\chapter}[hang]{\bf\huge}{\thechapter}{2pc}{}
% other way without showing the number before the chapter title:
%\titleformat{\chapter}[display]{\normalfont\bfseries}{}{0pt}{\Huge}


% citation
% numerical citations:
%\usepackage[numbers]{natbib}
% superscript citations:
%\usepackage[super]{natbib}
% author-year citations:
% \usepackage[round]{natbib}
% APA-style citing and bibiliography
\usepackage[natbibapa]{apacite}

% make the bibliography appear in the table of contents
\usepackage[nottoc,numbib]{tocbibind}

% package for nice tables
\usepackage{booktabs}

% units: provided as a bundle with the nicefrac package for typing fractions.
% Units uses nicefrac in typesetting physical units in a standard-looking way.
\usepackage{units}

% for signs ligh less/equl, greater/equal etc.
\usepackage{amssymb}

% how to handle links in the PDF
\usepackage[
    colorlinks=true,
    citecolor=black,
    urlcolor=black,
    linkcolor=black
]{hyperref}

\usepackage{verbatim}

% manage list of abbreviations
% acronyms/abbreviations are defined in the corresponding chapter (s. below)
% if a given acronym is given by \ac{my-acronym} for the first time in the text
% body its full name will be written and it will show up in the list of
% abbreviation
% \acs{} -> force short version
% \acl{} -> force long version
% \acp{} -> make it plural (adds an "s")
\usepackage[printonlyused]{acronym}

% make sure that acronyms get hyphenated
\usepackage{etoolbox}

\makeatletter
\patchcmd\@acf{\AC@acl}{\AC@foo}{}{}
\patchcmd\@acf{\AC@acl}{\AC@foo}{}{}
\patchcmd\@acf{\AC@foo}{\hskip\z@\AC@acl}{}{}
\patchcmd\@acf{\AC@foo}{\hskip\z@\AC@acl}{}{}
\makeatother


\begin{document}
% set pagestyle to 'no header' and 'page number at bottom in the middle'
\pagestyle{plain}
% don't show page number (for now)
\pagenumbering{gobble}
% set spacing between lines to 1.5
\begin{spacing}{1.5}




% create the title page
\begin{titlepage}
\begin{center}

From the Institute Institute of Neuroscience and Medicine,\\
Brain \& Behaviour (INM-7),\\
at Research Centre Jülich, Jülich, Germany

\vfill

\textbf{{\large Exploring naturalistic stimulus paradigms as an alternative to a
    task-based functional localizer paradigm}}

maybe add subtitle

\vfill

{\large Dissertation}

\vfill

to obtain the academic title of Doctor of Philosophy (PhD) in Medical Sciences\\
from the Faculty of Medicine at Heinrich Heine University Düsseldorf

\vfill

submitted by\\
\textbf{Christian~Olaf~Häusler}\\
(2022)

\end{center}
\end{titlepage}




%\chapter*{Examiner Details}
\newpage
%pagebreak[4]

\noindent This page 2 will not be included in the examination copies, but will
only be inserted in the copies for publication after the review process and
the oral examination. You will be notified of the exact text for page 2 once
print approval is granted.

\vfill
\noindent As an inaugural dissertation printed by permission of the\\
Faculty of Medicine at Heinrich Heine University Düsseldorf

\vspace*{\fill}
\noindent signed:\\
dean:\\
examiner: Name A\\
co-examiner: Name B



%\chapter*{Dedication}
\newpage
%pagebreak[4]

\begin{center}
\null\vspace{\stretch{1}}
    \textit{``Gladius ultor noster! Pectus amico, cuspis hosti!''}\\
    \hspace{0.3\textwidth} --- old studential motto
\vspace{\stretch{2}}\null
\end{center}




%\chapter*{List of Publications}
\newpage
%pagebreak[4]

%% Own publications are listed here.
%% Submitted manuscripts or manuscripts under revision are not listed here.
%% These will be listed in the “Application for authorization to the doctorate”
%% (Zulassungsantrag).
%% Please note that you must include a reference for all information you take
%% from your own publications.
%% The publications mentioned on page 4 must therefore also be listed in the
%% bibliography.
%% Direct quotations must be marked with quotation marks.

\todo[inline]{Legenden von Tabellen und Abbildungen sind kleinere Schriftgröße
(10pt-11pt)}

\todo[inline]{Abbildungen: im Text, nicht im Anhang; im Text erwähnt (Abb.1) und
erläutert; mit Titel ("Fig. x: \textbf{Titel.} weitere Informationen") und
Bildunterschrift; Bildunterschrift bzw. Legende: alle verwendeten Maßeinheiten
und Abkürzungen aufgeführt und erläutert}

\todo[inline]{Tabellen: Überschrift ("Table 1: \textbf{Überschrift.} Information,
die zu deren Verständnis erforderlich ist."); Legende ist wie bei Abbildungen
unterhalb der Tabelle}

\vspace*{\fill}

\noindent \textbf{Parts of this work have been published:}\\


\todo[inline]{adjust whole diss to required citation style!}

\todo[inline]{copy from reference into here; and corresponding chapters}

% just testing the bibliography
\noindent Christian Olaf Häusler and Michael Hanke. A studyforrest extension, an
annotation of spoken language in the German dubbed movie “Forrest Gump” and its
audio-description. F1000Research, 10(54):54, 2021.\\

\noindent Yaroslav O. Halchenko, Kyle Meyer, Benjamin Poldrack, Debanjum Singh
Solanky, Adina S. Wagner, Jason Gors, Dave MacFarlane, Dorian Pustina, Vanessa
Sochat, Satrajit S. Ghosh, Christian Mönch, Christopher J. Markiewicz, Laura
Waite, Ilya Shlyakhter, Alejandro de la Vega, Soichi Hayashi, Christian Olaf
Häusler, Jean-Baptiste Poline, Tobias Kadelka, Kusti Skytén, Dorota Jarecka,
David Kennedy, Ted Strauss, Matt Cieslak, Peter Vavra, Horea-Ioan Ioanas, Robin
Schneider, Mika Pflüger, James V. Haxby, Simon B. Eickhoff, and Michael Hanke.
DataLad: distributed system for joint management of code, data, and their
relationship. Journal of Open Source Software, 6(63):3262, 2021. doi:
10.21105/joss.03262.\\

\noindent Christian Olaf Häusler, Simon B. Eickhoff and Michael Hanke (2022).
Processing of visual and non-visual naturalistic spatial information in the
``parahippocampal place area'', Sci. Data, 9(1), 1--15.




\chapter*{Zusammenfassung}
% show page number, set it to Roman number
\pagenumbering{Roman}
% set page counter to 1
\setcounter{page}{1}

%% The German and English summary should be exactly one page long each.
%% The summary represents a coherent text. Please do not use subheadings.
%% should include the following content in a shortened form:
%% - scientific background and current state of research
%% - research question and objectives
%% - methodology
%% - results
%% - discussion and conclusions.

%% The summary is needed twice:
%% a) Bound into your dissertation.
%% b) As a pdf-document, submitted together with the application for
%%    admission to doctoral examination proceedings.

% take input from external file
% wissenschaftlicher Hintergrund
\textit{Functional localizers} sind fMRT-Experimente, die die funktionelle
Neuroanatomie von Individuen chrakterisieren sollen.
% aktueller Forschungsstand
Allerdings verwenden diese Paradigmen selektiv ausgewählte, experimentell
streng kontrollierte Reize, stützen sich auf die Folgebereitschaft des
jeweiligen Individuums, und üblicherweise nur eine Domäne von Gehirnfunktionen
abbilden.
%
Im Gegensatz dazu bieten naturalistische Reize wie Filme oder auditive
Erzählungen ein vereinnehmendes, aufgabenfreies Paradigma, das der Komplexität
und Vielfalt realer Erfahrungen näher kommt und eine viele Gehirnfunktionen
abdecken könnte.
% Fragestellung und Ziele
Der Schwerpunkt dieser Dissertation richtet sich auf das "Parahippocampal
Place" (PPA), ein funktionalles Areal höherer visueller Wahrnemung, das erhöhte
hämodynamische Aktivität aufweist, wenn Studienteilnehmer Bilder von
Landschaften oder Wahrzeichen betrachten, im Gegensatz zu anderen Reizen wie
Gesichtern oder Werkzeugen.
%
Unter Berücksichtigung der Prinzipien offener, transparenter und
reproduzierbarer Wissenschaft untersucht die Arbeit mit zwei methodischen
Ansätzen, ob ein Film und eine auditorische Erzählung einen visuellen Localizer
ersetzen könnten.

% Methodik 1
Als erster Ansatz führten wir eine modellgesteuerte Analyse der hämodynamischen
Aktivität während des Films "Forrest Gump" und seiner Audiodeskription durch,
die die Hauptstimuli des öffentlich zugänglichen "StudyForrest"-Datensatzes
(\href{www.studyforrest.org}{\url{studyforrest.org}}) sind.
%
Zunächste wurde eine umfassende Annotation der im Film und in der
Audiodeskription vorkommenden gesprochenen Sprache erstellt, um die Grundlage
für die Modellierung hämodynamischer Reaktionen zu schaffen und das Projekt
"StudyForrest" als offene Wissenschaftsressource zu erweitern.
%
Anschließend führten wir eine massenunivariate Analyse mit dem allgemeinen
linearen Modell (GLM) durch, um die PPA in Personen zu lokalisieren, die zuvor
bereits an einem Localizer-Experiment teilgenommen hatten.
% Ergebnisse 1
Die Ergebnisse legen nahe, dass eine modellgetriebene auf der Grundlage von
Annotationen eines Films oder eines ausschließlich auditorischen
naturalistischen Stimulus verwendet werden kann, um eine visuelles Areal in
Individuen lokalisieren zu können.


% Methodik 2
Als zweiten Ansatz untersuchte die Arbeit ein neuartiges, datengetriebenes Verfahren für die
funktionelle Lokalisierung, das es ermöglicht mittels \textit{functional
alignments}, die Lage des PPA in einem Individuum zu schätzen, indem es sich
Daten von anderen Individuum zu Nutze macht.
%
Unter Verwendung des \textit{shared response models} (SRM) haben wir einen
\textbf{common functional space} (CFS) und individuelle Transformationen
erstellt, um funktionalle Daten andere Individuen durch den CFS in den
Gehirnraum des zu untersuchenden Individuums zu projizieren.
%
Darüber hinaus untersuchten wir die Beziehung zwischen der Menge funktioneller
benutzt wurde, und der sich daraus ergebenden Schätzleistung.
% Ergebnisse 2
Die Ergebnisse legen nahe, dass eine auditorische Erzählung grundsätzlich dazu
verwendet werden kann, um die neuroanatomischen Position eines visuelles Areal
wie der PPA zu schätzen.
%
Darüber hinaus können Daten eines 15-minütigen Scans, während dem ein
Individuum einen Film schaut, hinreichend sein, um Gehirnmuster genauer zu
schätzen als ein Verfahren, das auf anatomischen Alignment beruht.

% Diskussion
Die Arbeit zeigt jedoch auch Hindernisse in der Entwicklung eines
multifunktionalen naturalistischen Localizers auf.
%
Daten naturalistischer Stimuli stellen eine Herausforderung für
modellgetriebene Analysen insofern da, weil sie physiologische und statistische
Modelannahmen strapazieren.
%
Außerdem zielen traditionelle Localizer darauf ab, die interindividuelle
Variabilität zu minimieren und funktionelle Areale in allen gesunden Personen
reliabel zu lokalisieren, hingegen naturalistische Stimuli jedoch höhere
Variabilität zulassen.

% Schlussfolgerungen
Daher hängt das Potenzial eines naturalistischen Stimulus, einen oder mehrere
traditionelle Localizer zu ersetzen, von weiteren Entwicklungen ab, die die
statistischen und methodologischen Herausforderungen angehen.
%
Dennoch könnte ein datengetriebener Ansatz auf Grundlage eines functional Alignments,
 der sich Daten eines natürlichen
basierend auf funktioneller Ausrichtung unter
Verwendung eines naturalistischen Stimulus ähnlicher Dauer wie bei einem
traditionellen Lokalisator potenziell die Ergebnisse vieler funktioneller
Lokalisatoren abschätzen.

Jedoch zeigt die Arbeit, dass ein datengetriebener Ansatz, der auf functional
alignment unter Verwendung eines Stimulus von ähnlicher Dauer wie der eines
traditionellen localizers beruht, das Potential hat, die Ergebnisse
funktioneller Localizer zu schätzen.





\chapter*{Summary}
% take input from external file
% scientific background
Functional localizers are fMRI experiment that aim to characterize the
functional neuroanatomy on the level of individuals.
%
However, these paradigms employ selectively sampled, tightly controlled
stimuli, rely on an individual's compliance, and can typically map only one
domain of brain functions.
% current state of research
In contrast, naturalistic stimuli, such as movies or auditory narratives,
provide an engaging, task-free paradigm that more closely resembles the
complexity and richness of real-life experiences, sampling a wide range of
brain functions.
% research question and objectives
This dissertation focuses on the parahippocampal place (PPA), a higher-visual
area, that exhibits increased hemodynamic activity when participants view
images of landscapes or landmarks, as opposed to other stimuli like faces or
tools.
% stimuli
Following the principles of open, transparent, and reproducible science, the
thesis explores whether a movie and an auditory narrative could replace a
visual localizer in two ways.

% methodology speech-anno: haeusler2021speechanno; ppa-paper:
% haeusler2022processing
As the first approach, we performed a model-driven analysis of hemodynamic
activity during the movie ``Forrest Gump'' and its audio-description, which are
the core stimuli of the publicly accessible studyforrest dataset
(\href{www.studyforrest.org}{\url{studyforrest.org}}).
%
An exhaustive annotation of speech occurring in the movie and audio-description
was created to establish the foundation for modeling hemodynamic responses and
to extend the studyforrest project as an open science resource for
investigating human brain functions under quasi-natural conditions.
% but technically similar to traditional
Subsequently, we performed a mass-univariate general linear model (GLM)
analysis to localize the PPA, which had previously been identified in the same
group of participants using a visual localizer.
% results
Results suggest that a model-driven analysis based on annotations of a movie or
an exclusively auditory naturalistic stimulus can be used to localize a
functional area on an individual level.

% SRM
As the second approach, we explored a novel functional alignment procedure that
allows to estimate the location of the PPA in an individual by leveraging data
collected from of a reference group.
%
Using a shared response model (SRM), we created a common functional space (CFS)
and subject-specific transformations to project functional data from the
reference through the CFS into an individual's brain space.
%
Additionally, we investigated the relationship between the quantity of data used
for functional alignment and the subsequent estimation performance.
%
Results suggest that an auditory narrative can in principle be used to estimate
a visual area such as the PPA.
%
Moreover, 15 to 30 minutes of functional scanning during movie watching can
generate a sufficient amount of data to estimate brain patterns
more accurately than a procedure based on anatomical alignment.

% discussion
The thesis also highlights obstacles in the pursuit of developing a
multi-functional naturalistic localizer.
%
Applying a model-driven analysis to naturalistic stimuli is challenging, as
these stimuli stress physiological and statistical assumptions.
%
Moreover, traditional localizers aim to minimize inter-subject variability and
reliably localize functional areas in all healthy individuals, whereas
naturalistic stimuli allow for more variability.
% conclusions
Therefore, the potential of a naturalistic stimulus to replace one or multiple
traditional localizers relies on further developments that address the
methodological and statistical challenges.
%
Nevertheless, a data-driven approach based on functional alignment using a
naturalistic stimulus of similar duration to that of one traditional localizer
could potentially estimate the results of many functional localizers.





\chapter*{List of Abbreviations}
%% The list of abbreviations contains and explains all abbreviations used in
%% the thesis (except for common linguistic abbreviations as defined in the
%% dictionary, e.g. Duden, Merriam-Webster).
%% In the case of physical or chemical quantities, it is also necessary to
%% specify the unit.
%% An alphabetical order is useful, a subdivision (e.g. SI units, own
%% abbreviations) is possible.
%% For example, you can format the page in two columns, the abbreviations in
%% bold and the corresponding explanations in non-bold.}



% decrease line spacing
\renewcommand{\baselinestretch}{0.75}\normalsize

% definition table of acronyms
% see documentation on how if your acronym needs special formatting, and
% the unit needs to be added to the list of abbreviations
% set space between acronym and full definition to [longest]
\begin{acronym}[longest]
    \acro{aa}[AA]{anatomical alignment}
    \acro{bold}[BOLD]{blood oxygen level-dependent}
    \acro{cms}[CMS]{common model space}
    \acro{eba}[EBA]{extrastriate body area}
    \acro{eeg}[EEG]{electroencephalography}
    \acro{fa}[FA]{functional alignment}
    \acro{ffa}[FFA]{fusiform face area}
    \acro{fov}[FoV]{field of view}
    \acro{glm}[GLM]{general linear model}
    \acro{fmri}[fMRI]{functional magnetic resonance imaging}
    \acro{loocv}[LOOCV]{leave-one-out cross-validation}
    \acro{loc}[LOC]{lateral occipital complex}
    \acro{meg}[MEG]{magnetoencephalography}
    \acro{ofa}[OFA]{occipital face area}
    \acro{pca}[PCA]{principal component analysis}
    \acro{ppa}[PPA]{parahippocampal place area}
    \acro{roi}[ROI]{region of interest}
    \acrodefplural{roi}[ROIs]{regions of interest}
    \acro{tr}[TR]{time of repetition}
    \acro{srm}[SRM]{shared response model}
    \acro{snr}[SNR]{signal-to-noise ratio}
\end{acronym}

% set line spacing back to normal
\renewcommand{\baselinestretch}{1}\normalsize




\tableofcontents
%% The entire table of contents should not be longer than 2 pages.
%% A chapter with only one subchapter is not allowed

% imo it makes more sense in following order:
% ToC, list of abbreviations, list of tables, list of figures, actual text



%\listoftables
%% List of tables is not mandatory
%% if not "not allowed"




%\listoffigures
%% List of figures is not mandatory
%% if not "not allowed"




\chapter{Introduction}
% set format of page numbers to arabic numbers
\pagenumbering{arabic}
% set page counter to 1
\setcounter{page}{1}

% take input from external file
%% The introduction begins with an overview of the topic

%
``A remarkable feature of the vertebrate brain is the anatomical specialization
of cortical regions for the processing of different types of information. Since
the late 19th century, it has been recognized that restricted lesions of the
human brain result in location-specific sensory, motor or cognitive deficits''
\citep{cohen1994localization}.


%
Thanks to technological advanced, brain imaging studies have extensively used
\ac{fmri} since the early 1990s to measure \ac{bold} activity in-vitro.
%
\textit{Human brain mapping} (e.g., \citep{raichle2009brief}) explores the
brain's topographic organization (e.g., \citep{eickhoff2018topographic}) and
attempts ``to specify in as much detail as possible the localisation of function
in the human brain'' \citep{savoy2001history}.
% higher visual areas
For example in the domain of higher-visual perception, replicated findings
suggest that category-selective brain regions like the \ac{ppa}
\citep{epstein1998ppa}, \ac{ffa} \citep{kanwisher1997ffa}), or the \ac{eba}
\citep{downing2001bodyarea}) exhibit significantly increased \ac{bold} activity
correlated with a ``preferred'' stimulus class.

\todo[inline]{practical and statistical reason are kind of the same; population
inference?}
%
Typical analyses of human brain mapping studies average data across subjects for
practical (e.g., limited scan time per subject), statistical (e.g., improved
of \ac{snr}) reasons, or to generalize from study subjects to a broader
population.
%
Nevertheless, the precise anatomical location of functionally defined regions of
the brain vary anatomically (measured in Talairach or MNI coordinates) across
individuals \citep{friston2006critique, saxe2006divide}.
%
Consequently, studies employing an averaging approach may just ``capture the
common denominator of each individual cognitive circuit and lose a large amount
of information'' \citep{pinel2007fast}.
%
However, ``interpretation of fMRI data at the level of individual brains is
essential for characterizing brain function in health and disease''
\citep{dubois2016building}.


\section{Functional localization}
%% Further introductory remarks describe the scientific background of the work
%% as precisely as possible. Cite the most important publications and avoid
%% extensive literature reviews.
% a.k.a. "state of research"

\todo[inline]{check: https://github.com/Parietal-INRIA/DiFuMo}

% individual level: localizers
On the level of individual subjects, \textit{functional localizer} experiments
(s. \citep{saxe2006divide, friston2006critique} for reviews) are conducted
during \ac{fmri} to characterize the location size, and shape of functional
areas correlating with perceptual or cognitive processes (e.g. perception of
object categories \citep{kanwisher1997ffa}, speech perception
\citep{fernandez2001language}, or theory of mind \citep{spunt2014validating}).
% purpose: ROI improve the statistical power of the main experiment's analysis
Functional localizers are often used as separate experiment to identify a
subject-specific functional \acp{roi} to ``guide, constrain or interpret results
from a main experiment \citep{saxe2006divide}.
% clinical application
They also promise to advance neuroimaging towards a clinical application (e.g.,
diagnosis prior to neurosurgery) because ``in the clinical setting making
diagnoses for single cases is imperative'' \citep{wegrzyn2018thought}.
% purpose: neurosurgery
For example, surgical procedures might impact the post-operative quality of life
so much (e.g. concerning cognitive control or speech production) that a surgical
intervention outweighs the therapeutic benefits.

% one localizer = one domain of functions
However, one localizer paradigm can usually map just one domain of brain
functions.
% detection power based on paradigms' design
Designed to maximize detection power, localizers employ carefully chosen,
tightly-controlled, and simplified stimuli presented in a block-wise manner
often accompanied with a task to keep study participants attentive.
% inefficiency -> localizer batteries
Since this approach becomes inefficient if one wants to map many different
processes with limited resources (time, availability and applicability of
diagnostic measures for an individual patient), researchers have developed more
time-efficient, multi-functional \textit{localizer batteries}
\citep{barch2013function, drobyshevsky2006rapid, pinho2018individual,
pinho2020individual, pinel2007fast}.
% localizer batteries: example
For example, \citet{pinel2007fast} employs a range of dedicated stimuli and
specific tasks in a 5-minute routine to map processes of ``auditory and visual
perception, motor actions, reading, language comprehension, and mental
calculation at an individual level'' \citep{pinel2007fast}.

% intro to contra localizer
But still, current localizer paradigms struggle to overcome especially two
challenges.
%
First, the localizers depend heavily on a participant's comprehension of the
task instruction and his/her compliance, a criterion that can be difficult to
meet in clinical or pediatric populations \citep{eickhoff2020towards,
vanderwal2015inscapes, vanderwal2019movies}.
% validity?
Second, the paradigms rely on selectively sampled, tightly-controlled stimuli
presented in blocks, and do not resemble how we perceive the real world outside
of the laboratory during everyday life.


\paragraph{excursion to current project}

\todo[inline]{better merge following excursion with ``Aims of thesis'' below??
(tough, a short excursion to current project makes sense before the naturalistic
stimulus inferno breaks loose)}

%
For this reason, this dissertation will explore whether a Hollywood movie and
its audio-only variant created for a visually impaired audience could, in
principle, substitute a traditional, task-based paradigm to localize functional
areas.
%
As a proof of concept, the dissertation focusses on the ``parahippocampal place
area'', a classic example of a higher-visual, functional area
\citep{epstein1998ppa, epstein1999parahippocampal} located in the ventral visual
pathway [check PPA paper for reference].
% what PPA is doing
Increased hemodynamic activity is observed in the \ac{ppa} when participants
view photos of landscapes, buildings or landmarks, compared to, e.g., photos of
faces or tools \citep[see reviews][]{epstein2014neural, aminoff2013role}.
%
These stimulus classes that are correlated with hemodynamic responses during
traditional paradigms might also be embedded as \textit{stimulus features} in
more naturalistic stimuli like a movie or spoken narrative.


\section{Naturalistic stimuli}
%
Since a major goal of cognitive neuroscience is not to reveal how the brain
responds to blocks of [tightly-controlled] stimuli presented in a laboratory
setting but how the brain processes information during everyday perception,
\textit{naturalistic stimuli} gained popularity in neuroscience.


\paragraph{Definition}

% definition quote
Naturalistic stimuli are ``a class of stimuli that aim to evoke more
naturalistic patterns of neural responses than traditional controlled artificial
stimuli. Naturalistic paradigms are typically complex and dynamic, and longer in
duration than many conventional stimuli.'' \citep{vanderwal2019movies}.
%
Therefore, naturalistic stimuli [promise to] ``sample a broad range of brain
states and engage multiple perceptual and cognitive systems in parallel''
\citep{haxby2020naturalistic}.
% movies & narratives
The most popular naturalistic stimuli in neuroscience are movies and auditory
narratives (s. \citep{jaaskelainen2021movies, jaaskelainen2020neural} for
reviews) that provide a time-locked event structure during a continuous, rich
and dynamic stimulation.
%
Consequently, naturalistic stimuli promise a higher ecological validity
\citep{zaki2009need, hasson2012future, hamilton2018revolution} because they more
closely mimic our rich visual and auditory experiences outside the scanner bore
in real-life \citep{hasson2008neurocinematics, haxby2020naturalistic}.
%
Naturalistic stimuli also promise a higher external validity because the
\textit{stimulus features} (i.e. the stimulus classes or variables in the
naturalistic stimulus) that are embedded in the naturalistic stimulus represent
a more random sample from the ``theoretical population of stimuli that might
have been used'' \citep{westfall2016fixing}.
%
% Carefully chosen stimulus sets ``selectively sample from the stimulus
% population leading to a stimulus-as-fixed-effect fallacy [Clarc, The
% language-as-fixed-effect fallacy: A critique of language statistics in
% psychological research]'' \citep{westfall2016fixing}. ``The conclusions cannot
% be generalized to a broader population of stimuli without risking inflated
% Type I error  [cf. Donnet S, Lavielle M, Poline JB: Are fMRI event-related
% response constant in time? A model selection answer]''
% \citep{westfall2016fixing}.


\paragraph{Early findings}
% reviews
Audio-visual movies and spoken narratives have been used during \ac{fmri}
(s.\citep{hamilton2018revolution, hasson2008neurocinematics,
sonkusare2019naturalistic, saarimaki2021naturalistic} for reviews), and \ac{eeg}
or \ac{meg} data acquisition (s. \citep{alday2019meg, kandylaki2019story} for
reviews).
%
Early studies have shown that watching a movie \citep{hasson2004intersubject,
hasson2008neurocinematics, hasson2010reliability} or listening to a narrative
\citep{lerner2011topographic, wilson2008beyond} reliably synchronize
spatiotemporal responses across multiple subjects in a large part of the brain
compared to, for example, an unedited video of a concert taken from a fixed
viewpoint \citep{hasson2004intersubject, hasson2008neurocinematics,
hasson2010reliability, lerner2011topographic, wilson2008beyond}.
% This finding could be attributed to a film director's goal to not only direct
% a movie, but also to capture and direct the audience's attention; this can be
% attributed to the way professional movies are shot and edited in order to
% intentionally manipulate the viewers' attentional focus and mental states
% \citep{brown2012cinematography, dancyger2011film-technique}.
%
Further, a pioneering study \citep{bartels2004mapping} suggests that functional
specialization of cortical areas is preserved during complex, life-like
stimulation.


\paragraph{Better compliance}
%
From a practical perspective, naturalistic stimuli promise improved subject
compliance regarding wakefulness and head motion due to minimal instruction
requirements (e.g. no fixation of eye gaze), task demands (no task except
enjoying the movie or auditory story).
%
This is especially the case for young children \citep{vanderwal2015inscapes},
and possibly psychiatric \citep{eickhoff2020towards} or elderly persons
resulting in increased data quality.
%
Since movies and spoke narratives are interesting and easy-to-follow stimuli
that are produced to be engaging and immersive, they also promise to put
participants at ease in the otherwise claustrophobic, uncomfortable, and noisy
fMRI scanner.
%
Lastly, spoken narratives are also appropriate for visually impaired persons
suffering from diminished or lack of eyesight.


\paragraph{contra: challenging data analysis; annotations needed}

%
However, the majority of naturalistic stimuli that have been used in
neuroscience have originally been designed for commercial purposes and not to
conduct research.
%
The temporal structure of stimulus features embedded in a naturalistic stimulus
are fixed and thus reproducible but initially not explicitly known.
%
Modeling brain activity correlating with the stimulus features embedded in the
time course is challenging \citep{saarimaki2021naturalistic, simony2020analysis}
because such models, like a traditional \ac{glm}, rely on the stimulus features
being annotated.
%
The lack of extensive annotations has led to a ``usage bottleneck''
\citep{aliko2020naturalistic} and might be the main reason why explicit models
of task or stimulus are ``notoriously'' \citep{richard2019fast}, if not
``prohibitively'' \citep{nastase2019measuring} difficult to construct.

%
Additionally if one wants to map only one domain of brain functions, a
full-length movie is an inadequate replacement for a traditional localizer
lasting about 15 minutes.

\todo[inline]{ask Michael: the modeled time course might not ``perfectly''
represent time-course of the correlating brain process}


\section{Predicting subject-specific topography from a reference group}

\todo[inline]{Write text in SRM chapter; put a shorter version in here}

\todo[inline]{necessary terms should be introduced here already}

\todo[inline]{needs to be a preparation for ``Aims of thesis''}

\todo[inline]{check \citep{poldrack2019establishment, yarkoni2017choosing}}

%
For that reason, this dissertation --- following a \ac{loocv} procedure --- also
explores whether just a part of a movie or auditory narrative can serve as a
``diagnostic'' run in order to estimate the location of the \ac{ppa} in an
``unknown'' subject's brain (i.e. a \textit{left-out subject} to test the
prediction performance of our model) from a reference group (i.e. the study
subjects providing the data to train our model).

\todo[inline]{add transition to alignment inferno}


\subsection{Anatomical alignment}


However, ``Anatomical variability and limited structure-function correspondence
across cortex [Paquola et al., 2019, Microstructural and functional gradients;
Vázquez-Rodríguez et al., 2019, Gradients of structure–function tethering...]
make this goal challenging [Rademacher et al., 1993, Topographical Variation of
the Human Primary Cortices; Thirion et al., 2006, Dealing with the shortcomings
of spatial normalization]'' \citep{bazeille2021empirical}.

%
``Even after state-of-the-art anatomical normalization to a standard space, we
still observe differences in individual-level functional activation patterns
that hinder cross-subject comparisons [Langs et al., 2010, Functional geometry
alignment and localization; Sabuncu et al., 2010, Function-based intersubject
alignment]'' \citep{bazeille2021empirical}.

%
Works better with basic perception, less (if at all) with cognition: ``more
sophisticated strategies for registering individual brains to an average
cortical surface provide better alignment across subjects for retinotopic visual
areas [Fischl et al., 1999], but they do not do much better than Talairach
co-ordinates for category-selective regions in the temporal lobe [Spiridon et
al., 2005]'' \citep{saxe2006divide}.

%
``cortical folding-based inter-subject alignment [Fischl et al., 1999,
Inflation, flattening; Yeo et al., 2010, Spherical demons] has been shown to
somewhat reduce functional mismatch [\citep{klein2010evaluation,
frost2012measuring}] [but see \citep{langers2014assessment}]''
\citep{dubois2016building}.

%
``Variability in functional-anatomical correspondence across individuals means
that even high-performing anatomical alignment does not ensure fine-grained
functional alignment [e.g., \citet{frost2012measuring}]''
\citep{kumar2020brainiak}.

\subsection{Functional alignment}


\subsubsection{Hyperalignment}


\subsubsection{Shared response model}


\subsubsection{interim summary}
% interim summary
In summary, naturalistic stimuli ``impose a meaningful timecourse across
subjects while still allowing for individual variation in brain activity and
behavioral responses, and lend themselves to a broader set of analyses than
either pure rest or pure event-related task designs'' \citep{finn2017can}.

\todo[inline]{add short sentence repeating challenging annotations / analyses}


\section{Aims of thesis}

\todo[inline]{shift some parts into excursions (``this dissertation'') above}

%% At the end of the introduction, add a subchapter on the Aims of Thesis in
%% which you describe the research question and the objectives of your work on
%% a maximum of two pages

%
Previous work on a group-average level has shown that it is possible to combine
\ac{bold} \ac{fmri} with naturalistic stimulation in order to localize brain
areas whose increased hemodynamic activity is correlated with psychological
processes \citep{bartels2004mapping}.
%
This dissertation explores --- while following the principles of open,
transparent, and reproducible science --- whether a movie and the movie's
audio-description that was produced for an visually impaired audience could, in
principle, substitute an established localizer paradigm.
%
Naturalistic stimuli would have substantial advantages (e.g. task demands
compliance, and data quality) over simplified stimuli presently used in
block-design localizer paradigms.

\todo[inline]{add focus on PPA here}

%
Moreover, the time-courses of rich, full-length naturalistic stimuli are
correlating with a variety of different brain functions ranging from low-level
perception (e.g., luminance) to high-level cognition (e.g., social cognition).
%
Thus, a naturalistic stimulus could eventually replace multiple dedicated
localizers in the future to provide a more comprehensive diagnostic routine.
%
However, a two-hour paradigm would be inefficient plus unsuitable for a clinical
population.
%
For that reason, this dissertation also explores whether just a part of a movie
or auditory narrative can be used to estimate individual topography of a
left-out subject from a data of reference group.


\subsection{Open, transparent, and reproducible science}

% reproducibility crisis
Over the last decade, there has been a growing awareness that results of
scientific publications are not reproducible or general scientific findings are
not replicable letting some authors speak of a ``reproducibility crisis'' or
``replication crisis'' in the sciences \citep{baker2016reproducibility,
plesser2018reproducibility, stupple2019reproducibility, nosek2022replicability}.
% reproducibility: definition
``A study is reproducible if all of the code and data used to generate the
numbers and figures in the paper are available and exactly produce the published
results'' \citep{leek2017most}.
% replicability: definition
A study is replicable if the same analysis of an equivalent experiment's data
leads to consistent results \citep{dubois2016building, leek2017most}.
%
Hence, on a metalevel, this dissertation aims to meet both the requirements of
open, accessible, shared, and transparent science \citep{watson2015will,
fecher2014open} as well as the requirements of a reproducible and replicable
research project:
%
the dissertation follows guidelines and best practices for a) coding and
scientific computing \citep{wilson2014best}, b) procedures and data analyses
\citep{nichols2017best, poldrack2017scanning, poldrack2019establishment}, and c)
sharing code, created data, and results \citep{eglen2017toward, nichols2017best,
pernet2015improving}.


\paragraph{input data}

% open data \citep{eglen2017toward}
First, my work is build on top of publicly and freely available \ac{fmri} data
that are part of the \textit{studyforrest} project
(\href{www.studyforrest.org}{studyforrest.org}).
%
The studyforrest project is an open science project that aims to provide a
versatile resource for investigating human brain function under quasi-natural
conditions.
%
The core of this dataset are two-hour long \ac{bold} \ac{fmri} scans of
participants watching the movie Forrest Gump \citep{ForrestGumpMovie} and
listening to the movie's audio-description that was created for a visually
impaired audience by adding a narrator to the movie's audio track.
%
Since its first publication in 2014 \citep{hanke2014audiomovie}, the
studyforrest project has served as a resource of raw (and preprocessed) data for
international working groups to conduct and publish independent, peer-reviewed
research (s.
\href{www.studyforrest.org/publications.html}{studyforrest.org/publications.html}).
%
The stimulus annotations that have been created over the course of the
dissertation are version-controlled and published in a standardized file format
\citep{haeusler2021speechanno}, and therefore contribute to the studyforrest
project as a resource for the scientific community.


\paragraph{code, analyses, output}

Further, all code is shared to improve reproducibility current results and to
facilitate replicability of findings on other datasets.
% automatization
Therefore, data analyses pipelines designed in a way that enables automated
processing.
%
Analyses pipelines are not be implemented in commercial, proprietary software
but in freely available and, if possible, open-source software.
% which tools to choose why?
Among potential software packages, we chose the tools that offer the most solid
documentation, and basis of developers and maintainers to ensure long-term
support.
% my code
Custom code written by myself is written in open-source programming languages
(Python and Bash), is version-controlled, documented, and released publicly and
freely accessible.
%
All input data, custom code, analysis steps and output data are accessible in
standardized \textit{DataLad} (\href{www.datalad.org}{datalad.org}) datasets.
Since DataLad provides a free and open-source software solution that manages
provenience, distribution, and version-control of code and data
\citep{halchenko2021datalad}, all executed steps from downloading the input data
to visualizing the results can be rerun to check and validate the dissertation's
results.


\paragraph{publications}
% open-access publishing
Last, because ``nature abhors a paywall'' \citep{dupre2020nature}, publications
describing generated data, reasoning of methodological choices, analysis steps,
and results are published in open-access journals.
% neurovault
Unthresholded statistical maps of all computed statistical $t$-contrasts are
additionally published at Neurovault
(\href{https://neurovault.org/}{neurovault.org}).


\subsection{Specific objectives and hypotheses}

\todo[inline]{This section is waaaaay longer than in other dissertations; here,
it is not just aims and hypotheses but also (still) an overview}

\todo[inline]{Here should be some random text between section titles}


\subsubsection{A studyforrest extension, an annotation of spoken language in the
German dubbed movie ``Forrest Gump'' and its audio-description}


\paragraph{Intro}

\todo[inline]{revise; cf. cons of naturalistic stimuli above}

% traditionally
``In contrast to stimuli designed to trigger a perceptual process of interest,
while controlling for confounding variables (e.g., color and luminance),
% our approach: annotation and regressors
naturalistic stimuli have a fixed but initially unknown temporal structure of
stimulus features of interest, as well as an equally unknown confound
structure'' \citep{haeusler2021speechanno}.

%
In order to build a model of hemodynamic activity, a researcher needs to a)
inform the model about the time courses of events that are assumed to correlate
with psychological and cognitive processes of interest, and b) model hemodynamic
responses (e.g., as nuisance regressors in a \ac{glm}) that are correlated with
the event structure of potentially confounding variables.

%
Hence, I created an extensive annotation of speech occurring in the movie and
the audio-description.
%
In general, the current annotation of speech extends the studyforrest dataset
as a public resource for independent research, and complements formerly
published annotations of portrayed emotions \citep{labs2015portrayed}, perceived
emotions \citep{lettieri2019emotionotopy}, as well as cuts and locations
depicted in the movie \citep{haeusler2016cutanno}.
%
Specifically for this dissertation, the annotation of speech serves as the basis
for the targeted analyses in study 2.


\paragraph{What I did}
% cloud-based annotations
State of the art cloud-based speech-to-text services (Google Cloud, IBM Watson)
fell short to deliver satisfactory scaffolds of an annotation.
% procedure
Consequently, I revised a preliminary transcription of language spoken by
actors, actresses, and the narrator, and submitted it to a forced aligner
(\href{https://github.com/MontrealCorpusTools/Montreal-Forced-Aligner}{Montreal
Forced Aligner} v1.0.1 \citep{mcauliffe2017montreal}) that identified the exact
onset and offset of each word and phoneme. The forced aligner's output was
extensively cleaned and extended both algorithmically as well as manually.


\paragraph{Annotation content}
%
Consequently as planned, the annotation's content substantially exceeds the
groundwork that was needed to conduct the study reported in Chapter 3 (s.
\citep{haeusler2022processing}), and contributes to the studyforrest project for
public use:
% content
the annotation provides ``information about the exact timing of each of the more
than 2500 spoken sentences, 16000 words (including 202 non-speech
vocalizations), 66000 phonemes, and their corresponding speaker''
\citep{haeusler2021speechanno}.
%
For every word, the annotation additionally provides ``lemmatization, a simple
part-of-speech-tagging (15 grammatical categories), a detailed part-of-speech
tagging (43 grammatical categories), syntactic dependencies, and a semantic
analysis based on word embedding which represents each word in a 300-dimensional
semantic space'' \citep{haeusler2021speechanno}.


\paragraph{Validation}

% what we did
We created a canonical \ac{glm} based on information drawn from the annotation
to model hemodynamic brain activity to validate the dataset's quality.
% results in line with previous studies
As hypothesized, results revealed statistically significant increased
hemodynamic activity in a bilateral cortical network including temporal,
parietal and frontal regions related to processing spoken language.
% results replicate
Our exploratory analysis replicates results of studies that employed
tightly-controlled stimuli (s. \citep{friederici2011brain,
hickok2007cortical,price2012twentyyears} for reviews), and studies that employed
data-driven methods to analyze \ac{fmri} data from auditory naturalistic stimuli
\citep{honey2012not, lerner2011topographic, silbert2014coupled}
% "logic" inference
Consequently, results suggest that the annotation's content and quality enables
independent ``researchers to model hemodynamic brain responses that correlate
with a variety of aspects of spoken language'' \citep{haeusler2021speechanno}
under more ecologically valid conditions.
% transition to study 2
Last, the results encouraged us to use the annotation as a groundwork to be
adapted to our specific needs in study 2.


\subsubsection{Processing of visual and non-visual naturalistic spatial
information in the ``parahippocampal place area''}

\todo[inline]{following is (deliberately) still too long and ``bumpy''}

\todo[inline]{will be cleaned iteratively during next 1000 iterations}


\paragraph{Intro}

% study in one sentence
In this study, we investigated whether it is possible to localize a visual area,
namely the \ac{ppa}, usually identified via contrasting blocks of pictures in a
task-based functional localizer, via two naturalistic stimuli.
% hypo a: group
We hypothesized that a conventional model-based statistical analysis would
reveal increased hemodynamic activity in medial temporal regions that were
functionally identified as the PPA by a previously published study
\citep{sengupta2016extension} that performed a functional localizer in the same
set of participants.
% hypo b: individuals
``We hypothesized further that a purely auditory stimulus could, in principle,
localize the PPA as an example of a ``visual area'' in individual persons,
%
and may offer an alternative paradigm to assess brain functions in visually
impaired individuals'' \citep{haeusler2022processing}.


\paragraph{Literature review}

% #1 previous studies
``Several studies have reported increased hemodynamic activity in the PPA
attributed to the processing of scene-related, spatial information, for example,
when participants were watching static pictures of landscapes compared to
pictures of faces or objects \citep{epstein1998ppa,
epstein1999parahippocampal}'' \citep{haeusler2022processing}.
% auditory semantics unclear
However, reports regarding the correlates of processing spatial information in
verbal stimulation are less clear \citep{aziz2008modulation}.

% rationale
In case the stimulus classes that are correlated with hemodynamic responses
during traditional paradigms are represented by [sampled by?] stimulus features
embedded in the movie ``Forrest Gump'' and its audio-description, it should also
be possible to annotate these features and model feature-related hemodynamic
responses during naturalistic stimulation in order to localize the \ac{ppa}.


\paragraph{Annotation}

% AV stimulus
``The movie stimulus shares the stimulation in the visual domain with classical
localizer stimuli, while featuring real-life-like visual complexity and
naturalistic auditory stimulation'' \citep{haeusler2022processing}.
% AV anno
``For the analysis of the movie stimulus, we took advantage of a previously
published annotation movie cuts and the depicted location after each cut
\citep{haeusler2016cutanno}'' \citep{haeusler2022processing}.

% AD stimulus
``The audio-description maintains the naturalistic nature of the movie stimulus,
but limited to the auditory domain'' \citep{haeusler2022processing}.
% AD annotation
For the analysis of the audio-description stimulus, we extended the published
annotation of speech that was created in study 1 \citep{haeusler2021speechanno}.
%
Nouns that the narrator uses to describe the movie's absent visual content were
semantically categorized in case the narrator used them to describe locations,
buildings, rooms, faces or bodies, and so on.

% statistical analysis
We then performed a canonical model-based, mass-univariate analysis supposed to
closely resemble the previously published data analysis of a traditional
block-design localizer performed with the same set of participants
\citep{sengupta2016extension}.
% events & contrasts
Hemodynamic responses for each stimulus were modeled based on events that should
correlate with the perception of spatial information and contrasted with events
that should correlate with non-spatial perception to a lesser degree, not at all
\citep{haeusler2022processing}.


\paragraph{Results}

% group AV
On a group-average level, significantly increased activation correlating with
visual spatial information occurring in the movie is overlapping with a
traditionally localized \ac{ppa} but also extending into earlier visual
cortices.
% group AD
``Activation correlating with semantic spatial information occurring in the
audio-description is more restricted to the anterior \ac{ppa}''
\citep{haeusler2022processing}.
% individual AD
``On an individual level, we find significant bilateral activity in the PPA of
nine individuals and unilateral activity in one individual''
\citep{haeusler2022processing}.
% individual level
``Bilateral clusters in 9 of 14 participants (of which \texttt{sub-04} shows
only a right-lateralized PPA in the block-design localizer results), and a
unilateral significant cluster in one participant, indicate that the
group-average results are representative for the majority of individual
participants'' \citep{haeusler2022processing}.


\paragraph{Discussion}

% generalization
Findings demonstrate that increased activation in the PPA during the perception
of static pictures generalizes to the perception of spatial information embedded
in a movie or a purely auditory stimulus \citep{haeusler2022processing}.
% model-based
Our results add further evidence that a model-driven GLM analysis based on
annotations can be applied to a naturalistic paradigm to localize functional
areas correlating with pre-defined perceptual processes.
%
Same but different: The present results are evidence that a functionally defined
region, such as the PPA, can be localized using a model-driven \ac{glm} analysis
that is based on a naturalistic stimulus' annotated temporal structure with
respect to a particular hypothesized cognitive or perceptual function.

% auditory localizer
``Finally, the presented evidence on the in-principle suitability of a naturally
engaging, purely auditory paradigm for localizing the PPA may offer a path to
the development of diagnostic procedures more suitable for individuals with
visual impairments or conditions like nystagmus''
\citep{haeusler2022processing}.


\paragraph{Transition to study 3}

\todo[inline]{I am still lost here; depends text in SRM study}


%
``Our results provide further evidence that the PPA can be divided into
functional subregions that coactivate during the perception of visual scenes''
\citep{haeusler2022processing}.

%
Results suggest that you can ``localize an auditory PPA'' but not the ``visual
PPA'' using an auditory narrative.

%
Still, there are responses in the parahippocampal cortex correlating with
spatial information.

%
Still, the AO response might be too different for ``sufficient alignment'' and
estimate the localizer results.

%
Our event structure and model is just an approximation of the ``real'' events
structure that is correlated with the response series in the parahippocampal
cortex (in AO but especially in the AV!).
%
However, exact modeling of that process is not necessary for data-driven
alignment in study 3.

%
Problem: we have a couple of subjects seem to simply not give a fuck about
spatial information in audio-description


\subsubsection{DataLad: distributed system for joint management of code, data,
and their relationship}

\todo[inline]{well, does not really fit in}


\subsubsection{Using varying amount of data from naturalistic stimulation for
functional alignment to predict results from a task-based functional localizer
paradigm}

\todo[inline]{following part is a draft}

\todo[inline]{will probably change a lot after text in study 3 is written}

% intro
\textit{Intro:} In order to map perceptual or cognitive functions onto the brain
anatomy of study participants, researchers usually conduct dedicated experiments
\textit{functional localizers} often accompanied with a task.

% problem
\textit{Current approach \& problem:} Nevertheless, the approach ``one paradigm
to map one domain of brain functions'' becomes impractical if a variety of
domains is supposed to be mapped in a time-efficient manner:
%
a clinical application must aim to minimize the scan time/cost but still provide
valid results

% therefore
\textit{Therefore:} In the current study, we explore a method and the quantity
of data needed to predict individual functional topographies by projecting
results of a localizer experiment (statistical $Z$-maps) from a reference group
into the brain anatomy of individual participants.

%
fMRI data acquired from naturalistic stimulation are used to align a subject's
voxel space with a common representational/functional reference space.
%
We test whether functional alignment can be achieved with a task-free, natural
stimulation ``calibration'' scan that requires no more acquisition time than a
conventional localizer paradigm.
%
Once aligned, the brain activity of the reference group can be used to predict
the activity of another subject.

%
We will estimate the trade-off between diagnostic quality and required effective
scan time by progressively reducing the duration of input BOLD fMRI data and
comparing the results of the reduced model to the reference computed from the
full length scan.


\textit{Method:}
% data
During functional magnetic resonance imaging (fMRI), participants ($N=14$) took
part in a task-based, block-design visual localizer and two naturalistic stimuli
paradigms: an audio-visual movie and the movie's audio-description, both
paradigms free of any task.

% cms creation
Based on these response time series, we first created a common model space
employing a shared response model \citep{chen2015reduced} following
leave-one-subject out cross-validation, a process that also computed
transformation matrices for the subjects that provided the data for the creation
of the model space.

% alignment
Then, we aligned left-out subjects with the common [via Procusted
transformation?] to derive transformation matrices for the left-out-subjects by
also varying the quantity of functional response time series to perform the
alignment.

% prediction
Lastly, the acquired transformation matrices were used to project the functional
topographies from the anatomy of the reference group into the common model
space, and from the common model space into the anatomy of the left-out
subjects.

%
Results of the conventional localizer are projected into the common space.
%
The transpose of subject-specific transformation matrices is used to project
functional properties of the common reference into that individual's voxel
space.
%
Then, data are projected into the voxel space of the left-out individual for
comparison with the localizer results for that individual.


% stimulus length?
[We assessed the relationship between length of naturalistic stimulation used
for a \textit{partial functional alignment} and the performance of predicting
empirical $Z$-maps.] by...

%
\textit{Results} suggest that ``a subject's idiosyncratic functional topography
can be estimated with high fidelity from that subject's fMRI data obtained while
watching a naturalistic movie using hyperalignment to project other subjects’
localizer data into that subject's idiosyncratic cortical anatomy''
\citep{jiahui2020predicting}.

%
\textit{Discussion}: stimulus length: 15-30 min vs. 2h;  results of auditory
stimulus to predict visual localizer are ``modest''.

%
\textit{Conclusion}: ``These findings lay the foundation for developing an
efficient tool for mapping functional topographies for a wide range of
perceptual and cognitive functions in new subjects based only on fMRI data
collected while watching an engaging, naturalistic stimulus and other subjects'
localizer data from a normative sample'' \citep{jiahui2020predicting}.


\paragraph{just some notes}

\todo[inline]{other contrasts that were tested over the course of the
dissertation: phonemes, grammatical tags, prosody, sex}






\chapter{DataLad: distributed system for joint management of code, data, and
their relationship}
% take input from external file
This part of the dissertation has been published:

\bigbreak

\noindent
%
Halchenko, Y. O.,
Meyer, K.,
Poldrack, B.,
Solanky, D. S.,
Wagner, A. S.,
Gors, J.,
MacFarlane, D.,
Pustina, D.,
Sochat, V.,
Ghosh, S. S.,
Mönch, C.,
Markiewicz, C. J.,
Waite, L.,
Shlyakhter, I.,
de la Vega, A.,
Hayashi, S.,
Häusler, C. O.,
Poline, J.-P.,
Kadelka, T.,
Skytén, K.,
Jarecka, D.,
Kennedy, D.,
Strauss, T.,
Cieslak, M.,
Vavra, P.,
Ioanas, H.-I.,
Schneider, R.,
Pflüger, M.,
Haxby, J. V.,
Eickhoff, S. B.,
\& Hanke, M.
%
(2021).
%
DataLad: distributed system for joint management of code, data, and their
relationship.
%
Journal of Open Source Software, 6(63), 3262.
%
doi: \href{https://doi.org/10.21105/joss.03262} {10.21105/joss.03262}.

% Yaroslav O. Halchenko, Kyle Meyer, Benjamin Poldrack, Debanjum Singh Solanky,
% Adina S. Wagner, Jason Gors, Dave MacFarlane, Dorian Pustina, Vanessa Sochat,
% Satrajit S. Ghosh, Christian Mönch, Christopher J. Markiewicz, Laura Waite,
% Ilya Shlyakhter, Alejandro de la Vega, Soichi Hayashi, Christian Olaf
% Häusler, Jean-Baptiste Poline, Tobias Kadelka, Kusti Skytén, Dorota Jarecka,
% David Kennedy, Ted Strauss, Matt Cieslak, Peter Vavra, Horea-Ioan Ioanas,
% Robin Schneider, Mika Pflüger, James V. Haxby, Simon B. Eickhoff, and Michael
% Hanke.  DataLad: distributed system for joint management of code, data, and
% their relationship. Journal of Open Source Software, 6(63):3262, 2021. doi:
% 10.21105/joss.03262.






\chapter{A studyforrest extension, an annotation of spoken language in the
German dubbed movie ``Forrest Gump'' and its audio-description}
% take input from external file

\section{Recapitulate aims here}

From TeaP talk:
%
What is still kind of unclear is if these results generalize from static
pictures to results from a more ecologically valid stimuli like a movie.
%
What is still unclear is if these results generalize to the auditory domain.  To
frame it as a question: Does increased hemodynamic activity in the PPA correlate
with auditory spatial information?
%
The goals of your study were the following: We asked ourselves:
%
1) How can we operationalize the perception of spatial information that is
embedded in our naturalistic stimuli?
%
2) Further, we wanted to compare our results to a classic visual localizer
experiment that used blocks of pictures.
%
And lastly, we wanted to explore if an auditory narrative could substitute a
visual experiment to do individual diagnostics
%
Which means: can we localize the PPA in individual persons using a more engaging
\& entertaining, but also exclusively auditory stimulus
%
To operationalize the perception of spatial information, we looked at time
points in the stimuli that should correlate with the perception of spatial
information.
%
And we looked for time points correlating with other perceptual processes that
we needed to build general linear model contrasts.
%
For the movie, we used the movie cuts that basically re-orient the observer in
space.
%
In the context of naturalistic stimuli - not just data-driven but also
model-driven analyses can be applied to data from naturalistic stimuli.
%
- but: to model event-related hemodynamic activity you need to know the temporal
structure of the stimulus
%
- and because naturalistic stimuli are so versatile and trigger so many
perceptual and cognitive processes you can re-use already existing data to
answer new research questions.

\section{How is this ground work for Study 4}
%
results suggest that you can ``localize a auditory PPA'' but not a the ``visual
PPA'' using an auditory narrative; It's similar but different; SRM study will
quantify the prediction performance






\chapter{Processing of visual and non-visual naturalistic spatial information in
the ``parahippocampal place area''}
% take input from external file
\todo[inline]{I do not know if it makes sense to point to the paper in dedicated
chapter; maybe, just pointing it out at the beginning of the dissertation
(``parts of this dissertation have been published'') is enough.}

\todo[inline]{otherwise, this chapter could be first ``real'' chapter following
the general introduction or last chapter; but order should be similar to
how introduction and general discussion are structured}






\chapter{Estimating results of functional localizer's statistical contrast in
individual subjects from data of a reference group}
% take input from external file
\section{Introduction}

% higher visual areas higher visual areas
In the domain of higher-visual perception, functionally defined
category-selective brain regions, such as the \ac{ppa} \citep{epstein1998ppa},
the \ac{ffa} \citep{kanwisher1997ffa}, or \ac{eba} \citep{downing2001bodyarea}
exhibit significantly increased \ac{bold} activity correlated with a
``preferred'' \citep{debeck2008interpreting} stimulus class.
%
While the topographies (i.e. the location, size and shape) of these
category-selective areas are similarly distributed across individuals, the exact
topographies vary interindividually \citep{rosenke2021probabilistic,
zhen2017quantifying, zhen2015quantifying, frost2012measuring}.
% definition of localizer
To identify the topography of functional areas in individual persons,
block-design \textit{functional localizer} paradigms are traditionally used that
contrast modeled hemodynamic responses correlating with the corresponding
stimulus classes, such as landscapes, faces, or bodies.
% problem: one localizer for one domain
Functional localizers are designed to maximize detection power and thus limited
to mapping only one domain of brain functions, such as retinotopic visual areas
\citep{wang2015probabilistic}, category-selective regions
\citep{stigliani2015temporal}, theory of mind \citep{spunt2014validating}, or
semantic processes \citep{fedorenko2010new, fernandez2001language}.
% which gets messy
However, when mapping multiple functional domains in a limited amount of time is
desired, the "one paradigm for one domain of functions" approach becomes
impractical.
% localizer batteries: intro
To address this issue, researchers have attempted to create time-efficient,
multi-functional localizer batteries \textit{localizer batteries}
\citep[e.g.,][]{barch2013function, drobyshevsky2006rapid, pinel2007fast}.
% task based = shit
Nevertheless, the diagnostic quality of localizer paradigms heavily depends on a
participant's comprehension of the task instructions and general compliance, a
criterion that can be difficult to meet in clinical or pediatric populations
\citep{eickhoff2020towards, vanderwal2019movies}.

% ppa via audio-description Results also suggest that a naturally engaging,
% purely auditory paradigm like an audio-description could, in principle,
% substitute a visual localizer as a diagnostic procedure to assess brain
% functions in visually impaired % individuals \citep{haeusler2022processing}.
In \citet{haeusler2022processing}, we demonstrated that a functionally defined
region such as the \ac{ppa} can be localized using a model-driven \ac{glm}
analysis that is based on the annotated temporal structure of a two-hour long
naturalistic stimulus.
% full feature film is too long
However, for practical and monetary reasons, a two-hour long paradigm is
unsuitable for clinical applications.
% hence, predict from reference
One approach to reduce the time and costs is to identify a functional area in an
individual person's brain anatomy based on data collected from an independent
sample of different individuals (i.e. data from a \textit{reference group}).
% intro: estimation via common anatomical space
Previous studies have estimated the most probable location of a functional area
in an individual from a reference group by performing either a volume-based
\citep[e.g.,][]{zhen2017quantifying, zhen2015quantifying} or a surface-based
\citep[e.g.,][]{frost2012measuring, weiner2018defining,
rosenke2021probabilistic, wang2015probabilistic} \textit{anatomical alignment}.
%
First, in order to address the issue of anatomical variability across persons,
functional data of persons in the reference group are anatomically aligned to
(i.e.  projected into) a \textit{common anatomical space} (e.g., Montreal
Neurological Institute brain atlas; \citep[MNI152,][]{fonov2011unbiased}).
% project into test subject to estimate
Then, data are projected from the common anatomical space into\todo{?} the
individual person's brain anatomy to provide an estimate a functional region's
location.
% volume-based alignment in one sentence
Volume-based anatomical alignment \citep[s.][for a review]{klein2009evaluation}
aligns voxels to a three-dimensional common anatomical space \citep[e.g., MNI152
atlas;][]{fonov2011unbiased}.
% surface-based alignment in one sentence
Surface-based anatomical alignment \citep{fischl1999cortical, yeo2009spherical}
aligns vertices to a two-dimensional common anatomical space \citep[e.g.,
FreeSurfer's fsaverage template;][]{fischl1999high}.
% difference in one sentence
Whereas volume-based alignment does not account for individual sulcal and gyral
folding patterns, surface-based alignment respects interindividual variability
of the cortical surface.
% surface-based estimation works better
Consequently, previous studies that compared volume-based and surface-based
alignment to estimate the location of functional regions have shown that
surface-based alignment reduces inter-subject variability and improves
estimation performance \citep{rosenke2021probabilistic, frost2012measuring,
wang2015probabilistic, weiner2018defining}.
% remaining variability after surface-based alignment
However, even after surface-based alignment, the anatomical location of
functional regions varies between individuals \citep[e.g.,][]{coalson2018impact,
benson2014correction, natu2021sulcal, wang2015probabilistic, frost2012measuring,
langers2014assessment, weiner2014mid, rosenke2021probabilistic}.
% frost as an example
\citet{frost2012measuring}, for example, localized 13 functional areas of the
high-level visual cortex and ``found a large variability in the degree to which
functional areas respect macro-anatomical boundaries'' \citep[][p.
1369]{frost2012measuring}.
% functional--anatomical correspondence
The remaining variability indicates that functional areas a not necessarily
bound to anatomical landmarks, and reflects the degree of
\textit{functional--anatomical correspondence} between a brain function and its
underlying anatomical location.

% case of PPA cf. also \citet{frost2012measuring, rosenke2021probabilistic}
% \citet{weiner2018defining} showed ``that cortical folding patterns and
% probabilistic predictions reliably identify place-selective voxels in medial
% VTC across individuals and experiments''.
%
% However, ``this structural-functional coupling is not always perfect and there
% is inter-subject variability as to how much the place-selective voxels extend
% within the parahippocampal gyrus, as well as the lingual gyrus and medial
% aspects of the fusiform gyrus.
%
% Despite this inter-subject variability, place-selective voxels are always
%located within the collateral sulcus across participants.''
%\citep{weiner2018defining}.
In order to address the issue of functional-anatomical variability across
subjects, \textit{functional alignment} algorithms, such as
\textit{hyperalignment} \citep{haxby2011common, guntupalli2016model} or the
\textit{shared response model} (SRM) \citep{chen2015reduced,
zhang2016searchlight}, have been developed.
%
Whereas anatomical alignment aligns voxels (or vertices) that share the same
anatomical location to a common anatomical space, functional alignment aligns
voxels (or vertices) that share similar functional properties to a
\textit{common functional space} (CFS).
%
Functional alignment algorithms are typically used to compute both a
high-dimensional, functional brain template (i.e. the \ac{cfs}) subject-specific
transformations from the functional data of a study's participants.
%
A subject-specific transformation allows to project functional data from a
subject's three-dimensional voxel space into\todo{?} the \ac{cfs} or vice versa
\citep{haxby2020hyperalignment, kumar2020brainiak}.
%
The \ac{cfs} and transformations are calculated (i.e. \textit{trained}) by
maximizing the inter-subject similarity of \ac{bold} response time series
correlating with a time-locked external stimulation \citep{haxby2011common,
chen2015reduced, sabuncu2010function}, or by maximizing the inter-subject
similarity of connectivity profiles \citep{feilong2018reliable,
guntupalli2018computational, nastase2019leveraging}.
%
Whereas connectivity-based functional alignment better aligns connectivity
profiles, response-based functional alignment better aligns response time-series
\citep{guntupalli2018computational}.
%
Although functional alignment algorithms can be applied to \ac{fmri} time series
data from paradigms employing simplified stimuli, data from naturalistic stimuli
provide
%
improved generalizability of the \ac{cfs}
%
and transformation matrices
%
to novel stimuli or tasks.
%
This is presumably because naturalistic stimuli sample a broader range of brain
states \citep{haxby2011common, guntupalli2016model}.

\todo[inline]{imo, validity (\& generalizability) of CFS and matrices are
non-separable claims!}

A more recent procedure \citep[e.g., ][]{jiahui2020predicting,
guntupalli2016model, haxby2011common} to estimate the most probable location of
a functional area in an individual from a reference performs an functional
alignment.
% solve functional-anatomical variability
First, the functional data of individuals in the reference group are
anatomically aligned to a common anatomical space.
%
Second, to address the issue of functional-anatomical variability across
individuals, the functional data are functionally aligned (i.e. projected into)
a \ac{cfs}.
%
Finally, data are projected from the \ac{cfs} into the\todo{?} individual's
brain anatomy, serving as an estimate of a functional region's location.
% Example: Jiahui (2020)
For instance, \citet{jiahui2020predicting} used surface-based hyperalignment to
calculate \acp{cfs} and transformations based on data from
%
the movie ``Grand Budapest Hotel'' ($\approx$\unit[50]{min}; \ac{tr}=\unit[1]{s}),
and
%
the movie ``Forrest Gump'' ($\approx$\unit[120]{min}; \ac{tr}=\unit[2]{s}).
%
\citet{jiahui2020predicting} then estimated $t$-contrast maps from a visual
localizer that aimed at identifying the \ac{ffa} by projecting the $t$-contrast
maps from a reference group through each \acp{cfs} into an individual's brain
anatomy.
%
Results showed that $t$-contrast maps from the visual localizer correlated more
highly with contrast maps that were estimated via hyperalignment than contrast
maps that were estimated via surface-based anatomical alignment.


\todo[inline]{\textit{criterion} and \textit{predictors} are defined here to
make the discussion easier; however, it's not predictor / criterion in a
strict sense as typically used (as referring to single variables)}

% focus: ppa
Here again, we focus on the \ac{ppa} \citep[e.g.,][for
reviews]{epstein2014neural, aminoff2013role}, and investigate whether we can
estimate the results of $t$-contrasts (i.e. $Z$-maps) that were created to
identify the \ac{ppa} using functional data from three different paradigms as
the to predicted \textit{criteria}:
%
(a) a classic visual localizer \citep{sengupta2016extension} as the assumed
``gold standard'',
%
(b) a movie \citep{haeusler2022processing}, and
%
(c) an auditory narrative \citep{haeusler2022processing}.
% math stuff from citep{vodrahalli2018mapping} ``SRM learns $N$ maps $W_{i}$
% with orthogonal columns such that $||X_{i}-W_{i}S||_{F}$ is minimized over
% $\left\{ W_{i}\right\} _{i=1}^{N},S$, where $X_{i}\in\mathbb{R}^{v\times{T}}$
% is the $i^{th}$ subject's fMRI response ($v$ voxels by $T$ repetition times)
% and $S\in\mathbb{R}^{k\times{T}}$ is a feature time-series in a
% $k$-dimensional shared space'' \citep{vodrahalli2018mapping}.  Inverse vs.
% transpose of a matrix: for orthogonal transformations (like we should have
% here, i.e. only rotation, expansion) these two are one and the same thing:
% https://www.quora.com/When-is-the-inverse-of-a-matrix-equal-to-its-transpose
% why SRM
Our volume-based functional alignment approach utilizes the \ac{srm} algorithm
\citep{chen2015reduced, richard2019fast} as implemented in the open-source
software package BrainIAK \citep[Brain Imaging Analysis Kit;
\href{https://brainiak.org}{\url{brainiak.org}};][]{kumar2020brainiak,
kumar2020brainiaktutorial}.
% general overview of SRM
The \ac{srm} is an unsupervised probabilistic latent-factor model that
decomposes \ac{bold} \ac{fmri} response time series of participants who have
experienced the same stimulus into a \ac{cfs} of \textit{shared features}
\citep[also known as ``\textit{shared feature space}'';][]{chen2015reduced} and
subject-specific linear transformations.
% math stuff
More specifically, the \ac{srm} algorithm uses each $n^{th}$ subject's response
time series represented as matrix $X_{n}$ ({$v$} voxels by $t$ time points) to
compute the \ac{cfs} $C$ ($k$ shared responses by $t$ time points) and
subject-specific transformation matrices $W_{n}$ ($v$ voxels by $k$ shared
responses) with orthonormal columns ($W_{n}^{T}W_{n}=I_{k}$).
% iteratively fitted
The algorithm randomly initializes and fits the transformation matrices over
iterations to minimize the error in explaining the participants' data, while
also learning the time course of the shared responses (cf.
\href{https://brainiak.org/tutorials/11-SRM/}{\url{brainiak.org/tutorials/11-SRM}}).
% number of dimensions
In contrast to hyperalignment, the number of dimensions of the \ac{cfs} is not
set by the number of voxels, but rather it is determined by the researcher to a
number lower than the number of voxels, a procedure that also filters out noise
and reduces overfitting \citep{chen2015reduced}.
% phrase math in words
As a result, each shared feature can be thought of as a weighted sum of many
voxels across subjects \citep{kumar2020brainiak}.
% result = alignment
A subject-specific transformation matrix can be thought of as the weight of each
voxel in a subject's voxel space on each shared feature, and allows a
subject's\todo{?} functionally data to be aligned to the \ac{cfs} by projecting
responses within the voxels into\todo{?} the $k$-dimensional \ac{cfs}.


\todo[inline]{check if $W_{n}^{T}W_{n}=I_{k}$ is True}

\todo[inline]{biggest issue (for discussion, too): how to separate validity /
generalizability of CFS from validity / generalizability of transformation
matrices?}

% multi-paradigm model
In contrast to previous studies \citep[e.g.][]{jiahui2020predicting,
guntupalli2016model, haxby2011common} that calculated a \ac{cfs} based on data
from a single paradigm, we calculated a \textit{multi-paradigm \ac{cfs}} based
on data from three paradigms.
% cross-validation
We followed an exhaustive leave-one-subject-out cross-validation (N$=$14
subjects) to train a shared feature space (i.e. the \ac{cfs}) based on
concatenated response time series of
%
the movie ``Forrest Gump'' ($\approx$\unit[120]{min}; \ac{tr}=\unit[2]{s}),
%
the movie's audio-description  that was produced for a visually impaired
audience ($\approx$\unit[120]{min}; \ac{tr}=\unit[2]{s}), and
%
a visual localizer ($\approx$\unit[21]{min}; \ac{tr}=\unit[2]{s})
%
from $N-1$ \textit{training subjects} as seen in
Fig.~\ref{fig:multi-stimulus-cfs}.
% four aspects to explore
The purpose of this study was to investigate four aspects.
%
First, we explored the validity and generalizability of our multi-paradigm
\ac{cfs} by predicting a left-out \textit{test subject}'s results from the
analysis of
%
(a) the localizer \citep{sengupta2016extension} as the assumed ``gold
standard'',
%
(b) the movie \citep{haeusler2022processing}, and
%
(c) the auditory narrative \citep{haeusler2022processing}
%
serving as the criteria.
% three predictors
Second, we use a test subject's response time series from each of the three
paradigms separately in order to align the test subject with the corresponding
\acp{tr} within the \ac{cfs}.
%
This \textit{partial alignment} lets us assess the validity and generalizability
of the three paradigms serving as \textit{predictors} (i.e. one
\textit{cross-subject-within-paradigm prediction}, and two
\textit{cross-subject-cross-paradigm predictions}).
% partial alignment
Third, considering the impracticality of using a complete naturalistic stimulus
in a clinical setting to align a test subject, we also explored the relationship
between the estimation performance of the results from each of the three
paradigms and the quantity of data from each of the three paradigms used to
functionally align the subject with the multi-paradigm \ac{cfs}.
% benchmark: anatomical alignment the criteria
Fourth, we compared the performance of our volume-based, functional alignment
procedures to the performance of a volume-based, anatomical alignment approach
that serves as a benchmark.

\todo[inline]{add 2-3 sentences stating the results}
%
% Our results provide evidence that transformation matrices calculated based
% on data from naturalistic stimuli promise an increased validity of derived
% transformation for functional alignment over transformation matrices based on
% data (of the same!) paradigm based on simplified stimuli.

\todo[inline]{add 2-3 sentences stating a conclusion \& vision}
% Our results suggest that it is possible to ``scan once, estimate many

\begin{figure*}[tbp]
\centering
\includegraphics[width=\linewidth]{figures/multi-stimulus-cfs.pdf}
\caption{
%
    \textbf{Overview of the shared response model (SRM).
}
    %
    For each fold of the leave-one-subject-out cross-validation, each training
    subject's response time series of
    %
    the movie ($\approx$\unit[120]{min}; \ac{tr}=\unit[2]{s}),
    %
    the movie's audio-description ($\approx$\unit[120]{min};
    \ac{tr}=\unit[2]{s}),
    %
    and the visual localizer ($\approx$\unit[21]{min}; \ac{tr}=\unit[2]{s})
    %
    were concatenated to serve as input for the \ac{srm} algorithm.
    %
    From these response time series represented as matrix $X_{n}$ ({$v$} voxels
    by $t$ time points), the algorithm calculates the common functional
    space (CFS) $C$ ($k$ shared features by $t$ time points) and
    subject-specific transformation matrices $W_{n}$
    ($v$ voxels by $k$ shared features) with orthonormal columns
    ($W_{n}^{T}W_{n}=I_{k}$).
} \label{fig:multi-stimulus-cfs} \end{figure*}





\section{Methods}

% we get the data from the naturalistic PPA paper (its subdataset) datalad get
% -n inputs/studyforrest-ppa-analysis/inputs/studyforrest-data-aligned datalad
%  get
%  inputs/studyforrest-ppa-analysis/inputs/studyforrest-data-aligned/sub-??/in\_bold3Tp2/sub-??\_task-a?movie\_run-?\_bold*.*

% reference to PPA-Paper
For the current study, we used the same subset of the studyforrest dataset that
we previously used in \citet{haeusler2022processing}:
%
The sample included the same fourteen participants who
% VIS
(a) participated in a dedicated six-category block-design visual localizer
\citep{sengupta2016extension},
% AV
(b) watched the audio-visual movie ``Forrest Gump''
\citep{hanke2016simultaneous}, and
% AD
(c) listened to the movie's audio-description \citep{hanke2014audiomovie}.
% see corresponding papers for details
An exhaustive description of the participants, stimulus creation, procedure,
stimulation setup, and fMRI acquisition can be found in the corresponding
publications, while a summary is provided in \citet{haeusler2022processing}.



\subsection{Preprocessing}

% data sources
The analyses in this study were conducted on the same preprocessed fMRI data he
current analyses were carried out on the same preprocessed fMRI data (s.
\href{https://github.com/psychoinformatics-de/studyforrest-data-aligned
}{\url{github.com/psychoinformatics-de/studyforrest-data-aligned}}) that were
used for
%
(a) the technical validation of the dataset \citep{hanke2016simultaneous},
%
(b) the localization of higher-visual areas \citep{sengupta2016extension}, and
%
(c) the investigation of responses of the \ac{ppa} correlating with naturalistic
spatial information in \citep{haeusler2022processing}.
%
We reran the preprocessing and analyses steps performed in
\citet{sengupta2016extension} and \citet{haeusler2022processing} using FEAT
v6.00 \citep[FMRI Expert Analysis Tool;][]{woolrich2001autocorr} as shipped with
FSL v5.0.9 \citep[\href{https://www.fmrib.ox.ac.uk/fsl}{FMRIB's Software
Library;}][]{smith2004fsl} to reproduce the time series that served as input for
the previous statistical analyses and their results (i.e. the statistical
$Z$-maps).
% temporal filtering
The preprocessing steps included high-pass temporal filtering (using a
Gaussian-weighted least-squares straight line) for every run of the visual
localizer (cutoff period of \unit[100]{s}), and every segment of the movie and
audio-description (cutoff period of \unit[150]{s}).
% brain extraction & spatial smoothing
Brain extraction was performed using BET \citep{smith2002bet}., and data from
all three paradigms were spatially smoothed using a Gaussian kernel with a full
width at half maximum of \unit[4.0]{mm}.
% grand mean normalization
A grand-mean intensity normalization was applied to each run of the functional
localizer and each segment of the naturalistic stimuli.
%
Further analyses on these reproduced times series were performed using Python
scripts that relied on
%
NiBabel v3.2.1 (\href{https://nipy.org}{\url{nipy.org}}),
%
NumPy v1.20.2 (\href{https://numpy.org}{\url{numpy.org}}),
%
Pandas v1.2.3 (\href{https://pandas.pydata.org}{\url{pandas.pydata.org}}),
%
Scipy v1.6.2 (\href{https://scipy.org}{\url{scipy.org}}),
%
scikit-learn v1.0 (\href{https://scikit-learn.org}{\url{scikit-learn.org}}),
%
BrainIAK v0.11
\citep[\href{https://brainiak.org}{\url{brainiak.org}}][]{kumar2020brainiak,
kumar2020brainiaktutorial},
%
Matplotlib v3.4.0 (\href{https://matplotlib.org}{\url{matplotlib.org}}),
%
seaborn v0.11.2 (\href{https://seaborn.pydata.org}{\url{seaborn.pydata.org}}),
%
and calling command line functions of FSL.

%\paragraph{Fixing FSL output}

% grand_mean_for_4d.py (formerly: data_normalize_4d.py):
% is not necessary anymore: FSL has applied grand mean scaling to
% 'filtered_func_data.nii.gz'

% input: 'sub-*/run-?.feat/filtered_func_data.nii.gz' (of VIS, AO & AV)
% output: saved to 'sub-??_task-*_run-?_bold_filtered.nii.gz'

% FSL adds back the mean value for each voxel's time course at the end of the
% preprocessing;
% hence, the script substracts that mean again but multiplies it by 10000
% (like FSL does it, too)

% definition of grand mean scaling for 4d data:
% voxel values in every image are divided by the average global mean
% intensity of the whole session. This effectively removes any mean global
% differences in intensity between sessions.

% FSL User Guide:
% filtered_func_data will normally have been temporally high-pass filtered,
% it is not zero mean; the mean value for each voxel's time course has been
% added back in for various practical reasons.
% When FILM begins the linear modeling, it starts by removing this mean.

\todo[inline]{restriction to ROIs is not mentioned in the intro. Is that okay?}

% masks-from-mni-to-bold3Tp2.py:
% - merges unilateral ROIs overlaps (already in MNI) to bilateral ROI
% - output: 'masks/in_mni/PPA_overlap_prob.nii.gz'
% - warps union of ROIs from MNI into each subjects space
% output: 'sub-*/masks/in_bold3Tp2/grp_PPA_bin.nii.gz' + audio_fov.nii.gz dilate
% the ROI masks by 1 voxel; output: 'grp_PPA_bin_dil.nii.gz'

% masks-from-mni-to-bold3Tp2.py:
% warp MNI masks into individual bold3Tp2 spaces

% masks-from-t1w-to-bold3Tp2.py:
% transforms 'inputs/tnt/sub-*/t1w/brain_seg*.nii.gz'
% into individual's bold3Tp2
% output: 'sub-*/masks/in_bold3Tp2/brain_seg*.nii.gz'

% mask-builder-voxel-counter.py:
% builds different individual masks by dilating, merging other masks
% creates a FoV of AO stimulus for every subject from 4d time-series of AO run
% output: sub-*/masks/in_bold3Tp2/audio_fov.nii.gz'
% counts the voxels
% long story short: we cannot used all gyri that contain PPA to some degree
% even if the mask by FoV of AO stimulus and individual gray matter mask

% data_mask_concat_runs.py:
% masks are not dilated and not masked with subject-specific gray matter mask
% outputs:
% 'sub-*_task_aomovie-avmovie_run-1-8_bold-filtered.npy
% 'sub-*_task_visloc_run-1-4_bold-filtered.npy'

% reason why we do it
The \ac{srm} requires that the number of samples (i.e. the number of \acp{tr})
exceed the number of features (the number of voxels).
%
In order to restrict the number of voxels, we created bilateral \acp{roi} for
each subject.
%
Specifically, we warped the union of individual \acp{ppa}
citep[s.][]{haeusler2022processing} from MNI space into each subject's voxel
space using subject-specific, non-linear transformation matrices that were
previously computed
\citep[][\href{https://github.com/psychoinformatics-de/studyforrest-data-templatetransforms
}{\url{github.com/psychoinformatics-de/studyforrest-data-templatetransforms}}]{hanke2014audiomovie}.
% applying masks
The time series data of each subject were then masked in their native voxel
space by the union of individual \acp{ppa} and the subject-specific \ac{fov} of
the audio-description.
% voxels = [1665, 1732, 1400, 1575, 1664, 1951, 1376, 1383, 1683, 1887, 1441,
% 1729, 1369, 1437] median = 1619.5
The number of remaining voxels per subject (range 1369--1951,
$\overline{X}=1592$, $SD=188$) can be seen in Fig.~\ref{fig:plot_voxel-counts}.
% normalization
Data of each run were normalized ($z$-scored) to a mean of zero
($\overline{X}=0$) and a standard deviation of one ($SD=1$).
%
Due to an image reconstruction problem \citep[cf.][]{hanke2014audiomovie}, the
last 75 \acp{tr} of the audio-description were missing in subject 04.
%
The \ac{srm} allows for different numbers of voxels across subjects, but the
number of \acp{tr} must be the same.
%
Consequently, we removed the last 75 \acp{tr} of the audio-description from the
time series of all other subjects.
% summary; AO + AV = 7123 TRs (not 7198 TRs anymore); localizer has 4 x 156 TRs
As a result, the data used to fit the \ac{srm} in the next step included 3599
\acp{tr} from the movie, 3524 \acp{tr} from the audio-description, and 624
\acp{tr} from the visual localizer experiment (7747 \acp{tr} in total).
%% concatenate and z-score
The time series of all three paradigms were concatenated and $z$-scored.

\begin{figure*}[tbp]
\centering
\includegraphics[width=\linewidth]{figures/plot_voxel-counts.pdf}
\caption{
%
\textbf{Number of voxels in the bilateral regions of interest (ROIs)
of each subject.}
%
In order to reduce the number of voxels, we warped the union of
individual \acp{ppa} \citep[cf. Fig. 1 in][]{haeusler2022processing} from
MNI152 space into each subject's native voxel space.
%
The remaining voxels of each subject were further constrained to those
voxels that are included in the respective subject's \ac{fov} of the
audio-description \citep[cf.][]{hanke2014audiomovie}.
}
\label{fig:plot_voxel-counts}
\end{figure*}


\begin{comment}

The number of remaining voxels per subject can be seen in Table
\ref{tab:ppamaskvoxels} (range 1369--1951, $\overline{X}=1592$, $SD=188$).


\begin{table*}[btp] \caption{
%
\textbf{Table heading.}
%
The number of remaining voxels after masking time series data of each paradigm
and subject with the union of individual \acp{ppa} warped from MNI space
into each individual's subjects-space and the subject-specific field of view
of audio-description.
    }

\label{tab:ppamaskvoxels}
\begin{tabular}{ll}
\toprule
\textbf{Subject} & \textbf{no. of voxels} \\
\midrule
sub-01 & 1665 \tabularnewline
sub-02 & 1732 \tabularnewline
sub-03 & 1400 \tabularnewline
sub-04 & 1575 \tabularnewline
sub-05 & 1664 \tabularnewline
sub-06 & 1951 \tabularnewline
sub-14 & 1376 \tabularnewline
sub-09 & 1383 \tabularnewline
sub-15 & 1683 \tabularnewline
sub-16 & 1887 \tabularnewline
sub-17 & 1441 \tabularnewline
sub-18 & 1729 \tabularnewline
sub-19 & 1369 \tabularnewline
sub-20 & 1437 \tabularnewline
\bottomrule
\end{tabular}
\caption*{The legend text goes here.}
\end{table*}

\end{comment}




\subsection{Estimation via functional alignment}


\subsubsection{Overview}
%
To estimate the empirical $Z$-maps for $t$-contrasts using functional alignment,
we followed a four-step procedure.
% create CFS and training subjects' matrices
First, for every fold of a leave-one-out cross-validation (N$=$14 subjects), we
trained a \ac{srm} on $N-1$ training subjects' response time series of the
movie, the audio-description, and the visual localizer.
% results in...
This step generated a \ac{cfs} for each fold of the cross-validation and an
orthonormal transformation matrix for each training subject.
% align test subject
Second, we aligned the test subject to the corresponding \acp{tr} within the
\ac{cfs} using time series data from the visual localizer, the movie, or the
audio-description.
%
Second, we aligned the test subject's time series data from the movie,
audio-description, and visual localizer paradigms separately to the
corresponding TRs within the \ac{cfs}.
%
This step produced different transformation matrices for the test subject based
on data from different paradigms.
% quantity vs. performance
In order to examine the relationship between estimation performance and the
amount of data used to generate a transformation matrix, we also varied the
number of runs of the paradigms.
%
This step produced transformation matrices based on an increasing number of runs
per paradigm.
%
In the third step, we mapped the training subjects' empirical $Z$-maps from
their voxel space into the \ac{cfs} using their transformation matrices.
% project from CFS into test subject
Finally, we projected the training subjects' $Z$-maps from the \ac{cfs} into the
test subject's voxel space using the transpose of the test subject's
transformation matrix.
% actual prediction
We obtained the test subject's predicted $Z$-maps by calculating the arithmetic
mean of the projected Z-maps



\subsubsection{Fitting the SRM}
%
In order to obtain the \ac{cfs} and the training subjects' transformation
matrices, we used the probabilistic \ac{srm} algorithm that is implemented in
BrainIAK v.11 \citep[Brain Imaging Analysis Kit;][]{kumar2020brainiak,
kumar2020brainiaktutorial}, and approximates the \ac{srm} based on the
Expectation Maximization (EM) algorithm as proposed by \citet{chen2015reduced}
and optimized by \citet{anderson2016enabling}.
% number of dimensions / features ``The effect of number of PCs on BSC was
% similar for models that were based only on Princeton (n = 10) or Dartmouth (n
% = 11) data, suggesting that this estimate of dimensionality is robust across
% differences in scanning hardware and scanning parameters''
% \citep{haxby2011common}.
%
% ``These dimensionality estimates are a function of the spatial and temporal
% resolution of fMRI and the number and variety of response vectors used to
% derive the common space'' \citep{guntupalli2016model}.
%
% ``The true dimensionality of representation in human cortex surely involves
% vastly more distinct tuning functions. Estimates of the dimensionality of
% cortical representation, therefore, will almost certainly be much higher as
% data with higher spatial and temporal resolution for larger and more varied
% samples of response vectors are used to build new common models''
% \citep{guntupalli2016model}.
We chose a value of $k=10$ for the number of shared features (i.e. the number of
dimensions in the \ac{cfs}) based on the temporal and spatial resolution of our
data (\ac{tr} = \unit[2]{s}; \unit[2.5 $\times$ 2.5 $\times$ 2.5]{mm}), the
average number of voxels per \ac{roi}, and findings from
\citet{haxby2011common}.
%
\citet{haxby2011common} used hyperalignment to create a \ac{cfs} of 1,000
dimensions based of functional data (\ac{tr} = \unit[3]{s}) of voxels (\unit[3
$\times$ 3 $\times$ 3]{mm}) located in the ventral temporal cortex.
%
Then, \citet{haxby2011common} reduced the dimensionality of the \ac{cfs} by
applying a \ac{pca} in order to determine the subspace that is sufficient to
capture the full range of response-pattern distinctions.

They then applied principal component analysis \ac{pca} to reduce the
dimensionality of the \ac{cfs} to determine the subspace that sufficiently
captured the range of response-pattern distinctions.
%
Furthermore, the cortical topographies of category-selective brain regions were
preserved in the 35-dimensional \ac{cfs}.
% ...as judged by visual inspection
In the present study, we also computed \acp{cfs} of $k=5, 20, 30, 40, 50$ but
prediction performance based on these \acp{cfs} barely varied from a
10-dimensional \ac{cfs}.
% iterations:
The algorithm was set to iterate 30 times to minimize the error.

% correlations of regressors
In order to visualize characteristics of the \ac{cfs}, we calculated the Pearson
correlation coefficients between the shared responses and the regressors that
were previously modeled \citep[cf.][]{sengupta2016extension,
haeusler2022processing} to investigate hemodynamic responses during the three
paradigms.
%
As an example, we chose the \ac{cfs} that was created in the first fold of the
cross-validation from $N-1$ subjects to estimate $Z$-maps of subject 01.
%
The time series of the shared features were trimmed to match the corresponding
\acp{tr} of the respective paradigms.
%
Fig.~\ref{fig:corr-vis-reg-srm} shows the correlations between regressors
created to model hemodynamic responses during the visual localizer and shared
responses (trimmed to \acp{tr} that match the visual localizer).
Fig.~\ref{fig:corr-av-reg-srm} shows the correlations between regressors created
to model hemodynamic responses during the movie \citep[cf. Table 3
in][]{haeusler2022processing} and shared responses, while
Fig.~\ref{fig:corr-ao-reg-srm} shows the correlations between regressors created
to model hemodynamic responses during the audio-description \citep[cf. Table 3
in][]{haeusler2022processing} and shared responses.


\todo[inline]{what do the plots suggest?}

\todo[inline]{make colors of non-used TRs more transparent (alpha=15)}

% shuffle runs
As a negative control, we randomly shuffled the order of runs of the visual
localizer and the segments of the naturalistic stimuli separately for each
paradigm and training subject. We then concatenated the time series, fit the
\ac{srm}, and calculated the Pearson correlation coefficients.
%
We expected that the \ac{srm} algorithm would fail to fit ``meaningful''
\todo{??} shared responses to randomly shuffled training data.
%
As hypothesized, the results based on shuffled time series revealed no or only
minor correlations between the shared responses and regressors, as shown in
Fig.~\ref{fig:corr-vis-reg-srm-shuffled},
Fig.~\ref{fig:corr-av-reg-srm-shuffled}, and
Fig.~\ref{fig:corr-ao-reg-srm-shuffled}.


\todo[inline]{Plots are located in the appendix; add representation of the model
to the plots}


\begin{figure*}[tbp]
\centering
\includegraphics[width=\linewidth]{figures/corr_vis-regressors-vs-cfs_sub-01_srm-ao-av-vis_feat10-iter30_7123-7747.pdf}
\caption{
%
\textbf{Pearson correlation coefficients between regressors of the visual
localizer and shared features.}
%
The time series of the shared features within the multi-paradigm \ac{cfs}
%
(as calculated for subject 01 in the first fold of the cross-validation)
%
were trimmed to match the corresponding \acp{tr} of the visual localizer
paradigm \citep{sengupta2016extension}.
%
The six regressors of the visual localizer model hemodynamic responses to
the six categories of pictures that were presented in blocks.
}
\label{fig:corr-vis-reg-srm}
\end{figure*}


\begin{figure*}[tbp]
\centering
\includegraphics[width=\linewidth]{figures/corr_av-regressors-vs-cfs_sub-01_srm-ao-av-vis_feat10-iter30_3524-7123.pdf}
\caption{
%
\textbf{Pearson correlation coefficients between regressors of the movie
and shared features}
%
The time series of the shared features within the multi-paradigm \ac{cfs}
%
(as calculated for subject 01 in the first fold of the cross-validation)
%
were trimmed to match the corresponding \acp{tr} of the movie
\citep{hanke2016simultaneous}.
%
The regressors \texttt{vse\_new} to \texttt{vno\_cut} are based on
annotations movie frames, whereas the regressors
\texttt{fg\_av\_ger\_lr} to \texttt{fg\_av\_ger\_ud} represent low-level
visual or auditory confounds
\citep[cf. Table 3 in][]{haeusler2022processing}.
}
\label{fig:corr-av-reg-srm}
\end{figure*}


\begin{figure*}[tbp]
\centering
\includegraphics[width=\linewidth]{figures/corr_ao-regressors-vs-cfs_sub-01_srm-ao-av-vis_feat10-iter30_0-3524.pdf}
\caption{
%
\textbf{Pearson correlation coefficients between regressors of the
audio-description and shared features.}
%
The time series of the shared features within the multi-paradigm \ac{cfs}
%
(as calculated for subject 01 in the first fold of the cross-validation)
%
were trimmed to match the corresponding \acp{tr} of the
audio-description \citep{hanke2014audiomovie}.
%
The regressors \texttt{body} to \texttt{sex\_m} are based on
annotations of nouns spoken by the audio-description's narrator,
whereas the regressors \texttt{fg\_ad\_ger\_lrdiff} and
\texttt{fg\_ad\_ger\_rms} represent low-level auditory confounds
\citep[cf. Table 3 in][]{haeusler2022processing}.
%
\texttt{geo\&groom} is a combination of
regressors as used on the positive side of the primary contrasts aimed to
localize the \ac{ppa} \citep[cf. Table 5 in][]{haeusler2022processing}.
}
\label{fig:corr-ao-reg-srm}
\end{figure*}


\subsubsection{Alignment of the test subject}

% AO: 0-451, 0-892, 0-1330, 0-1818, 0-2280, 0-2719, 0-3261, 0-3524
% AV: 3524-3975, 3524-4416, 3524-4854, 3524-5342, 3524-5804, 3524-6243,
%     3524-6785, 3524-7123
% AO+AV: 0-7123

% in-code documentation of
% https://github.com/brainiak/brainiak/blob/master/brainiak/funcalign/srm.py
% says: # Solve the Procrustes problem; A =
% subjectFMRIdata.dot(SharedResponses.T) U, \_, V = np.linalg.svd(A,
% full\_matrices = False) return U.dot(V)

%
We aligned the test subject's response time series from the visual localizer,
the movie, or the audio-description to the corresponding \acp{tr} within the
\ac{cfs} by factorizing the response time series data via singular value
decomposition.
%
This step produced an orthonormal transformation matrix $W_{n}$ each paradigm
that allow a mapping of data from a test subject's voxel space into the
\ac{cfs}.
%
To investigate how the amount of data used to acquire a transformation matrix
affects estimation performance, we also varied the number of runs per paradigm.
%
Specifically, we we used one up to four runs (each lasting \unit[5.2]{min}) of the
visual localizer, and one up to eight segments (each lasting
$\approx$\unit[15]{min}) to align the test subject to the corresponding \acp{tr}
within the \ac{cfs}.
%
Therefore, for each test subject, we obtained four matrices based on data from
the visual localizer and eight different matrices per naturalistic stimulus,
each transformation matrix having a size of $v$ voxels by $k$ shared responses
but being based on an increasing amount of data used to calculate the mapping.


\subsubsection{Estimation of $t$-contrasts' results}

% overview
We estimated the results of the three $t$-contrast (i.e. the \textit{empirical
$Z$-maps}) of the test subject by projecting the empirical $Z$-maps of all
training subjects trough the \ac{cfs} into the test subject's voxel space:
% functional alignment; into CFS (calling srm.transform(masked\_zmaps))
First, we masked the empirical $Z$-maps of the training subjects' empirical
$Z$-maps with the same subject-specific masks used to generate the \ac{cfs} from
the time series data.
%
Then,  we used the transformation matrices derived during the training of the
\ac{cfs} to map the masked empirical $Z$-maps from each training subject's voxel
space into the \ac{cfs}.
%
Next, we used the transpose of a transformation matrix obtained from the
alignment of the test subject to project the $Z$-maps from the \ac{cfs} into the
test subject's voxel space.
% take the mean
For each of the three $t$-contrasts, we computed the arithmetic mean of the
respective projected $Z$-maps, which served as an estimation of the test
subject's empirical $Z$-maps (hence, a \textit{predicted $Z$-maps}).

\subsection{Estimation via anatomical alignment}

\todo[inline]{strictly speaking data were not projected through the MNI152 atlas
but through a study-specific brain template co-registered to the MNI152 atlas,
wasn't it?}

\todo[inline]{was the matrix the transpose or a "totally" different one?}

%
As a baseline, we used an anatomical alignment procedure to estimate the results
of $t$-contrast of each paradigm.
%
We predicted a test subject's empirical $Z$-maps from the analysis of
%
the visual localizer \citep{sengupta2016extension},
%
the movie \citep{haeusler2022processing}, and
%
the audio-description \citep{haeusler2022processing}
%
using the training subjects' results of the same paradigm (hence,
cross-subject-within-paradigm predictions).
%
To do this, we projected the masked empirical $Z$-maps of each paradigm and each
subject were projected from their native voxel space into the MNI space via
previously computed subject-specific transformation matrices
\citep[][\href{https://github.com/psychoinformatics-de/studyforrest-data-templatetransforms}{\url{github.com/psychoinformatics-de/studyforrest-data-templatetransforms}}]{hanke2014audiomovie}
% from MNI into subject
We then used the transpose of the transformation matrix to project the data from
the MNI space into the test subject's voxel space.
% take the mean
Similar to our functional alignment procedure, we obtained an estimation of the
test subject's empirical $Z$-maps by taking the arithmetic mean of the projected
$Z$-maps.



\subsection{Cronbach's alpha}

\todo[inline]{I have no clue where \citet{jiahui2020predicting, jiahui2022cross}
have the statement about what Cronbach's expresses from...}

\todo[inline]{shift results stated here to actual result section? In particular,
in case a t-test involving Cronbach's is calculated (which is not the case so
far)?}



%
We calculated Cronbach's $\alpha$ as a measure of reliability and the amount of
measurement error \citep{cronbach1951coefficient, cortina1993coefficient}
present in the empirical $Z$-maps of each paradigm and subject.
%
Cronbach's $\alpha$ expresses the expected correlation between the currently
used empirical $Z$-maps and an additional set of empirical $Z$-maps calculated
based on data of a hypothetical independent dataset collected from the same
paradigm and subjects \citep{jiahui2020predicting, jiahui2022cross}.
%
These expected correlations, represented by Cronbach's $\alpha$, were calculated
based on the first-level \ac{glm} $Z$-maps (four in case of the visual
localizer; eight in case of the naturalistic stimuli) that were averaged in the
second-level \ac{glm} analyses of \citet{sengupta2016extension} and
\citet{haeusler2022processing}.
%
Cronbach's $\alpha$ of empirical (i.e. second-level) $Z$-maps for each subject
and paradigm can be seen in Fig.~\ref{fig:cronbachs}, descriptive statistics
across subjects for each paradigm can be seen in Table~\ref{fig:cronbachs}.

%Visual localizer:  mean=0.899990, std=0.087051, min=0.658523,
%25\%=0.906643, 50\%=0.934019, 75\%=0.947906, max=0.963065.
%
%Movie: mean=0.611332, std=0.137878, min=0.284440,
%25\%=0.555529, 50\%=0.627240, 75\%=0.676353, max=0.800254.
%
%Audio-description: mean=0.476194, std=0.358019, min=-0.526626,
%25\%=0.428975, 50\%=0.627799, 75\%=0.679987, max=0.823584.

\todo[inline]{movie's outlier: sub-06 (0.28); but when movie PPA is predicted he
/ she is not the outlier}

\todo[inline]{audio-description's outlier: sub-05 (\textbf{-0.5!!}), sub-02
(0.0), sub-20 (0.27); when audio PPA is predicted, these three subjects are the
outliers}

\todo[inline]{imo, descriptive statistics given in a text are not necessary;
i.e. plot and, maybe, the additional table, are better}

The Cronbach's $\alpha$ values for each paradigm are as follows:
%
for the visual localizer,
%
the mean is 0.899990 with a standard deviation of 0.087051.
%
The minimum value is 0.658523, the 25th percentile is 0.906643, the median is
0.934019, the 75th percentile is 0.947906, and the maximum is 0.963065.
%
For the movie paradigm,
%
the mean is 0.611332 with a standard deviation of 0.137878.
%
The minimum value is 0.284440, the 25th percentile is 0.555529, the median is
0.627240, the 75th percentile is 0.676353, and the maximum is 0.800254.
%
Finally, for the audio-description paradigm, the mean is 0.476194 with a
standard deviation of 0.358019.
%
The minimum value is -0.526626, the 25th percentile is 0.428975, the median is
0.627799, the 75th percentile is 0.679987, and the maximum is 0.823584.


\begin{figure*}[tbp] \centering
    \includegraphics[width=\linewidth]{figures/plot_cronbachs.pdf}
    \caption{\textbf{Cronbach's $\alpha$ of the empirical $Z$-maps for each
    paradigm and subject.}
    %
    Cronbach's $\alpha$ was calculated based on the $Z$-maps yielded by the
    first-level \ac{glm} analyses of the visual localizer
    \citep{sengupta2016extension} (four runs) and naturalistic stimuli
    \citep{haeusler2022processing} (eight segments each) respectively.
    %
    The second-level \ac{glm} analyses across runs / segments yielded the
    empirical $Z$-maps that were estimated in the present study.
    }
    \label{fig:cronbachs}
\end{figure*}


\begin{table*}[btp]
\centering
    \caption{
    %
    \textbf{Descriptive statistics of Cronbach's $\alpha$ across subjects.}
    %
    Cronbach's $\alpha$ of the empirical $Z$-maps that are the result of the
    second-level \ac{glm} analyses performed in
    \citet{sengupta2016extension} and \citet{haeusler2022processing}. Values of
    Cronbach's $\alpha$ were calculated based on the first-level $Z$-maps (four
    in case of the visual localizer; eight in case of the naturalistic stimuli)}
\label{tab:cronbachs}
\begin{tabular}{llll}
    \toprule
    \textbf{statistic} & \textbf{localizer} & \textbf{movie} & \textbf{audio-description} \\
    \midrule
    mean & 0.89999 & 0.611332 & 0.476194 \tabularnewline
    std & 0.087051 & 0.137878 & 0.358019 \tabularnewline
    min & 0.658523 & 0.28444 & -0.526626 \tabularnewline
    25\% & 0.906643 & 0.555529 & 0.428975 \tabularnewline
    50\% & 0.934019 & 0.62724 & 0.627799 \tabularnewline
    75\% & 0.947906 & 0.676353 & 0.679987 \tabularnewline
    max & 0.963065 & 0.800254 & 0.823584 \tabularnewline
    \bottomrule
\end{tabular}
     % \caption*{Legend would be here}
\end{table*}



\section{Results}


\begin{comment}

``Because the localizer task comprises several scanning runs, we calculated the
reliability of the localizer across runs with Cronbach's $\alpha$ to provide an
estimate of the noise ceiling for these correlations'' \citep{jiahui2022cross}.

\end{comment}


\todo[inline]{report descriptive statistics of the (un)transformed
correlations?}
%
%The mean Pearson correlation values [not yet Fisher transformed] between
%empirical $Z$-maps and predicted $Z$-maps via anatomical alignment were 0.xx
%($N=14$, $\overline{X}=0.xx$, $SD=0.xx$, range [?], median
% [9], 25\%, 50\%, 75\%).

\todo[inline]{I tested all samples of Fisher transformed correlations for
normality via Shapiro-Wilk test (imo, the most appropriate test here)}

\todo[inline]{of all samples that were part of one of the eleven t-test, just
one sample (4 runs of localizer in order to estimate localizer) accepted H1
(i.e. the sample is NOT drawn from normal distribution); compute Wilcoxon
signed-rank test (which is for dependent samples)?}

\todo[inline]{t-test assumes equal variances which is probably often not
fulfilled; alternative: two sample permutation test?}

\todo[inline]{correct the alpha-level (as it is done now) or the p-values?}

\todo[inline]{give one-sentence interpretations à la "Results suggest..."? imo,
that's not good}

\todo[inline]{so far, I did not test any difference(s) to Cronbach's $\alpha$;
s.  comments below where it would makes sense to test difference to Cronbach's}

\todo[inline]{round the values or just write $p$<.0001}

The unthresholded $Z$-maps in their respective subject's voxel space can be
accessed at
\href{https://identifiers.org/neurovault.collection:12340}{\url{neurovault.org/collections/12340}}.
\todo{still empty}
%
We calculated the correlation between each individual's empirical $Z$-maps
obtained from previous analyses \citep{haeusler2022processing,
sengupta2016extension} and their respective predicted $Z$-maps (cf.
Fig.~\ref{fig:stripplot}) to assess the performance of the alignment procedures.
%
In general, the mean Pearson correlation coefficients vary depending on the
criterion being estimated (i.e. $Z$-maps of the visual localizer, movie, or
audio-description), as well as the functional alignment procedure (anatomical
vs. functional alignment).
%
In the case of functional alignment, the quantity of the paradigm's data used as
a preditor and to align a test subject to the \ac{cfs} also affects the
correlation coefficients, as shown in Fig.~\ref{fig:stripplot}.
%
However, the functional alignment procedure consistently shows an increasing
estimation performance across criteria as more data of a predictor is used to
align the test subjects.
%
In order to investigate potential differences between some conditions, we
conducted several pairwise comparisons using Fisher z-transformed correlation
values.
%
These comparisons were not pre-planned, but rather were selected later as
examples for further exploration.
%
We used a Bonferroni correction to adjust the alpha level to $\alpha$ of $0.05 /
11 = 0.00\overline{45}$ for multiple comparisons.


\begin{figure*}[tbp] \centering
    \includegraphics[width=\linewidth]{figures/plot_corr-emp-vs-estimation.pdf}
    \caption{
    %
    \textbf{Correlations between empirical and predicted
    \textit{\textbf{Z}}-maps for each paradigm and subject.}
    %
    Functional alignment was performed based on an increasing amount of
    functional data used to align a test subject to the common functional space
    (CFS): runs of the visual localizer paradigm lasted \unit[5.2]{min}; segments
    of the naturalistic stimuli lasted $\approx$\unit[15]{min}.
    %
    Solid horizontal lines:
    %
    median of Cronbach's $\alpha$ across subjects for empirical $Z$-maps of the
    respectively estimated paradigm (cf. Fig.~\ref{fig:cronbachs}).
    %
    Dotted horizontal lines:
    %
    mean of Cronbach's $\alpha$ across subjects for empirical $Z$-maps of the
    respectively estimated paradigm (cf. Fig.~\ref{fig:cronbachs});
    %
    Grey dots:
    %
    correlations between empirical $Z$-maps and an estimation using anatomical
    alignment.
    %
    A left-out subject's $Z$-map was estimated by projecting the training
    subjects' $Z$-maps ($N=13$) from their respective voxel space through the
    MNI152 space into the test subject's voxel space, then averaging the values
    across projected $Z$-maps;
    %
    Green dots:
    %
    correlations between empirical $Z$-map and an estimation using functional
    alignment based an transformation matrices calculated from one up to four
    runs (each lasting \unit[5.2]{min}) of the visual localizer.
    %
    Red dots:
    %
    correlations between empirical $Z$-map and an estimation using functional
    alignment based an transformation matrices calculated from one up to eight
    segments (each lasting $\approx$\unit[15]{min}) of the movie.
    %
    Blue dots:
    %
    correlations between empirical $Z$-map and an estimation using functional
    alignment based an transformation matrices calculated from one up to eight
    segments (each lasting $\approx$\unit[15]{min}) of the audio-description.
    %
    \textbf{to do: keep both, median and mean of Cronbach's? Improve legend
    somehow without crowding figure / subplots?}
   }
    \label{fig:stripplot}
\end{figure*}




\paragraph{When estimating the $Z$-maps of the visual localizer,...}
%
When estimating the $Z$-maps of the visual localizer, the correlations between
empirical $Z$-maps and $Z$-maps predicted via the first movie segment were
significantly higher than the correlations between empirical Z-maps and Z-maps
predicted via anatomical alignment (Fisher z-transformed, t(14)=6.3525802,
$p$=0.0000253).
%
However, the functional alignment based on data of the visual localizer
(within-paradigm prediction) and audio-description was lower than anatomical
alignment, independent of the number of runs/segments used to align the test
subjects to the \ac{cfs}. This was evident from visual inspection.
%
The functional alignment via the first movie segment (451 \acp{tr}) yielded
significantly higher correlations than functional alignment via four localizer
runs lasting 624 \acp{tr} (Fisher z-transformed, t(14)=5.8905545,
$p$<0.0000532),
%
The mean correlation between empirical $Z$-maps and $Z$-maps predicted via the
first movie segment were significantly lower than the correlations between
empirical $Z$-maps and $Z$-maps predicted via the first two movie segments
(Fisher z-transformed, t(14)= -5.4946197, $p$=0.0001031).
%
However, the difference between mean correlations based on two movie segments
compared to the mean correlation based on three movie segments was not
significant (Fisher z-transformed, t(14)= -0.1293547, $p$=0.8990569).
%
The prediction performance based on data of the audio-description increased with
more data, as evident by visual inspection.
%
However, when comparing a similar amount of \acp{tr}, the functional alignment
via the first segment of the audio-description yielded significantly lower
correlations than the functional alignment via four localizer runs (Fisher
z-transformed, t(14)=-2.3000009, $p$<0.0386588). \todo{only uncorrected}

\todo[inline]{possibly run the test that compares 8 movie segments to Cronbach's
alpha of localizer? a.k.a. 8 runs of movie for functional alignment are "not as
good" as Cronbach's alpha}


\paragraph{When estimating the $Z$-maps of the movie,...}
%
When estimating the $Z$-maps of the movie, the correlations between empirical
$Z$-maps and predicted $Z$-maps via the first movie segment were significantly
higher than the correlations between empirical $Z$-maps and predicted $Z$-maps
via anatomical alignment (Fisher z-transformed, t(14)= 5.7754451,
$p$=0.0000643).
%
The functional alignment via the first movie segment (451 \acp{tr}) yielded
significantly higher correlation than a functional alignment via four localizer
runs lasting 624 \acp{tr} (Fisher z-transformed, t(14)=6.8532349,
$p$<0.0000116).
%
The mean correlation between empirical $Z$-maps and predicted $Z$-maps via the
first segment alone of the movie were significantly lower than the correlations
between empirical $Z$-maps and predicted $Z$-maps predicted $Z$-maps via the
first two movie segments (Fisher z-transformed, t(14)= -3.7454592,
$p$=0.0024485).
%
The difference between mean correlations based on two movie segments in
comparison the mean correlation based on three movie segments was [not]
different [when adjusted alpha-level used] (Fisher z-transformed, t(14)=
-2.5759899, $p$=0.02303).
%
The prediction performance based on data of the audio-description increased with
more data as evident by visual inspection.
%
However, when comparing a similar amount of \acp{tr}, the functional alignment
via the first segment of the audio-description yielded no significant difference
in correlations compared to functional alignment via four localizer runs (Fisher
z-transformed, t(14)=-1.8674144, $p$<0.084551).

\todo[inline]{possibly run the test that compares x movie segments for alignment
to Cronbach's alpha of movie? a.k.a functional alignment will "sooner or later
be better" than Cronbach.}


\paragraph{When estimating the Z-maps of the audio-description,...}

\todo[inline]{This part is the ugly one; all samples will probably reject H0
assuming normality}

\todo[inline]{the three outliers are always sub-02, sub-05, sub-20
(always=anatomical alignment, 4 runs of localizer, 8 runs of movie, 8 runs of
audio-description)}

\todo[inline]{run a test that compares <whatever> to Cronbach's a? for example:
audio-description for alignment will sooner or later be significantly higher
than Cronbach's; might be a cue that SRM does a denoising of the e.g., map (and
does not estimate "reliable patterns deviant from the norm" incorrectly}

%
When estimating the Z-maps of the audio-description, correlations between
empirical $Z$-maps and predicted $Z$-maps via the first movie segment were
significantly lower than the correlations between empirical $Z$-maps and
predicted $Z$-maps via anatomical alignment (Fisher z-transformed, t(14)=
-4.2004329, $p$=0.0010387, Bonferroni corrected).









\pagebreak


\section{Discussion}

\todo[inline]{generally, I think it's best to abstract A LOT from the specific
results (other studies discuss general stuff only, too; fucking joke, anyway)}

\todo[inline]{Two possibilities to structure the discussion:}

\todo[inline]{1. discuss concrete examples (a.k.a. the tests) \& draw the
general inference (how it is done now), or:}

\todo[inline]{2. write general shit \& give some concrete examples (a.k.a.
statistical tests) to make the point (which would correspond to the structure of
results section}

\todo[inline]{Delete captions of subsubsections}

\subsection{Introductory paragraph}

%
We investigated the performance of a functional alignment procedure that we used
to predict results of $t$-contrasts (i.e. the criteria) that were created in
previous studies \citep{sengupta2016extension, haeusler2022processing} in order
to localize the \ac{ppa}.
%
Following a leave-one-subject-out cross-validation, we fit a shared response
model \citep{chen2015reduced} to the training subjects' concatenated response
time series of three different paradigms in order to create a multi-paradigm
\ac{cfs} and the training subjects' subject-specific transformation matrices.
%
In order to acquire the test subject's transformation matrix, we used an
increasing amount of the test subject's response time series of the three
paradigms (i.e. the predictors)  separately by functionally aligning the test
subject to the corresponding \acp{tr} of the shared feature space (i.e.
\ac{cfs}).
%
The empirical $Z$-maps of each training subject were projected from their
respective voxel space through the \ac{cfs} into the test subject's voxel space
to yield the test subject's predicted $Z$-maps.
%
In case of the anatomical alignment approach that served as a baseline, the
training subjects' $Z$-maps were projected via nonlinear, volume-based
transformation through the MNI152 brain atlas into the test subject's voxel
space.


\subsubsection{Four aspects under investigation}

\todo[inline]{this needs to be in line with paragraph "In contrast to previous
studies..." in intro}

\todo[inline]{biggest issue: how to separate validity / generalizability of
\ac{cfs} from validity / generalizability of transformation matrices?}

\todo[inline]{in any case: within-paradigm prediction is more about validity of
model? cross-paradigm prediction is more generalizability of matrices?}
%
First, we assess the \textbf{validity of the \ac{cfs}} by estimating $Z$-maps
from the analysis (i.e. $t$-contrast) of same paradigms used to align the test
subject (cross-subject-\textbf{within-paradigm} prediction).
%
Second, we assess the \textbf{generalizability of the transformation matrices}
by estimating $Z$-maps from the analysis (i.e. $t$-contrast) of paradigms that
were not used to align the test subject (cross-subject-\textbf{cross-paradigm}
prediction).
%
Third, we explore the relationship between the estimation performance of the
t-contrast's results and the quantity of data from each of the three paradigms
used to functionally align the subject with the \ac{cfs}.
%
Last, we compared the performance of our volume-based, functional alignment
procedures to that of a volume-based, anatomical alignment approach, which
served as a benchmark.


\subsubsection{Results + Conclusion}
%
We found that using a partial alignment approach with a multi-paradigm \ac{cfs}
derived from concatenated time series of multiple paradigms is effective.
%
The prediction performance improves as more data from the paradigms used to
align a test subject included.
%
However, one movie segment outperformed an estimation using anatomical
alignment.
%
This opens up the possibility of estimating results from many localizer
paradigms using a naturalistic stimulus of similar duration of one traditional
localizer paradigm.



\subsection{Anatomical alignment vs. functional alignment}

\todo[inline]{it is actually not that bad}

\todo[inline]{it's consistent across paradigms, too (always a median correlation
of about 0.6)}

%
When estimating the visual paradigms (localizer and movie PPA), using just one
movie segment with partial alignment outperformed anatomical alignment.
%
When estimating audio PPA, as more movie data were included, the prediction
performance of the multi-paradigm CFS improved the more data of the movie are
included, and eventually became as good as anatomical alignment
%
However, using just one movie segment resulted in worse performance than
anatomical alignment.

%
In contrast, localizer runs were not effective for estimating the localizer
$Z$-maps for within-paradigm prediction and movie PPA for cross-paradigm
prediction compared to anatomical alignment.
%
Surprisingly, when estimating audio PPA (cross-paradigm), the localizer for
alignment came close to anatomical alignment, indicating that audio PPA is the
criterion that the visual localizer can estimate the best.


The performance of audio-description is inferior to anatomical alignment for
cross-paradigm estimation.
%
However, it improves when more segments are used for alignment.
%
In cross-paradigm prediction, the gap between movie and audio-description
narrows as more data of the audio-description is used.
%
In within-paradigm prediction, audio-description eventually outperforms
anatomical alignment when more than 4-6 segments are used.
%
Nevertheless, for every alignment procedure (both functional and anatomical), it
is difficult to estimate the outliers of Cronbach's $\alpha$ of audio PPA.



\subsection{Functional alignment}




\subsubsection{Movie for functional alignment}

\todo[inline]{a.k.a. movie is best (which is in line with previous research)}

\todo[inline]{paragraph focuses on estimation of localizer though}

%
When estimating the visual localizer, results indicate that $\approx$15 minutes
(\ac{tr}=2s) of movie watching used for functional alignment outperform
prediction using anatomical alignment.
%
Prediction performance further increases when $\approx$30 minutes of movie data
submitted to the algorithm to calculate the subject-specific transformation
matrices.
%
However, three segments ($\approx$45 minutes) do not lead to an significantly
increased estimation performance suggesting a decreasing benefit of longer
scanning time than 30 minutes during audio-visual naturalistic stimulation.



\subsubsection{Visual localizer for functional alignment}

\todo[inline]{a.k.a. why is it bad? which is not in line with previous research}

\todo[inline]{is it our model or the transformation matrices?}

%
Unexpectedly, the prediction of the localizer using localizer runs (i.e.,
within-experiment prediction) was worse than anatomical alignment.
%
It is possible that our naturalistic stimuli (about 7000 TRs) may have
"disrupted" our shared responses during the 450 \acp{tr} within the SRM?
%
However, the correlations of regressors with shared responses (cf.
Fig.~\ref{fig:corr-vis-reg-srm}) indicate that the shared responses show the
highest correlation with regressors of the localizer.
%
Therefore, the matrices based on the localizer runs are less reliable for
estimation compared to anatomical alignment?



\subsubsection{Audio-description for functional alignment}

\todo[inline]{a.k.a. it's totally different from different previous research}

%
Using the audio-description for functional alignment to estimate visual
paradigms/cross-modal by using data from a non-visual paradigm is a challenging
task.
%
However, the estimation performance increases with the availability of
more data.
%
One possible explanation could be that the audio-description lacks visual
stimulation and is not as rich as the movie.
%
Another explanation is that the response in the PPA to a auditory paradigm is
(too) different from a response in the PPA to a visual paradigm.
%
Nevertheless, the data collected during listening to an audio-description, which
is richer than a mere narrative, could potentially be used to estimate a visual
category-selective area. However, the current approach requires impractical
amounts of data.
%
It would be interesting to estimate the results from a controlled speech
paradigm, i.e., another same-modality criterion.


\paragraph{audio PPA as criterion: the issue of reliability}

\todo[inline]{also true (but less severe) in case of movie PPA}

\todo[inline]{needs to be in line with general discussion's pros \& cons of
naturalistic stimuli}

%
When estimating audio PPA, the participants with poor Cronbach's $\alpha$ have
the lowest correlation between predicted and empirical $Z$-maps.
%
When estimating audio PPA, participants with poor Cronbach's $\alpha$ have the
lowest correlation between predicted and empirical $Z$-maps.
%
These participants are not "reliable outliers deviant from the norm" but may be
considered as unreliable outliers due to their noisy data.
%
Therefore, the SRM may be performing some level of denoising instead of modeling
these unreliable outliers incorrectly.

%
Was the construct "perception of auditory spatial information" modeled
insufficiently in \citet{haeusler2022processing}?
%
The primary audio-description contrast yielded bilateral clusters in nine
participants that overlapped with the block-design localizer results, and one
cluster in the left-hemispheric PPA for one participant (sub-09). Thus, it seems
that the construct was sufficiently modeled for more than 50\%.
%
The fact that it worked for most subjects suggests that the issue is not a
methodological decision and that the audio-description does sample the responses
to spatial information.
%
Rather, it is more likely an issue with the naturalistic paradigm, which allows
for more individual variation over the course of the two-hour long stimulation,
possibly influenced by an unbalanced number of events per stimulus segment.
%
Although the shared responses show the highest correlation with the regressors
of the visual localizer, the plots of shared responses and regressors of the
audio-description (cf. Fig.~\ref{fig:corr-ao-reg-srm}) suggest that the shared
responses during TRs of the audio-description are not entirely noise but that
the shared features during TRs of the movie and audio-description seem to be
more abstract.
%
Final point: given that the within-paradigm prediction works well for most
subjects, outliers of Cronbach's $\alpha$ do not significantly degrade the
model.

%
Perhaps the large individual variability observed in some subjects during the
audio-description paradigm is due to the paradigm itself, which allows for more
variation over the two-hour long stimulation.
%
Another possibility is that the 2-hour long auditory stimulation is not as
immersive or engaging as the multi-modal audio-visual movie.
%
A potential challenge in controlling for these factors is the difficulty in
judging or controlling attentional focus compared to eye-tracking during a
movie.
%
However, it may be possible to measure alertness using EEG recordings.
%
Alternatively, it could be related to the auditory domain in general, although
this is difficult to determine precisely and needs a dedicated paradigm, imo.


\paragraph{Just a bold inference}

\todo[inline]{add text about SRM as denoising technique?}
%
The current results suggest that the processing of auditory spatial information
is reliably engaged across subjects during the audio-description and in such a
way that the ``SRM will improve sensitivity for detecting a cognitive process of
interest in the test data'' \citep{cohen2017computational}.

%
Present results support evidence that results in \citet{haeusler2022processing}
that are restricted to the anterior part of the localizer PPA are not based on
the methodological decisions, and that the responses to auditory spatial
information lead to different activity patterns than visual stimulation.
%
Further studies using controlled paradigms are needed to investigate auditory
spatial information and studies using auditory narratives should employ more
sophisticated models of event structure in order to assess the suitability of
naturalistic stimuli as a "true naturalistic localizer".
%
It is important to note that the use of naturalistic stimuli in research can
provide more ecologically valid results, but it is also important to consider
potential confounds and limitations leading to problems we have to struggle with
here when trying to interpret the results.




\subsubsection{Interim summary (on functional alignment)}

\todo[inline]{how to separate validity \& generalizability of \ac{cfs} from
validity \& generalizability of matrices?}

\todo[inline]{if \citet{haxby2011common} is discussed (a lot), it should
probably be primed in the intro (right before "Here, we...")}

\todo[inline]{problem: \citet{feilong2022individualized} already assessed data
quantity; results suggest 30m are "good" in case of their model; plus, they
model individual component.}

%
Our results provide evidence that transformation matrices calculated based on
data from naturalistic stimuli promise an increased validity and [or?]
generalizability for functional alignment over transformation matrices based on
data of a controlled paradigm based on simplified stimuli.
%
This is the case for both within-paradigm prediction (e.g., audio-description
for alignment to estimate $Z$-maps from the audio-description's $t$-contrast)
and cross-paradigm prediction (e.g. audio-description for alignment to estimate
$Z$-maps from the visual localizer's $t$-contrast).
%
Our results are different results of \citet{haxby2011common} who found that the
prediction of $Z$-maps from a controlled paradigm via matrices \& \ac{cfs} based
on the same controlled paradigm was as good or better than prediction (of the
same) $Z$-maps via matrices \& \ac{cfs} based on movie data.
%
In summary, the multi-stimulus \ac{cfs} generalizes over paradigms to be
estimated but performance depends on the paradigm used to align the test subject
to the \ac{cfs}.



\subsection{Vision}

% A shared calibration scan across datasets could be used to transfer data
% between datasets, a procedure that is easier to accomplish than shared
% subjects across datasets \citep[cf.][an extension of the \ac{srm} for shared
% subjects across datasets]{zhang2018transfer}.

\todo[inline]{needs to be in line with general discussion's "Vision" (which
is commented out at the moment)}

\todo[inline]{Maybe, merge part here into general discussion? the part here is
not long anyway}


\subsubsection{Intro}

\todo[inline]{maybe, repeat phrasing from introduction (cf. localizer
paradigms)}

%
Our results suggest that 15 minutes functional scanning using an engaging
naturalistic stimulus could provide sufficient data for a \textit{calibration
scan}.
%
A standardized calibration scan could be used to align a new subject to a
\ac{cfs} that was derived from extensive scans of a reference group.
%
The reference group's scans would include data collected from both naturalistic
paradigms and controlled paradigms.
%
The controlled paradigms could include functional localizers that are
specifically designed to reliably map brain processes involved in, for example,
low-level auditory or visual perception, as well as higher cognition, such as
theory of mind or \citep{spunt2014validating} or semantic processes
\citep{fedorenko2010new, fernandez2001language}.
%
The diagnostic run would be based on a multifaceted naturalistic stimulus that
samples a broad range of brain states in order to allow for a valid alignment to
the reference \ac{cfs}.
%
Compared to a diagnostic run based on a controlled paradigm, a naturalistic
stimulus would have the additional benefits of higher engagement and better
compliance \citep{vanderwal2015inscapes, eickhoff2020towards}, especially in
children or patients.


\paragraph{Application: estimate \& quantify regular vs. deviant pattern}

%
Once a new subject is aligned to the \ac{cfs}, the functional patterns collected
in a reference group could be mapped through the \ac{cfs} into the new subject's
voxel space.
%
This would allow for the estimation of regular patterns in the new subject when
additional functional scans are not feasible due to scanner availability, time
limitations, monetary constraints, or compliance issues.
%
Moreover, this approach would allow for the quantification of the similarity (or
difference) of a new subject's actual pattern (i.e., an empirical $Z$-map) to a
pattern estimated from a healthy or clinical reference group.


\todo[inline]{essentially quoting the abstract of \citet{yates2021emergence}:}
%
For instance, a recent study by \citet{yates2021emergence} ``employed shared
response modeling to investigate the presence and localization of adult
functions in children.
%
The authors learned a feature space from fMRI activity of adults watching a
movie and then translated the shared features into the anatomical brain space of
children aged 3 to 12 years old.
%
They found reliable correlations between reconstructed activity and children's
actual fMRI activity as they watched the same movie.
%
The strength of the correlation in the precuneus, inferior frontal gyrus, and
lateral occipital cortex predicted chronological age.
%
This approach demonstrates the potential to generalize findings across different
populations and ages, further highlighting the importance of establishing a
reliable and accurate \ac{cfs}''.



\subsection{Future questions}

\todo[inline]{just some ideas...}


\subsubsection{Other ROIs than category-selective areas}

\todo[inline]{topic for general discussion: limit of naturalistic stimuli
(passively consumed)?}


\begin{comment}

\paragraph{Results of \citet{frost2012measuring}}
%
``We localized 13 widely studied functional areas and found a large variability
in the degree to which functional areas respect macro-anatomical boundaries
across the cortex.
%
The percent gain in overlap [after surface-based alignment] differed greatly
across the different functional regions throughout the cortex.
%
There is a strong structural-functional correspondence in some areas whilst in
others the spatial location of the functional area is not tightly bound to
anatomical landmarks and varies greatly across subjects within a cortical area''
\citep{frost2012measuring}.
%
Language areas were found to vary greatly across subjects whilst a high degree
of overlap was observed in sensory and motor areas.
%
PPA gained 17.6\% in the left and 27.7\% in the right hemisphere.
%
The FFA gained 44.1\% in the left and 12\% in the right hemisphere.
%
The FFA did not exhibit the same strong structural-functional correspondence and
saw more modest increases in overlap after macro-anatomical alignment.
%
FFA varies in its location along the length of the fusiform gyrus even though
the gyri themselves are well aligned across subjects.
%
LOC gained 62.7\% in the left and 38.4\% in the right hemisphere''
\citep{frost2012measuring}.

% category-specific areas
Similar to nonlinear volume-based alignment, similarity across person is higher
after surface-based alignment ``for retinotopically defined regions, with
character-selective regions showing the lowest consistency for both alignments,
closely followed by mFus- and IOG-faces'' \citep{rosenke2021probabilistic}.

\end{comment}


%
In the present study, we focused on the \ac{ppa} as a classic example of a
higher-visual area.
%
It is important to note that the alignment procedure may perform differently
when estimating other functional areas.
% varying \textit{functional--anatomical correspondence}
Previous studies have shown that there is a varying degree of correspondence
between a brain function and its underlying anatomical location.
%
Previous studies that used either volume-based \citep{zhen2017quantifying,
zhen2015quantifying} or surface-based alignment \citep{rosenke2021probabilistic,
frost2012measuring} to estimate the location of functional areas have ``found a
large variability in the degree to which functional areas respect
macro-anatomical boundaries across the cortex'' \citep{frost2012measuring}.
%
In some areas, there is a strong structural-functional correspondence, while in
others, the spatial location of the functional area varies greatly across
subjects within a cortical area \citep{frost2012measuring}
%
For example, even within in the domain of category-selective areas, the
interindividual variability varies across functional areas
\citep{zhen2017quantifying, zhen2015quantifying, frost2012measuring}:
%
For instance, in the domain of category-selective areas, interindividual
variability varies across functional areas , with scene-selective regions
showing larger variability in spatial topography compared to face-selective
regions \citep{zhen2017quantifying, frost2012measuring}.


\paragraph{Inference}

\todo[inline]{cite a review that summarizes research the range of brain studied
using naturalistic stimuli}

%
Therefore, future studies could investigate the extent to which functional
alignment leads to an increased estimation performance compared to anatomical
alignment, especially in areas where the functional-anatomical correspondence is
known to vary across subjects.
%
Naturalistic stimuli have been used to investigate a variety of domains like XYZ
[check for, e.g., visual or auditory perception, spatial cognition; emotion;
music, speech or social perception], and possibly allow valid mapping of
functional data in all of these domains.


\todo[inline]{Paraphrase:}
%
``There are numerous other functional localizers in other perceptual and
cognitive domains, such as simple visual motion, biological motion, tonotopy,
voice perception, music perception, language, calculations, working memory, and
theory of mind.
%
Because naturalistic movies include people, human actions, conversations, social
interactions, background music etc., we predict functional alignment based on
movies also will work for localizers of functional topographies for audition,
language, and social cognition.
%
Some high-level cognitive processes, such as calculation, working memory, and
logical reasoning, may be less well sampled by movie viewing, and further work
is necessary to test whether hyperalignment based on movie-viewing data can be
used to estimate topographies for these other domains of high-level cognition.
%
Researchers may need to select or develop movies that involve more calculations
or logical inference-making to afford accurate estimation of topographies in
these domains'' \citep{jiahui2020predicting}.


\paragraph{Studyforrest dataset's other localizer t-contrasts}

\todo[inline]{it is such a low-hanging fruit; better drop it?}

For instance, the visual localizer of the studyforrest dataset also contains
contrast that aimed at localizing the \ac{ffa} and \ac{ofa} that are associated
with face perception \citep{kanwisher1997ffa, pitcher2011occipitalfacearea}, the
\ac{eba} that is associated with the perception of human bodies
\citep{downing2001bodyarea}, and the \ac{loc} that is associated with the
perception of (small) objects (like tools or toys) \citep{malach1995loc},
%
The subject-specific \acp{roi} masks for these areas that where created by
\citep{sengupta2016extension} and our analysis pipeline we used in
\citet{haeusler2022processing} provide an opportunity for future studies (e.g.,
a master's thesis or part of a PhD project) to explore the hemodynamic responses
that correlate with auditory information related to faces, body parts, or small
objects.



\subsubsection{Other functional alignment algorithms}

\todo[inline]{a big treatise on functional alignment algorithms should
definitively be avoided}


\paragraph{SRM}



The SRM algorithm used in this study has a drawback in that it only models
responses that are common across individuals, without including an individual
component.
%
Although a person's idiosyncratic responses are excluded from the shared
response model, they ``are not necessarily noise and may in fact be highly
reliable within participants'' \citep{cohen2017computational}.
%
Based on the results of this study, it appears that we estimate a regular
pattern in noisy data of a couple of participants  (i.e., we perform an
\ac{srm}-based denoising).
%
However, to predict a reliably deviant pattern, it is necessary to use a
matching CFS from a deviant reference, or preferably, a model that incorporates
a shared response, an individual component, and noise.
%
Future studies should assess the performance of various functional alignment
algorithms in estimating reliably deviant patterns, e.g. atypical topographies
of language areas.
%
Does the algorithm estimate a regular pattern, which allows for quantifying the
difference of a deviant actual pattern from a norm, or does it directly estimate
the deviant pattern from a regular reference?

\todo[inline]{if this topic is supposed to be discussed, check outsourced
templates for atypical language topographies (in clinical populations)}

\todo[inline]{at least two of \citet{feilong2022individualized, jiahui2022cross,
turek2018capturing} have modeled an individual component.}




% citep{cohen2017computational} on SRM and individual residuals
\begin{comment}

\paragraph{SRM and individual differences}
%
``SRM can be used to isolate participant-unique responses by examining the
residuals after removing shared group responses, or it can be applied
hierarchically to the residuals to identify subgroups [\citet{chen2017shared}]
'' \citep{cohen2017computational}.
%
``Recognizing that signal exists beyond the average or shared response of a
group, such studies exploit idiosyncratic but stable responses to account for
previously unexplained variance in brain function, behavioral performance and
clinical measures [e.g., Finn (2015). Functional fingerprinting (based on
connectivity)]'' \citep{cohen2017computational}.
%
``In cases where each subject's unique response is of more interest than the
shared signal, SRM can be used to factor out the shared component thereby
isolating the idiosyncratic response for each subject
[\citep{chen2015reduced}]'' \citep{kumar2020brainiak}.

\end{comment}




\subsubsection{ROI vs. whole-brain}

\todo[inline]{searchlight SRM: \citet{zhang2016searchlight}; negative:
more (hyper) parameters to vary}

``Searchlight functional alignment \citep{zhang2016searchlight,
guntupalli2016model} learns local transformations and aggregates them into a
single large-scale alignment.
%
The searchlight scheme [Kriegeskorte, 2006, Information-based...] , popular in
brain imaging \citep{guntupalli2018computational, guntupalli2016model} has been
used as a way to divide the cortex into small overlapping spheres of a field
radius.
%
This method allows researchers to remain agnostic as to the location of
functional or anatomical boundaries, such as those suggested by
parcellation-based approaches.
%
A local transform can then be learned in each sphere and the full alignment is
obtained by aggregating (e.g. summing as in \citep{guntupalli2016model} or
averaging) across overlapping transforms.
%
The aggregated transformation produced is no longer guaranteed to bear the type
of regularity (e.g orthogonality, isometry, or diffeomorphicity enforced during
the local neighborhood fit)..
%
``In the case of searchlight Procrustes, we selected searchlight parameters to
match those used in Guntupalli et al. (2016):
%
each searchlight had 5 voxel radius, with a 3 voxel distance between searchlight
centers'' \citep{bazeille2021empirical}.


\subsubsection{Volume- vs surface-based}

\todo[inline]{An "issue" for reviewers but, imo, it's not a real issues}

\todo[inline]{opportunity costs: $Z$-maps of localizer were calculated in voxel
space; we are nearer on the raw data (less error accumulation) because we work
with voxels (surface vertices need additional mapping of $Z$-maps calculated
from smoothed data); probably "difficult" to adapt to subcortical structures}

We compare volume-based anatomical alignment to volume-based functional
alignment.
%
Future work could compare surface-based alignment that respects cortical folding
structure -- that out-performs predictions based on [affine] volume-based
anatomical alignment \citep{weiner2018defining} -- to surface-based functional
alignment.


\subsubsection{Time series vs connectivity-based}

\todo[inline]{\citet{jiahui2022cross} do connectivity-based 1-step
hyperalignment across different movie datasets which is as good as response
hyperalignment; however, it's a shitty procedure to scale because it needs a
transformation matrix for every pair of subjects (i.e no \ac{cfs})}

\todo[inline]{just some templates:}

%
``Derivation of this common space can be based on either neural response
profiles (e.g. data collected during tasks, such as movie viewing (Haxby et al.,
2011)) or functional connectivity profiles files
\citep{guntupalli2018computational}.
%
Response-based hyperalignment maps data from the anatomical space to a common
information space based on time-point response patterns across cortical
vertices.
%
Connectivity-based hyperalignment maps data from the anatomical space to a
common information space based on functional connectivity patterns''
\citep{busch2021hybrid}.

%
``The number of voxels that can be considered simultaneously for functional BOLD
response time series alignment is limited by the number of timepoints in the
calibration scan (about 300-400 voxels for a 15min scan with a 2s TR,
corresponding to a local cortical neighborhood of about 1cm in diameter for a
standard resolution).
%
This limitation does not exist in this form for a functional alignment that is
based on connectivity vectors.
%
The length of these connectivity vectors is determined by the number of
reference (or seed) regions in the brain'' [project proposal].


% Kumar on Nastase's ugly mofo paper
``Estimating the SRM from functional connectivity data rather than response time
series circumvents the need for a single shared stimulus across subjects.
%
Connectivity SRM allows us to derive a single shared response space across
different stimuli with a shared connectivity profile
\citep{nastase2019leveraging}'' \citep{kumar2020brainiak}.
%
``The sampling of connectivity vector space is defined by the selection of
connectivity targets, but the richness and reliability of connectivity estimates
depends on the variety of brain states over which connectivity is estimated''
\citep{haxby2020hyperalignment}.


%
``Response-based hyperalignment was shown to align response-based data better
than connectivity-based hyperalignment, whereas connectivity-based
hyperalignment was shown to better align connectivity-based data than
response-based hyperalignment
%
Response-based common spaces better align response data, whereas
connectivity-based common spaces better align connectivity data
\citep{guntupalli2018computational}'' \citep{busch2021hybrid}.


\subsection{Short comings}

\todo[inline]{Any other short comings? Most is (or can be implicitly) mentioned in "Future Questions")}

\subsubsection{Bigger sample size}
%
The SRM is computationally less demanding [= "computationally more efficient"?]
than hyperalignment, an advantage for scientists who want to replicate our
results but do not have access to a high-performance computer cluster [or
"high-throughput computer cluster" or simply "specialized hardware"?].
%
Moreover, it should scale pretty well to bigger sample sizes; compared to, for
example, \citet{jiahui2020predicting, jiahui2022cross} 1-step hyperalignment
alignment that needs pair-wise matrices for subjects (because no CFS).


\paragraph{Small sample size lead to varying \ac{cfs} across subjects}
% "unstable" CFS
The correlations of shared responses within the \acp{cfs} created from $N-1$
training subjects varied across the folds of the cross-validation.
% interpretation
That means, a change of 1/13 of the data for every subject's analysis is causing
the estimates to vary [how much?].
% conclusion
Future studies, should create a \ac{cfs} based on data from more subjects and
investigate the relationship between number of participants, variability of
parameters, and estimation performance.


\subsection{Conclusion}

God bless America!


\section{Data Availability}

\todo[inline]{all from PPA-Paper but with new GIN link leading to an empty repo}

% \href{https://gin.g-node.org/chaeusler/studyforrest-ppa-analysis}{\url{gin.g-node.org/chaeusler/studyforrest-ppa-analysis}}

% new; PPA analysis
All fMRI data and results are available as Datalad \citep{halchenko2021datalad}
datasets, published to or linked from the \emph{G-Node GIN} repository
(\href{https://gin.g-node.org/chaeusler/studyforrest-ppa-srm}{\url{gin.g-node.org/chaeusler/studyforrest-ppa-srm}}).
% original
Raw data of the audio-description, movie and visual localizer were originally
published on the \emph{OpenfMRI} portal
(\url{https://legacy.openfmri.org/dataset/ds000113}; \citep{Hanke2014ds000113},
\space \url{https://legacy.openfmri.org/dataset/ds000113d};
\citep{hanke2016ds000113d}).
% visual localizer
Results from the localization of higher visual areas are available as Datalad
datasets at \emph{GitHub}
(\href{https://github.com/psychoinformatics-de/studyforrest-data-visualrois}{\url{github.com/psychoinformatics-de/studyforrest-data-visualrois}}).
% raw data
The realigned participant-specific time series that were used in the current
analyses were derived from the raw data releases and are available as Datalad
datasets at \emph{GitHub}
(\href{https://github.com/psychoinformatics-de/studyforrest-data-aligned}{\url{github.com/psychoinformatics-de/studyforrest-data-aligned}}).
% OpenNeuro
The same data are available in a modified and merged form on OpenNeuro at
\url{https://openneuro.org/datasets/ds000113}.
% NeuroVault for z-maps of SRM
Unthresholded $Z$-maps of all contrasts can be found at
\href{https://identifiers.org/neurovault.collection:12340}{\url{neurovault.org/collections/12340}}.


\section*{Code Availability}

Scripts to generate the results as Datalad \citep{halchenko2021datalad} datasets
are available in a \emph{G-Node GIN} repository
(\href{https://gin.g-node.org/chaeusler/studyforrest-ppa-srm}{\url{gin.g-node.org/chaeusler/studyforrest-ppa-srm}}).






\pagebreak

\section{Backup of texts regarding "done but not mentioned"}


\subsection{Calculate $Z$-maps mean in the common space already}
%
I also tested averaging $Z$-maps in the \ac{cfs} (i.e.: not in the test
subject's voxel space): similar results
%
In case of anatomical alignment, I did
not test averaging data in MNI152 space.


\subsection{Calculate $Z$-map from training subjects' TRs in FEAT}

iirc, I projected all subjects' localizer time series through
model space into test subject voxel space; then, calculated the contrast
with these data (s. scripts 'test/data\_denoise-vis.py' \&
'test/data\_srm-vis-to-ind.py').
%

The problem was: if one wants to test the different transformation matrices (I
only did it with one; imo, based on alignment using the whole audio-description)
it gets totally messy \& computational intensive.
%
Results were similar to the original procedure if not slightly worse.


\subsection{Leakage of test data in union of individual \acp{ppa}}

%
We used the union of individual \acp{ppa} as spatial constrain for $Z$-maps.
%
But we have a leakage of test data (test subject is in data for the mask).
%
We might miss some voxels (of some participants) at the borders of the \ac{roi},
because the subject-specific, binary masks are based on a ("titrated")
threshold.  \citep{sengupta2016extension}
%
In the future, an independent probabilistic atlas should be used, the \ac{roi}
dilated, [and a separate model calculated for each hemisphere].






\chapter{General discussion}
%% In the [general] discussion, a critical evaluation of your own results is
%% made against the background of the literature. At this point, further
%% results that have not already been listed in the results section can also be
%% discussed.

% take input from external file
\todo[inline]{Include SRM / functional alignment in pro \& contra nat. stim.}

%
Traditionally, human brain mapping studies have averaged \ac{fmri} data across
participants.
%
To advance the field towards a clinical application, data from single persons
need to be assessed.
% functional localizer
Functional localizers are an established method to describe the topography (i.e.
the location, size, and shape) of functional areas on the level of individuals.
% contra localizers
However, traditional localizer paradigms rely on selectively sampled, tightly
controlled stimuli and participant compliance, and can usually map only one
domain of brain functions.
% movies & narratives
Naturalistic stimuli like movies and auditory narratives offer a time-locked
event structure that samples a variety of brain functions ranging from low-level
perception (e.g., luminance) to high-level cognition (e.g., social cognition).
%
A localizer based on a naturalistic stimulus could offer a higher external
validity and potentially map a variety of brain functions.
% goal of thesis
Consequently, the purpose of this thesis was---while adhering to the principles
of open, transparent, and reproducible science---to investigate whether a movie
and the movie's audio-description may, in principle, replace a traditional
localizer paradigm.
% PPA as proof of concept
As a proof of concept, we focused on the \ac{ppa}, a ``classic'' higher visual
area.
%
The \ac{ppa} exhibits increased hemodynamic activity when participants view
photos of landscapes, buildings or landmarks, compared to, for instance, photos
of faces or tools \citep[cf,][for reviews]{epstein2014neural, aminoff2013role}.
%
Moreover, results of \citet{aziz2008modulation}, who compared hemodynamic
activity levels in the \ac{ppa} correlated with different categories presented
in spoken sentences, revealed that semantic scene-related information also
modulates the \ac{ppa}'s activity level.
%
We assessed the potential of both the movie and the audio-description to replace
a visual localizer by modifying the procedure for the analysis of data from
localizer paradigms to data from the two naturalistic stimuli.


%
First, hemodynamic responses correlating with the temporal structure of
annotated stimulus features \citep[cf.][]{haeusler2016cutanno,
haeusler2021speechanno} were modeled in order to create \ac{glm} $t$-contrasts
that aimed to localize the \ac{ppa}.
% estimation
Second, we applied functional alignment procedure as a novel method in order to
estimate results from the visual localizer \citep[cf.][]{sengupta2016extension},
movie and audio-description \citep[cf.][]{haeusler2022processing} in one
participant from results of participants in a reference group.




\section{Open Science}

% intro
In recent years, there has been a growing movement towards open science, which
seeks to make scientific research more accessible, transparent, and
reproducible.
%
Open science encompasses a range of practices, including open data, open-source
software, and open access publishing.
%
The overarching objective of this dissertation was to meet the standards of
open, shared, accessible, and transparent science in general, as well as the
standards of a reproducible and replicable research project specifically
\citep[cf.][]{watson2015will, fecher2014open}.
%
This objective included using open data and open-source software, and publishing
data, materials, code, and results openly available.


\subsection{Using open data and open-source software}

%
The first subgoal was to use open data, open materials, and open-source
software.
%
To achieve this, the thesis utilized publicly available
%
\ac{fmri} data \citep{hanke2014audiomovie, hanke2016simultaneous,
sengupta2016extension},
%
subject-specific \acp{roi} \citep{sengupta2016extension} and
%
stimulus annotations \citep{haeusler2016cutanno}
%
that are part of the studyforrest project
(\href{www.studyforrest.org}{\url{studyforrest.org}}).
%
The analyses were implemented in freely available and, where possible,
open-source software to prevent creating an ``artificial paywall'' for running
the analyses again on the initially openly accessible data.
%
The thesis benefited from established free software such as
%
Python and
%
FSL \citep[\href{https://www.fmrib.ox.ac.uk/fsl}{FMRIB's Software
Library;}][]{smith2004fsl}, which have been developed and debugged for years
through collaborative efforts,
%
but also from scientific software packages like
%
DataLad
\citep[\href{www.datalad.org}{\url{datalad.org}};][]{halchenko2021datalad} or
%
BrainIAK
\citep[\href{https://brainiak.org}{\url{brainiak.org}};][]{kumar2020brainiak,
kumar2020brainiaktutorial}
%
that emerged recently.
%
The use of pre-existing software packages, data, and results from previous
analyses enabled the project to shift time and resources from software
development and data collection to subsequent stages of the project.
%
The available data proved to be extremely valuable during the COVID-19 pandemic
as acquiring new fMRI data from participants became impossible, which led to a
need to revise the project plan.

% issue 1: data quality
A first issue with open data that is frequently overlooked is that openly
accessible data do not exempt the data consumer from the responsibility of
carefully scrutinizing the quality of the data and the underlying experimental
paradigm (e.g., stimuli and code).
%
It is too tempting to ``simply push the data through an analysis pipeline''
without carefully assessing the quality of the data first.
%
Researchers must ensure that any potential errors in the data are identified and
addressed before proceeding with their analysis.
% check the data
Consumers of datasets must assume that anything not clearly specified in the
dataset's description has not been taken into account.
% standards may differ
This is particularly important as the standards (e.g., quality, formats,
parameters) and open sciences practices (e.g., documenting) may differ across
scientific fields or even within a scientific field depending on a working
group's expertise and rigor.
% laugh with many, don't trust any
Even if the data are provided by a reputable source, researchers who consider
using third-party data should also consider themselves to be obliged to test and
validate a dataset's quality as if it was created by themselves, and in
accordance with their standards and particular use case.
%
In this regard, open data, despite being collected to the best of knowledge and
belief, can be compared to open-source software:
% super fancy literal quote
data will contain noise, errors, or artifacts, just as software will contain
bugs, but ``given enough eyeballs, all bugs are shallow'' \citep[][p.
30]{raymond1999cathedral}.

% issue 2: decisions were made
A second issue of open data is that the decisions made during data collection,
preprocessing, and further analyses may influence or even limit subsequently
performed analyses.
%
Researchers must carefully consider these decisions when deciding whether to use
pre-existing data, or to collect or preprocess data themselves.
%
For instance, we and \citet{sengupta2016extension} selected the \ac{ppa} among
possible candidates of scene-selective area because the \ac{ppa} was the first
scene-selective area to be discovered and is the most reliably activated region
across studies that investigate visual scene perception.
%
However, other areas, such as the \ac{rsc} and \ac{opa}, have repeatedly been
shown to be involved in visual spatial perception and navigation
\citep{chrastil2018heterogeneity, bettencourt2013role, dilks2013occipital,
epstein2019scene}.
%
Although we did not explicitly hypothesize it, we assumed that at least the
analysis of the movie may yield significant clusters in the medial parietal and
lateral occipital cortex that may correspond to the functionally defined
\ac{rsc} and \ac{opa} respectively.
%
Indeed, results revealed significantly increased activity in the medial parietal
and lateral occipital cortex, and provide an incentive for further studies.
% \citep[cf. algorithmic procedure in, e.g.,][]{julian2012algorithmic}.
However, in the case of results from \citep{sengupta2016extension}, one would
have to replicate the non-automatized procedure of \citet{sengupta2016extension}
in order to create the corresponding masks.
%
This example shows how decisions made during data collection, preprocessing or
preceding investigations, despite being state-of-the-art at the time of being
published, are affecting subsequent studies.
% pro & contra: opportunity costs
Hence, when considering using open data, researchers need to weigh the costs and
benefits of one option (such as using preprocessed data as provided) relative to
an alternative option (such as preprocessing raw data differently than
provided), and then choose the option that will yield the highest net return.
%
In summary, any previous step that required human intervention or was not fully
automated influences the degree to which data or materials can be replicated,
updated, or extended.




\subsection{Publishing data, materials, code, and results}

\todo[inline]{speech anno paper \& ppa paper are on github and public}

\todo[inline]{thesis is on github but not public}

\todo[inline]{state that and the URLs here?}
%
%The annotation comprises, e.g., time-stamps of phonemes, words and sentences of
%all speakers, a grammatical tagging, and an annotation of syntactic
%dependencies and semantics.

%
The second subgoal was to publish the data, materials, and results openly
available.
%
The data and custom code created for this dissertation are version-controlled,
meaning that any changes made were logged and documented, to promote
transparency.
%
To ensure reproducibility, processing steps, ranging from downloading input data
to plotting figures, are implemented in scripts that can be rerun from the
command line.
%
For example, the created annotation of speech has been published freely
accessible \citep{haeusler2021speechanno}.
%
Its content goes beyond what was required to conduct the analyses in
\citet{haeusler2022processing}, serves as an extension of the studyforrest
project and widens the ``annotation bottleneck'' \citep[][p.
16]{aliko2020naturalistic} of two naturalistic stimuli.
%
In \citet{haeusler2022processing}, we used open \ac{fmri} data to investigate a
research subject that was not anticipated when the data were initially made
available.
%
The results of \citet{haeusler2022processing} indicate that increased
hemodynamic activity in the \ac{ppa} generalizes from blocks of images to
spatial information embedded in a movie and an auditory narrative.
%
These results published in a peer-reviewed journal highlight the benefits of
sharing and reusing data to explore research questions and to generate new
insights.
% contra
From a negative perspective, creating data, materials, and code to be published
requires a considerable amount of time and effort.
%
To encourage third parties to reuse the data, dataset creators must anticipate
potential use cases, collect the data with appropriate extent and rigor, convert
the data into a standardized format (taking into account, for example, naming
conventions and folder structure).
% automation
Analyses pipelines need to be designed and tested in a way that they can
reliably replicate every stage stage of a dataset.
% publication: findable, accessible, interoperable, reusable
Additionally, dataset creators must take into account legal matters (such as
intellectual property rights, use licenses, statements of agreement, and
anonymization of participant data), facilitate discovery by humans and web bots
(e.g., by including detailed descriptions and machine-readable metadata), and
guarantee long-term curation and accessibility.

%
From a positive perspective, creating a dataset that is intended to be published
has immediate benefits for researchers.
%
Meticulously recording each step and commenting the advantages and disadvantages
of alternate procedural options leads to deeper understanding of the scientific
area, its practices, and methods.
%
The version-control of every step reduces the likelihood of a look-ahead bias.
%
Tracking and extensively documenting each stage of the data and code from the
beginning to the final results can also be thought of a lab protocol that
comprises structured information for writing the corresponding scientific
article.
%
Hence, creating a dataset supposed to be published encourages precise work
habits and good scientific practices in general.


\subsection{Interim summary}

% Open access publications might receive more citations than paywalled
% publications [\citep{piwowar2018state}], open data might get cited, and
% promote new collaborations [\citep{popkin2019data}].

% incentives like ``professional recognition or the allocation of extra funding
% [Kidwell et al., 2016; Fecher et al., 2015; Ali-Khan et al., 2018];
% Funding agencies already require publication of findings in OA schemes and
% data-sharing plans [Neylon, 2017]'' \citep{toribio2021early}.

%
In summary, pursuing the objective of conducting an open and reproducible
research project was not a requirement for submitting the thesis but required
considerable additional work and time.
%
Since open science practices are not yet covered in graduate or PhD curricula,
learning about principles and standards, as well as putting them into practice,
relied on self-initiative and self-learning.
%
The constantly emerging standards and principles, and the steadily developing
software packages to apply these standards, made it difficult to put theoretical
knowledge into practice.
%
In my opinion, the time and effort needed for open science practices greatly
exceed any short-term advantages.
%
The lengthy procedures are not justified by ``gambling'' on being cited in the
event that released data are reused, or by merely pursuing the ``higher
purpose'' of addressing the replication crisis.
%
Particularly, designing and testing fully automated analysis pipelines or
scripts to plot complex figures without any manual finishing for the perceived
sole purpose of reproducibility is out of proportion to the immediate benefit of
easier bug tracking or simply ``higher confidence in one's own work''.
%
On the contrary, PhD students that pursue a career in science may be concerned
about exposing themselves to critique owing to a maximum of voluntary
transparency and possibly (and blamelessly?) overlooked errors.
%
Another concern is the potential for being ``scooped'', which refers to the risk
that another working group is using the same data for a similar research
question at the same time and eventually claiming priority to the research idea
and its findings \citep[cf.][]{laine2017afraid}.
%
This risk is aggravated in case of early-career scientists that created and
maintain a public dataset, pre-registered studies based on a public dataset, or
have to adhere to inflexible project plans.
%
Hence, undergraduate programs should teach the benefits and best practices, and
risks of open science, and provide practical training in related software
packages.
%
Postgraduate programs should create incentives to conduct open science projects.
% Instead of compulsory requirements from funders, which might only lead
% researchers to show minimal compliance [Neylon, 2017].
%
After all, open sciences is a suitable tool to
%
(a) hold researchers responsible for collecting, storing, documenting,
processing, and publishing data and materials in accordance with best practices,
%
(b) increase the reproducibility of results and reliability of findings,
%
(c) make knowledge and technologies widely accessible,
%
(d) therefore increase the efficiency of the scientific progress and promote
innovation, and
%
(e) ultimately increase the public's trust in the scientific process and its
findings.


\section{Naturalistic stimuli for functional localization}

%
The traditional method for identifying the \ac{ppa} is to contrast hemodynamic
responses to blocks of images of scenes or landscapes with blocks of images of
tools or faces.
%
While the exact outline of the \ac{ppa} varies depending on the type of stimuli,
task, and contrast (as well as statistical threshold), the traditional localizer
approach can reliably delineate the \ac{ppa} bilaterally in a large proportion
of subjects \citep{zhen2017quantifying}.
%
\citet{sengupta2016extension}, for instance, were able to identify the
left-hemispheric \ac{ppa} in 12 of 14 subjects and right-hemispheric \ac{ppa} in
14 of 14 subjects based on localizer data.
% study in one sentence
In \citet{haeusler2022processing}, we investigated whether the \ac{ppa}, as
previously identified in the same group of participants of
\citet{sengupta2016extension}, could be localized using an audio-visual
naturalistic stimulus and an exclusively auditory naturalistic stimulus.
% stress similarity to localizer approach
We adapted the traditional localizer approach and modeled hemodynamic responses
to events in the two naturalistic stimuli to create $t$-contrasts that aimed to
localize the \ac{ppa}.
% AV operationalization
For the statistical analysis (i.e. \ac{glm}) of the movie, we utilized an
annotation of movie cuts and depicted locations \citep{haeusler2016cutanno}.
% AD operationalization
For the \ac{glm} of the audio-description, we extended the annotation of speech
that we created and validated in \citet{haeusler2021speechanno} by further
annotating nouns that the audio-description's narrator uses to describe the
lacking visual content.
% group results: specifically
On a group-average level, results demonstrate that increased hemodynamic
activation in the \ac{ppa} during the perception of static images generalizes to
the perception of spatial information embedded in a movie and an auditory
stimulus.
%
We have shown that a model-driven \ac{glm} analysis based on a naturalistic
stimulus' annotation can replicate findings of studies that employed traditional
paradigms.
%
Our results provide further evidence \citep[cf.][]{bartels2004mapping} that
functional specialization of cortical areas is maintained during naturalistic
stimulation.
% individual results
On an individual level, our analysis of the movie revealed bilateral clusters of
increased hemodynamic activity in the \ac{ppa} of five participants and a
unilateral cluster in seven participants.
%
The analysis of the audio-description revealed bilateral clusters in nine
participants and one unilateral cluster in one participant.
% conclusion
These findings suggest that a naturalistic stimulus, whether visual or auditory,
could potentially replace a traditional localizer to assess brain functions in
individuals.



\subsection{Current challenges and limitations}

\todo[inline]{add that engagement is particularly hard to asses during
audio-only}

%
The current thesis highlights obstacles in the pursuit of developing a
multi-functional localizer.
%
Traditional experimental designs typically involve presenting participants with
simple, well-controlled stimuli that are carefully designed to elicit
clearly-defined responses in targeted brain regions.
%
This allows for modeling of hemodynamic responses that can predict and explain
the neural activity observed in response to these stimuli.
%
The currently dominant analysis approach is the mass-univariate \ac{glm}, which
has its roots in \ac{pet} research, and is tailored to analyze data of
parametric experimental designs that manipulate isolated experimental variables
of interest.
%
The \ac{glm} requires the researcher to specify which stimulus features are
presumed to be correlated with the brain process under investigation.
%
Then, the researcher needs to model a hypothesized hemodynamic time course that
is fit to the data in order to predict the observed hemodynamic activity and
contrast parameter estimates \citep{friston1998event}.
%
Naturalistic stimuli, however, are continuous and complex, with a multitude of
sensory features that can activate different brain regions simultaneously.
%
Applying the traditional analysis approach is challenging and can lead to
difficulties in isolating specific neural correlates of perceptual or cognitive
processes.
%
The properties of naturalistic stimuli can stress physiological assumptions of
the traditional GLM approach, such as cognitive subtraction
\citep{friston1996trouble}, the consistency of hemodynamic responses across
events \citep[the rationale behind \textit{trial-averaging};
cf.][]{dale1997selective}, and the linearity of hemodynamic responses
\citep{cohen1997parametric, boynton1996linear, dale1999optimal}.  \todo{if
points are correct, extend}

%
The properties of naturalistic stimuli also stress statistical assumptions such
as the absence of collinearity among variables.
%
The lack of experimental control, however, can be alleviated by detailed
annotations that allow modeling and statistically controlling potentially
confounding variables \citep[e.g.,][]{deniz2019representation}.
%
Low-level visual features, such as brightness, and low-level auditory features,
such as root-mean square power (i.e., volume) can be automatically extracted on
a low temporal scale (e.g., per movie frame; cf.
\href{https://github.com/psychoinformatics-de/studyforrest-data-confoundsannotation
}{\url{github.com/psychoinformatics-de/studyforrest-data-confoundsannotation}}
for low-level annotations of the studyforrest project).
%
Recent advances in machine learning have led to the creation tools, such as the
``pliers'' \citep{mcnamara2017developing} implemented in the neuroscout platform
\citep[\href{https://neuroscout.org/}{\url{neuroscout.org}};][]{delavega2022neuroscout},
that can automatically extract higher-level features like semantics or clearly
defined object categories.
%
Such tools can replace time-consuming manual annotations or provide a
provisional scaffold that reduces manual labor for a growing number of stimulus
features.
%
However, variables that are difficult to define, hard to consistently label,
have multiple interpretations, fluctuate on longer time scales, or
subject-related variables, such as the level of engagement or felt emotions
\citep[cf.][]{lettieri2019emotionotopy, saarimaki2021naturalistic}, still defy
an automatic annotation.
%
Hence, merely assessing the confound structure of a naturalistic stimulus as a
possible candidate for a prospective model-driven analysis might become
time-consuming.
%
Moreover, a large amount of annotated features may lead to annotations that are
``hard to use'' \citep[][p. 2]{richard2019fast} or ``high-dimensional,
cumbersome models'' \citep[][p. 2]{richard2019fast} eventually resulting in a
lack of statistical power due to insufficient data samples.



\subsection{Interim summary}

%
Given the unsolved issues and current limitations, it is perhaps not surprising
that, even 20 years after the group-level findings of
\citet{bartels2004mapping}, no functional localizer based on a movie or an
auditory narrative exists to localize higher visual areas.
%
The most ecologically valid visual localizers currently available are
\textit{dynamic localizers} \citep[e.g.,][]{pitcher2011differential,
fox2009defining}, which use blocks of short videos (each video lasting
$\approx$\unit[2-3]{s}) of scenes, faces, etc., making them well-suited for
traditional modeling procedures.

%
Even naturalistic stimuli, despite being labeled as ``naturalistic''
and ubiquitously referred to as ``more ecologically valid'', are not strictly
``naturalistic'', but rather an approximation of real-life presented on a screen
or via headphones in the laboratory.
% entertain, inform, or advertise.
Similar to traditionally used stimuli that have been carefully designed by
researchers to probe specific brain processes, most naturalistic stimuli used in
neuroscience have been carefully designed by professional media creators to
appeal to their target audience.
%
Film directors intentionally manipulate the viewers' attentional focus and
mental states using a variety of techniques like camera-movement, composition,
movie editing, voice-overs \citep{brown2012cinematography,
dancyger2011film-technique, katz1991film, mercado2011filmmakers} that, when used
correctly, largely occur unnoticed.
%
For example, participants asked to spot movie cuts miss between 10\% and 50\% of
them depending on the type of cut \citep{smith2008edit}.
%
While these techniques reduce individual variation and lead to reliably
synchronized spatial-temporal responses across subjects in a large part of the
cortex \citep{hasson2008neurocinematics}, they also introduce a confounding
factor in the form of ``naturalistic stimulus statistics''.

%
Although naturalistic stimuli are intended to be inherently engaging and offer a
task-free paradigm, sustained attention is still required by participants when
``freely viewing'' a movie or ``freely listening'' to an auditory narrative,
which may become challenging over longer periods of time.
%
The occasional disengagement of single subjects may have a negligible effect on
group-average results, especially when statistics are calculated over long
stimulus intervals.
%
However, when a study aims to investigate a brain process that is expected to be
universal across humans on an individual level, a varying engagement may impair
the reliability of individual-level results across stimulus segments and
potentially mask effects of interest.
%
Nonetheless, a varying level of engagement may reflect a personal preference
towards the presented stimulus or elicit, given the low demand characteristics
\citep[cf.][]{orne1962social} of naturalistic paradigms, a more natural behavior
reflecting a personal trait.



\subsection{Functional localization via estimation from reference group}

\todo[inline]{write when SRM part is finished}

\todo[inline]{explicitly (re)state aims}

%
Results from \citet{haeusler2022processing} suggest that the movie as well as
the audio-description sample hemodynamic responses that correlate with the
occurrence of spatial information.
%
Therefore, we did the following ...

% goal 1: new procedure
We estimated results of a dedicated localizer \citep{sengupta2016extension} via
functional alignment from results of a reference group.
% the problem
Considering practical and monetary constraints in a clinical context, a paradigm
lasting 90 to 120 minutes is inappropriate for even an extensive individual
diagnostic procedure, we also assessed the relationship between length of
naturalistic stimulation used to align the test participant to the fixed
\ac{cfs} and the estimation performance.



\subsubsection{Interim summary and future studies}

\todo[inline]{revise here with SRM part and vice versa}

\todo[inline]{imo, it fits better into the SRM part; general discussion is more
about pro \& contra of naturalistic stimuli}

\todo[inline]{Following are some "templates" outlining the general idea}

% examples of probabilistic atlasses: \citet{rosenke2021probabilistic}:
% Cortical atlases have been developed, which allow localization of visual areas
% ``in new subjects by leveraging ROI data from an independent set of typical
% participants: Frost and Goebel 2012;
% ventral temporal cortex (VTC) category selectivity: Julian et al. 2012,
% Zhen et al. 2017, Weiner et al. 2018; visual field maps: Benson et al. 2012,
% Benson and Winawer 2018; Wang et al. 2015''.


\paragraph{Problem space}

%
``Identifying all of the currently known topographic regions of the human visual
system requires multiple scanning sessions'' \citep{wang2015probabilistic}.
%
``Given the expense and availability of fMRI, this is not always practical''
\citep{wang2015probabilistic}.
%
``Our approach has the potential to estimate an unlimited variety of functional
topographies at the individual level based on the responses to a single
naturalistic, dynamic stimulus'' \citep{jiahui2020predicting}.


\paragraph{Database}

%
``A normative database of participants who were scanned during movie viewing and
during functional localizers for different perceptual and cognitive functions
would serve as a reference'' \citep{jiahui2020predicting}.
%
``There are numerous other functional localizers in other perceptual and
cognitive domains, such as simple visual motion, biological motion, tonotopy,
voice perception, music perception, language, calculations, working memory, and
theory of mind'' \citep{jiahui2020predicting}.

%
``A database from a normative group could allow researchers to estimate new
subjects' functional topographies by collecting only a movie-viewing data set
and then deriving the individualized topographies with the normative database''
\citep{jiahui2020predicting}.
%
The database ``may prove especially useful for predicting functional patterns in
case no localizer data are available, saving scanning time and expenses''
\citep{rosenke2021probabilistic}.


\paragraph{Calibration}
%
A naturalistic stimulus like a move or audio-description could be used to align
a test subject to a \ac{cfs} created from data of a normative reference group.
%
Naturalistic stimuli ``engage in parallel multiple neural systems for vision,
audition, language, person perception, social cognition, and other functions''
\citep{jiahui2020predicting} and offer higher generalizability [and provide
higher validity?] of transformations matrices.

%
''Some high-level cognitive processes, such as calculation, working memory, and
logical reasoning, may be less well sampled by movie viewing, and further work
is necessary to test whether hyperalignment based on movie-viewing data can be
used to estimate topographies for these other domains of high-level cognition.
'' \citep{jiahui2020predicting}.


\paragraph{Application: estimation}

%
Once a valid alignment is established, known functional properties of the
(normative) reference can then be projected into the respective individual voxel
space by mapping a variety $Z$-maps created from a variety of $t$-contrast from
a normative reference group onto an individual subjects and thus potentially
substitute a variety of localizers.
%
``A new subject's functional topographies could be estimated based only on that
subject's movie data and other subjects' localizer data from the normative
database that could be projected into that subject's cortical anatomy using
hyperalignment transformation matrices derived from movie data and could replace
tedious functional localizers with an engaging movie''
\citep{jiahui2020predicting}.

%
``Investigators would need to scan their participants only during movie viewing
and a wide range of idiosyncratic functional topographies could then be
estimated individually based on localizer data projected from the brains in the
normative sample into the new participants' cortical anatomies''
\citep{jiahui2020predicting}.
%
``Functional topographies could be mapped from a database containing a wide
range of perceptual and cognitive functions to new subjects based only on fMRI
data collected while watching an engaging, naturalistic stimulus and other
subjects' localizer data from a normative sample'' \citep{jiahui2020predicting}.

%
''Because naturalistic movies include people, human actions, conversations,
social interactions, background music etc., we predict that hyperalignment
transformation matrices based on these movies also will work for localizers of
functional topographies for audition, language, and social cognition''
\citep{jiahui2020predicting}.
%
``From a single movie dataset multiple functional topographies can be estimated
\citep{guntupalli2016model}, whereas different localizers are typically required
to map different functional topographies, making a thorough mapping of selective
topographies time-consuming and inefficient'' \citep{jiahui2020predicting}.


\paragraph{Application: deviation (a.k.a. clinical application)}

\todo[inline]{cf. general introduction (on "individual neuroscience")}

\todo[inline]{cf. SRM discussion: quantify deviation from a norm vs.  predict
deviant pattern?}


\todo[inline]{\citet{silva2018challenges, szaflarski2017practice}}

%
That reference would enable an qualitative and quantitative description of an
individual's brain function with respect to such a norm, and consequently
progress the field towards neuroimaging studies of individual differences that
more closely resemble their psychological counterparts.


``In the clinical context, fMRI plays an important role for planning surgery in
patients with tumors or epilepsies, as it aids the understanding of which parts
of the brain need to be spared in order to preserve sensory, motor or cognitive
abilities'' \citep{wegrzyn2018thought}.


\paragraph{Language lateralization}

%
For example \ac{fmri} could be used as an noninvasive alternative to map
language areas and potentially assess lateralization (or hemispheric asymmetry)
of functional brain topography related to language (sub)functions, in order to
guide pre- and perioperative assessment of neurosurgery, e.g., in case of
epilepsy.


\subsubsection{SRM}

The SRM algorithm used in this study has a drawback in that it only models
responses that are common across individuals, without including an individual
component.
%
Although a person's idiosyncratic responses are excluded from the shared
response model, they ``are not necessarily noise and may in fact be highly
reliable within participants'' \citep{cohen2017computational}.
%
Based on the results of this study, it appears that we estimate a regular
pattern in noisy data of a couple of participants  (i.e., we perform an
\ac{srm}-based denoising).
%
However, to predict a reliably deviant pattern, it is necessary to use a
matching CFS from a deviant reference, or preferably, a model that incorporates
a shared response, an individual component, and noise.
%
Future studies should assess the performance of various functional alignment
algorithms in estimating reliably deviant patterns, e.g. atypical topographies
of language areas.
%
Does the algorithm estimate a regular pattern, which allows for quantifying the
difference of a deviant actual pattern from a norm, or does it directly estimate
the deviant pattern from a regular reference?

\todo[inline]{at least two of \citet{feilong2022individualized, jiahui2022cross,
turek2018capturing} have modeled an individual component.}

\paragraph{SRM and individual differences}
%
``SRM can be used to isolate participant-unique responses by examining the
residuals after removing shared group responses, or it can be applied
hierarchically to the residuals to identify subgroups [\citet{chen2017shared}]
'' \citep{cohen2017computational}.
%
``Recognizing that signal exists beyond the average or shared response of a
group, such studies exploit idiosyncratic but stable responses to account for
previously unexplained variance in brain function, behavioral performance and
clinical measures [e.g., Finn (2015). Functional fingerprinting (based on
connectivity)]'' \citep{cohen2017computational}.
%
``In cases where each subject's unique response is of more interest than the
shared signal, SRM can be used to factor out the shared component thereby
isolating the idiosyncratic response for each subject
[\citep{chen2015reduced}]'' \citep{kumar2020brainiak}.



\section{Conclusion}

%
Traditional localizer paradigms are designed to identify functional brain
regions that are assumed to be selectively engaged by certain cognitive or
perceptual processes.
%
In contrast, naturalistic stimuli capture the complexity and richness of
real-world experiences, leading to responses in multiple brain regions
simultaneously.
%
As a proof of concept the current thesis focused on the \ac{ppa}, a classic
visual area, to investigate the potential of two naturalistic stimuli to replace
a traditional localizer paradigm.
%
We found that an audio-visual and exclusively auditory naturalistic stimulus can
produce results that are consistent with previous studies using a traditional
localizer paradigm.
%
However, the current thesis also highlights challenges of naturalistic stimuli
as a potential naturalistic localizer, such as the medium-specific confound
structure.
%
Moreover, the lack of an explicit task allows for more individual variation in
engagement in the paradigm.
% further studies
Further studies are needed to investigate whether findings of this thesis
generalize to other higher-visual areas and other domains of brain functions.
%
With further investigation and refinement of analysis methods, it may be
possible to localize multiple concise functional areas and networks that
correlate with specific brain processes using only one naturalistic stimulus.


%
Naturalistic stimuli are versatile and offer a wealth of information that can be
used to investigate various research questions.
%
However, researchers must carefully consider the strengths and limitations of
naturalistic stimuli and use appropriate methods to address these challenges.
%
Researchers should familiarize themselves with media techniques to make informed
decisions when choosing potential stimulus candidates, and quantify the confound
structure of each candidate before collecting data.
% consider sharing
Creating a dataset with the intention of sharing it with the scientific
community can promote the progress of the field and ensure a high-quality
dataset by adhering to best practices of collecting, documenting, storing, and
processing data.
%
To address the challenge of individually varying levels of engagement, \ac{fmri}
scanning sessions could be accompanied by \ac{eeg}, eye-tracking, physiological
recordings, such as skin conductance response and heart rate, and followed by
self-reports to evaluate, e.g., a participant's alertness and audio track
audibility within the noisy scanner.
%
These additional recordings and measures may complicate the application of
naturalistic stimuli in clinical settings but are necessary at the current
stage.
% create and provide annotations
The additional creation and sharing of stimulus annotations are crucial to
overcoming the usage bottleneck of naturalistic stimuli.
%
Stimulus annotations can be used to check physiological assumptions, address
methodological or statistical issues, and foster the development of new analysis
methods.
%
Currently, neuroscience is still in the research and development stage, and
needs to adapt traditional methods and develop new methods developed to analyze
data from naturalistic stimuli.
%
Model-driven analyses based on extensive annotations may reveal which stimulus
features are driving which brain responses, enabling researchers to more
effectively choose an appropriate naturalistic stimulus for a given research
question or targeted population.

%
Naturalistic stimuli should not seek to replace traditional paradigms entirely.
%
Instead, they should be used in conjunction to test findings under more
ecologically valid conditions, generate new hypotheses, and advance our
understanding of the brain \citep[cf.][]{jaaskelainen2021movies,
sonkusare2019naturalistic}.
%
In case of functional localizers, it is important to note that the naturalistic
stimuli may not yield the same results as traditional paradigms.
%
Traditional localizers aim to minimize individual variation and localize
functional areas in all healthy individuals, while naturalistic stimuli lack a
specific task and may result in more individual variation.
%
Even in a non-clinical population, analyses from naturalistic stimuli may not
reproduce results from traditional localizers because naturalistic stimuli probe
more naturalistic behavior.
%
The lack of demand characteristics may enable participants to behave more
naturally during an experimental paradigm, potentially revealing individual
differences in perception and cognition that may also affect their engagement
with and processing of information in the real world.
%
Although the lack of control over engagement levels in naturalistic paradigms
may present challenges in investigating universal brain processes on an
individual level, it also provides a unique opportunity to explore individual
differences and their relationship to cognitive and personality traits.
%
Individual variation can be considered a confound when searching for a pattern
of brain activity common to all humans, or can be considered to be an advantage
when individual variation is critical to differentiate between subjects or
groups.
%
Therefore, if the minimization of inter-subject variability and reliability in
all subjects is the primary goal, traditional localizers should be used.

%
In conclusion, naturalistic stimuli can provide an engaging and ecologically
valid alternative to traditional localizer paradigms, but it is important for
researchers to acknowledge the distinct purposes and limitations of both
paradigms.
%
The potential of naturalistic stimuli to replace one if not multiple traditional
localizers should be approached with caution, particularly as their potential
depends on the specific domain of brain function being studied.
%
Therefore, neuroscience research should continue to explore the use of
naturalistic stimuli and focus on developing analysis methods that better
control confounding variables and individual variations in engagement.
%
Caution should be exercised before applying these stimuli in practical
applications to draw inferences about health and disease of individuals.


\begin{comment}



\pagebreak



\section{Some backups}


\subsection{auditory PPA is in anterior visual PPA}

\todo[inline]{at the moment, that is not mentioned in the general discussion;
imo, it's more a bycatch and does not really fit into the main topic}
%
However, the usability of a purely auditory paradigm to localize the \ac{ppa} as
it is defined by a visual paradigm might be limited by our findings that suggest
that increased hemodynamic activity during auditory stimulation is spatially
restricted to the anterior \ac{ppa}.
%
Our results provide further evidence to studies in the field of
visual perception that suggest that the \ac{ppa} can be divided into
functionally subregions that are coactivate during the perception of visual
scenes but process different stimulus features
\citep{aminoff2007parahippocampal, baldassano2013differential}.
%
On the other hand, results encourage future studies to investigate response in
the parahippocampal region to auditory stimuli under more controlled conditions.
%
In case the controlled paradigm reveals less variation in reliability across
participants, future studies employing naturalistic stimuli could then
investigate factors that might have influenced reliability of individual results
in \citet{haeusler2022processing}:
%
a) stimulus-related factors like the number available events that can be modeled
and contrasted per stimulus segment, or the general confound structure,
%
b) modeling-related factors like the assumed shape of the hemodynamic response
%
c) subject-specific factors like alertness or engagement over the course of the
experiment.




\subsection{Templates for Cronbach's $\alpha$}


\paragraph{\citet{jiahui2020predicting}}
%
``We compared the correlations between maps estimated from a participant's own
data and maps estimated from other participants' data to the reliability of the
localizer (across the four localizer runs).
%
The mean Cronbach's Alpha between the four localizer runs was 0.60 (N = 15, S.D.
= 0.14)'' \citep{jiahui2020predicting}.
%
``Results mean that if we scan each participant for another 4 localizer
runs, and compute the correlation between the two maps (4 runs vs. 4 runs), the
correlation would be 0.60 on average in the studyforrest''
\citep{jiahui2020predicting}.
%
``Cronbach's alpha indicates that the predicted contrast map based on
hyperalignment is close to or as good as the real contrast map based on four
localizer runs (studyforrest: t(14) = 0.61, p = 0.55; Grand Budapest: t(20) =
3.02, p = 0.007)'' \citep{jiahui2020predicting}.
%
``The predicted contrast map based on hyperalignment was better than the
contrast map based on data from three out of four localizer runs in other
participants (t(14) = 2.36, p = 0.03) and in Grand Budapest, the predicted
contrast map was comparable to the contrast map based on three localizer runs in
other participants (t(20) = 0.48, p = 0.63)''
\citep{jiahui2020predicting}.


\paragraph{\citet{jiahui2022cross}}
%
``Because the localizer task comprises several scanning runs, we calculated the
reliability of the localizer across runs with Cronbach's alpha to provide an
estimate of the noise ceiling for these correlations'' \citep{jiahui2022cross}.
%
``We compared the correlations between topographies estimated from a
participant's own localizer data and those from other participants' data to the
reliability of the localizer, calculated with Cronbach's alpha.
%
Predictions made with hyperalignment were close to and sometimes even exceeded
the reliability values (Figure 1B).
%
This indicates that the predicted category-selective topographies from other
participants' data using hyperalignment were as precise and sometimes even
better than the topographies estimated with their own localizer data''
\citep{jiahui2022cross}.


\paragraph{\citet{feilong2022individualized}}
%
``Due to the presence of noise in localizer data, the estimated face-selectivity
map is a combination of a “true” face-selectivity map of the participant and
some noise.
%
The component from the “true” map is supposed to be shared by all localizer
runs, and thus the data quality and the level of noise can be estimated based on
the similarity between the 4 maps (i.e., one from each run).
%
We used Cronbach's alpha to estimate the reliability  of the average map of the
4 runs.
%
We estimated the reliability of the localizer-based map using Cronbach's alpha,
which is the expected correlation between two average maps, each based on 4 runs
of independent data.
%
If we were to collect another 4 localizer runs from the participant and get a
new average map based on the 4 new runs (i.e., independent data), then the
expected correlation between the two average maps would be Cronbach's alpha.
%
In other words, if the correlation between the model-predicted map and the
localizer-based (average) map is higher than Cronbach's alpha, then the
model-predicted map is more accurate than the average map based on 4 runs''
\citep{feilong2022individualized}.

%
``Based on the similarity between these four maps, we computed the Cronbach's
alpha coefficient for each participant, which estimates the reliability of the
average map.
%
That is, if we were to scan the participant for another four localizer runs and
correlate the new average map with the current average map, the expected
correlation would be Cronbach's alpha.'' \citep{feilong2022individualized}

%
``For both datasets, the localizer-based and model-predicted face-selectivity
maps were highly correlated (Forrest: r = 0.618 ± 0.089 [mean ± SD], Raiders: r
= 0.716 ± 0.074), and the correlations were higher than our previous
state-of-the-art model using the same dataset and hyperalignment (Jiahui et al.,
2020).
%
Across all participants, the average Cronbach's alpha was 0.606 ± 0.126 for
Forrest, and 0.764 ± 0.089 for Raiders.
%
For approximately a third of the participants (Forrest: 6 out of 15, 40\%;
Raiders: 6 out of 20, 30\%), the correlation exceeded the Cronbach's alpha of
localizer-based maps.
%
In other words, for these participants, the predicted map based on our model can
be more accurate than the map based on a typical localizer scanning session
comprising four runs'' \citep{feilong2022individualized}.

%
``Note that for the Forrest dataset, the similarity sometimes exceeded
Cronbach's alpha, which means the model-predicted map is more accurate than a
map based on 4 localizer runs (21 minutes).
%
The quality of localizer-based maps increases with more localizer data, which
can be estimated using the Spearman–Brown prediction formula [Brown, 1910;
Spearman, 1910].
%
Based on the Spearman–Brown prediction formula, we can estimate how Cronbach's
alpha changes with the amount of data (i.e., the number of localizer runs), and
correspondingly, how much localizer data is needed to achieve the quality of the
model-predicted map
%
For the Forrest dataset, the maps predicted by 15, 30, 60, and 120 minutes of
movie data were as accurate as 17, 22, 26, and 30 minutes of localizer data,
respectively.
%
For the Raiders dataset, the maps predicted by 15, 28, and 56 minutes of movie
data were as accurate as 10, 11 and 12 minutes of localizer data, respectively''
\citep{feilong2022individualized}

\subsection{Templates from \citet{haxby2011common} on validity \&
    generalizability}

%
``The algorithm also can be applied to simpler, controlled experimental data but
our previous results showed that the sampling of response vectors from these
experiments is impoverished and produces a model representational space that
does not generalize well to new stimuli in other experiments
\citep{haxby2020hyperalignment}'' \citep{guntupalli2016model}.

%
``We base the derivation of the transformation matrices and the common space on
responses to the movie---a complex, naturalistic, dynamic stimulus.
%
Although the algorithm also can be applied to fMRI data from more controlled
experiments, we found that a common model based on such data has greatly
diminished general validity \citep{haxby2011common}, presumably because,
relative to a rich and dynamic naturalistic stimulus, such experiments sample an
impoverished range of brain states \citep{guntupalli2016model}''.

``The general validity of the model across the varied stimulus sets that we
tested could be achieved only when hyperalignment was based on responses to the
movie.
%
Common models based on responses to smaller, more controlled stimulus
sets---still images of a limited number of categories---were valid only for
restricted stimulus domains, indicating that these models captured only a
subspace of the substantially larger representational space in VT cortex''
\citep{haxby2011common}.

``We used a complex and dynamic natural stimulus---a full-length action
movie---to sample a diverse variety of representational states.
%
The results show that hyperalignment based on responses to this stimulus affords
a single model of VT cortex with general validity across a broad range of
stimuli, whereas hyperalignment based on responses to still images in more
controlled, conventional experiments does not.
%
Thus, by virtue of the rich diversity of a complex, natural stimulus, our model
of the representational space in VT cortex also has general validity across
stimuli''\citep{haxby2011common}.

``We also derived common models based on responses to the face and object
categories in ten subjects and on responses to the pictures of animals in 11
subjects.
%
These alternative common models afforded high levels of accuracy for BSC of the
stimulus categories used to derive the common space but did not generalize to
BSC for the movie time segments.
%
Thus, models based on hyperalignment of responses to a limited number of
stimulus categories align only a small subspace within the representational
space in VT cortex and are, therefore, inadequate as general models of that
space.
%
On the positive side, these results also show that hyperalignment can be used
for BSC of an fMRI experiment without data from movie viewing''
\citep{haxby2011common}.

%
``Further analyses revealed other desirable properties of the movie as a
stimulus for model derivation.
%
The movie evoked responses in VT cortex that were more distinctive than were
responses to the still images in the category perception experiments.
%
Moreover, the general validity of the model based on the responses to the movie
is not dependent on responses to stimuli that are in both the movie and the
category perception experiments but, rather, appears to rest on stimulus
properties that are more abstract and of more general utility''
\citep{haxby2011common}.

%
``We investigated whether hyperalignment of the face and object data and
hyperalignment of the animal species data would afford high levels of BSC
accuracy using only the data from those experiments.
%
In each experiment, we derived a common space based on all runs but one. We
transformed the data from all runs, including the left-out run, into this common
space.
%
We trained the classifier on those runs used for hyperalignment in all subjects
but one and tested the classifier on the data from the left-out run in the
left-out subject (i.e. the test data for determining classifier accuracy played
no role either in hyperalignment or in classifier training [Kriegeskorte et al.,
2009]).
%
BSC of face and object categories after hyperalignment based on data from that
experiment was equivalent to BSC after movie-based hyperalignment (62.9\% ±
2.9\% versus 63.9\% ± 2.2\%, respectively; Figure 4).
%
BSC of the animal species after hyperalignment based on data from that
experiment was significantly better than BSC after movie-based hyperalignment
(76.2\% ± 3.7\% versus 68.0\% ± 2.8\%, respectively; p < 0.05; Figure 4).
%
Result suggests that the validity for a model of a specific subspace may be
enhanced by designing a stimulus paradigm that samples the brain states in that
subspace more extensively'' \citep{haxby2011common}.

%
``We next asked whether hyperalignment based on these simpler stimulus sets was
sufficient to derive a common space with general validity across a wider array
of complex stimuli.
%
We applied the hyperalignment parameters derived from the face and object data
to the movie data in the ten Princeton subjects and the hyperalignment
parameters derived from the animal species data to the movie data in the 11
Dartmouth subjects.
%
BSC of 18s movie time segments after hyperalignment based on category perception
experiment data was markedly worse than BSC after hyperalignment based on movie
data (17.6\% ± 1.3\% versus 65.8\% ± 2.7\% for Princeton subjects; 28.3\% ±
2.8\% versus 74.9\% ± 4.1\% for Dartmouth subjects; p < 0.001 in both cases;
Figure 4).
%
Thus, hyperalignment of data using a set of stimuli that is less diverse than
the movie is effective, but the resultant common space has validity that is
limited to a small subspace of the representational space in VT cortex''
\citep{haxby2011common}.

%
``We also tested whether the general validity of the model space reflects
responses to stimuli that are in both the movie and the category perception
experiments or reflects stimulus properties that are not specific to these
stimuli.
%
We recomputed the common model after removing all movie time points in which a
monkey, a dog, an insect, or a bird appeared. We also removed time points for
the 30 s that followed such episodes to factor out effects of delayed
hemodynamic responses.
%
BSC of the face and object and animal species categories, including distinctions
among monkeys, dogs, insects, and birds, was not affected by removing these time
points from the movie data [65.0\% ± 1.9\% versus 64.8\% ± 2.3\% for faces and
objects; 67.1\% ± 3.0\% versus 67.6\% ± 3.1\% for animal species; Figure S4B].
%
This result suggests that the movie-based hyperalignment parameters that afford
generalization to these stimuli are not stimulus specific but, rather, reflect
stimulus properties that are more abstract and of more general utility for
object representations'' \citep{haxby2011common}.


\subsection{Dynamic localizers}
%
``Previous reports showing significant differences between topographies
estimated by static and dynamic localizers, especially in superior temporal and
frontal cortices [Fox et al., 2009; Pitcher et al., 2011]''
\citep{jiahui2022cross}.
%
``For all categories, the dynamic localizer elicited stronger and broader
category-selective activations than the static localizer''
\citep{jiahui2022cross}.
%
``For example, for the face-selective topographies, the dynamic localizer
activated more areas than the static localizer (e.g., in superior temporal and
frontal cortices)'' \citep{jiahui2022cross}.
%
``The searchlight analysis showed that the dynamic localizer had higher
reliabilities across the cortex, especially in regions that were selectively
responsive to the target category'' \citep{jiahui2022cross}.
%
``The low correlations were not because the prediction method failed but
reflected the difference in the topographies activated by different types of
localizers.'' \citep{jiahui2022cross}.


\subsection{Brain \& behavior: "fingerprints"}
%
Maybe might be stable individual differences in cognitive tendencies or
cognitive abilities like susceptibility / predisposition [?] to attend to, [or]
recognize [or process] auditory spatial information.
% Kanai
In case the pattern is stable within individual subjects / is ``highly
consistent across different sessions [or experiments], then they are
characteristics of the individuals and may reflect differences in their brain
function'' \citep{kanai2011structural} [on structural diffences].
%
``Individual differences in topology (i.e. location, size, shape of functional
areas) and the activity within functional areas can also be considered to be
interesting cases of inter-individual variability to understand the neural basis
of human cognition and behavior, brain-phenotype relationships'', and ``present
useful phenotypes or biomarkers \citep{glasser2016multi,
vanhorn2008individual}''


\subsubsection{Brain \& behavior: example studies}

\todo[inline]{cf. also \citep{gordon2017individual, gordon2017precision}}
%
For example, \citet{kong2019spatial} suggested based on resting-state functional
connectivity measures ``that individual-specific network topography (i.e.,
location and spatial arrangement) might serve as a fingerprint of human behavior
that can predict behavioral phenotypes across cognition, personality, and
emotion'' \citep{kong2019spatial} [with modest accuary, comparable to previous
reports predicting phenotypes based on connectivity strength].

%
\citep{bijsterbosch2018relationship}'s ``results indicate that spatial variation
in the topography of functional regions across individuals is strongly
associated with behaviour'' \citep{bijsterbosch2018relationship}.
%
\citet{bijsterbosch2018relationship} found ``that the spatial arrangement of
functional regions is strongly predictive of non-imaging measures of behavior
and lifestyle'' [however shape \& exact location of brain regions interacted
strongly with  modeling of brain connectivity].
%
\citet{bijsterbosch2018relationship} found ``that individual differences in the
size, shape and exact position of the brain regions [as identified by
resting-state functional connectivity measures] was strongly linked to
individual differences in behavioral tests and questionnaires [including
intelligence, life satisfaction, drug use and aggression problems]''
\citep{bijsterbosch2018relationship}.

%
``The variations in spatial topographical features captured a more direct and
unique representation of subject variability than temporal correlations between
regions defined by group parcellation approaches (coupling).
%
Hence, the cross-subject information represented in commonly adopted
'connectivity fingerprints' could largely reflect spatial variability in the
location of functional regions across individuals, rather than variability in
coupling strength (at least for methods that directly map group-level
parcellations onto individual data)'' \citep{bijsterbosch2018relationship}.

%
``Depending on the employed spatial alignment algorithm and the amount of
removed spatial intersubject variability, the degree to which spatial
information may influence FC estimates possibly varies considerably across
studies.
%
In recent years, significant efforts have gone into the methods that more
accurately estimate the spatial location of functional parcels in individual
subjects [Chong et al., 2017; Glasser et al., 2016; Gordon et al., 2016; Hacker
et al., 2013; Harrison et al., 2015; Varoquaux et al., 2011; Wang et al., 2015],
and into advanced hyperalignment approaches [Chen et al., 2015; Guntupalli et
al., 2016; Guntupalli and Haxby, 2017]'' \citep{bijsterbosch2018relationship}.

\end{comment}





%% References
% \bibliographystyle{unsrtnat}
\bibliographystyle{apacite}
\bibliography{references}

\todo[inline]{if APA-style is used, show all (i.e. more than 7) authors; ggf.
andere APA-style-Vorlage oder anpassen}




\chapter{Appendix}
%% The appendix contains:\\
%% a) any existing survey materials, e.g. questionnaires\\
%% b) detailed description of formulas and derivations\\
%% Please do not include signed documents
%% (e.g. the vote of the Ethics Committee)}
%\appendix
\todo[inline]{Update with plots (and text) showing mean correlations across 1000
models}

\todo[inline]{manuall add representation of time / model (cf. plots in SRM part)}

\todo[inline]{finalize text and note that is shuffled (cf. text in SRM part}

\begin{figure*}[tbp]
\centering
\includegraphics[width=\linewidth]{figures/corr_vis-regressors-vs-cfs_sub-01_srm-ao-av-vis-shuffled_feat10-iter30_7123-7747.pdf}
    \caption{
	%
	\textbf{Similarity of hemodynamic reponses modeled for the analysis of
    visual localizer in \citet{sengupta2016extension} and shared features
    calculated by the shared response model (SRM) for subject 01 in the first
    fold of the cross-validation.}
    %
    Before calculating the Pearson correlation coefficients plotted in the
    figure, the time series of the shared features within the multi-paradigm
    \ac{cfs} were trimmed to match the corresponding \acp{tr} of the visual
    localizer paradigm \citep{sengupta2016extension}.
    %
    The modeled hemodynamic responses represent predicted responses to
    the six categories of pictures that were presented in blocks.
    }

\label{fig:corr-vis-reg-srm-shuffled}
\end{figure*}


\begin{figure*}[tbp]
\centering
    \includegraphics[width=\linewidth]{figures/corr_av-regressors-vs-cfs_sub-01_srm-ao-av-vis-shuffled_feat10-iter30_3524-7123.pdf}
    \caption{
    %
    \textbf{Similarity of hemodynamic reponses modeled for the analysis of the
    movie in \citet{haeusler2022processing}
    and shared features calculated by
    the shared response model (SRM) for subject 01 in the first fold of the
    cross-validation.}
    %
    Before calculating the Pearson correlation coefficients plotted in the
    figure, the time series of the shared features within the multi-paradigm
    \ac{cfs} were trimmed to match the corresponding \acp{tr} of the movie
    \citep{hanke2016simultaneous}.
    %
    The modeled shared responses (i.e. regressors) \texttt{vse\_new} to 
    \texttt{vno\_cut} are based on
    annotations of movie frames, whereas the regressors
    \texttt{fg\_av\_ger\_lr} to \texttt{fg\_av\_ger\_ud} represent low-level
    visual or auditory confounds
    \citep[cf. Table 3 in][]{haeusler2022processing}.
    %
    \texttt{vse\_new}: change of the camera position to a setting not depicted
    before;
    \texttt{vse\_old}: change of the camera position to a recurring setting;
    %
    \texttt{vlo\_ch}: change of the camera position to another locale within
    the same setting;
    %
    \texttt{vpe\_new}: change of the camera position within a locale not
    depicted before;
    %
    \texttt{vpe\_old}: change of the camera position within a recurring locale;
    %
    \texttt{vno\_cut}: a pseudorandomly selected frames within a continuous
    movie shot;
    %
    \texttt{fg\_av\_ger\_lr}: left-right luminance difference;
    %
    \texttt{fg\_av\_ger\_lrdiff}: left-right volume difference;
    %
    \texttt{fg\_av\_ger\_ml}: mean luminance;
    %
    \texttt{fg\_av\_ger\_pd}: perceptual difference;
    %
    \texttt{fg\_av\_ger\_rms}: root mean square volume;
    %
    \texttt{fg\_av\_ger\_ud}: upper-lower luminance difference.
    }
\label{fig:corr-av-reg-srm-shuffled}
\end{figure*}



\begin{figure*}[tbp]
\centering
    \includegraphics[width=\linewidth]{figures/corr_ao-regressors-vs-cfs_sub-01_srm-ao-av-vis-shuffled_feat10-iter30_0-3524.pdf}
    \caption{
    %
    \textbf{Similarity of hemodynamic reponses modeled for the analysis of the
    audio-description in \citet{haeusler2022processing}
    and shared features calculated by
    the shared response model (SRM) for subject 01 in the first fold of the
    cross-validation.}
    %
    Before calculating the Pearson correlation coefficients plotted in the
    figure, the time series of the shared features within the multi-paradigm
    \ac{cfs} were trimmed to match the corresponding \acp{tr} of the
    audio-description \citep{hanke2014audiomovie}.
    %
    The modeled shared responses (i.e. regressors) \texttt{body} to 
    \texttt{sex\_m} are based on
    annotated categories of nouns spoken by the audio-description's narrator,
    whereas the regressors \texttt{fg\_ad\_ger\_lrdiff} and
    \texttt{fg\_ad\_ger\_rms} represent low-level auditory confounds
    \citep[cf. Table 3 in][]{haeusler2022processing}.
    %
    \texttt{body}: trunk of the body; overlaid clothes;
    %
    \texttt{bpart}: limbs and trousers;
    %
    \texttt{fahead}: (parts) of the face or head;
    %
    \texttt{furn}: moveable furniture (insides \& outsides);
    %
    \texttt{geo}: immobile landmarks;
    %
    \texttt{groom}: rooms \& locales or geometry-defining elements;
    %
    \texttt{object}: moveable and countable entities with firm boundaries;
    %
    \texttt{se\_new}: a setting occurring for the first time;
    %
    \texttt{se\_old}: a recurring setting;
    %
    \texttt{sex\_f}: female name, female person(s);
    %
    \texttt{sex\_m}: male name, male person(s);
    %
    \texttt{fg\_ad\_lrdiff}: left-right volume difference;
    %
    \texttt{fg\_ad\_rms}: root mean square volume.
    %
    \texttt{geo\&groom} is a combination of regressors as used on the positive
    side of the primary contrasts aimed to localize the \ac{ppa}
    \citep[cf. Table 5 in][]{haeusler2022processing}.
    }
\label{fig:corr-ao-reg-srm-shuffled}
\end{figure*}






\chapter*{Acknowledgments}
% don't show page number
\pagenumbering{gobble}

% take input from external file
% text
I want to thank...
% family
Vadda; Gabriele; Nico, Anna, Antonia, Julius; Annelie, Johannes;
%
Niklas Lampe, Ken Waigel, Brahim Sokoli, Stefan Mysliwietz, Ilja Rabinovitch,
Gavin Theren, Madhi; Ruwen Jeffe; Brahim Sokoli
% colleagues
Adina Wagner, Alex and Laura Waite, Tobias Kadela, Julia Mahns, Markus Thoma;
% Mitbewohner
Christian Fröhlich, Kyesam Jung, Binny Davis;
% landlord
Mr. Halking; Valeri Kippes
% job
Anika Völkel;
% eating
Troja Pizzeria;
% bands
Heaven Shall Burn, Black Sun Empire, Hatebreed, The Ghost Inside, Emmure,
Babymetal, Lorna Shore ("...And I Return To Nothingness EP", wtf?);
%
And my thanks do not go to the media posting incredibly badly written news articles
but to all the pet guardians who post videos of their adorable cats: you made
me laugh and let me fall asleep with a good mood after long days.
% bosses
Simon Eickhoff, Gerhard Jocham, foremost Michael Hanke
%






%\chapter*{Declaration of authorship}
%% The affidavit (declaration of Authorship) is signed in the
%% "Application for authorization to the doctorate" (Zulassungsantrag).
%% According to the PO, the affidavit is no longer inserted in the dissertation




\end{spacing}
\end{document}
