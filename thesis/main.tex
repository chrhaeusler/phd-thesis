%% LaTex template for a monographic dissertation
%% at the Faculty of Medicine at Heinrich Heine University Düsseldorf

%% author: Christian Olaf Häusler
%% date: January 2022
%% license:
%% Creative Commons Attribution 4.0 International Public License
%% https://creativecommons.org/licenses/by/4.0/

%% This template was created to meet the requirements as provided by:
%% ENG-PhD_Ordnung_vom_05.03.2018.pdf
%% DE-PhD_Ordnung_vom_05.03.2018.pdf
%% 2020-06-19_GZ-Guidelines_for_preparation_of_Dissertations_-_Monograph.pdf
%% 2021-04-01_GZ-Leitfaden_zur_Erstellung_von_klassischen_Dissertationen.pdf


%% requirements that did not fit anywhere else:


%% Copyright information: For all illustrations and tables that you have not
%% created yourself, but have taken over or modified from other sources, a
%% source reference, analogous to the citation of text sources, must be given
%% in the caption. The source reference must be included in the reference
%% list/bibliography.

%% Please note, especially in view of the fact that your dissertation will be
%% published online, that the reproduction of illustrations and tables is subject
%% to copyright. [...] you should always ask the responsible publisher for
%% permission. This also applies to illustrations from publications in which
%% you are listed as the author, as the reproduction rights are usually
%% transferred to the publisher. However, most publishers allow the re-use of
%% parts of your own publications. In the case of journals, this is usually
%% stipulated in a contract. Therefore, please refer to the websites of the
%% respective publishers for information on author rights and licenses. The
%% granting of a licence by a publisher may also be subject to fees.

% font size 11-12pt
\documentclass[english,12pt]{report}
\usepackage[utf8]{inputenc}

% allow abstracts (to be placed at the beginning of chapters)<w
\usepackage{lipsum}
\makeatletter
\newenvironment{chapterabstract}
               {\begin{center}\textbf{Abstract}\end{center}
                 \list{}{
                        %\listparindent 1em%
                        %\setlength{\leftmargin}{<value>} adjust if you need
                        \itemindent\listparindent
                        \rightmargin\leftmargin
                        \parsep\z@ \@plus\p@}%
                \item\relax}
               {\endlist}

% use ä, ü, ...
\usepackage[T1]{fontenc}

% language settings
\usepackage[USenglish]{babel}

% use a common font like Times New Roman or Arial:
% keep LaTex' standard ("Computer Modern")

% margins of the pages;
% left: 30-35mm (due to binding)
% right, top, bottom: 20-25mm
\usepackage[
    a4paper,
    left=30mm,
    right=20mm,
    top=20mm,
    bottom=20mm
]
{geometry}


% manage header, footers, and page numbers
\usepackage{fancyhdr}

% add to dos as comments
\usepackage{todonotes}

% manage spacing between lines
\usepackage{setspace}

% format the chapters' header
\usepackage{titlesec}
\titleformat{\chapter}[hang]{\bf\huge}{\thechapter}{2pc}{}
% other way without showing the number before the chapter title:
%\titleformat{\chapter}[display]{\normalfont\bfseries}{}{0pt}{\Huge}


% citation
% numerical citations:
%\usepackage[numbers]{natbib}
% superscript citations:
%\usepackage[super]{natbib}
% author-year citations:
% \usepackage[round]{natbib}
% APA-style citing and bibiliography
\usepackage[natbibapa]{apacite}

% make the bibliography appear in the table of contents
\usepackage[nottoc,numbib]{tocbibind}

% package for nice tables
\usepackage{booktabs}

% Abbreviate "Figure x" to "Fig. x"
\usepackage{caption}
% \captionsetup[figure]{labelfont={bf},name={Fig.},labelsep=colon}
\captionsetup[figure]{name={Fig.},labelsep=colon}

\captionsetup[table]{name={Table.},labelsep=colon}

% units: provided as a bundle with the nicefrac package for typing fractions.
% Units uses nicefrac in typesetting physical units in a standard-looking way.
\usepackage{units}

% for signs ligh less/equl, greater/equal etc.
\usepackage{amssymb}

% how to handle links in the PDF
\usepackage[
    colorlinks=true,
    citecolor=black,
    urlcolor=black,
    linkcolor=black
]{hyperref}

% enable support for "\begin{comment}" and "\end{comment}"
\usepackage{verbatim}

% manage list of abbreviations
% acronyms/abbreviations are defined in the corresponding chapter (s. below)
% if a given acronym is given by \ac{my-acronym} for the first time in the text
% body its full name will be written and it will show up in the list of
% abbreviation
% \acs{} -> force short version
% \acl{} -> force long version
% \acp{} -> make it plural (adds an "s")
\usepackage[printonlyused]{acronym}

% prevent widows and orphans
% cf. https://en.wikipedia.org/wiki/Widows_and_orphans
\usepackage[all]{nowidow}

% enable rounding of numpers
% e.g. \nprounddigits{3}\numprint{13.9978}$ equals "13.998" in the pdf
\usepackage{numprint}


\begin{document}
% set pagestyle to 'no header' and 'page number at bottom in the middle'
\pagestyle{plain}
% don't show page number (for now)
\pagenumbering{gobble}
% set spacing between lines to 1.5
\begin{spacing}{1.5}




% create the title page
\begin{titlepage}
\begin{center}

From the Institute Institute of Neuroscience and Medicine,\\
Brain \& Behaviour (INM-7),\\
at Research Centre Jülich, Jülich, Germany

\vfill

\textbf{{\large Exploring naturalistic stimulus paradigms as an alternative to a
    task-based functional localizer paradigm}}

maybe add subtitle

\vfill

{\large Dissertation}

\vfill

to obtain the academic title of Doctor of Philosophy (PhD) in Medical Sciences\\
from the Faculty of Medicine at Heinrich Heine University Düsseldorf

\vfill

submitted by\\
\textbf{Christian~Olaf~Häusler}\\
(2023)

\end{center}
\end{titlepage}




%\chapter*{Examiner Details}
\newpage
%pagebreak[4]

\noindent This page 2 will not be included in the examination copies, but will
only be inserted in the copies for publication after the review process and
the oral examination. You will be notified of the exact text for page 2 once
print approval is granted.

\vfill
\noindent As an inaugural dissertation printed by permission of the\\
Faculty of Medicine at Heinrich Heine University Düsseldorf

\vspace*{\fill}
\noindent signed:\\
dean:\\
examiner: Name A\\
co-examiner: Name B



%\chapter*{Dedication}
\newpage
%pagebreak[4]

\begin{center}
\null\vspace{\stretch{1}}
    \textit{``Gladius ultor noster! Pectus amico, cuspis hosti!''}\\
    \hspace{0.3\textwidth} --- old studential motto
\vspace{\stretch{2}}\null
\end{center}




%\chapter*{List of Publications}
\newpage
%pagebreak[4]

%% Own publications are listed here.
%% Submitted manuscripts or manuscripts under revision are not listed here.
%% These will be listed in the “Application for authorization to the doctorate”
%% (Zulassungsantrag).
%% Please note that you must include a reference for all information you take
%% from your own publications.
%% The publications mentioned on page 4 must therefore also be listed in the
%% bibliography.
%% Direct quotations must be marked with quotation marks.

\vspace*{\fill}

\noindent \textbf{Parts of this work have been published:}\\

\noindent
%
Halchenko, Y. O.,
Meyer, K.,
Poldrack, B.,
Solanky, D. S.,
Wagner, A. S.,
Gors, J.,
MacFarlane, D.,
Pustina, D.,
Sochat, V.,
Ghosh, S. S.,
Mönch, C.,
Markiewicz, C. J.,
Waite, L.,
Shlyakhter, I.,
de la Vega, A.,
Hayashi, S.,
Häusler, C. O.,
Poline, J.-P.,
Kadelka, T.,
Skytén, K.,
Jarecka, D.,
Kennedy, D.,
Strauss, T.,
Cieslak, M.,
Vavra, P.,
Ioanas, H.-I.,
Schneider, R.,
Pflüger, M.,
Haxby, J. V.,
Eickhoff, S. B.,
\& Hanke, M.
%
(2021).
%
DataLad: distributed system for joint management of code, data, and their
relationship.
%
Journal of Open Source Software, 6(63), 3262.
%
doi: \href{https://doi.org/10.21105/joss.03262} {10.21105/joss.03262}.\\

\noindent
%
Häusler, C. O.,
%
\& Hanke, M.
%
(2021).
%
A studyforrest extension, an annotation of spoken language in the German dubbed
movie ``Forrest Gump'' and its audio-description.
%
F1000Research, 10(54).
%
doi: \href{https://doi.org/10.12688/f1000research.27621.1}
{10.12688/f1000research.27621.1}.\\

\noindent
%
Häusler, C. O.,
%
Eickhoff, S. B.,
%
\& Hanke, M.
%
(2022).
%
Processing of visual and non-visual naturalistic spatial information in the
``parahippocampal place area''.
%
Scientific Data, 9(1).
%
doi: \href{https://doi.org/10.1038/s41597-022-01250-4}
{10.1038/s41597-022-01250-4}.




\chapter*{Zusammenfassung}
% show page number, set it to Roman number
\pagenumbering{Roman}
% set page counter to 1
\setcounter{page}{1}

%% The German and English summary should be exactly one page long each.
%% The summary represents a coherent text. Please do not use subheadings.
%% should include the following content in a shortened form:
%% - scientific background and current state of research
%% - research question and objectives
%% - methodology
%% - results
%% - discussion and conclusions.

%% The summary is needed twice:
%% a) Bound into your dissertation.
%% b) As a pdf-document, submitted together with the application for
%%    admission to doctoral examination proceedings.

% take input from external file
% wissenschaftlicher Hintergrund
Lore ipsum...
% aktueller Forschungsstand

% Fragestellung und Ziele

% Methodik

% Ergebnisse

% Diskussion

% Schlussfolgerungen





\chapter*{Summary}
% take input from external file
\todo[inline]{to be written}

% scientific background
text in summary-en.tex
% current state of research

% research question and objectives

% methodology

% results

% discussion

% conclusions

% lore ipsum






\chapter*{List of Abbreviations}
%% The list of abbreviations contains and explains all abbreviations used in
%% the thesis (except for common linguistic abbreviations as defined in the
%% dictionary, e.g. Duden, Merriam-Webster).
%% In the case of physical or chemical quantities, it is also necessary to
%% specify the unit.
%% An alphabetical order is useful, a subdivision (e.g. SI units, own
%% abbreviations) is possible.
%% For example, you can format the page in two columns, the abbreviations in
%% bold and the corresponding explanations in non-bold.}



% decrease line spacing
\renewcommand{\baselinestretch}{0.75}\normalsize

% definition table of acronyms
% see documentation on how if your acronym needs special formatting, and
% the unit needs to be added to the list of abbreviations
% set space between acronym and full definition to [longest]
\begin{acronym}[longest]
    \acro{aa}[AA]{anatomical alignment}
    \acro{bold}[BOLD]{blood oxygen level-dependent}
    \acro{cas}[CAS]{common anatomical space}
    \acro{cms}[CMS]{common model space}
    \acro{cfs}[CFS]{common functional space}
    \acro{eba}[EBA]{extrastriate body area}
    \acro{eeg}[EEG]{electroencephalography}
    \acro{fa}[FA]{functional alignment}
    \acro{ffa}[FFA]{fusiform face area}
    \acro{fov}[FoV]{field of view}
    \acro{glm}[GLM]{general linear model}
    \acro{fmri}[fMRI]{functional magnetic resonance imaging}
    \acro{hrf}[HRF]{haemodynamic response function}
    \acro{loocv}[LOOCV]{leave-one-out cross-validation}
    \acro{loc}[LOC]{lateral occipital complex}
    \acro{meg}[MEG]{magneto\-encephalo\-graphy}
    \acro{ofa}[OFA]{occipital face area}
    \acro{opa}[OPA]{occipital place area}
    \acro{pca}[PCA]{principal component analysis}
    \acro{pet}[PET]{positron emission tomography}
    \acro{ppa}[PPA]{parahippocampal place area}
    \acro{roi}[ROI]{region of interest}
    \acro{rsc}[RSC]{retrosplenial complex}
    \acrodefplural{roi}[ROIs]{regions of interest}
    \acro{tr}[TR]{time of repetition}
    \acro{srm}[SRM]{shared response model}
    \acro{snr}[SNR]{signal-to-noise ratio}
\end{acronym}



% set line spacing back to normal
\renewcommand{\baselinestretch}{1}\normalsize




\tableofcontents
%% The entire table of contents should not be longer than 2 pages.
%% A chapter with only one subchapter is not allowed

% imo it makes more sense in following order:
% ToC, list of abbreviations, list of tables, list of figures, actual text



%\listoftables
%% List of tables is not mandatory
%% if not "not allowed"




%\listoffigures
%% List of figures is not mandatory
%% if not "not allowed"




\chapter{Introduction}
% set format of page numbers to arabic numbers
\pagenumbering{arabic}
% set page counter to 1
\setcounter{page}{1}

% take input from external file
\section{Overview}
%% The introduction begins with an overview of the topic

\todo[inline]{some sentences are kinda quotes from other paper; e.g. Dubois?}

\todo[inline]{other contrasts: phonemes, grammatical tags, prosody, sex}

\todo[inline]{sections are supposed to be numbered}

%
Brain imaging with \ac{bold} \ac{fmri} has been used extensively for almost
three decades to investigate perceptual and cognitive brain functions.
%
Typical analysis procedures average (voxel-wise) data of at least 10-15 subjects
to improve the \ac{snr}.
%
Consequently, these studies do not characterize brain function at the level of
an individual.
%
It is plausible to assume that models of brain functions that are based on a
lowest common denominator approach only capture a fraction of individual
functional brain properties.
%
Therefore, those models are an incomplete foundation for an investigation of
inter-individual differences.
%
However, characterization of individual brain function is by far the most
important application of BOLD fMRI in a clinical context.
%
For example, for pre-surgical screening, or diagnosis of brain function in
health and disease.
%
The goal of the proposed project is to pave the way towards adopting an approach
to the investigation of individual brain function that is based on individual
differences with respect to large normative samples --- a proven strategy that
has been standard in psychological diagnostics and other clinical research for a
long time.\todo{give examples}


\section{Introductory remarks}
%% introductory remarks describe the scientific background of the work as
%% precisely as possible. Cite the most important publications and avoid
%% extensive literature reviews.


\subsection{State of research}


\subsubsection{Functional localization}

\todo[inline]{we don't do speech lateralization anymore; imo the neurosurgery
thing should not be mentioned in the intro anymore; better come up with it in
the general discussion}

%
The most frequently employed paradigm for characterizing individual brains with
BOLD fMRI are so called functional localizers.
%
Functional localizers aim at isolating and localizing brain activity correlated
with specific perceptual processes (e.g. different object categories;
\citet{kanwisher1997ffa}) or cognitive processes (e.g. theory of mind;
\citet{spunt2014validating}).
%
Typically, localizers are dedicated measurements that are used to define
individual \acp{roi}.
%
For example to improve the statistical power of the main experiment's analysis
or to locate brain functions prior to neurosurgery.
% side-effects of neurosurgery
Surgical procedures might impact the post-operative quality of life so much
(e.g. concerning cognitive control or speech production) that it potentially
outweighs the therapeutic benefits.
%
The challenge is to precisely localize relevant brain areas with limited
resources (time, availability and applicability of diagnostic measures for an
individual patient) in order to correctly predict the impact of the planned
procedure.
%
Importantly, functional localizers, despite being tuned for detection power,
quickly become inefficient if one wants to map many different processes in a
limited amount of time.
%
One example of a time-efficient multi-functional localizer for reading, language
comprehension, calculation, motor response, and basic retinotopy was developed
by \citep{pinel2007fast}).\todo{Thirion's work?}
%
It employs a range of dedicated stimuli and specific tasks participants have to
perform in a 5-minute routine.
%
The diagnostic quality of such paradigms relies heavily on participants'
compliance and comprehension of the task instructions, a criterion that can be
difficult to meet in certain target populations, such as patients with dementia.

\todo[inline]{s. \citep{vanderwal2015inscapes}}

\subsubsection{Anatomical alignment (in order to predict)}

The currently dominating approach in neuroimaging group analyses relies on
topological constraints defined by an three-dimensional, anatomical reference
space (e.g. the MNI152 template brain).
%
Surface-based: e.g. \citep{weiner2018defining}.


\subsubsection{Functional alignment (in order to predict)}

\paragraph{Functional alignment in general}
%
An alternative approach to individual localization has been proposed by
\citet{haxby2011common}.
%
They (and e.g. \citet{jiahui2019predicting}) also predicted the location, form,
and size of  target brain areas in ventral temporal cortex from dedicated
localizer scans of other individuals.
%
The key difference of this approach is to rely on similarity of representational
geometry of brain activity patterns and aligning individual brains into a
multi-dimensional a group space.

\todo[inline]{shift from focussing on (connectivity) hyperaligment more to
shared response model}


\paragraph{Hyperalignment}

\todo[inline]{why didn't we used connectivity alignment in the first place?}
%
\citet{haxby2011common} used BOLD response patterns evoked by a 2h action movie
to derive a common representational space.
%
The algorithm, namely hyperalignment, derives this representation using a
variant of Procrustes analysis and computes invertible (orthonormal)
transformations from each individual brain’s voxel-space into this common
reference space.
%
Importantly, the study also showed that an individual's \ac{ffa} or the
\ac{ppa}, can be localized precisely based on data from a reference group.
\todo{explain FFA, PPA}
%
The same authors later showed that this approach can be extended to predict
functional organization across large proportions of the cortical surface, for
example to predict the represented visual field coordinate in visual cortex
based on retinotopic mapping scans of other individuals
\citep{guntupalli2016model}.

\todo[inline]{we do not do connectivity hyperalignment but SRM now!}
%
In his doctoral thesis, recently submitted to the Faculty of Natural Sciences in
Magdeburg, Falko Kaule showed that congruent time-locked BOLD responses across
subjects (i.e. all subjects watching the exact same full-length movie) as used
by Haxby and colleagues are not required to derive a valid alignment of
individuals with a common representational space \citep{kaule2017examination}.
%
Comparable prediction performance can be achieved by using \textbf{functional
connectivity patterns} (correlation of a voxel's time series with reference
regions in the same brain).
%
This finding enables, in principle, the use of different ``calibration'' scans
to determine an alignment with a common representational space, for example with
an age-appropriate stimulus, or a shortened scan time to fit into a particular
clinical schedule.
%
Once a valid alignment is established, known functional properties of a
(normative) reference, derived from extensive scans and analysis of other
subjects, can then be projected into the respective individual voxel space (s.
Fig. 1 in \citep{nishimoto2016lining}).


\paragraph{Shared response model}

\citep{chen2015reduced}




\subsubsection{Naturalistic stimuli in neuroscience}

\todo[inline]{cf. PuG talk}


\paragraph{Intro}

One major goal of cognitive neuroscience is to reveal how the brain processes
information during everyday perception.
%
Traditionally, human brain mapping studies used carefully controlled experiments
and rudimentary stimuli that lack ecological validity.
%
For example, previous studies presented isolated higher-level visual features
such as scenes, faces, human bodies or tools.
%
Results suggest that domain-specic modules like the \ac{ppa}
\citep{epstein1998ppa}, \ac{ffa} \citep{kanwisher1997ffa}), the occipital face
area \ac{ofa} \citep{pitcher2011occipitalfacearea}, the \ac{eba}
\citep{downing2001bodyarea}), and the \ac{loc} \citet{malach1995loc} exist in
the human brain.
%
As a consequence of investigating perceptual and cognitive functions by
utilizing isolated stimuli, studies subdivided the cerebral cortex into
distinctive functional areas whose \ac{bold} activity is specifically correlated
with one particular simplied stimulus type.
%
However, the question remains how those functional areas behave in lifelike
situations and how they might interact.
%
After all, we do not experience the
world around us as separated into small unidimensional stimuli, but perceive ---
through different senses --- a seemingly continuous and unified world.
%
To address
this open question, usage of so called naturalistic stimuli have gained
popularity.


 This


\paragraph{Definition}

\todo[inline]{time-locked events}

\todo[inline]{shorten but add narrative studies; synchronization is also true
for narratives}
%
Naturalistic stimuli are ``a class of stimuli that aim to evoke more
naturalistic patterns of neural responses than traditional controlled artificial
stimuli. Naturalistic paradigms are typically complex and dynamic, and longer
in duration than many conventional stimuli.'' \citep{vanderwal2019movies}.

%
They ``impose a meaningful timecourse across subjects while still allowing for
individual variation in brain activity and behavioral responses, and lend
themselves to a broader set of analyses than either pure rest or pure
event-related task designs.'' \citep{finn2017can}

%
Numerous studies have shown that watching a movie leads to correlated
time-locked brain responses across subjects in many brain regions, and to
synchronized eye movements \citep{hasson2010reliability, lankinen2014isc-meg}.
%
This can be attributed to the way professional movies are shot and edited in
order to intentionally manipulate the viewers' attentional focus and mental
states \citep{brown2012cinematography, dancyger2011film-technique}.

Movies have been used during \ac{fmri} \citep{bartels2004mapping,
hasson2004intersubject}, \ac{eeg} \citep{dmochowski2014audience,
krause2000relative}, simultaneous \ac{eeg}-\ac{frmi}
\citep{whittingstall2010integration}, or \ac{meg} \citep{lankinen2014isc-meg,
luo2010auditory}.
%
The underlying assumption is that movies watched in a laboratory setting offer a
more complex and continuous stimulation that better mimics our natural dynamic
environment.
%
Indeed, studies have shown that freely watching a movie leads to synchronized
spatiotemporal responses across multiple subjects in a large part of the brain
\citep{hasson2010reliability, lankinen2014isc-meg}.\todo{but still different}
%
Professionally produced movies evoke inter-subject correlations (ISC) in more
parts of the brain than, for example, an unedited video of a concert, taken from
a fixed viewpoint \citep{hasson2010reliabilitiy}.
%
This finding could be attributed to a film director's goal to not only direct a
movie, but also to capture and direct the audience's attention
\citep{brown2012cinematography, dancyger2011film-technique}.


\todo[inline]{reviews on narratives: \citep{hamilton2018revolution}, more M/EEG
\citep{alday2019meg}}


\paragraph{Higher validity}
%
Findings suggest that naturalistic stimuli offer a higher ecological validity
\citep{hasson2004intersubject} because they better mimic statistics in our
natural dynamic environment.
%
Further they offer a higher external validity \citep{westfall2016fixing})
because they sample the stimulus space ``better''.\todo{iykwim}
%
''a RSM is only indicated when the stimuli used in the study do not fully
exhaust the theoretical population of stimuli that might have been used \citep{westfall2016fixing}.
%
because selectively sample from the stimulus population leading to a
stimulus-as-fixed-effect fallacy \citep{westfall2016fixing}.(Clarc, The
language-as-fixed-effect fallacy: A critique of language statistics in
psychological research).
%
The conclusions cannot be generalized to a broader population of stimuli without
risking inflated Type I error  (cf. Donnet S, Lavielle
M, Poline JB: Are fMRI event-related response constant in time? A model
selection answer\citep{westfall2016fixing)}.
%
While the functional alignment can also be applied to fMRI data from stimulation
paradigms with simplified stimuli, the transformations for functional alignment
have greatly diminished general validity \citep{haxby2011common}, presumably
because such experiments sample a sparser range of brain states
\citep{guntupalli2016model}.

%
Lastly, increased validity of derived transformation for functional alignment by
sampling a more diverse set of mental states that reflect (confound) statistics
of the natural environment, and enable investigation of the acquired data for a
variety of research questions (e.g. visual or auditory perception, spatial
cognition; emotion; music, speech or social perception)


\paragraph{Better compliance}

\todo{check also \citep{eickhoff2020towards}}
%
The to be diagnosed individuals were scanned while they were watching a movie
without any explicit task or.
%
Improved subject compliance and compatibility due to minimal instruction
requirements (e.g., no fixation of eye gaze) and task demands (no task except
enjoying the movie or audiobook).
%
These minimal instructions make naturalistic paradigms appropriate especially
for elderly or visually impaired persons.
%
Lastly, naturalistic stimuli offer improved data quality, as an interesting and
easy-to-follow stimulus is more capable of putting a participant at ease in the
otherwise claustrophobic, uncomfortable and noisy fMRI scanner.

``Relative to traditional fMRI experiments that typically use highly controlled
stimuli, naturalistic stimuli are more ecologically valid (Zaki and Ochsner,
2009; Hasson and Honey, 2012; Adolphs et al., 2016; Hamilton and Huth, 2018),
convey rich perceptual and semantic information (Bartels and Zeki, 2004; Huth et
al., 2012, 2016) and more fully sample neural representational space (Haxby et
al., 2011, 2014)''\citep{nastase2019measuring}.

``Recent work (Vanderwal et al., 2015) also suggests that naturalistic stimuli
may improve subject compliance (in terms of wakefulness and head motion relative
to, e.g. rest), which is particularly important when scanning patient
populations and children. As mentioned previously, different stimuli will
variably synchronize different brain systems; for example, engaging,
Hollywood-style movies may yield greater, more widespread ISCs than real-life,
unedited videos (Hasson et al., 2010; Cohen et al., 2017)''
\citep{nastase2019measuring}.


\todo[inline]{imo following part can heavily be shortened}

\paragraph{studyforrest}
%
The studyforrest project is an open science project that aims at providing a
versatile resource for investigating human brain function under quasi-natural
conditions.
%
The core of this dataset are two hour long BOLD fMRI scans of participants
watching the movie Forrest Gump (and also listening to a version for the blind
in another scan of equal length).
%
Since its first publication in 2014 \citep{hanke2014audiomovie}, this resources
has led to eight independent studies of international research groups outside
Magdeburg that were published in peer-reviewed journals
(http://studyforrest.org).
%
Based on my knowledge about cinematographic editing techniques, I annotated the
~870 movie cuts with respect to depicted major locations, scene settings,
within-scene rooms and perspectives \citep{haeusler2016cutanno}.
%
The annotation served as prerequisite to investigate cognitive functions, such
as spatial reorientation, perspective taking, and memory retrieval for known
spatial layouts in terms of their occurrence in the movie stimulus.

For 15 participants in the studyforrest datasets two different full-length movie
scans are readily available: one with the normal audio-visual movie
\citep{hanke2016simultaneous}, and a second one with an audio-only movie variant
(originally produced for a visually impaired audience) that is time-locked to
the audio-visual version \citep{hanke2014audiomovie}.

``A case study for successfully sharing naturalistic stimuli is studyforres t.org
(Hanke et al., 2014, 2016), a dataset which includes audio-only and audio-visual
viewings of the movie Forrest Gump during fMRI acquisi- tion. Study Forrest is a
data collection and curation effort designed to serve as a community resource
for new discoveries, in the tradition of distributed science collaborations such
as the International Genetically Engineered Machine competitions. As of October
2019, 29 unique studies had been published using the studyforrest.org dataset,
17 of which were published without any of the original authors of the data
release. This is possible in large part thanks to the richness of naturalistic
stimuli, where the same movie can be used for both task-free as well as
stimulus-driven analyses, with the original stimulus re-annotated for particular
features of interest. For example, studyforrest.org has been used to test
cerebro- vascular biomarkers (Voss et al., 2017) but, among other features, was
also annotated for expressed emotion (Labs et al., 2015) which later informed a
study on emotion encoding gradients in the brain (Lettieri et al., 2019).A case
study for successfully sharing naturalistic stimuli is studyforres t.org (Hanke
et al., 2014, 2016), a dataset which includes audio-only and audio-visual
viewings of the movie Forrest Gump during fMRI acquisi- tion. Study Forrest is a
data collection and curation effort designed to serve as a community resource
for new discoveries, in the tradition of distributed science collaborations such
as the International Genetically Engineered Machine competitions. As of October
2019, 29 unique studies had been published using the studyforrest.org dataset,
17 of which were published without any of the original authors of the data
release. This is possible in large part thanks to the richness of naturalistic
stimuli, where the same movie can be used for both task-free as well as
stimulus-driven analyses, with the original stimulus re-annotated for particular
features of interest. For example, studyforrest.org has been used to test
cerebro- vascular biomarkers (Voss et al., 2017) but, among other features, was
also annotated for expressed emotion (Labs et al., 2015) which later informed a
study on emotion encoding gradients in the brain (Lettieri et al., 2019)''
\citep{dupre2020nature}.


\section{Aims of thesis}
%% At the end of the introduction, add a subchapter on the Aims of Thesis in
%% which you describe the research question and the objectives of your work on
%% a maximum of two pages


\subsection{Overview of aims}
%
Previous work has shown that it is, in principle, possible to combine BOLD fMRI
with a rich naturalistic stimulus for the purpose of localizing areas associated
with particular brain functions \citep{bartels2004mapping}.
%
This can be approached by either modeling specific stimulus features, or by
using the high-dimensional nature of such a stimulus to derive a common
representational space for aligning functional properties of cortex.
%
However, there has been no study aimed at replacing an established localizer
paradigm with a naturalistic stimulus as a short diagnostic routine.
%
A part of a movie of the same length as a dedicated localizer experiment would
have substantial advantages (e.g. task demands compliance, and data quality)
over simplified stimuli presently used in localizer paradigms.
%
Given that naturalistic stimuli offer a rich stimulation correlating with a
variety of different brain functions, they could replace multiple dedicated
localizers and thus offering a more comprehensive and efficient diagnostic in a
similar or even less amount of time.



\subsubsection{Auditory stimulus to localize visual area}
%
This investigation will also provide insight if it is feasible to produce an
audio-only localizer paradigm as an alternative for studies where no visual
stimulation is feasible or desired.
%
Previous results show that there are significant voxel-wise BOLD response time
series correlations between the two datasets in brain areas associated with
speech and story processing (s. Figure 3 in \citep{hanke2016simultaneous}).

%
\todo[inline]{following is actually not true anymore}

For this purpose, I want to evaluate two strategies:

\begin{itemize}
    \item direct modeling of a natural stimulus and
    \item prediction via functional alignment to a reference population
\end{itemize}


\subsubsection{Reproducibility, transparency, openly shared}

\todo{transparency, reproducibility, sharing, check \citep{halchenko2021datalad}}
%
All required algorithms and analyses will be implemented in a way that enables
automated processing.
%
This will facilitate documentation and efficient processing of large number of
datasets.
%
Implementations will be based on open-source software tools to guarantee a
maximum level of reproducibility, and relative ease of long-term maintenance
\citep{eglen2017toward}.
%
The goal is to provide researchers with efficient, validated, ready to use
solutions (stimulation, MR sequence configuration, analysis software) for
functional localization that can be incorporated into their study protocols.
%
We share results and code in standardized file and data formats und use free and
open-source software (FSL, Python packages like ...). Only use publicly
available input data (. \citep{glen2017toward}).


``Naturalistic approaches have a strong potential to further transparent and
openly shared neuroscientific research.  The richness of the data sets collected
inherently favours the analysis of data to address multiple questions or their
reanalysis to address questions other than those exam- ined in the initial
analysis. For example, if one researcher conducts an experiment using stories
and is mainly inter- ested in syntactic processes, the nature of the stimuli
opens up the possibility for other researchers to model phonological processes,
for example, if the data is openly accessible and annotated appropriately. A
stan- dardised way of sharing data from naturalistic exper- imental paradigms is
needed, in order to ensure an easy navigation through the “maze” of openly
shared data and an informed decision regarding which datasets are suitable for
answering specific hypotheses. Ideally, researchers would not only share the
neuroimaging/ electrophysiological data, but also the details of the para- digm
including specific time indications, task descrip- tions and the stimulus files as
presented to the participant. This proposal is different from previous task
ontologies (Poldrack & Gorgolewski, 2014; Turner & Laird, 2012) in that it
captures the specifics of a more eco- logically valid approach and the use of a
natural task, a design which has not yet been incorporated into existing task
ontologies'' \citep{kandylaki2019story}.

%
``A visionary aim of openly sharing data and meta-data of more ecologically
valid designs is the holistic under- standing of human brain function.
Researchers would be able to choose carefully from correctly tagged data- sets
and model brain responses using big data. Then, assisted by current methods in
computer science such as machine learning and artificial neural networks,
researchers could attempt to re-construct brain function in an ecologically
valid manner (see also Hasson et al., 2018)'' \citep{kandylaki2019story}.


\subsubsection{Study 1: Annotation of audio-description}


\todo[inline]{why we need an annotation of speech}

%
Movies are designed to entertain the audience and not to conduct research.
%
Variables (i.e. the ``features'' embedded in the naturalistic stimuli) might be
confounded.
%
Moreover, features of interest might be highly correlated, making a it
impossible to controlling them statistically.
%
Hence, researchers often rely on data-driven methods to analyze the data
``because they do not require an explicit model of the task or stimulus'' and/or
``constructing such a model may be prohibitively difficult.''
\citep{nastase2019measuring}.
%
Which is true but the wrong mindset that lead to a lack of annotation in most
datasets derived from naturalistic paradigms.
%
From data-driven approaches, we gained a lot of knowledge, but data-driven
approaches essentially fall short in case you want to correlate discovered brain
activation patterns with psychological processes.
%
``annotation bottleneck'' \citep{aliko2020naturalistic}i.

``Nevertheless, naturalistic stimuli add additional layers of complexity to the
technical difficulty of data sharing. Some of these – such as recording
acquisition and presentation timings as well as annotating stimuli for features
of interest – are well-recognized from the task-based neuroimaging literature
and implemented in existing standards such as BIDS'' \citep{dupre2020nature}.


\subsubsection{Study 2: task-based vs. visual vs. auditory}
%
It is likely that a rich natural stimulation evokes more widespread
network activity, compared to a simplified stimulus that additionally requires
the subject to perform a task.
%
Hence: compare task-free localizer to dedicated task-based speech localizer
%
compare diagnostic performance of the task-free movie / audio-description fMRI
recordings will be compared to the results of the localizer paradigm.
%
a specific diagnostic contrasts can be selected.

\subsubsection{Study 3: Functional alignment \& stimulus length}
%
All analyses up to this point will have been performed on 2h-long
scans, but any clinical application must aim to minimize the scan time/cost.
%
We test wether reliable functional alignment can be achieved with a task-free,
natural stimulation ``calibration” scan that requires no more acquisition time
than a conventional localizer paradigm.
%
We will estimate the trade-off between diagnostic quality
and required effective scan time by progressively reducing the duration of input
BOLD fMRI data and comparing the results of the reduced model to the reference
computed from the full length scan.

\paragraph{Predict ROI from a reference group}
%
fMRI data acquired from an individual will not be analyzed directly regarding a
specific cognitive function, but will be used to align that individual brain's
voxel space with a common high-dimensional representational reference space.
%
Each axis in this common space can be seen as a kind of cortical tuning
function.
%
The orthonormal transformation of an individual voxel space into this common
reference reflects the particular linear combination of voxel response time
series with respect to each common space component.
%
Using a leave-one-subject-out strategy, individual results of the conventional
localizer will then be projected into the common space, aggregated, and
re-projected into the voxel space of the left-out individual for comparison with
the localizer results for that individual (see Figure 2).
%
Using movie-evoked brain activity, hyperalignment procedure learns subject-wise
optimal transformations of brain activity into a common representational space.
%
Once aligned, the brain activity of the reference group can be used to predict
the activity of another subject.
%
The localization of speech areas will also be performed in the common reference
space directly (the stimulation time axis is preserved in the common space), and
the results will be projected into the voxel space of the left-out individual,
where they can be compared with the localisation result derived from that
subject’s movie scan.
%
Once a valid alignment is established, the inverse transformation is then used
to project functional properties of the common reference into that individual's
voxel space.


\paragraph{minimum length of stimulus; ``calibration scan''}
%
minimum data requirement for reliable functional alignment (within-subject
test-retest???
%
estimate the minimum scan time requirement for deriving a valid functional
alignment by further reducing the amount of input data of the left-out
individual.
\paragraph{just uses an intersecting time-series between stimulus sets}
%
The joint dataset will enable me to study whether a functional alignment to the
common reference space can be performed based on a short scan.






\chapter{DataLad: distributed system for joint management of code, data, and
their relationship}
% take input from external file
\todo[inline]{what is supposed to be written here?}

\todo[inline]{Xiaojin: introduction ends on page 10, then one page with
references to the paper, discussion start with page number 14?}

\todo[inline]{Anna Plachti: Mathematisch-Naturwissenschaftliche Fakultät hat Text und Abbildungen der Paper 1 zu 1 (?) in die Diss kopiert; ist dann natürlich anders Formatiert als in der Veröffentlichun}






\chapter{A studyforrest extension, an annotation of spoken language in the
German dubbed movie ``Forrest Gump'' and its audio-description}
% take input from external file
text in study2.tex






\chapter{Processing of visual and non-visual naturalistic spatial information in
the ``parahippocampal place area''}
% take input from external file
\section{Abstract}
%
Usually researchers conduct dedicated experiments often
accompanied with a task (so called "functional localizer") to map perceptual or
cognitive functions onto brain areas of study participants.
%
Nevertheless, the approach "one paradigm, one brain function" becomes
unfeasible if one wants to map a variety of functions in a time-efficient
manner.
%
Currently, we explore a way to project brain mapping data (statistical z-maps)
from a reference group onto individual brains of study participants after
performing a functional aligning of participants with a common model space.
%
Data were obtained from conducting one functional localizer paradigm and two
paradigms using naturalistic stimulation during functional magnetic resonance
imaging (fMRI): participants took part in a task-based, block-design visual
localizer, and participants were watching an audio-visual movie and listening
to the movie's audio-description free of any task.
%
Based on these data, we created a common model space employing a shared
response model (SRM; Chen, 2015).
%
On the one hand, the common model space allows denoising data from individuals
that were used to create the common model space.
%
On the other hand, data from left-out subjects can be aligned with the common
model space by using a (preferably short) segment of a naturalistic stimulus as
a "diagnostic run", a process that provides a subject-specific transformation
matrices.
%
The inverse of the acquired transformation matrices can then be used to project
data from other paradigms aligned in common space onto the brain of the
left-out subjects.
%
The general goal of the project is to assess the required length of the
diagnostic run, and to compare empirical z-maps with z-maps that were predicted
from other participants' data.
%
We present preliminary results of predicted z-maps gained from the same
paradigm as the diagnostic run as well as across paradigms.


\section{Introduction}


\todo[inline]{vgl.bash-history}

main idea: predict location from other participants' data,problem: individual
differences in, functional-anatomical correspondence, does functional alignment
improve the prediction? is a short diagnostic run "sufficient" for alignment?
%
Why does it makes sense to predict the location of a functional area in the
brain of one subject, based on the location of that area in a reference group?
%
How can this prediction be improved if you do not just rely on an anatomical
aligment but on a functional alignment

%
Topographic mapping:
%
Brain mapping maps brain functions, perceptual or cognitive processes, onto
brain areas.
%
Usually, this is done by letting study participants perform a task during a so
called ``functional localizer''.
%
The first problem with these localizers is: they mostly need the participants
to perform a task, and they are just plain boring which often leads to a
diminished compliance.
%
The second problem is: you usually need one localizer for one domain of brain
functions
%
Which means: if you want to map a variety of domains, you need a variety of
localizers, and then the whole approach gets time-consuming and inefficient
(localizer batteries; e.g. Thirion's work)
%
focussing on the domain of so called ``higher-visual, category-selective
areas''
%
How do you usually map these category-selective areas onto brains of individual
study participants?
%
You conduct a functional localizer experiments, lasting about 20 minutes,
%
You let the participants watch pictures of different categories, while also
letting them perform a task to force them to pay attention to picture, after
picture, after picture
%
After having collecting the data you do some statistics, and as a result you
get statistical maps, t- or z-map, that essentially tell you where the
higher-visual areas are located
%
If we need about 20 minutes for just one domain, why don’t we predict the
location of a functional area in one person from the location that we found in
other persons?



\subsection{Prediction from anatomy and CMS}

We can do this in basically two ways:
a) we do this by performing an anatomical alignment,
or
which is a new a approach:
b) we can do this by performing a functional alignment

First, the case of an anatomical alignment,
- anatomical alignment aligna vertices or voxels
- in 2-dimensional or 3-dimensional, anatomical space
- which means, that we transform the shape of an individual brain
  into the shape of an average, standard brain

but when we want to predict functional areas from anatomically aligned brains,
we are running into the problem of poor
functional-anatomical correspondence.

Poor functional-anatomical correspondence means:
even if we assume that two persons have anatomically identical brains, still,
the location, size \& shape of the functional area is probably different.

-------------------------------------
Here, as an example the location of the „PPA“,
the Parahippocampal Place Area.
which is more active when we look at pictures of scenes or landscapes.

Here, we see the location of the PPA in 14 subjects,
and the brighter the voxel the more subjects have their PPA in that location.

You can see that the location is roughly similar across persons, but still,
the plot shows that there is individual variation.

Speaking of individual variation as a side note:
In case you are interested in individual differences,
poor functional-anatomical might be an issue:

- do you really find differences in individual brain patterns or
  do differences stem from differences in
  functional-anatomical correspondence?
- and vice versa: we might find no differences because of
  poor functional-anatomical correspondence

but back to the current case:
we do not want to predict brain anatomy,
but we want to predict the location of brain functions,
so wouldn’t it make more sense to not
perform an anatomical alignment but a functional alignment?


\subsection{functional alignment}

well, it does makes sense.

What functional alignment does, is the following.
Functional alignment...
- aligns cortical patterns, meaning time-series (or connectivity profiles)
- in a multi-dimensional function space
- and here the reference is not an average brain anatomy
  but a so called „common model space“ or „CMS“ for short

There are two algorithms dominating the field.
And these are algorithms are
- Hyperalignment, and the
- Shared Response Model, that we use in the current project

Here again as a side note:
If you want to know more about functional alignment in context of more
fine-grained individual differences.
You should really take a look at these two papers
Don’t worry, I will show the references again at the end


To give you a rough understanding how functional alignment works:

You let study participants watch a naturalistic stimulus like a movie or an audio-book...
...which are continuous, naturally engaging,
and are triggering a wide variety of time-locked brain functions

From that stimuli, you get the time-series per voxel and per subject,
and what the algorithm essentially does is:
it learns responses that are shared across participants
(which is the common model space)

but more importantly, the algorithm also learns individual transformation matrices.

These transformation matrices can be used to project data into
and out of the Common Model Space.


\subsection{Aims \& hypotheses}

We wanted to predict the location of the PPA based on data from a reference group

And we wanted to compare the prediction performance based on
anatomical alignment \& functional alignment



\section{Methods}


- 14 participants, 3 experiments with 3 tesla, TR = 2 sec
- functional localizer (4 runs; 650 TRs) as "ground truth"
- movie \& audiobook (2x8 runs; > 7200 TRs) for model space
- model space: shared response model (Chen et. al 2015)
- limited to voxels within ROIs based on other subjects' data
- goal: align left-out subject to model space using
  varying amount of data of movie or audiobook

- transform localizer results into left-out subject


Which data did we use ?

---------------------------------
We used data from the studyforrest dataset
that provides fMRI data
- from the movie „Forrest Gump“
- from the audio-description of the movie produced for a visually-impaired audience, and
- from a localizer experiment for higher-visual areas

Taken together, for 14 subjects,
we have roughly 7500 data points to let the algorithm learn the shared responses
which is the common model space,
and it also learns the transformation matrices.

Given that each brain has far more than 7500 voxels,
we needed to restrict our analysis to voxels within a region of interest
that we defined anatomically

Now, how do we predict functional areas?
First, the easier example: prediction using anatomical alignment

-------------------------------------
- you conduct the localizer experiment to get the brain responses in the reference group
- then you align the brains in that reference group anatomically
  to the standard brain
- and you also align your left-out subject to the standard brain,
  which gives you a transformation matrix
- then you use the inverse of that transformation matrix to project the data
  from the reference group into the anatomy of the left-out subject,
  which is essentially the prediction

Second, the prediction using functional alignment

--------------------------------------
For the prediction through a common model space,
we first needed to create it

We let participants watch the movie and audio-description of Forrest Gump,
assuming that the naturalistic stimuli trigger, among others,
brain responses that a similar to those triggered by the functional localizer

Usually when aligning brain responses (and not connectivity profiles) you would let the left-out subject watch the same and the whole stimuli
that were used to create the Common Model Space to get the transformation matrices.

But this is not very time-efficient, right?

------------------------------------------------
Hence, the real question that we asked yourselves was:
How many minutes of a naturalistic stimulus are necessary to perform what we call a „partial alignment“.

So, we only used parts of the naturalistic stimuli to let the algorithm learn the transformation matrices for the left-out subject

and we tested
which amount of data is needed to do an alignment to the common model space
that provides transformation matrices that outperform a prediction using just an anatomical alignment.



\section{Results}


For every subject, we see the correlation of z-maps that tell us the, quote „real“ unquote, PPA and predicted PPA

In green,
we see the correlations between empirical values from the localizer
\& the predicted values using anatomical alignment

In orange,
we see the correlations between empirical values \&
the predicted values using parts of the movie

In blue,
we see the correlations between empirical values \&
the predicted values using parts of the audio-description

What we can see is,
- that 15 minutes of movie watching outperform an anatomical alignment,
- and 30 minutes of movie watching outperform 15 minutes of movie watching
- more than 30 minutes do not lead to a significantly improved prediction performance.

I marked subject 4, because I want to show you how results look like in a horizontal slice of the brain of subject 4

---------------------------
We have these nice blurry EPI-images and all z-maps are threshold
at a value of bigger than 2.3.

always in red,
we can see the z-map from the localizer experiment across the whole brain,

the region of interest that we used is white
and the predicted values, are blue

The prediction using anatomical alignment
and the prediction using 15 minutes of movie data show a correlation of about .7

the prediction using 15 minutes of the audio-description correlates about 0 with the empirical z-map

The last one is the extreme case,
but it can give you an idea of how the z-maps look in a slice of the brain



\section{Discussion}


- 15 min of movie watching
  used for functional alignment
  outperform prediction using
  anatomical alignment

- proof of concept
- ROIs vs. whole brain
  via searchlight?
- predict other t-contrasts?
- predict other localizers?
  e.g. retinotopy, language areas

As a summary just three points

1) 15 minutes of movie watching used for functional alignment outperform a prediction using anatomical alignment

2) more than 30 minutes do not increase prediction performance

3) and please rememeber that the visual localizer is considered to be the „ground truth“ which is a kind of a big assumption

at least to me, far more interesting is not the discussion
but what still needs to be done in the current project

1) how do we test the differences in performance?
a recent paper used dependent t-tests
to test for differences in the correlations

2) how do we calculate the reliability of the localizer as kind of a „ceiling“

3) lastly, we currently apply another method to get the „z-map template“
    on which the prediction is based

Lastly, what is out if scope of the current project:

First,
we currently have a cross-subject and cross-experiment prediction,
but we do not have a cross-scanner prediction

hence, we would like to create a CMS from another experiment’s data,
using another scanner and hopefully more subjects

In case of an alignment of time-series,
that experiment needs, at least, a part of Forrest Gump as an intersection

Second,
we do not want to restrict ourselves to a region of interest,
but perform a whole-brain alignment,
probably using a search-light algorithm

and finally,
what is all that leading to?

It leads to the creation of a functional atlas

Imagine you scan a new, unknown subject for just 15 minutes more, and you additionally get results from a whole variety of other paradigms mapped onto that brain
Results from localizers of low-level perceptual processes, but also higher-level cognitive processes like language, memory, emotions and so on

And I did not write that onto the slides because it is just a vague idea:
If you have different common model spaces for different subgroups,
you can investigate which alignment
onto which „subgroup common model space“
results in less error,
that lets you classify to which subgroup your new subject might belong.


\section{Conclusion}


% SRM
%\begin{comment}


\chapter{Assessing the quantity of data for functional alignment to estimate
    responses in the ``parahippocampal place area''}
% take input from external file
\todo[inline]{I use "prediction" and "estimation" pretty much interchangeably}

\section{Introduction}

% higher visual areas higher visual areas
In the domain of higher-visual perception, functionally defined
category-selective brain regions, such as the \ac{ppa} \citep{epstein1998ppa},
the \ac{ffa} \citep{kanwisher1997ffa}, or \ac{eba} \citep{downing2001bodyarea}
exhibit significantly increased \ac{bold} activity correlated with a
``preferred'' \citep{debeck2008interpreting} stimulus class.
%
While the topographies (i.e. the location, size and shape) of these
category-selective areas are similarly distributed across individuals, the exact
topographies vary interindividually \citep{rosenke2021probabilistic,
zhen2017quantifying, zhen2015quantifying, frost2012measuring}.
% definition of localizer
To identify the topography of functional areas in individual persons,
block-design \textit{functional localizer} paradigms are traditionally used that
contrast modeled hemodynamic responses correlating with the corresponding
stimulus classes, such as landscapes, faces, or bodies.
% problem: one localizer for one domain
Functional localizers are designed to maximize detection power and thus limited
to mapping only one domain of brain functions, such as retinotopic visual areas
\citep{wang2015probabilistic}, category-selective regions
\citep{stigliani2015temporal}, theory of mind \citep{spunt2014validating}, or
semantic processes \citep{fedorenko2010new, fernandez2001language}.
% which gets messy
However, when mapping multiple functional domains in a limited amount of time is
desired, the "one paradigm for one domain of functions" approach becomes
impractical.
% localizer batteries: intro
To address this issue, researchers have attempted to create time-efficient,
multi-functional localizer batteries \textit{localizer batteries}
\citep[e.g.,][]{barch2013function, drobyshevsky2006rapid, pinel2007fast}.
% task based = shit
Nevertheless, the diagnostic quality of localizer paradigms heavily depends on a
participant's comprehension of the task instructions and general compliance, a
criterion that can be difficult to meet in clinical or pediatric populations
\citep{eickhoff2020towards, vanderwal2019movies}.

% ppa via audio-description Results also suggest that a naturally engaging,
% purely auditory paradigm like an audio-description could, in principle,
% substitute a visual localizer as a diagnostic procedure to assess brain
% functions in visually impaired % individuals \citep{haeusler2022processing}.
In \citet{haeusler2022processing}, we demonstrated that a functionally defined
region such as the \ac{ppa} can be localized using a model-driven \ac{glm}
analysis that is based on the annotated temporal structure of a two-hour long
naturalistic stimulus.
% full feature film is too long
However, for practical and monetary reasons, a two-hour long paradigm is
unsuitable for clinical applications.
% hence, predict from reference
One approach to reduce the time and costs is to identify a functional area in an
individual person's brain anatomy based on data collected from an independent
sample of different individuals (i.e. data from a \textit{reference group}).
% intro: estimation via common anatomical space
Previous studies have estimated the most probable location of a functional area
in an individual from a reference group by performing either a volume-based
\citep[e.g.,][]{zhen2017quantifying, zhen2015quantifying} or a surface-based
\citep[e.g.,][]{frost2012measuring, weiner2018defining,
rosenke2021probabilistic, wang2015probabilistic} \textit{anatomical alignment}.
%
First, in order to address the issue of anatomical variability across persons,
functional data of persons in the reference group are anatomically aligned to
(i.e.  projected into) a \textit{common anatomical space} (e.g., Montreal
Neurological Institute brain atlas; \citep[MNI152,][]{fonov2011unbiased}).
% project into test subject to estimate
Then, data are projected from the common anatomical space into the individual
person's brain anatomy to provide an estimate a functional region's location.
% volume-based alignment in one sentence
Volume-based anatomical alignment \citep[s.][for a review]{klein2009evaluation}
aligns voxels to a three-dimensional common anatomical space \citep[e.g., MNI152
atlas;][]{fonov2011unbiased}.
% surface-based alignment in one sentence
Surface-based anatomical alignment \citep{fischl1999cortical, yeo2009spherical}
aligns vertices to a two-dimensional common anatomical space \citep[e.g.,
FreeSurfer's fsaverage template;][]{fischl1999high}.
% difference in one sentence
Whereas volume-based alignment does not account for individual sulcal and gyral
folding patterns, surface-based alignment respects interindividual variability
of the cortical surface.
% surface-based estimation works better
Consequently, previous studies that compared volume-based and surface-based
alignment to estimate the location of functional regions have shown that
surface-based alignment reduces inter-subject variability and improves
estimation performance \citep{rosenke2021probabilistic, frost2012measuring,
wang2015probabilistic, weiner2018defining}.
% remaining variability after surface-based alignment
However, even after surface-based alignment, the anatomical location of
functional regions varies between individuals \citep[e.g.,][]{coalson2018impact,
benson2014correction, natu2021sulcal, wang2015probabilistic, frost2012measuring,
langers2014assessment, weiner2014mid, rosenke2021probabilistic}.
% frost as an example
\citet{frost2012measuring}, for example, localized 13 functional areas of the
high-level visual cortex and ``found a large variability in the degree to which
functional areas respect macro-anatomical boundaries'' \citep[][p.
1369]{frost2012measuring}.
% functional--anatomical correspondence
The remaining variability indicates that functional areas a not necessarily
bound to anatomical landmarks, and reflects the degree of
\textit{functional--anatomical correspondence} between a brain function and its
underlying anatomical location.

% case of PPA cf. also \citet{frost2012measuring, rosenke2021probabilistic}
% \citet{weiner2018defining} showed ``that cortical folding patterns and
% probabilistic predictions reliably identify place-selective voxels in medial
% VTC across individuals and experiments''.
%
% However, ``this structural-functional coupling is not always perfect and there
% is inter-subject variability as to how much the place-selective voxels extend
% within the parahippocampal gyrus, as well as the lingual gyrus and medial
% aspects of the fusiform gyrus.
%
% Despite this inter-subject variability, place-selective voxels are always
%located within the collateral sulcus across participants.''
%\citep{weiner2018defining}.
In order to address the issue of functional-anatomical variability across
subjects, \textit{functional alignment} algorithms, such as
\textit{hyperalignment} \citep{haxby2011common, guntupalli2016model} or the
\textit{shared response model} \citep{chen2015reduced, zhang2016searchlight},
have been developed.
%
Whereas anatomical alignment aligns voxels (or vertices) that share the same
anatomical location to a common anatomical space, functional alignment aligns
voxels (or vertices) that share similar functional properties to a
\textit{common functional space} (CFS).
%
Functional alignment algorithms are typically used to compute both a
high-dimensional, functional brain template (i.e. the \ac{cfs}) subject-specific
transformations from the functional data of a study's participants.
%
A subject-specific transformation allows to project functional data from a
subject's three-dimensional voxel space into the \ac{cfs} or vice versa
\citep{haxby2020hyperalignment, kumar2020brainiak}.
%
The \ac{cfs} and transformations are calculated (i.e. \textit{trained}) by
maximizing the inter-subject similarity of \ac{bold} response time series
correlating with a time-locked external stimulation \citep{haxby2011common,
chen2015reduced, sabuncu2010function}, or by maximizing the inter-subject
similarity of connectivity profiles \citep{feilong2018reliable,
guntupalli2018computational, nastase2019leveraging}.
%
Whereas connectivity-based functional alignment better aligns connectivity
profiles, response-based functional alignment better aligns response time-series
\citep{guntupalli2018computational}.
%
Although functional alignment algorithms can be applied to \ac{fmri} time series
data from paradigms employing simplified stimuli, data from naturalistic stimuli
provide
%
improved generalizability of the \ac{cfs}
%
and transformation matrices
%
to novel stimuli or tasks.
%
This is presumably because naturalistic stimuli sample a broader range of brain
states \citep{haxby2011common, guntupalli2016model}.

\todo[inline]{imo, validity (\& generalizability) of CFS and matrices are
non-separable claims!}

A more recent procedure \citep[e.g., ][]{jiahui2020predicting,
guntupalli2016model, haxby2011common} to estimate the most probable location of
a functional area in an individual from a reference performs an functional
alignment.
% solve functional-anatomical variability
First, the functional data of individuals in the reference group are
anatomically aligned to a common anatomical space.
%
Second, to address the issue of functional-anatomical variability across
individuals, the functional data are functionally aligned (i.e. projected into)
a \ac{cfs}.
%
Finally, data are projected from the \ac{cfs} into the individual's brain
anatomy, serving as an estimate of a functional region's location.
% Example: Jiahui (2020)
For instance, \citet{jiahui2020predicting} used surface-based hyperalignment to
calculate \acp{cfs} and transformations based on data from
%
the movie ``Grand Budapest Hotel'' ($\approx$\unit[50]{m}; \ac{tr}=\unit[1]{s}),
and
%
the movie ``Forrest Gump'' ($\approx$\unit[120]{m}; \ac{tr}=\unit[2]{s}).
%
\citet{jiahui2020predicting} then estimated $t$-contrast maps from a visual
localizer that aimed at identifying the \ac{ffa} by projecting the $t$-contrast
maps from a reference group through each \acp{cfs} into an individual's brain
anatomy.
%
Results showed that $t$-contrast maps from the visual localizer correlated more
highly with contrast maps that were estimated via hyperalignment than contrast
maps that were estimated via surface-based anatomical alignment.


\todo[inline]{\textit{criterion} and \textit{predictors} are defined here to
make the discussion easier; however, it's not predictor / criterion in a
strict sense as typically used (as referring to single variables)}

% focus: ppa
Here again, we focus on the \ac{ppa} \citep[e.g.,][for
reviews]{epstein2014neural, aminoff2013role}, and investigate whether we can
estimate the results of $t$-contrasts (i.e. $Z$-maps) that were created to
identify the \ac{ppa} using functional data from three different paradigms as
the to predicted \textit{criteria}:
%
(a) a classic visual localizer \citep{sengupta2016extension} as the assumed
``gold standard'',
%
(b) a movie \citep{haeusler2022processing}, and
%
(c) an auditory narrative \citep{haeusler2022processing}.
% math stuff from citep{vodrahalli2018mapping}
% ``SRM learns $N$ maps $W_{i}$ with orthogonal columns such that
% $||X_{i}-W_{i}S||_{F}$ is minimized over $\left\{ W_{i}\right\} _{i=1}^{N},S$,
% where $X_{i}\in\mathbb{R}^{v\times{T}}$ is the $i^{th}$ subject's fMRI
% response ($v$ voxels by $T$ repetition times) and
% $S\in\mathbb{R}^{k\times{T}}$ is a feature time-series in a $k$-dimensional
% shared space'' \citep{vodrahalli2018mapping}.
% Inverse vs. transpose of a matrix:
% for orthogonal transformations (like we should have here, i.e. only rotation,
% expansion) these two are one and the same thing:
% https://www.quora.com/When-is-the-inverse-of-a-matrix-equal-to-its-transpose
% why SRM
Our volume-based functional alignment approach utilizes the \ac{srm} algorithm
\citep{chen2015reduced, richard2019fast} as implemented in the open-source
software package BrainIAK \citep[Brain Imaging Analysis Kit;
\href{https://brainiak.org}{\url{brainiak.org}};][]{kumar2020brainiak,
kumar2020brainiaktutorial}.
% general overview of SRM
The \ac{srm} is an unsupervised probabilistic latent-factor model that
decomposes \ac{bold} \ac{fmri} response time series of participants who have
experienced the same stimulus into a \ac{cfs} of \textit{shared features}
\citep[also known as ``\textit{shared feature space}'';][]{chen2015reduced} and
subject-specific linear transformations matrices.
% math stuff
More specifically, the \ac{srm} algorithm uses each $n^{th}$ subject's response
time series represented as matrix $X_{n}$ ({$v$} voxels by $t$ time points) to
compute the \ac{cfs} $C$ ($k$ shared responses by $t$ time points) and
subject-specific transformation matrices $W_{n}$ ($v$ voxels by $k$ shared
responses) with orthonormal columns ($W_{n}^{T}W_{n}=I_{k}$).
% iteratively fitted
The algorithm randomly initializes and fits the transformation matrices over
iterations to minimize the error in explaining the participants' data, while
also learning the time course of the shared responses (s.
\href{https://brainiak.org/tutorials/11-SRM/}{\url{brainiak.org/tutorials/11-SRM}}).
% number of dimensions
In contrast to hyperalignment, the number of dimensions of the \ac{cfs} is not
set by the number of voxels, but rather it is determined by the researcher to a
number lower than the number of voxels, a procedure that also filters out noise
and reduces overfitting \citep{chen2015reduced}.
% phrase math in words
As a result, each shared feature can be thought of as a weighted sum of many
voxels across subjects \citep{kumar2020brainiak}.
% result = alignment
A subject-specific transformation matrix can thought of as the weight of each
voxel in a subject's voxel space on each shared feature, and allows a subject's
functionally data to be aligned to the \ac{cfs} by projecting responses within
the voxels into the $k$-dimensional \ac{cfs}.


\todo[inline]{check if $W_{n}^{T}W_{n}=I_{k}$ is True}

\todo[inline]{biggest issue (for discussion, too): how to separate validity /
generalizability of CFS from validity / generalizability of transformation
matrices?}

% multi-paradigm model
In contrast to previous studies \citep[e.g.][]{jiahui2020predicting,
guntupalli2016model, haxby2011common} that calculated a \ac{cfs} based on data
from a single paradigm, we calculated a \textit{multi-paradigm \ac{cfs}} based
on data from three paradigms.
% cross-validation
We followed an exhaustive leave-one-subject-out cross-validation (N$=$14
subjects) to train a shared feature space (i.e. the \ac{cfs}) based on
concatenated response time series of
%
the movie ``Forrest Gump'' ($\approx$\unit[120]{m}; \ac{tr}=\unit[2]{s}),
%
the movie's audio-description  that was produced for a visually impaired
audience ($\approx$\unit[120]{m}; \ac{tr}=\unit[2]{s}), and
%
a visual localizer ($\approx$\unit[21]{m}; \ac{tr}=\unit[2]{s})
%
from $N-1$ \textit{training subjects} as seen in
Fig.~\ref{fig:multi-stimulus-cfs}.
% four aspects to explore
The purpose of this study was to investigate four aspects.
%
First, we explored the validity and generalizability of our multi-paradigm
\ac{cfs} by predicting a left-out \textit{test subject}'s results from the
analysis of
%
(a) the localizer \citep{sengupta2016extension} as the assumed ``gold
standard'',
%
(b) the movie \citep{haeusler2022processing}, and
%
(c) the auditory narrative \citep{haeusler2022processing}
%
serving as the criteria.
% three predictors
Second, we use a test subject's response time series from each of the three
paradigms separately in order to align the test subject with the corresponding
\acp{tr} within the \ac{cfs}.
%
This \textit{partial alignment} lets us assess the validity and generalizability
of the three paradigms serving as \textit{predictors} (i.e. one
\textit{cross-subject-within-paradigm prediction}, and two
\textit{cross-subject-cross-paradigm predictions}).
% partial alignment
Third, considering the impracticality of using a complete naturalistic stimulus
in a clinical setting to align a test subject, we also explored the relationship
between the estimation performance of the results from each of the three
paradigms and the quantity of data from each of the three paradigms used to
functionally align the subject with the multi-paradigm \ac{cfs}.
% benchmark: anatomical alignment the criteria
Fourth, we compared the performance of our volume-based, functional alignment
procedures to the performance of a volume-based, anatomical alignment approach
that serves as a benchmark.

\todo[inline]{add 2-3 sentences stating the results}
%
% Our results provide evidence that transformation matrices calculated based
% on data from naturalistic stimuli promise an increased validity of derived
% transformation for functional alignment over transformation matrices based on
% data (of the same!) paradigm based on simplified stimuli.

\todo[inline]{add 2-3 sentences stating a conclusion \& vision}
% Our results suggest that it is possible to ``scan once, estimate many

\begin{figure*}[tbp]
\centering
\includegraphics[width=\linewidth]{figures/multi-stimulus-cfs.pdf}
\caption{
%
\textbf{Overview of the shared response model.
}
    %
    For each fold of the leave-one-subject-out cross-validation, each training
    subject's response time series of
    %
    a movie ($\approx$\unit[120]{m}; \ac{tr}=\unit[2]{s}),
    %
    the movie's audio-description ($\approx$\unit[120]{m}; \ac{tr}=\unit[2]{s}),
    %
    and the visual localizer ($\approx$\unit[21]{m}; \ac{tr}=\unit[2]{s})
    %
    were concatenated to serve as input for the \ac{srm} algorithm.
    %
    From these response time series represented as matrix $X_{n}$ ({$v$} voxels
    by $t$ time points), the algorithm calculates the common functional
    space (CFS) $C$ ($k$ shared features by $t$ time points) and
    subject-specific, transformation matrices $W_{n}$
    ($v$ voxels by $k$ shared features) with orthonormal columns
    ($W_{n}^{T}W_{n}=I_{k}$).
} \label{fig:multi-stimulus-cfs} \end{figure*}





\section{Methods}

% we get the data from the naturalistic PPA paper (its subdataset) datalad get
% -n inputs/studyforrest-ppa-analysis/inputs/studyforrest-data-aligned datalad
%  get
%  inputs/studyforrest-ppa-analysis/inputs/studyforrest-data-aligned/sub-??/in\_bold3Tp2/sub-??\_task-a?movie\_run-?\_bold*.*

% reference to PPA-Paper
For the current study, we used the same subset of the studyforrest dataset that
we previously used in \citet{haeusler2022processing}:
%
The sample included the same fourteen participants who
% VIS
(a) participated in a dedicated six-category block-design visual localizer
\citep{sengupta2016extension},
% AV
(b) watched the audio-visual movie ``Forrest Gump''
\citep{hanke2016simultaneous}, and
% AD
(c) listened to the movie's audio-description \citep{hanke2014audiomovie}.
% see corresponding papers for details
An exhaustive description of the participants, stimulus creation, procedure,
stimulation setup, and fMRI acquisition can be found in the corresponding
publications, while a summary is provided in \citet{haeusler2022processing}.



\subsection{Preprocessing}

% data sources
The analyses in this study were conducted on the same preprocessed fMRI data he
current analyses were carried out on the same preprocessed fMRI data (s.
\href{https://github.com/psychoinformatics-de/studyforrest-data-aligned
}{\url{github.com/psychoinformatics-de/studyforrest-data-aligned}}) that were
used for
%
(a) the technical validation of the dataset \citep{hanke2016simultaneous},
%
(b) the localization of higher-visual areas \citep{sengupta2016extension}, and
%
(c) the investigation of responses of the \ac{ppa} correlating with naturalistic
spatial information in \citep{haeusler2022processing}.
%
We reran the preprocessing and analyses steps performed in
\citet{sengupta2016extension} and \citet{haeusler2022processing} using FEAT
v6.00 \citep[FMRI Expert Analysis Tool;][]{woolrich2001autocorr} as shipped with
FSL v5.0.9 \citep[\href{https://www.fmrib.ox.ac.uk/fsl}{FMRIB's Software
Library;}][]{smith2004fsl} to reproduce the time series that served as input for
the previous statistical analyses and their results (i.e. the statistical
$Z$-maps).
% temporal filtering
The preprocessing steps included high-pass temporal filtering (using a
Gaussian-weighted least-squares straight line) for every run of the visual
localizer (cutoff period of \unit[100]{s}), and every segment of the movie and
audio-description (cutoff period of \unit[150]{s}).
% brain extraction & spatial smoothing
Brain extraction was performed using BET \citep{smith2002bet}., and data from
all three paradigms were spatially smoothed using a Gaussian kernel with a full
width at half maximum of \unit[4.0]{mm}.
% grand mean normalization
A grand-mean intensity normalization was applied to each run of the functional
localizer and each segment of the naturalistic stimuli.
%
Further analyses on these reproduced times series were performed using Python
scripts that relied on
%
NiBabel v3.2.1 (\href{https://nipy.org}{\url{nipy.org}}),
%
NumPy v1.20.2 (\href{https://numpy.org}{\url{numpy.org}}),
%
Pandas v1.2.3 (\href{https://pandas.pydata.org}{\url{pandas.pydata.org}}),
%
Scipy v1.6.2 (\href{https://scipy.org}{\url{scipy.org}}),
%
scikit-learn v1.0 (\href{https://scikit-learn.org}{\url{scikit-learn.org}}),
%
BrainIAK v0.11
\citep[\href{https://brainiak.org}{\url{brainiak.org}}][]{kumar2020brainiak,
kumar2020brainiaktutorial},
%
Matplotlib v3.4.0 (\href{https://matplotlib.org}{\url{matplotlib.org}}),
%
seaborn v0.11.2 (\href{https://seaborn.pydata.org}{\url{seaborn.pydata.org}}),
%
and calling command line functions of FSL.

%\paragraph{Fixing FSL output}

% grand_mean_for_4d.py (formerly: data_normalize_4d.py):
% is not necessary anymore: FSL has applied grand mean scaling to
% 'filtered_func_data.nii.gz'

% input: 'sub-*/run-?.feat/filtered_func_data.nii.gz' (of VIS, AO & AV)
% output: saved to 'sub-??_task-*_run-?_bold_filtered.nii.gz'

% FSL adds back the mean value for each voxel's time course at the end of the
% preprocessing;
% hence, the script substracts that mean again but multiplies it by 10000
% (like FSL does it, too)

% definition of grand mean scaling for 4d data:
% voxel values in every image are divided by the average global mean
% intensity of the whole session. This effectively removes any mean global
% differences in intensity between sessions.

% FSL User Guide:
% filtered_func_data will normally have been temporally high-pass filtered,
% it is not zero mean; the mean value for each voxel's time course has been
% added back in for various practical reasons.
% When FILM begins the linear modeling, it starts by removing this mean.

\todo[inline]{restriction to ROIs is not mentioned in the intro. Is that okay?}

% masks-from-mni-to-bold3Tp2.py:
% - merges unilateral ROIs overlaps (already in MNI) to bilateral ROI
% - output: 'masks/in_mni/PPA_overlap_prob.nii.gz'
% - warps union of ROIs from MNI into each subjects space
% output: 'sub-*/masks/in_bold3Tp2/grp_PPA_bin.nii.gz' + audio_fov.nii.gz dilate
% the ROI masks by 1 voxel; output: 'grp_PPA_bin_dil.nii.gz'

% masks-from-mni-to-bold3Tp2.py:
% warp MNI masks into individual bold3Tp2 spaces

% masks-from-t1w-to-bold3Tp2.py:
% transforms 'inputs/tnt/sub-*/t1w/brain_seg*.nii.gz'
% into individual's bold3Tp2
% output: 'sub-*/masks/in_bold3Tp2/brain_seg*.nii.gz'

% mask-builder-voxel-counter.py:
% builds different individual masks by dilating, merging other masks
% creates a FoV of AO stimulus for every subject from 4d time-series of AO run
% output: sub-*/masks/in_bold3Tp2/audio_fov.nii.gz'
% counts the voxels
% long story short: we cannot used all gyri that contain PPA to some degree
% even if the mask by FoV of AO stimulus and individual gray matter mask

% data_mask_concat_runs.py:
% masks are not dilated and not masked with subject-specific gray matter mask
% outputs:
% 'sub-*_task_aomovie-avmovie_run-1-8_bold-filtered.npy
% 'sub-*_task_visloc_run-1-4_bold-filtered.npy'

% reason why we do it
The \ac{srm} requires that the number of samples (i.e. the number of \acp{tr})
exceed the number of features (the number of voxels).
%
In order to restrict the number of voxels, we created bilateral \acp{roi} for
each subject.
%
Specifically, we warped the union of individual \acp{ppa}
citep[s.][]{haeusler2022processing} from MNI space into each subject's voxel
space using subject-specific, non-linear transformation matrices that were
previously computed
\citep[][\href{https://github.com/psychoinformatics-de/studyforrest-data-templatetransforms
}{\url{github.com/psychoinformatics-de/studyforrest-data-templatetransforms}}]{hanke2014audiomovie}.
% applying masks
The time series data of each subject were then masked in their native voxel
space by the union of individual \acp{ppa} and the subject-specific \ac{fov} of
the audio-description.
% voxels = [1665, 1732, 1400, 1575, 1664, 1951, 1376, 1383, 1683, 1887, 1441,
% 1729, 1369, 1437] median = 1619.5
The number of remaining voxels per subject (range 1369--1951,
$\overline{X}=1592$, $SD=188$) can be seen in Fig.~\ref{fig:plot_voxel-counts}.
% normalization
Data of each run were normalized ($z$-scored) to a mean of zero
($\overline{X}=0$) and a standard deviation of one ($SD=1$).
%
Due to an image reconstruction problem \citep[s.][]{hanke2014audiomovie}, the
last 75 \acp{tr} of the audio-description were missing in subject 04.
%
The \ac{srm} allows for different numbers of voxels across subjects, but the
number of \acp{tr} must be the same.
%
Consequently, we removed the last 75 \acp{tr} of the audio-description from the
time series of all other subjects.
% summary; AO + AV = 7123 TRs (not 7198 TRs anymore); localizer has 4 x 156 TRs
As a result, the data used to fit the \ac{srm} in the next step included 3599
\acp{tr} from the movie, 3524 \acp{tr} from the audio-description, and 624
\acp{tr} from the visual localizer experiment (7747 \acp{tr} in total).
%% concatenate and z-score
The time series of all three paradigms were concatenated and $z$-scored.

\begin{figure*}[tbp]
\centering
\includegraphics[width=\linewidth]{figures/plot_voxel-counts.pdf}
\caption{
%
\textbf{Number of voxels in the bilateral regions of interest (ROIs)
of each subject.}
%
In order to reduce the number of voxels, we warped the union of
individual \acp{ppa} \citep[cf. Fig. 1 in][]{haeusler2022processing} from
MNI152 space into each subject's native voxel space.
%
The remaining voxels of each subject were further constrained to those
voxels that are included in the respective subject's \ac{fov} of the
audio-description \citep[cf.][]{hanke2014audiomovie}.
}
\label{fig:plot_voxel-counts}
\end{figure*}


\begin{comment}

The number of remaining voxels per subject can be seen in Table
\ref{tab:ppamaskvoxels} (range 1369--1951, $\overline{X}=1592$, $SD=188$).


\begin{table*}[btp] \caption{
%
\textbf{Table heading.}
%
The number of remaining voxels after masking time series data of each paradigm
and subject with the union of individual \acp{ppa} warped from MNI space
into each individual's subjects-space and the subject-specific field of view
of audio-description.
    }

\label{tab:ppamaskvoxels}
\begin{tabular}{ll}
\toprule
\textbf{Subject} & \textbf{no. of voxels} \\
\midrule
sub-01 & 1665 \tabularnewline
sub-02 & 1732 \tabularnewline
sub-03 & 1400 \tabularnewline
sub-04 & 1575 \tabularnewline
sub-05 & 1664 \tabularnewline
sub-06 & 1951 \tabularnewline
sub-14 & 1376 \tabularnewline
sub-09 & 1383 \tabularnewline
sub-15 & 1683 \tabularnewline
sub-16 & 1887 \tabularnewline
sub-17 & 1441 \tabularnewline
sub-18 & 1729 \tabularnewline
sub-19 & 1369 \tabularnewline
sub-20 & 1437 \tabularnewline
\bottomrule
\end{tabular}
\caption*{The legend text goes here.}
\end{table*}

\end{comment}




\subsection{Estimation via functional alignment}


\subsubsection{Overview}
%
To estimate the empirical $Z$-maps for $t$-contrasts using functional alignment,
we followed a four-step procedure.
% create CFS and training subjects' matrices
First, for every fold of a leave-one-out cross-validation (N$=$14 subjects), we
trained a \ac{srm} on $N-1$ training subjects' response time series of the
movie, the audio-description, and the visual localizer.
% results in...
This step generated a \ac{cfs} for each fold of the cross-validation and an
orthonormal transformation matrix for each training subject.
% align test subject
Second, we aligned the test subject to the corresponding \acp{tr} within the
\ac{cfs} using time series data from the visual localizer, the movie, or the
audio-description.
%
Second, we aligned the test subject's time series data from the movie,
audio-description, and visual localizer paradigms separately to the
corresponding TRs within the \ac{cfs}.
%
This step produced different transformation matrices for the test subject based
on data from different paradigms.
% quantity vs. performance
In order to examine the relationship between estimation performance and the
amount of data used to generate a transformation matrix, we also varied the
number of runs of the paradigms.
%
This step produced transformation matrices based on an increasing number of runs
per paradigm.
%
In the third step, we mapped the training subjects' empirical $Z$-maps from
their voxel space into the \ac{cfs} using their transformation matrices.
% project from CFS into test subject
Finally, we projected the training subjects' $Z$-maps from the \ac{cfs} into the
test subject's voxel space using the transpose of the test subject's
transformation matrix
% actual prediction
We obtained the test subject's predicted $Z$-maps by calculating the arithmetic
mean of the projected Z-maps



\subsubsection{Fitting the SRM}
%
In order to obtain the \ac{cfs} and the training subjects' transformation
matrices, we used the probabilistic \ac{srm} algorithm that is implemented in
BrainIAK v.11 \citep[Brain Imaging Analysis Kit;][]{kumar2020brainiak,
kumar2020brainiaktutorial}, and approximates the \ac{srm} based on the
Expectation Maximization (EM) algorithm as proposed by \citet{chen2015reduced}
and optimized by \citet{anderson2016enabling}.
% number of dimensions / features ``The effect of number of PCs on BSC was
% similar for models that were based only on Princeton (n = 10) or Dartmouth (n
% = 11) data, suggesting that this estimate of dimensionality is robust across
% differences in scanning hardware and scanning parameters''
% \citep{haxby2011common}.
%
% ``These dimensionality estimates are a function of the spatial and temporal
% resolution of fMRI and the number and variety of response vectors used to
% derive the common space'' \citep{guntupalli2016model}.
%
% ``The true dimensionality of representation in human cortex surely involves
% vastly more distinct tuning functions. Estimates of the dimensionality of
% cortical representation, therefore, will almost certainly be much higher as
% data with higher spatial and temporal resolution for larger and more varied
% samples of response vectors are used to build new common models''
% \citep{guntupalli2016model}.
We chose a value of $k=10$ for the number of shared features (i.e. the number of
dimensions in the \ac{cfs}) based on the temporal and spatial resolution of our
data (\ac{tr} = \unit[2]{s}; \unit[2.5 $\times$ 2.5 $\times$ 2.5]{mm}), the
average number of voxels per \ac{roi}, and findings from
\citet{haxby2011common}.
%
\citet{haxby2011common} used hyperalignment to create a \ac{cfs} of 1,000
dimensions based of functional data (\ac{tr} = \unit[3]{s}) of voxels (\unit[3
$\times$ 3 $\times$ 3]{mm}) located in the ventral temporal cortex.
%
Then, \citet{haxby2011common} reduced the dimensionality of the \ac{cfs} by
applying a \ac{pca} in order to determine the subspace that is sufficient to
capture the full range of response-pattern distinctions.

They then applied principal component analysis \ac{pca} to reduce the
dimensionality of the \ac{cfs} to determine the subspace that sufficiently
captured the range of response-pattern distinctions.
%
Furthermore, the cortical topographies of category-selective brain regions were
preserved in the 35-dimensional \ac{cfs}.
% ...as judged by visual inspection
In the present study, we also computed \acp{cfs} of $k=5, 20, 30, 40, 50$ but
prediction performance based on these \acp{cfs} barely varied from a
10-dimensional \ac{cfs}.
% iterations:
The algorithm was set to iterate 30 times to minimize the error.

% correlations of regressors
In order to visualize characteristics of the \ac{cfs}, we calculated the Pearson
correlation coefficients between the shared responses and the regressors that
were previously modeled \citep[cf.][]{sengupta2016extension,
haeusler2022processing} to investigate hemodynamic responses during the three
paradigms.
%
As an example, we chose the \ac{cfs} that was created in the first fold of the
cross-validation from $N-1$ subjects to estimate $Z$-maps of subject 01.
%
The time series of the shared features were trimmed to match the corresponding
\acp{tr} of the respective paradigms.
%
Fig.~\ref{fig:corr-vis-reg-srm} shows the correlations between regressors
created to model hemodynamic responses during the visual localizer and shared
responses (trimmed to \acp{tr} that match the visual localizer).
Fig.~\ref{fig:corr-av-reg-srm} shows the correlations between regressors created
to model hemodynamic responses during the movie \citep[cf. Table 3
in][]{haeusler2022processing} and shared responses, while
Fig.~\ref{fig:corr-ao-reg-srm} shows the correlations between regressors created
to model hemodynamic responses during the audio-description \citep[cf. Table 3
in][]{haeusler2022processing} and shared responses.


\todo[inline]{What do the plots suggest?}

\todo[inline]{make colors of non-used TRs more transparent (alpha=15)}

% shuffle runs
As a negative control, we randomly shuffled the order of runs of the visual
localizer and the segments of the naturalistic stimuli separately for each
paradigm and training subject. We then concatenated the time series, fit the
\ac{srm}, and calculated the Pearson correlation coefficients.
%
We expected that the \ac{srm} algorithm would fail to fit ``meaningful''
\todo{??} shared responses to randomly shuffled training data.
%
As hypothesized, the results based on shuffled time series revealed no or only
minor correlations between the shared responses and regressors, as shown in
Fig.~\ref{fig:corr-vis-reg-srm-shuffled},
Fig.~\ref{fig:corr-av-reg-srm-shuffled}, and
Fig.~\ref{fig:corr-ao-reg-srm-shuffled}.


\todo[inline]{Plots are located in the appendix; add representation of the model
to the plots}


\begin{figure*}[tbp]
\centering
\includegraphics[width=\linewidth]{figures/corr_vis-regressors-vs-cfs_sub-01_srm-ao-av-vis_feat10-iter30_7123-7747.pdf}
\caption{
%
\textbf{Pearson correlation coefficients between regressors of the visual
localizer and shared features.}
%
The time series of the shared features within the multi-paradigm \ac{cfs}
%
(as calculated for subject 01 in the first fold of the cross-validation)
%
were trimmed to match the corresponding \acp{tr} of the visual localizer
paradigm \citep{sengupta2016extension}.
%
The six regressors of the visual localizer model hemodynamic responses to
the six categories of pictures that were presented in blocks.
}
\label{fig:corr-vis-reg-srm}
\end{figure*}


\begin{figure*}[tbp]
\centering
\includegraphics[width=\linewidth]{figures/corr_av-regressors-vs-cfs_sub-01_srm-ao-av-vis_feat10-iter30_3524-7123.pdf}
\caption{
%
\textbf{Pearson correlation coefficients between regressors of the movie
and shared features}
%
The time series of the shared features within the multi-paradigm \ac{cfs}
%
(as calculated for subject 01 in the first fold of the cross-validation)
%
were trimmed to match the corresponding \acp{tr} of the movie
\citep{hanke2016simultaneous}.
%
The regressors \texttt{vse\_new} to \texttt{vno\_cut} are based on
annotations movie frames, whereas the regressors
\texttt{fg\_av\_ger\_lr} to \texttt{fg\_av\_ger\_ud} represent low-level
visual or auditory confounds
\citep[cf. Table 3 in][]{haeusler2022processing}.
}
\label{fig:corr-av-reg-srm}
\end{figure*}


\begin{figure*}[tbp]
\centering
\includegraphics[width=\linewidth]{figures/corr_ao-regressors-vs-cfs_sub-01_srm-ao-av-vis_feat10-iter30_0-3524.pdf}
\caption{
%
\textbf{Pearson correlation coefficients between regressors of the
audio-description and shared features.}
%
The time series of the shared features within the multi-paradigm \ac{cfs}
%
(as calculated for subject 01 in the first fold of the cross-validation)
%
were trimmed to match the corresponding \acp{tr} of the
audio-description \citep{hanke2014audiomovie}.
%
The regressors \texttt{body} to \texttt{sex\_m} are based on
annotations of nouns spoken by the audio-description's narrator,
whereas the regressors \texttt{fg\_ad\_ger\_lrdiff} and
\texttt{fg\_ad\_ger\_rms} represent low-level auditory confounds
\citep[cf. Table 3 in][]{haeusler2022processing}.
%
\texttt{geo\&groom} is a combination of
regressors as used on the positive side of the primary contrasts aimed to
localize the \ac{ppa} \citep[cf. Table 5 in][]{haeusler2022processing}.
}
\label{fig:corr-ao-reg-srm}
\end{figure*}


\subsubsection{Alignment of the test subject}

% AO: 0-451, 0-892, 0-1330, 0-1818, 0-2280, 0-2719, 0-3261, 0-3524
% AV: 3524-3975, 3524-4416, 3524-4854, 3524-5342, 3524-5804, 3524-6243,
%     3524-6785, 3524-7123
% AO+AV: 0-7123

% in-code documentation of
% https://github.com/brainiak/brainiak/blob/master/brainiak/funcalign/srm.py
% says: # Solve the Procrustes problem; A =
% subjectFMRIdata.dot(SharedResponses.T) U, \_, V = np.linalg.svd(A,
% full\_matrices = False) return U.dot(V)

%
We aligned the test subject's response time series from the visual localizer,
the movie, or the audio-description to the corresponding \acp{tr} within the
\ac{cfs} by factorizing the response time series data via singular value
decomposition.
%
This step produced an orthonormal transformation matrix $W_{n}$ each paradigm
that allow a mapping of data from a test subject's voxel space into the
\ac{cfs}.
%
To investigate how the amount of data used to acquire a transformation matrix
affects estimation performance, we also varied the number of runs per paradigm.
%
Specifically, we we used one up to four runs (each lasting \unit[5.2]{m}) of the
visual localizer, and one up to eight segments (each lasting
$\approx$\unit[15]{m}) to align the test subject to the corresponding \acp{tr}
within the \ac{cfs}.
%
Therefore, for each test subject, we obtained four matrices based on data from
the visual localizer and eight different matrices per naturalistic stimulus,
each transformation matrix having a size of $v$ voxels by $k$ shared responses
but being based on an increasing amount of data used to calculate the mapping.


\subsubsection{Estimation of $t$-contrasts' results}

% overview
to estimate the results of
the three $t$-contrast (i.e. the \textit{empirical $Z$-maps}).


We estimated the results of the three $t$-contrast (i.e. the \textit{empirical
$Z$-maps}) of the test subject by projecting the empirical $Z$-maps of all
training subjects trough the \ac{cfs} into the test subject's voxel space:
% functional alignment; into CFS (calling srm.transform(masked\_zmaps))
First, we masked the empirical $Z$-maps of the training subjects' empirical
$Z$-maps with the same subject-specific masks used to generate the \ac{cfs} from
the time series data.
%
Then,  we used the transformation matrices derived during the training of the
\ac{cfs} to map the masked empirical $Z$-maps from each training subject's voxel
space into the \ac{cfs}.
%
Next, we used the transpose of a transformation matrix obtained from the
alignment of the test subject to project the $Z$-maps from the \ac{cfs} into the
test subject's voxel space.
% take the mean
For each of the three $t$-contrasts, we computed the arithmetic mean of the
respective projected $Z$-maps, which served as an estimation of the test
subject's empirical $Z$-maps (hence, a \textit{predicted $Z$-maps}).

\subsection{Estimation via anatomical alignment}

\todo[inline]{strictly speaking data were not projected through the MNI152 atlas
but through a study-specific brain template co-registered to the MNI152 atlas,
wasn't it?}

\todo[inline]{was the matrix the transpose or a "totally" different one?}

%
As a baseline, we used an anatomical alignment procedure to estimate the results
of $t$-contrast of each paradigm.
%
We predicted a test subject's empirical $Z$-maps from the analysis of
%
the visual localizer \citep{sengupta2016extension},
%
the movie \citep{haeusler2022processing}, and
%
the audio-description \citep{haeusler2022processing}
%
using the training subjects' results of the same paradigm (hence,
cross-subject-within-paradigm predictions).
%
To do this, we projected the masked empirical $Z$-maps of each paradigm and each
subject were projected from their native voxel space into the MNI space via
previously computed subject-specific transformation matrices
\citep[][\href{https://github.com/psychoinformatics-de/studyforrest-data-templatetransforms}{\url{github.com/psychoinformatics-de/studyforrest-data-templatetransforms}}]{hanke2014audiomovie}
% from MNI into subject
We then used the transpose of the transformation matrix to project the data from
the MNI space into the test subject's voxel space.
% take the mean
Similar to our functional alignment procedure, we obtained an estimation of the
test subject's empirical $Z$-maps by taking the arithmetic mean of the projected
$Z$-maps.



\subsection{Cronbach's alpha}

\todo[inline]{I have no clue where \citet{jiahui2020predicting, jiahui2022cross}
have the statement about what Cronbach's expresses from...}

\todo[inline]{Shift results stated here to actual result section? In particular,
in case a t-test involving Cronbach's is calculated (which is not the case so
far)?}



%
We calculated Cronbach's $\alpha$ as a measure of reliability and the amount of
measurement error \citep{cronbach1951coefficient, cortina1993coefficient}
present in the empirical $Z$-maps of each paradigm and subject.
%
Cronbach's $\alpha$ expresses the expected correlation between the currently
used empirical $Z$-maps and an additional set of empirical $Z$-maps calculated
based on data of a hypothetical independent dataset collected from the same
paradigm and subjects \citep{jiahui2020predicting, jiahui2022cross}.
%
These expected correlations, represented by Cronbach's $\alpha$, were calculated
based on the first-level \ac{glm} $Z$-maps (four in case of the visual
localizer; eight in case of the naturalistic stimuli) that were averaged in the
second-level \ac{glm} analyses of \citet{sengupta2016extension} and
\citet{haeusler2022processing}.
%
Cronbach's $\alpha$ of empirical (i.e. second-level) $Z$-maps for each subject
and paradigm can be seen in Fig.~\ref{fig:cronbachs}, descriptive statistics
across subjects for each paradigm can be seen in Table~\ref{fig:cronbachs}.

\todo[inline]{imo, descriptive statistics given in a text are not necessary;
i.e. plot and/or table are better}

%Visual localizer:  mean=0.899990, std=0.087051, min=0.658523,
%25\%=0.906643, 50\%=0.934019, 75\%=0.947906, max=0.963065.
%
%Movie: mean=0.611332, std=0.137878, min=0.284440,
%25\%=0.555529, 50\%=0.627240, 75\%=0.676353, max=0.800254.
%
%Audio-description: mean=0.476194, std=0.358019, min=-0.526626,
%25\%=0.428975, 50\%=0.627799, 75\%=0.679987, max=0.823584.

\todo[inline]{movie's outlier: sub-06 (0.28); but when movie PPA is predicted he
/ she is not the outlier}

\todo[inline]{audio-description's outlier: sub-05 (\textbf{-0.5!!}), sub-02
(0.0), sub-20 (0.27); when audio PPA is predicted (based on max of functional
data per paradigm), these three subjects are the outliers}

The Cronbach's $\alpha$ values for each paradigm are as follows:
%
for the visual localizer,
%
the mean is 0.899990 with a standard deviation of 0.087051.
%
The minimum value is 0.658523, the 25th percentile is 0.906643, the median is
0.934019, the 75th percentile is 0.947906, and the maximum is 0.963065.
%
For the movie paradigm,
%
the mean is 0.611332 with a standard deviation of 0.137878.
%
The minimum value is 0.284440, the 25th percentile is 0.555529, the median is
0.627240, the 75th percentile is 0.676353, and the maximum is 0.800254.
%
Finally, for the audio-description paradigm, the mean is 0.476194 with a
standard deviation of 0.358019.
%
The minimum value is -0.526626, the 25th percentile is 0.428975, the median is
0.627799, the 75th percentile is 0.679987, and the maximum is 0.823584.




\begin{table*}[btp]
\centering
    \caption{
    %
    \textbf{Descriptive statistics of Cronbach's $\alpha$ across subjects.}
    %
    Imo, not super necessary to provide these numbers. Stripplot + boxplots
    could be sufficient. If table is supposed to be kept, round the numbers,
    write a good description for the table.}
\label{tab:cronbachs}
\begin{tabular}{llll}
    \toprule
    \textbf{statistic} & \textbf{localizer} & \textbf{movie} & \textbf{audio-description} \\
    \midrule
    mean & 0.89999 & 0.611332 & 0.476194 \tabularnewline
    std & 0.087051 & 0.137878 & 0.358019 \tabularnewline
    min & 0.658523 & 0.28444 & -0.526626 \tabularnewline
    25\% & 0.906643 & 0.555529 & 0.428975 \tabularnewline
    50\% & 0.934019 & 0.62724 & 0.627799 \tabularnewline
    75\% & 0.947906 & 0.676353 & 0.679987 \tabularnewline
    max & 0.963065 & 0.800254 & 0.823584 \tabularnewline
    \bottomrule
\end{tabular}
\caption*{The legend text goes here.}
\end{table*}


\begin{figure*}[tbp] \centering
    \includegraphics[width=\linewidth]{figures/plot_cronbachs.pdf}
    \caption{\textbf{Cronbach's $\alpha$ of the empirical $Z$-maps for each
    paradigm and subject.}
    %
    Cronbach's $\alpha$ was calculated based on the $Z$-maps yielded by the
    first-level \ac{glm} analyses of the visual localizer
    \citep{sengupta2016extension} (four runs) and naturalistic stimuli
    \citep{haeusler2022processing} (eight segments each) respectively.
    %
    The second-level \ac{glm} analyses across runs / segments yielded the
    empirical $Z$-maps that were estimated in the present study.
    }
    \label{fig:cronbachs}
\end{figure*}



\pagebreak



\section{Results}


\begin{comment}

``Because the localizer task comprises several scanning runs, we calculated the
reliability of the localizer across runs with Cronbach's alpha to provide an
estimate of the noise ceiling for these correlations'' \citep{jiahui2022cross}.

\end{comment}


%
In order to quantify the estimation performance of the alignment procedures, we
correlated each individual's empirical $Z$-maps gained from previous analyses
\citep{sengupta2016extension, haeusler2022processing} with their respective
predicted $Z$-map (cf. Fig.~\ref{fig:stripplot}).
%
Unthresholded empirical and predicted $Z$-maps in their respective subject's
voxel space can be found at
\href{https://identifiers.org/neurovault.collection:12340}{\url{neurovault.org/collections/12340}}\todo{still
empty}.


\begin{figure*}[tbp] \centering
    \includegraphics[width=\linewidth]{figures/plot_corr-emp-vs-estimation.pdf}
    \caption{
    %
    \textbf{Correlations between empirical and predicted
    \textit{\textbf{Z}}-maps for each paradigm and subject.}
    %
    Functional alignment was performed based on an increasing amount of
    functional data used to align a test subject to the common functional space
    (CFS): runs of the visual localizer paradigm lasted \unit[5.2]{m}; segments
    of the naturalistic stimuli lasted $\approx$\unit[15]{m}.
    %
    Solid horizontal lines:
    %
    median of Cronbach's $\alpha$ across subjects for empirical $Z$-maps of the
    respectively estimated paradigm (cf. Fig.~\ref{fig:cronbachs}).
    %
    Dotted horizontal lines:
    %
    mean of Cronbach's $\alpha$ across subjects for empirical $Z$-maps of the
    respectively estimated paradigm (cf. Fig.~\ref{fig:cronbachs});
    %
    Grey dots:
    %
    correlations between empirical $Z$-maps and an estimation using anatomical
    alignment.
    %
    A left-out subject's $Z$-map was estimated by projecting the training
    subjects' $Z$-maps ($N=13$) from their respective voxel space through the
    MNI152 space into the test subject's voxel space, then averaging the values
    across projected $Z$-maps;
    %
    Green dots:
    %
    correlations between empirical $Z$-map and an estimation using functional
    alignment based an transformation matrices calculated from one up to four
    runs (each lasting \unit[5.2]{m}) of the visual localizer.
    %
    Red dots:
    %
    correlations between empirical $Z$-map and an estimation using functional
    alignment based an transformation matrices calculated from one up to eight
    segments (each lasting $\approx$\unit[15]{m}) of the movie.
    %
    Blue dots:
    %
    correlations between empirical $Z$-map and an estimation using functional
    alignment based an transformation matrices calculated from one up to eight
    segments (each lasting $\approx$\unit[15]{m}) of the audio-description.
    %
    \textbf{to do: keep both, median and mean of Cronbach's? Improve legend
    somehow without crowding figure / subplots?}
   }
    \label{fig:stripplot}
\end{figure*}


\subsection{General trends in the results}

\todo[inline]{imo, results should start with describing general trends according
to the "four aspects under investigation" stated in the intro}

%
The [mean] Pearson correlation coefficients vary depending on the criterion to
be estimated (i.e. $Z$-maps of the visual localizer, movie, or
audio-description), the functional alignment procedure (anatomical vs.
functional alignment), and---in case of functional alignment---on the quantity
of a paradigm's data used to align a test subject to the \ac{cfs} as can be seen
in Fig.~\ref{fig:stripplot}.
%
However, the functional alignment procedure across criteria and predictors
reveals a monotonically increasing estimation performance the more functional
data are used to align the test subjects.


\subsection{The exemplary tests}

\todo[inline]{report descriptive statistics of the (un)transformed
correlations?}
%
%The mean Pearson correlation values [not yet Fisher transformed] between
%empirical $Z$-maps and predicted $Z$-maps via anatomical alignment were 0.xx
%($N=14$, $\overline{X}=0.xx$, $SD=0.xx$, range [?], median
% [9], 25\%, 50\%, 75\%).

\todo[inline]{I tested all samples of Fisher transformed correlations for
normality via Shapiro-Wilk test (imo, the most appropriate test here)}

\todo[inline]{of all samples that were part of one of the eleven t-test, just
one sample (4 runs of localizer in order to estimate localizer) accepted H1
(i.e. the sample is NOT drawn from normal distribution); compute Wilcoxon
signed-rank test (which is for dependent samples)?}

\todo[inline]{t-test assumes equal variances which is probably often not
fulfilled; alternative: two sample permutation test?}

\todo[inline]{correct the alpha-level (as it is done now) or the p-values?}

\todo[inline]{report descriptive statistics of the correlations?}

\todo[inline]{round the values or just write $p$<.0001}

\todo[inline]{give one-sentence interpretations / summary à la "Results
suggest..."? imo, that's not good}

\todo[inline]{so far, I did not test any difference(s) to Cronbach's; s.
comments below where it would makes sense to test difference to Cronbach's}

%
In order to investigate the differences between some conditions [well, if you
know what I mean] more closely, we post-hoc\todo{?? it's not post-hoc an ANOVA}
performed eleven [final number?] paired t-tests on Fisher z-transformed
correlation values, and set the $\alpha$-level to a Bonferroni corrected
$\alpha$ of $0.05 / 11 = 0.00\overline{45}$.

%
\textbf{For example, in case of estimating the $Z$-maps of the visual
localizer},
%
the correlations between empirical $Z$-maps and $Z$-maps predicted
via the first movie segment were significantly higher than the correlations
between empirical $Z$-maps and $Z$-maps predicted via anatomical alignment
(Fisher z-transformed, t(14)= 6.3525802, $p$=0.0000253).
%
However, functional alignment based on data of the visual localizer
(within-paradigm prediction) and audio-description was lower than anatomical
alignment independent of the number of runs / segments used to align the test
subjects to the \ac{cfs} [not tested; clear from eyeballing].
%
Whereas the functional alignment via the first movie segment (451 \acp{tr})
yielded significantly higher correlations than a functional alignment via four
localizer runs lasting 624 \acp{tr} (Fisher z-transformed, t(14)=5.8905545,
$p$<0.0000532),
%
the functional alignment via the first segment of the audio-description yielded
significantly lower correlations than the functional alignment via four
localizer runs (Fisher z-transformed, t(14)=-2.3000009, $p$<0.0386588, will not
be significant when Bonferroni corrected).
%
The mean correlation between empirical $Z$-maps and $Z$-maps predicted via the
first movie segment were significantly lower than the correlations between
empirical $Z$-maps and $Z$-maps predicted via the first two movie segments
(Fisher z-transformed, t(14)= -5.4946197, $p$=0.0001031, Bonferroni corrected).
%
The difference between mean correlations based on two movie segments in
comparison the mean correlation based on three movie segments was not
significant (Fisher z-transformed, t(14)= -0.1293547, $p$=0.8990569).

\todo[inline]{possibly run test that compares 8 movie segments to Cronbach's
alpha? a.k.a. 8 runs of movie for functional alignment are "not as good" as
Cronbach's alpha}



\textbf{In case of estimating the $Z$-maps of movie,...}

\todo[inline]{following phrasing is just copied from above; values are adjusted
accordingly}

...the correlations between empirical $Z$-maps and predicted $Z$-maps via the
first movie segment were significantly higher than the correlations between
empirical $Z$-maps and predicted $Z$-maps via anatomical alignment (Fisher
z-transformed, t(14)= 5.7754451, $p$=0.0000643).
%
Whereas the functional alignment via the first movie segment (451 \acp{tr})
yielded significantly higher correlation than a functional alignment via four
localizer runs lasting 624 \acp{tr} (Fisher z-transformed, t(14)=6.8532349,
$p$<0.0000116),
%
the functional alignment via the first segment of the audio-description yielded
no significantly different correlations than the functional alignment via four
localizer runs (Fisher z-transformed, t(14)=-1.8674144, $p$<0.084551).
%
The mean correlation between empirical $Z$-maps and predicted $Z$-maps via the
first segment alone of the movie were significantly lower than the correlations
between empirical $Z$-maps and predicted $Z$-maps predicted $Z$-maps via the
first two movie segments (Fisher z-transformed, t(14)= -3.7454592,
$p$=0.0024485).
%
The prediction performance based on data of the audio-description increases with
more data. However, the "relevant" test compared "similar" amount of \acp{tr}:
%
The difference between mean correlations based on two movie segments in
comparison the mean correlation based on three movie segments was [not]
different [when adjusted alpha-level used] (Fisher z-transformed, t(14)=
-2.5759899, $p$=0.02303).

\todo[inline]{possibly run test that compares x movie segments for alignment to
Cronbach's alpha of movie's empirical $Z$-map? a.k.a functional alignment will
"sooner or later be better" than Cronbach.}


\textbf{In case of estimating the $Z$-maps of the audio-description,...}

\todo[inline]{This part is the ugly one; all samples will probably reject H0
assuming normality}

%
...correlations between empirical $Z$-maps and predicted $Z$-maps via the first
movie segment were significantly lower than the correlations between empirical
$Z$-maps and predicted $Z$-maps via anatomical alignment (Fisher z-transformed,
t(14)= -4.2004329, $p$=0.0010387, Bonferroni corrected).

\todo[inline]{test a differences compared to Cronbach's a? e.g.,
audio-description for alignment will sooner or later be significantly higher
than Cronbach's; might be a cue that SRM does a denoising of the e.g., map (and
does not estimate "reliable patterns deviant from the norm" wrongly)}

\todo[inline]{the three outliers are always sub-02, sub-05, sub-20 (always=
anatomical alignment, 4 runs of localizer, 8 runs of movie, 8 runs of
audio-description)}





\pagebreak


\section{Discussion}

\todo[inline]{generally, I think it's best to abstract A LOT from the specific
results (other studies discuss general stuff only, too; fucking joke, anyway)}

\todo[inline]{Therefor, two possibilities: a) write general shit \& give some
concrete examples (a.k.a. statistical tests) to make the point, or b) discuss
concrete examples (a.k.a. the tests) \& draw the general inference (how it is
done now)}

\todo[inline]{Problem: \citet{feilong2022individualized} assess data quantity,
too; results suggest 30m are "good" in case of their model.}

\todo[inline]{point for the discussion: what makes the prediction of the
localizer based on localizer runs for alignment so "bad"? Is it our model or the
transformation matrices? In any case, our results are different from
\citep{haxby2011common}: CFS \& transformation matrices based on controlled
paradigms were as good as / better than CFS \& transformation matrices based on
movie when estimating results of controlled paradigm; if it is discussed
(lengthy), then add findings from \citet{haxby2011common} that compare (within-
and across-paradigm) prediction based on naturalistic \ac{cfs} (more TRs) to
non-naturalistic \ac{cfs} (less TRs) here; if this is mentioned here, then also
define  within- and across-paradigm prediction here already and not below in
"Here, we..."}

\todo[inline]{point for the discussion: the highest correlations between
individual regressors \& shared responses can be seen in the visual localizer
(the model cannot be that bad for the localizer TRs?) => but prediction via
localizer runs suck when localizer results are predicted: interpretation?}

\todo[inline]{point for the discussion:  when audio PPA is estimated, the lowest
correlation between predicted and empirical $Z$-maps are in those participants
that have a poor Cronbach's (and no significant cluster for the audio PPA in
\citet{haeusler2022processing}); imo, they not "reliable outliers" being
different from the norm but probably just noisy asses; hence, SRM should
probably do more of a denoising as opposed to modeling (reliable) outliers
incorrectly}

\todo[inline]{even if the \ac{glm} model is freaking awesome, does a
naturalistic stimulus provide enough events sufficiently balanced across
segments when $t$-contrasts for whatever functions are supposed to be modeled?
Naturalistic stimuli are not the panacea...!}


\subsection{Short summary of aims \& hypotheses, method, results}

\todo[inline]{if necessary, write similar text at the beginning of results}

\todo[inline]{Try to start with aims \& goals before reviewing methods}

\todo[inline]{in general, it's probably (still) too long}


\paragraph{Methods}
%
In the present study, we estimated the results of $t$-contrasts that were
created in previous studies \citep{sengupta2016extension,
haeusler2022processing} in order to localize the \ac{ppa}.
%
Following a leave-one-subject-out cross-validation, the $t$-contrasts' empirical
$Z$-maps serving as the to be predicted criteria were estimated using an
anatomical alignment approach \citep[cf.][]{zhen2015quantifying,
zhen2017quantifying} and a functional alignment approach.
%
In case of the functional alignment approach, we fit a shared response model
\citep{chen2015reduced} to the training subjects' concatenated response time
series of three different paradigms in order to acquire a \ac{cfs} and the
training subjects' subject-specific transformation matrices.
%
In order to acquire the test subject's transformation matrix, we used an
increasing amount of the test subject's response time series of the three
paradigms separately as predictors by functionally aligning the test subject
to the corresponding \acp{tr} of the shared feature space (i.e. \ac{cfs}).
%
The empirical $Z$-maps of each training subject were projected from their
respective voxel space through the \ac{cfs} into the test subject's voxel space
to yield the test subject's predicted $Z$-maps.
%
In case of the anatomical alignment approach, the training subjects' $Z$-maps
were projected via nonlinear, volume-based transformation through the MNI152
brain atlas \todo{kind of incorrect} into the test subject's voxel space.

%
The estimation performance of the alignment approaches was quantified by
correlating the empirical $Z$-maps with their respective predicted $Z$-map
(which can be found at
\href{https://identifiers.org/neurovault.collection:12340}{\url{neurovault.org/collections/12340}}).


\paragraph{Three aspects under investigation}

\todo[inline]{strictly speaking, with comparison to anatomical alignment it's
four aspects}

\todo[inline]{biggest issue for discussion: how to separate validity /
generalizability of \ac{cfs} from validity / generalizability of transformation
matrices? If I do not find a solution for that the intro should be adjusted (to,
maybe, just two points?)}

\todo[inline]{the highest correlations between individual regressors and shared
responses can be seen in the visual localizer => but the matrices based on
localizer runs suck the most for estimation; interpretation?}

\todo[inline]{in any case: within-paradigm prediction is more about validity of
model (and matrices?); cross-paradigm prediction is more generalizability of
(model and) matrices}

%
Zero, we compare anatomical to functional alignment.
%
First, we assess the \textbf{validity of the \ac{cfs}} by estimating $Z$-maps
from the analysis (i.e. $t$-contrast) of same paradigms used to align the test
subject (cross-subject-\textbf{within-paradigm} prediction).
%
Second, we assess the \textbf{generalizability of the transformation matrices}
by estimating $Z$-maps from the analysis (i.e. $t$-contrast) of paradigms that
were not used to align the test subject (cross-subject-\textbf{cross-paradigm}
prediction).
%
Third, we explore the relationship between the estimation performance of the
t-contrast's results and the quantity of data from each of the three paradigms
used to functionally align the subject with the \ac{cfs}.


\paragraph{Results}

\todo[inline]{if a high-level summary of results is supposed to be written here
(at the moment I think it does not make a lot of sense), the summary heavily
depends on the "final" phrasing of the discussion}

\todo[inline]{Maybe, it's simply better to write a stupid high-level discussion
and ---to make the point--- cherry-pick some statistical tests as concrete
examples}

Results suggest that "partial alignment" works for multi-paradigm \ac{cfs}
derived from concatenated time series of multiple paradigms.
%
For all criteria, the prediction performance improves continuously with more
data of the paradigms used as predictors.


\subsection{Functional alignment vs. anatomical alignment}
%
When the visual paradigms (i.e. localizer and movie PPA) are estimated, the
movie beats anatomical alignment with just one movie segment (which is
statistically tested).
%
Movie gets (sooner or later) as good as anatomical alignment when audio PPA is
estimated (but is worse when just one movie segment is used).

%
Localizer runs suck compared to anatomical alignment when used to estimate
localizer $Z$-map (within-paradigm prediction) and movie PPA (cross-paradigm
prediction).
%
Surprisingly, when audio PPA is estimated (cross-paradigm), localizer for
alignment gets close to anatomical alignment; i.e. audio PPA is the criterion
the visual localizer can estimate the best.\todo{but why???}

%
Audio-description sucks---compared to anatomical alignment---for cross-paradigm
estimation.
%
Though, it gets better when more / all segments are used for alignment:
%
For within-paradigm prediction, audio-description eventually outperforms
anatomical alignment (when more than 4-6 segments are used)
%
Generally, the more data of the audio-description is used, the narrower the gap
between movie and audio-description gets.

%
For every alignment procedure (functional as well as anatomical), the outliers
of Cronbach's of audio PPA are difficult to estimate.


\paragraph{Summary \& conclusion on anatomical vs. functional alignment}

\todo[inline]{Results suggest...? Interpretation}

Anatomical alignment is actually not that bad.
%
It's pretty consistent across paradigms, too (always a median correlation of
about 0.6).



\subsection{Functional alignment across predictors \& criteria}

\todo[inline]{at the moment, I think it's best to order by paradigm used as
predictor; for every predictor, discuss each criterion and the quantity of
data}

\todo[inline]{problem: \citet{feilong2022individualized} already assessed data
quantity; results suggest 30m are "good" in case of their model; plus, they
model individual component.}

The multi-paradigm model is kind of valid, I guess...






\subsubsection{Validity \& generalizability of movie}

\todo[inline]{a.k.a. movie is best (which is in line with previous research)}

\paragraph{criterion: localizer}

\paragraph{criterion: movie PPA}

%
It's not a dynamic localizer employing video snippets of landscapes and faces
but a "real naturalistic localizer".

\paragraph{criterion: audio PPA}

\paragraph{general stuff about quantity of data}

%
Results indicate that $\approx$15 minutes (\ac{tr}=2s) of movie watching used
for functional alignment outperform prediction using anatomical alignment.
%
Prediction performance further increases when $\approx$30 minutes of movie data
submitted to the algorithm to calculate the subject-specific transformation
matrices.
%
However, three segments ($\approx$45 minutes) do not lead to an significantly
increased estimation performance suggesting a decreasing benefit of longer
scanning time than 30 minutes during audio-visual naturalistic stimulation.



\subsubsection{Validity \& generalizability of visual localizer}

\todo[inline]{a.k.a. why is it so "bad"???}

\todo[inline]{We have way more \acp{tr} in case of naturalistic stimuli; however
not necessarily equal number of events; and events are not "clean" events but
heavily confounded by other stuff}

\paragraph{criterion: localizer}

%
Surprisingly, prediction of localizer via localizer runs (i.e. within-experiment
prediction) is worse than anatomical alignment.
%
How does that come? At least the correlations of regressors with shared
responses show the "clearest" correlations of a regressor with a shared response
(which is not the case in the \acp{tr} of the movie or audio-description).
%
Maybe, our naturalistic stimuli (about 7000 TRs) fucked up our shared responses
in the 450 \acp{tr} within the SRM (imo, Fig.~\ref{fig:corr-vis-reg-srm}
suggests otherwise)


\paragraph{criterion: movie PPA}

\paragraph{criterion: audio PPA}

\paragraph{general stuff about quantity of data}




\subsubsection{Validity \& generalizability of audio-description}

\todo[inline]{a.k.a. it's totally different from different previous studies!}

\todo[inline]{here it is about audio-description as predictor!}

\todo[inline]{below is a paragraph about "audio PPA as criterion"; separate
points more clearly or somehow merge if the two topics have too many
intersecting points}


\paragraph{criterion: localizer}
%
A daring cross-modal prediction.
%
Kind of works, but you need a shit ton of data, i.e. it gets better, the more
data are available


\paragraph{criterion: movie PPA}
%
Another daring cross-modal prediction (similar pattern, eventually outperforms
visual localizer for alignment and gap between audio-description and movie for
alignment gets narrower the more data are used).

\paragraph{criterion: audio PPA}
%
Even with just one segment it outperforms (slightly) the functional alignment
using localizer runs or movie segments (not statistically tested though).
%
Eventually, audio-description based functional alignment outperforms anatomical
alignment.


\paragraph{general stuff about quantity of data}
%
It does not look like "garbage in, garbage out".

\paragraph{Inference}

\todo[inline]{yeah, what does that mean?}

%
Audio-description is lacking visual stimulation, audio-description is
"not-as-rich" as the movie.
%
Additionally, maybe, the auditory response in PPA is (too) different from visual
response?
%
Still, results show that data collected during listening to an audio-description
[which is richer than a mere narrative] could, in principle, be used to estimate
a visual category-selective area [but impractical amounts of data with current
approach].

%
Interesting would be estimation of results from a controlled speech
paradigm, i.e. another same-modality criterion.


\subsection{Interim summary: Validity \& generalizability of multi-stimulus
model and validity \& generalizability of matrices}

\todo[inline]{how to separate validity \& generalizability of \ac{cfs} from
validity \& generalizability of matrices?}

%
Our results provide evidence that transformation matrices calculated based on
data from naturalistic stimuli promise an increased validity and [or?]
generalizability for functional alignment over transformation matrices based on
data of a controlled paradigm based on simplified stimuli.



%
This is the case for both within-paradigm prediction (e.g., audio-description
for alignment to estimate $Z$-maps from the audio-description's $t$-contrast)
and cross-paradigm prediction (e.g. audio-description for alignment to estimate
$Z$-maps from the visual localizer's $t$-contrast).

%
One segment of audio-description is not statistically different from four
runs of localizer to estimate the localizer (after Bonferroni correction;
p=0.03).

\todo[inline]{probably discuss \citet{haxby2011common} here; cf. templates at
the very end of this chapter}

\todo[inline]{if \citet{haxby2011common} is discussed (a lot), it should
probably be primed before "Here, we..." in the introduction}

%
Our results are different results of \citet{haxby2011common} who found that the
prediction of $Z$-maps from a controlled paradigm via matrices \& \ac{cfs} based
on the same controlled paradigm was as good or better than prediction (of the
same) $Z$-maps via matrices \& \ac{cfs} based on movie data.

%
In summary, the multi-stimulus \ac{cfs} generalizes over paradigms
to be estimated but performance depends on the paradigm used to align the test
subject to the \ac{cfs}.



\subsection{More specific stuff}

\subsubsection{Localizer as criterion: "ground truth"?}

\todo[inline]{imo, this is more a topic for the general discussion}

\todo[inline]{cf. general discussion: pros \& cons of naturalistic stimuli}


\subsubsection{audio PPA as criterion: the issue of reliability}

\todo[inline]{also true (but less severe) in case of movie PPA}

\todo[inline]{topic for general discussion, too: pros \& cons of naturalistic}

\todo[inline]{the "deviant" participants in the audio-description are the ones
that have a poor Cronbach's; imo, they're not "reliable outliers" being
different from the norm but noisy asses}

\todo[inline]{but potential point to make: results could be interpreted as the
SRM doing some denoising (as opposed to modeling reliable outliers incorrectly)}

Results of \citet{haeusler2022processing} could be influenced by paradigm and
methodological choices.


\paragraph{Issues of methodological decisions}

\todo[inline]{a.k.a. operationalization \& construct validity}

%
\citet{haeusler2022processing} might have modeled the time course of "spatial
responses" "insufficiently".
%
However, the primary audio-description contrast in
\citet{haeusler2022processing} ``yielded bilateral clusters in nine participants
that are within or overlapping with the block-design localizer results.
%
For another participant (sub-09) the analysis yielded one cluster in the
left-hemispheric PPA'' \citep{haeusler2022processing}.
%
Hence, we should have modeled the construct "correctly" for >50\% of subjects in
\citet{haeusler2022processing}.


Is the number of events across segments too unbalanced?
%
Here again, the dead-end argument is that it worked for most subjects.


\paragraph{Issues of the paradigm}
%
Results of \citet{haeusler2022processing} suggest that the audio-description
samples the responses to (at least auditory) spatial information.
%
Moreover, the plots of shared responses \& AO regressors (cf.
Fig.~\ref{fig:corr-ao-reg-srm}) suggest that the shared responses during
\acp{tr} of the audio-description are not total garbage.
%
Shared responses are correlated with specific regressors of the visual
localizer, whereas shared features during \acp{tr} of the movie /
audio-description seem to be more abstract (or the regressors simply suck).
%
Finally, the within-paradigm prediction of audio PPA works for most subjects;
%
which also suggests that the Cronbach's-a-outliers do not degrade the model too
much.


\todo[inline]{No, I am not eager to look at the outliers' first-level
$Z$-maps in detail...}

%
So, maybe, the audio-description gave too much room for variations across the
time of the paradigm in some subjects?
%
It might be an issue auditory domain in general (whatever that is supposed to
mean)?
%
It might be the case that a 2h-long auditory stimulation is not as immersive /
engaging as the flashy multi-modal audio-visual movie?
%
Problem to control: the attentional focus is harder to judge / control (compared
to eye-tracking during movie);
%
alertness (via EEG) should be possible though.



\paragraph{Inference}

\todo[inline]{add text about SRM as denoising technique?}
%
But still, current results suggest that the audio-description engages the
process of auditory spatial information reliable across subjects in such a way
that the ``SRM will improve sensitivity for detecting a cognitive process of
interest in the test data'' \citep{cohen2017computational}.


\todo[inline]{Following are pretty bold statements}
%
Present results support evidence that results in \citet{haeusler2022processing}
that are restricted to the anterior part of the localizer PPA are not based on
the methodological decisions.
%
Further, present results add evidence that that the responses to auditory
spatial information lead to different activity patterns than visual stimulation
[cf. interpretation in \citet{haeusler2022processing}].
%
We need---of course!---further studies that use controlled paradigms to
investigate auditory spatial information, and studies that use auditory
narratives and employ more sophisticated models of event structure in order to
assess the suitability of naturalistic stimuli as "true naturalistic localizer".



\subsubsection{audio PPA as criterion: the issue of individual differences}

\todo[inline]{imo, this topic can be dropped because the differences are not
reliable}



\subsection{Vision: calibration scan + database}

% A shared calibration scan across datasets could be used to transfer data
% between datasets, a procedure that is easier to accomplish than shared
% subjects across datasets \citep[cf.][an extension of the \ac{srm} for shared
% subjects across datasets]{zhang2018transfer}.

\todo[inline]{text needs to match with general discussion's "Vision"}

\todo[inline]{focus on what we did in the present study}

\todo[inline]{especially, "clinical context" belongs to general discussion}



\subsubsection{Intro}

\todo[inline]{repeat phrasing from introduction (cf. localizer paradigms)}

\todo[inline]{do not open pandora's box of "functional atlas"!!}

%
Our results suggest that additional 15 minutes functional scanning using an
engaging naturalistic stimuli could provide sufficient data for a
\textit{calibration scan}.
%
A standardized calibration scan could be used to align a new subject to a
\ac{cfs} that was derived from extensive scans of a reference group.
%
Extensive scans of the reference group would comprise data collected from
naturalistic paradigms but also controlled paradigms considered to be the ``gold
standard'' to probe low-level auditory or visual perception, or higher cognition
like theory of mind \citep{spunt2014validating} or semantic processes
\citep{fedorenko2010new, fernandez2001language}.
%
The diagnostic run would be based on a multifaceted naturalistic stimulus that
samples a broad range of brain states in order to allow a valid alignment to the
reference \ac{cfs}.
%
Compared to a diagnostic run based on a controlled paradigm, a naturalistic
stimulus would have the additional benefit of higher engagement and better
compliance \citep{vanderwal2015inscapes, eickhoff2020towards} in, e.g., children
or a patients.


\paragraph{application: estimate \& quantify regular vs. deviant pattern}

%
Once a new subjects is aligned to the \ac{cfs}, functional patterns collected in
a reference group could be mapped through the \ac{cfs} into the new subject's
voxel space.
%
On the one hand, this would allow to estimate regular patterns in a new subject
when additional functional scans are not possible to scanner availability,
time-limitations or monetary constrains, or compliance issues.
%
On the other hand, this would allow to quantify the similarity (or difference)
of a new subject's actual pattern (i.e. a empirical $Z$-map) to a pattern
estimated from a healthy or clinical reference group.


\paragraph{Examples for clinical populations}

%
Patient populations, such as patients are blind
%
[Mahon et al. 2009; He et al., 2013; Striem-Amit et al. 2012b; van den Hurk et
al. 2017, Wolbers et al., 2011],
%

\paragraph{Example: \citet{yates2021emergence} quantify difference}

\todo[inline]{This is essentially the abstract of \citet{yates2021emergence}}

For example, \citet{yates2021emergence} tested ``the presence and localization
of adult functions in children using shared response modeling.
%
The feature space was learned from fMRI activity of adults watched a movie.
%
The shared features were then translated into the anatomical brain space of
children 3--12 years old.
%
The found reliable correlations between reconstructed activity and children's
actual fMRI activity as they watched the same movie.
%
The strength of the correlation in the precuneus, inferior frontal gyrus, and
lateral occipital cortex predicted chronological age''
\citep{yates2021emergence}.



\subsection{Shortcomings \& future questions}

\subsubsection{Bigger sample size}
%
The SRM is computationally less demanding [= "computationally more efficient"?]
than hyperalignment, an advantage for scientists who want to replicate our
results but do not have access to a high-performance computer cluster [or
"high-throughput computer cluster" or simply "specialized hardware"?].
%
Moreover, it should scale pretty well (compared that to
\citep{jiahui2020predicting, jiahui2022cross} 1-step alignment that needs
pair-wise matrices for subjects 'cause no \ac{cfs}; similarly,
\citep{busch2021hybrid})\todo{check Busch}.


\paragraph{...because our sample size was small with following effect}

\todo[inline]{topic might be dropped}

% what is the case
The correlations of shared responses within the \acp{cfs} created from $N-1$
training subjects varied across the folds of the cross-validation.
% interpretation
That means, a change of 1/13 of the data for every subject's analysis is causing
the estimates to vary [how much?].
% conclusion
Future studies, should create a \ac{cfs} based on data from more subjects and
investigate the relationship between number of participants, variability of
parameters, and estimation performance.



\subsubsection{Other ROIs than category-selective areas}

\todo[inline]{a.k.a. explore estimation performance in case of other functions}

\todo[inline]{don't write too much here}

\todo[inline]{more a topic for general discussion: naturalistic stimuli
might be limited (e.g., might not sample executive functions sufficiently)}

%
In the present study, we focused on the \ac{ppa} as an classic example of a
higher-visual area.

%
However, studies have shown a varying degree of \textit{functional--anatomical
correspondence} between a brain function and its underlying anatomical location.
%
Previous studies that used either volume-based \citep{zhen2017quantifying,
zhen2015quantifying} or surface-based alignment \citep{rosenke2021probabilistic,
frost2012measuring} in order to estimate the most probable location of
functional area in a test subject have ``found a large variability in the degree
to which functional areas respect macro-anatomical boundaries across the
cortex'' \citep{frost2012measuring}.
%
``There is a strong structural-functional correspondence in some areas whilst in
others the spatial location of the functional area varies greatly across
subjects within a cortical area'' \citep{frost2012measuring}.

%
For example, in domain of category-selective areas, the interindividual
variability varies across functional areas \citep{zhen2017quantifying,
zhen2015quantifying, frost2012measuring}:
%
``Scene-selective regions showed larger interindividual variability [after
nonlinear volume-based alignment] than the face-selective regions in spatial
topography'' \citep{zhen2017quantifying}.




\paragraph{Studyforrest dataset's other localizer t-contrasts}

\todo[inline]{does it make sense to discuss that? it's such a low-hanging fruit}

\todo[inline]{if it makes sense to be discussed, discuss it here and not in the
general discussion}

%
The studyforrest dataset's visual localizer \citep{sengupta2016extension} offers
other contrasts (and thus \acp{roi} masks) aimed at localizing the \ac{ffa} and
\ac{ofa} that are associated with face perception \citep{kanwisher1997ffa,
pitcher2011occipitalfacearea}, the \ac{eba} that is associated with the
perception of human bodies \citep{downing2001bodyarea}, and the \ac{loc} that is
associated with the perception of (small) objects (like tools or toys)
\citep{malach1995loc}.

Quickly shifted from general discussion to here:
%
The visual localizer performed by \citet{sengupta2016extension} employed images
from six categories (houses, landscapes, faces, bodies without heads, small
objects, and scrambled images).
%
As a results, the corresponding dataset provides subject-specific \acp{roi}
masks for higher visual areas besides the \ac{ppa}:
%
the fusiform face area (FFA) \citep{kanwisher1997ffa} and the occipital face
area (OFA) \citep{pitcher2011occipitalfacearea},
%
the extrastriate body area (EBA) \citep{downing2001bodyarea},
%
and the lateral occipital complex (LOC) \citep{malach1995loc}.
%
Future studies (e.g., a master's thesis or part of a PhD project) could adjust
our extension of the annotation of speech created in study 2 and the
corresponding analysis pipeline in order to explore hemodynamic responses
correlating with auditory information related to faces, body parts or small
objects.


\paragraph{Examples}

\todo[inline]{shorten this part}

% Zhen 2017
``Scene-selective regions showed larger interindividual variability [after
nonlinear volume-based alignment] than the face-selective regions in both
spatial topography and functional selectivity'' \citet{zhen2017quantifying}.


% category-specific areas
Similar to nonlinear volume-based alignment, similarity across person is higher
after surface-based alignment ``for retinotopically defined regions, with
character-selective regions showing the lowest consistency for both alignments,
closely followed by mFus- and IOG-faces'' \citep{rosenke2021probabilistic}.

%
``We localized 13 widely studied functional areas and found a large variability
in the degree to which functional areas respect macro-anatomical boundaries
across the cortex'' \citep{frost2012measuring}.
%
``The percent gain in overlap [after surface-based alignment] differed greatly
across the different functional regions throughout the cortex''
\citep{frost2012measuring}.
%
``There is a strong structural-functional correspondence in some areas whilst in
others the spatial location of the functional area is not tightly bound to
anatomical landmarks and varies greatly across subjects within a cortical area''
\citep{frost2012measuring}.
%
``Language areas were found to vary greatly across subjects whilst a high degree
of overlap was observed in sensory and motor areas'' \citep{frost2012measuring}.

%
``Area LOC also showed increased overlap after CBA with a 62.7\% gain in the
left hemisphere and 38.4\% on the right'' \citep{frost2012measuring}.
%
Finally PPA exhibit more gain in the right hemisphere with 27.7\% gain, than on
the left with 17.6\%'' \citep{frost2012measuring}.
%
``FFA varies in its location along the length of the fusiform gyrus even though
the gyri themselves are well aligned across subjects''
%
The FFA did not exhibit the same strong structural-functional correspondence and
saw more modest increases in overlap after macro-anatomical alignment with
44.1\% and 12\% gain for the left and right hemispheres''
\citep{frost2012measuring}.



\paragraph{More severe: language areas}

\todo[inline]{If this is mentioned at all, write half a sentence about speech
lateralization; problem: even in case whole-cortex alignment algo is used,
speech areas probably vary too much (search lights a probably too small in case
of atypical speech topographies)}

%
More severe: ``Language areas were found to vary greatly across subjects whilst
a high degree of overlap was observed in sensory and motor areas''
\citep{frost2012measuring}.


\paragraph{Inference}

\todo[inline]{Mention whole-cortex (searchlight) alignment here already?
...instead of writing a dedicated treatise about "Other functional alignment
algorithms" below?}

\todo[inline]{maybe, cite studies that did research on other brain functions
using naturalistic stimuli; or just cite a good review}

%
Hence, a new dataset should not just have more subjects but also more
localizers.
%
Then, future studies can investigate other functional domains.
%
Naturalistic stimuli have been used to investigate a variety of domains like XYZ
[check for, e.g., visual or auditory perception, spatial cognition; emotion;
music, speech or social perception], and possibly allow valid mapping of
functional data in all of these domains.


\subsubsection{\ac{srm} is problematic in case of atypical organization /
outliers}

\todo[inline]{Check \citet{feilong2022individualized, jiahui2022cross,
turek2018capturing}; at least two of them have modeled an individual component.}

%
A drawback of the SRM algorithm we employed here is that it models responses
that are common across persons without an individual component.
%
A person's idiosyncratic responses are excluded from the shared response model
but ``are not necessarily noise and may in fact be highly reliable within
participants'' \citep{cohen2017computational}.
%
In order to predict a reliably atypical pattern (and not just quantify the
deviation from norm), you need matching \ac{cfs} or---probably better---model
has to have a shared response + individual component + noise.
%
Assess performance of whatever functional alignment algo to [directly] estimate
[reliable] outliers in a sample.
%
Do you estimate the regular pattern (which allows you do quantify the difference
of a deviant actual pattern to a norm) or do you directly estimate the deviant
pattern from regular reference? Hm...

%
Imo, present results suggest estimation of regular pattern (a.k.a. denoising).


\paragraph{Templates that can probably be dropped}
%
``SRM can be used to isolate participant-unique responses by examining the
residuals after removing shared group responses, or it can be applied
hierarchically to the residuals to identify subgroups [\citet{chen2017shared}]
'' \citep{cohen2017computational}.
%
``In cases where each subject's unique response is of more interest than the
shared signal, SRM can be used to factor out the shared component thereby
isolating the idiosyncratic response for each subject
[\citep{chen2015reduced}]'' \citep{kumar2020brainiak}.
%
``Recognizing that signal exists beyond the average or shared response of a
group, such studies exploit idiosyncratic but stable responses to account for
previously unexplained variance in brain function, behavioral performance and
clinical measures [e.g., Finn (2015). Functional fingerprinting (based on
connectivity)]'' \citep{cohen2017computational}.




\subsubsection{Other functional alignment algorithms}

\todo[inline]{In general, a big treatise of pro \& cons of different algorithms
needs to be avoided}

\todo[inline]{Evtl. kann man alle Punkte dieser Subsubsection bereits oben
verwursten; hier kommen die Punkte eh so ein wenig zusammenhangslos als
Selbstzweck und ziehen allgemeine ganz am Ende die Ergebnisse runter (``a.k.a.
"warum hast du nicht gleich Algo XY benutzt?'')}


\subsubsection{Volume- vs surface-based}

\todo[inline]{An "issue" for reviewers but, imo, it's not a real issues}

\todo[inline]{$Z$-maps of localizer were calculated in voxel space; cf. general
discussion (opportunity costs)}

\todo[inline]{we are nearer on the raw data (less error accumulation?) 'cause we
work with voxels (surface vertices need additional mapping of $Z$-maps
calculated from smoothed data), and both our anatomical and functional alignment
use voxels as input (i.e. it is not mixed)}

We compare volume-based [nonlinear] anatomical alignment to volume-based
functional alignment.
%
Future work could compare surface-based alignment that respects cortical folding
structure -- that out-performs predictions based on [affine] volume-based
anatomical alignment \citep{weiner2018defining} -- to surface-based functional
alignment.


\subsubsection{ROI vs. whole-brain (i.e. searchlight)}

\todo[inline]{Maybe, shift that to "future studies on other ROIs"}

\todo[inline]{searchlight SRM \citep{zhang2016searchlight}}

\todo[inline]{negative: more parameters to vary and to assess}

``Searchlight functional alignment [\citep{zhang2016searchlight,
guntupalli2016model}] learns local transformations and aggregates them into a
single large-scale alignment.
%
The searchlight scheme [Kriegeskorte, 2006, Information-based functional brain
mapping], popular in brain imaging [Guntupalli et al., 2018; 2016], has been
used as a way to divide the cortex into small overlapping spheres of a field
radius.
%
This method allows researchers to remain agnostic as to the location of
functional or anatomical boundaries, such as those suggested by
parcellation-based approaches.
%
A local transform can then be learned in each sphere and the full alignment is
obtained by aggregating (e.g. summing as in \citep{guntupalli2016model} or
averaging) across overlapping transforms.
%
The aggregated transformation produced is no longer guaranteed to bear the type
of regularity (e.g orthogonality, isometry, or diffeomorphicity enforced during
the local neighborhood fit)'' \citep{bazeille2021empirical}.
%
``In the case of searchlight Procrustes, we selected searchlight parameters to
match those used in Guntupalli et al. (2016):
%
each searchlight had 5 voxel radius, with a 3 voxel distance between searchlight
centers'' \citep{bazeille2021empirical}.


\subsubsection{Time series vs connectivity-based}

\todo[inline]{kind of a killer cause you do not need intersection of
time series}

\todo[inline]{avoid a treatise by stating in the intro that response based
aligns responses better, connectivity-based aligns connectivity better}

\todo[inline]{what did \citep{nastase2019leveraging} do? As of now, I do not
care anymore tbh}

\todo[inline]{\citet{jiahui2022cross} do connectivity-based 1-step
hyperalignment across different movie datasets which is as good as response
hyperalignment; however, it's a shitty procedure to scale because the need a
transformation matrix for every pair of subjects (i.e no \ac{cfs})}


%
``Response-based hyperalignment (RHA) mapped data from the anatomical space to a
common information space based on time-point response patterns across cortical
vertices.
%
Connectivity-based hyperalignment (CHA) mapped data from the
anatomical space to a separate common information space based on functional
connectivity patterns derived from the movie response data''
\citep{busch2021hybrid}.

``Response-based hyperalignment was shown to align response-based data better
than connectivity-based hyperalignment, whereas connectivity-based
hyperalignment was shown to better align connectivity-based data than
response-based hyperalignment [Guntupalli et al., 2018]''
\citep{busch2021hybrid}.
%
``Response-based common spaces better align held-out response data, whereas
connectivity-based common spaces better align held-out connectivity data.
[Guntupalli et al., 2018] \citep{busch2021hybrid}.

%
``Both connectivity hyperalignment and response hyperalignment increased ISCs
and bsMVPC classification accuracies significantly over anatomy-based alignment,
but each algorithm achieves better alignment for the information that it uses to
derive a common model, namely connectivity profiles and patterns of response,
respectively'' \citep{guntupalli2018computational}.


%
In other words, RHA outperforms CHA on response-based metrics of alignment,
whereas CHA outperforms RHA on connectivity-based metrics'' ``Hyperalignment
projects cortical pattern vectors into a common, high-dimensional information
space \citep{haxby2020hyperalignment}.

%
Derivation of this common space can be based on either neural response profiles
(e.g. data collected during tasks, such as movie viewing (Haxby et al., 2011))
or functional connectivity profiles files \citep{guntupalli2018computational}''
\citep{busch2021hybrid}.

%
``The number of voxels that can be considered simultaneously for functional BOLD
response time series alignment is limited by the number of timepoints in the
calibration scan (about 300-400 voxels for a 15min scan with a 2s TR,
corresponding to a local cortical neighborhood of about 1cm in diameter for a
standard resolution).
%
This limitation does not exist in this form for a functional alignment that is
based on connectivity vectors.
%
The length of these connectivity vectors is determined by the number of
reference (or seed) regions in the brain'' [project proposal].

% Kumar on Nastase's ugly mofo paper
``Estimating the SRM from functional connectivity data rather than response time
series circumvents the need for a single shared stimulus across subjects.
%
Connectivity SRM allows us to derive a single shared response space across
different stimuli with a shared connectivity profile
\citep{nastase2019leveraging}'' \citep{kumar2020brainiak}.
%
``The sampling of connectivity vector space is defined by the selection of
connectivity targets, but the richness and reliability of connectivity estimates
depends on the variety of brain states over which connectivity is estimated''
\citep{haxby2020hyperalignment}.




\subsection{Conclusion}

God bless America!


\section{Data Availability}

\todo[inline]{all from PPA-Paper but with new GIN link leading to an empty repo}

% \href{https://gin.g-node.org/chaeusler/studyforrest-ppa-analysis}{\url{gin.g-node.org/chaeusler/studyforrest-ppa-analysis}}

% new; PPA analysis
All fMRI data and results are available as Datalad \citep{halchenko2021datalad}
datasets, published to or linked from the \emph{G-Node GIN} repository
(\href{https://gin.g-node.org/chaeusler/studyforrest-ppa-srm}{\url{gin.g-node.org/chaeusler/studyforrest-ppa-srm}}).
% original
Raw data of the audio-description, movie and visual localizer were originally
published on the \emph{OpenfMRI} portal
(\url{https://legacy.openfmri.org/dataset/ds000113}; \citep{Hanke2014ds000113},
\space \url{https://legacy.openfmri.org/dataset/ds000113d};
\citep{hanke2016ds000113d}).
% visual localizer
Results from the localization of higher visual areas are available as Datalad
datasets at \emph{GitHub}
(\href{https://github.com/psychoinformatics-de/studyforrest-data-visualrois}{\url{github.com/psychoinformatics-de/studyforrest-data-visualrois}}).
% raw data
The realigned participant-specific time series that were used in the current
analyses were derived from the raw data releases and are available as Datalad
datasets at \emph{GitHub}
(\href{https://github.com/psychoinformatics-de/studyforrest-data-aligned}{\url{github.com/psychoinformatics-de/studyforrest-data-aligned}}).
% OpenNeuro
The same data are available in a modified and merged form on OpenNeuro at
\url{https://openneuro.org/datasets/ds000113}.
% NeuroVault for z-maps of SRM
Unthresholded $Z$-maps of all contrasts can be found at
\href{https://identifiers.org/neurovault.collection:12340}{\url{neurovault.org/collections/12340}}.


\section*{Code Availability}

Scripts to generate the results as Datalad \citep{halchenko2021datalad} datasets
are available in a \emph{G-Node GIN} repository
(\href{https://gin.g-node.org/chaeusler/studyforrest-ppa-srm}{\url{gin.g-node.org/chaeusler/studyforrest-ppa-srm}}).






\pagebreak

\section{Backup of texts regarding "done but not mentioned"}


\subsection{Calculate $Z$-maps mean in the common space already}
%
I also tested averaging $Z$-maps in the \ac{cfs} (i.e.: not in the test
subject's voxel space): similar results
%
In case of anatomical alignment, I did
not test averaging data in MNI152 space.


\subsection{Calculate $Z$-map from training subjects' TRs in FEAT}

iirc, I projected all subjects' localizer time series through
model space into test subject voxel space; then, calculated the contrast
with these data (s. scripts 'test/data\_denoise-vis.py' \&
'test/data\_srm-vis-to-ind.py').
%

The problem was: if one wants to test the different transformation matrices (I
only did it with one; imo, based on alignment using the whole audio-description)
it gets totally messy \& computational intensive.
%
Results were similar to the original procedure if not slightly worse.


\subsection{Leakage of test data in union of individual \acp{ppa}}

%
We used the union of individual \acp{ppa} as spatial constrain for $Z$-maps.
%
But we have a leakage of test data (test subject is in data for the mask).
%
We might miss some voxels (of some participants) at the borders of the \ac{roi},
because the subject-specific, binary masks are based on a ("titrated")
threshold.  \citep{sengupta2016extension}
%
In the future, an independent probabilistic atlas should be used, the \ac{roi}
dilated, [and a separate model calculated for each hemisphere].



%\end{comment}


\chapter{General discussion}
%% In the [general] discussion, a critical evaluation of your own results is
%% made against the background of the literature. At this point, further
%% results that have not already been listed in the results section can also be
%% discussed.

% take input from external file
\todo[inline]{Phrasing here is still similar to phrasing in general intro}

Human brain mapping studies have traditionally averaged \ac{fmri} data across
participants.
%
However, data need to be assessed on the level of individual persons in order to
advance the field towards a clinical application.
% functional localizer
A promising tool to perform this advancement are/is functional localizers
[because localizers aim to characterize the location size, and shape of
functional areas on the level of individual subject].
% contra localizers
However, traditional localizer paradigms employ selectively sampled, tightly
controlled stimuli, rely heavily on a participant's compliance, and can usually
map just one domain of brain functions
% naturalistic stimuli could replace
Localizer paradigms based on naturalistic stimuli could provide higher
ecological as well as external validity, higher data quality due to increased
compliance, and potentially map a variety of brain functions [ranging from
low-level perception (e.g., luminance) to high-level cognition (e.g., social
cognition)] simultaneously.
% visually impaired
Lastly, an exclusively auditory stimulus like an audiobook or audio drama would
also be appropriate for visually impaired persons[, e.g., suffering from
nystagmus or lack of eyesight].

% PPA as proof of concept
Focussing on the \ac{ppa}, a ``classic'' higher-visual area,
\citep{epstein1998ppa}, the goal of this thesis was to explore whether a movie
and the movie's audio-description could, in principle, substitute a traditional
localizer paradigm.
% paragraph on open science
An additional goal of this dissertation was to perform all studies under the
principles of open, shared, and transparent science.
%
In order to enable independent researchers to validate current results and use
written code to replicate current findings in prospective studies, all created
data, code, analysis steps, and results are published as version-controlled
DataLad \citep[\href{www.datalad.org}{datalad.org};][]{halchenko2021datalad}
datasets.


\section{Recapitulation of work packages}

\todo[inline]{maybe, very short summary of parts here; momentarily it's a draft}

%
First, we extended the studyforrest dataset (study 1).
%
Second and similarly to traditional localizer paradigms, we modeled hemodynamic
activity based on annotated stimulus features embedded in the movie ``Forrest
Gump'' and its audio-description, and created \ac{glm} $t$-contrasts in order to
localize the \ac{ppa} (study 2).
%
Third, we estimated results of the localizer by projecting data through a
\ac{cfs} (study 3).

\todo[inline]{maybe, give an overview of how the remaining part of the thesis is
structured; at the moment: each study as such \& study in light of open science,
general discussion about open science across studies}


\subsection{Speech anno}


\subsubsection{Goal of speech anno}

% what we did in 1 sentence
In study 1 \citep{haeusler2021speechanno}, we created and validated an
annotation of speech occurring in the movie and its audio-description pursuing
two goals.
% aim #1: groundwork for PPA study
The first aim was to build the groundwork that enabled us to conduct study 2.
% aim #2: extend studyforrest
The second aim was to create an exhaustive annotation of speech that
substantially exceeds the groundwork necessary to conduct study 2 in order to
extend the studyforrest dataset as a public resource for independent research.


\subsubsection{Discussion of speech anno}

% validation analysis
We validated the annotation's quality in study 1 and performed a canonical
\ac{glm} analysis by contrasting regressors correlating with speech-related
events to a regressor correlating with events without speech.
% results
As hypothesized, results revealed statistically significant increased
hemodynamic activity in a bilateral cortical network known to be involved in the
perception of speech \citep[e.g.,][]{friederici2011brain, wilson2008beyond}.
% conclusion
These results encouraged us to a) use the annotation as the groundwork for study
2, b) publish the annotation as an extension of the studyforrest project.


\subsubsection{Open science in context of speech anno}

\todo[inline]{how "personal" am I supposed to get?}


\paragraph{Intro}

% additional effort 1
Pursuing the goal of creating a publication-worthy dataset led to additional
work that goes far beyond the work that was necessary to build the \ac{glm} in
study 2.
% additional effort 2
The published annotation provides, among others, time-stamps of phonemes, words
and sentences of all speakers, a grammatical tagging, and an annotation of
syntactic dependencies and semantics.


\paragraph{The self-flagellation retrospectively}
%
Great care was taken during the initial creation and subsequent iterative
corrections in order to provide accurate information to the scientific
community.
%
However, over the course of generating the dataset it became apparent that there
is no such thing as a ``perfect'' annotation:
%
As in human language in general, an annotation of speech will always contain
ambiguities.
%
Additionally, there is a trade-off that needed to be balanced between a) doing
the ``mere minimum'' and putting time and effort in creating additional
information that might not be fruitful, and b) providing a sound/substantial
groundwork for potential use-cases that needed to be anticipated.
%
Any further processing step might be based on a decision that might not match
the requirement of a specific use case.
%
For example, an annotation of semantics might be based on a current state-of-the
art language model that might be superseded by future language models.
%
Therefore, the published annotation does not only comprise the final outcome but
also the raw data and documented code that can automatically be rerun step by
step to reproduce the final outcome of both the annotation and its validation
analysis, all freely accessible in a version-controlled dataset.


\paragraph{Conclusion}
%
In summary, the annotation provides extensive information about the time course
of stimulus features, and therefore a headstart to independent researchers that
wish to ``model hemodynamic brain responses that correlate with a variety of
aspects of spoken language ranging from a speaker's identity, to phonetics,
grammar, syntax, and semantics'' \citep{haeusler2021speechanno} under more
real-life like conditions.
%
Consequently, the outcome of study 1 contributes to the studyforrest project as
a resource for the scientific community by further widening the ``annotation
bottleneck'' \citep{aliko2020naturalistic} of two naturalistic stimuli.


\subsection{PPA paper}

\subsubsection{Recapitulation}

\paragraph{Goal of PPA paper}
% study in one sentence
The goal of study 2 \citep{haeusler2022processing} was to explore whether an
audio-visual and an exclusively auditory naturalistic stimulus could be used in
order to localize the \ac{ppa} as it was previously identified in the same set
of participants by a traditional block-design functional localizer that employed
static pictures \citep{sengupta2016extension}.


\paragraph{Method \& results of PPA paper}
% AV operationalization
For the model-based mass-univariate statistical analysis (i.e.\ac{glm}) of the
movie's data, we operationalized the perception of visual spatial information
based on an annotation of movie cuts and depicted locations
\citep{haeusler2016cutanno}.
% AD operationalization
For the \ac{glm} of the audio-description's data, we extended the annotation of
speech \citep{haeusler2021speechanno} by further annotating nouns that the
narrator uses to describe the movie's absent visual content.

% group results: AV \& AD
On a group-average level, findings demonstrate that increased activation in the
\ac{ppa} generalizes to the perception of spatial information embedded in the
audio-visual movie and the audio-description.
% individual AD
On an individual level, semantic spatial information occurring in the
audio-description is correlated with significant activity in the anterior part
of the \ac{ppa} bilaterally in nine individuals and unilaterally in one
individual.


\paragraph{Discussion \& conclusion of PPA paper}

% conclusion 1
Results add evidence \citep[cf.][]{bartels2004mapping} that a functionally
defined region, such as the \ac{ppa}, can be localized using a model-driven
analysis that is based on a naturalistic stimulus' annotated temporal structure.
% conclusion 2
Further, results suggest that a purely auditory naturalistic stimulus like an
audio-description could potentially substitute a visual localizer as a
diagnostic procedure to assess brain functions in visually impaired individuals
[phrasing pretty similar to \citep{haeusler2022processing}].


\subsubsection{Future studies: PPA}

%
Two aspect revealed by our analyses invite further investigations on the
properties of the \ac{ppa}.
%
First, the responses correlating with an auditory stimulation are spatially
restricted to the anterior \ac{ppa}, and
%
Second, we observed higher intersubject variability of responses of the \ac{ppa}
to a naturalistic auditory stimulation compared to the visual stimulation during
the localizer and movie paradigm.


\paragraph{Just anterior PPA during auditory stimulation}

% our interpretation
Previous studies in the field of visual perception suggest that the \ac{ppa} can
be divided into functionally subregions that might process different stimulus
features.

\todo[inline]{paraphrase stuff from paper here}

%
Hence, we attributed the revealed pattern to different features inherent in the
visual stimuli compared to features inherent in the naturalistic auditory
stimulus [phrasing pretty similar to \citep{haeusler2022processing}].
%
However, our interpretation of the observed pattern ``can only be preliminary,
because the auditory stimulation dataset differs in key acquisition properties
(field-strength, resolution) from the datasets of the movie and visual localizer
representing a confound of undetermined impact'' [still pretty similar to
phrasing in \citep{haeusler2022processing}].
% conclusion
Future studies could employ controlled stimuli, maybe accompanied by a task, to
investigate in detail whether the observed differential activations during
visual and auditory stimulation are replicable.


\paragraph{Interindividual variability in response to auditory stimulation}

% statement
On an individual-level, we observed higher intersubject variability of responses
of the \ac{ppa} to a naturalistic auditory stimulation compared to the
audio-visual movie and visual localizer paradigms \citep[cf. Table 3
in][]{sengupta2016extension}.
% not necessarily noise
However, the divergent pattern from the group mean in four of fourteen
individuals should not necessarily be interpreted as measurement errors,
artefacts, or ``random noise'' \todo{choose just one term} but could also be
attributed to individual differences in responses to the task free, auditory
paradigm.
%
Our naturalistic auditory paradigm differs from block-design localizer paradigms
not just in the exclusively auditory stimulation but also in the accidental,
event-like presentation of spatial information, and the absence of a task which
leaves study participant naive to the investigated cognitive process.


\paragraph{Possibly correlated factors}

\todo[inline]{rephrase in a "readable" (but still preliminary) way}

%
The revealed pattern could correlated with [influenced by] situational factors
like the experimental design (stimulus type, no task), (transient) state of a
participant (e.g., alertness or engagement),  our simply our ``adventurous''
modeling approach[?].
%
\todo{well...}
%
More stable factors might be individual differences in cognitive tendencies or
cognitive abilities like susceptibility / predisposition [?] to attend to, [or]
recognize [or process] auditory spatial information.
% Conclusion
Future studies could employ both controlled and naturalistic stimuli to
investigate whether our results that revealed higher intersubject variability in
response to auditory spatial information are a) replicable across different
experiments and paradigms, and b) reliable within subjects.


\paragraph{Brain \& behavior: intro to "fingerprints"}

\todo[inline]{following is pretty speculative 'cause reliability of differences
is premise for "fingerprints"; hence, just a draft; does it make sense to
discuss it?}

% kanai
In case the pattern is stable within individual subjects / is ``highly
consistent across different sessions [or experiments], then they are
characteristics of the individuals and may reflect differences in their brain
function'' \citep{kanai2011structural} [on structural diffences].
%
``Individual differences in topology (i.e. location, size, shape of functional
areas) and the activity within functional areas can also be considered to be
interesting cases of inter-individual variability to understand the neural basis
of human cognition and behavior, brain-phenotype relationships'', and ``present
useful phenotypes or biomarkers \citep{glasser2016multi,
vanhorn2008individual}''
%
\todo{paraphrase}

\paragraph{Brain \& behavior: example studies}

\todo[inline]{shorten heavily or drop altogether}

%
For example, \citet{kong2019spatial} suggested based on resting-state functional
connectivity measures ``that individual-specific network topography (i.e.,
location and spatial arrangement) might serve as a fingerprint of human behavior
that can predict behavioral phenotypes across cognition, personality, and
emotion'' \citep{kong2019spatial} [with modest accuary, comparable to previous
reports predicting phenotypes based on connectivity strength].

%
\citep{bijsterbosch2018relationship}'s ``results indicate that spatial variation
in the topography of functional regions across individuals is strongly
associated with behaviour'' \citep{bijsterbosch2018relationship}.
%
\citet{bijsterbosch2018relationship} found ``that the spatial arrangement of
functional regions is strongly predictive of non-imaging measures of behavior
and lifestyle'' [however shape \& exact location of brain regions interacted
strongly with  modeling of brain connectivity].
%
\citet{bijsterbosch2018relationship} found ``that individual differences in the
size, shape and exact position of the brain regions [as identified by
resting-state functional connectivity measures] was strongly linked to
individual differences in behavioral tests and questionnaires [including
intelligence, life satisfaction, drug use and aggression problems]''
\citep{bijsterbosch2018relationship}.

%
``The variations in spatial topographical features captured a more direct and
unique representation of subject variability than temporal correlations between
regions defined by group parcellation approaches (coupling).
%
Hence, the cross-subject information represented in commonly adopted
'connectivity fingerprints' could largely reflect spatial variability in the
location of functional regions across individuals, rather than variability in
coupling strength (at least for methods that directly map group-level
parcellations onto individual data)'' \citep{bijsterbosch2018relationship}.

\todo[inline]{drop following paragraph; just here to better understand paragraph
above}
%
``Depending on the employed spatial alignment algorithm and the amount of
removed spatial intersubject variability, the degree to which spatial
information may influence FC estimates possibly varies considerably across
studies.
%
In recent years, significant efforts have gone into the methods that more
accurately estimate the spatial location of functional parcels in individual
subjects [Chong et al., 2017; Glasser et al., 2016; Gordon et al., 2016; Hacker
et al., 2013; Harrison et al., 2015; Varoquaux et al., 2011; Wang et al., 2015],
and into advanced hyperalignment approaches [Chen et al., 2015; Guntupalli et
al., 2016; Guntupalli and Haxby, 2017]'' \citep{bijsterbosch2018relationship}.


\subsubsection{Conclusion on future PPA studies}

\todo[inline]{draft; clean \& rephrase!}
%
In summary, our results invite further studies that investigate the properties
of the parahippocampal area in response to
%
a) in response to an auditory naturalistic stimulation (with or without a
simultaneous task to attend to spatial information, or with subsequent memory
task),
%
b) in response to a controlled paradigm (with or without any task).
%
The respective paradigms could elaborate whether interindividual variability of
responses in the parahippocampal area is related to cognitive processes like
``auditory scene perception'' [is that a valid term?], and to degree auditory
spatial information is utilized for, e.g., spatial orientation, way finding or
[planing, remembering, executing] navigation.


\subsubsection{Future studies: other scene-selective areas}

\todo[inline]{imo, RSC and OPA do not need to be discussed in general
discussion; could be shortly discussed in "open science in general" (->
opportunity costs)}


\subsubsection{Future studies: other functional areas (studyforrest dataset)}

\todo[inline]{imo, a side note; not necessarily needed to be discussed; hence,
draft}

The visual localizer performed by \citet{sengupta2016extension} employed images
from six categories (houses, landscapes, faces, bodies without heads, small
objects, and scrambled images).
%
As a results, the corresponding dataset provides subject-specific \acp{roi}
masks for higher visual areas besides the \ac{ppa}:
%
the fusiform face area (FFA) \citep{kanwisher1997ffa} and the occipital face
area (OFA) \citep{pitcher2011occipitalfacearea},
%
the extrastriate body area (EBA) \citep{downing2001bodyarea},
%
and the lateral occipital complex (LOC) \citep{malach1995loc}.
%
Future studies (e.g., a master's thesis or part of a PhD project) could adjust
our extension of the annotation of speech created in study 2 and the
corresponding analysis pipeline in order to explore hemodynamic responses
correlating with auditory information related to faces, body parts or small
objects.


\subsubsection{Open science in context of PPA paper}

% goal PPA study
Under the perspective of an open science project, the goal of study 2 was to use
three \ac{fmri} datasets \citep{hanke2014audiomovie, hanke2016simultaneous,
sengupta2016extension}, two stimulus annotations \citep{haeusler2021speechanno,
haeusler2016cutanno}, as well as previously published results
\citep{sengupta2016extension} for a new research question.


% skipped work
On the one hand, the availability of the \ac{fmri} data and the subject-specific
\acp{roi} enabled us to shift our focus from acquiring the raw data to
subsequent stages of a research project.
% example of skipped work
For example, the annotations that were created in a "general purpose state"
could be extended immediately to match the needs of study 2, followed by writing
the scripts that preprocessed and statistically analyzed the data.
% additional work
On the other hand, pursuing the goal of an open science project lead to
additional work.
% example of additional work
For example, code needed to be in a state worthy to be published and documented
for other readers, analyses pipelines needed to be in a state to be automatized,
every processing step documented, saved, protocolled, and shared to allow
reproducibility of results and facilitate replicability of finding.

\todo[inline]{shorten results/findings}

% results
Our results have been published in a peer-reviewed, open-access journal
%
``offer evidence that a model-driven GLM analysis based on annotations can be
applied to a naturalistic paradigm to localize concise functional areas and
networks correlating with specific perceptual processes''
\citep{haeusler2022processing},
%
``demonstrate that increased activation in the PPA during the perception of
static pictures generalizes to the perception of spatial information embedded in
a movie and an exclusively auditory stimulus \citep{haeusler2022processing}, and
%
``provide further evidence that the PPA can be divided into functional
subregions that coactivate during the perception of visual scenes''
\citep{haeusler2022processing}

% conclusion
In summary, we re-used existing data as a foundation for a new investigation in
order to generate novel findings encourage further studies, and illustrate the
benefits of publicly and freely available datasets.


\subsection{SRM study}

\todo[inline]{following parts are an old draft}

\subsubsection{Transition from study 2 to study 3}
%
Despite exploratory approach in study 2 (a.k.a. shitty modeling of subjectively
assessed events), results suggest:
%
the response to spatial information must be somewhere in within the response
time series and is detectable.
%
Hence, we might be able to use the response patterns measured during the
presentation of the audio-description in order to generate a \ac{cfs} [needs to
be defined above in overview] in study 3, and align an ``unknown'' test
participant to that \ac{cfs}.

% align left-out subject
Based on our findings in study 2 \citep{haeusler2022processing}, we assumed that
the event structure in both naturalistic stimuli would correlate, among others,
with brain responses that are similar to those correlating with the event
structure in a dedicated functional localizer.

% summary of study 2
Results of study 2 suggest that a naturalistic stimulus might provide an
engaging, task-free paradigm to localize brain functions in individual subjects.


\subsubsection{Goal of SRM study}
% the problem
Considering practical and monetary constraints in a clinical context, a paradigm
lasting 90 to 120 minutes is inappropriate for even an extensive individual
diagnostic procedure.

% goal 1: new procedure
The first goal was to assess a procedure to estimate results of a dedicated
localizer \citep{sengupta2016extension} based on data acquired during
naturalistic stimulation.
%
Following leave-one-subject out cross-validation, we estimated (i.e. predicted)
the results of the visual localizer experiment ($Z$-values of voxels within a
\ac{roi}) of a left-out test participant based on localizer results of a
reference group.

% goal 2: partial alignment
The second goal was to assessed the relationship between length of
naturalistic stimulation used to align the test participant to the fixed
\ac{cfs} and the estimation performance.
%
Lastly, the estimation performance of our new procedure based on \ac{fa} was
compared an estimation performance based on \ac{aa}.


\paragraph{Hypotheses}
%
We hypothesized that increased quantity of data used to calculate the
transformation matrices of the left-out subjects for a \ac{fa} would to increase
prediction performance.
%
Further, we hypothesized that \ac{fa} would eventually perform
``better'' than an estimation based on \ac{aa}.


\subsubsection{Discussion of SRM study}

\todo[inline]{...to be written}


\subsubsection{Future studies on SRM / functional alignment}

\todo[inline]{...to be written}


\subsubsection{Open science in context of SRM study}

\todo[inline]{...to be written}




\section{Open Science}

\todo[inline]{check: anything about open science of each paper missing here?}

\todo[inline]{Following is preliminary phrasing; part has way more captions than
necessary}

\todo[inline]{most importantly:
%
which points make sense to be brought up (and need to be phrased nicely)?
%
which points are missing?
%
which points should be left out?}


\subsection{Intro}
%
Following the practices of open and reproducible science was not mandatory for
submitting the thesis but required additional work and time.
%
``Open and reproducible research means you need to guarantee the accuracy of the
methods used and to explicitly describe and document all stages of the
scientific process to ensure its transparency and traceability''.
%
Standards to follow are not yet fully established and corresponding software
tools are still emerging.
%
Since the best practices are not yet part of a graduate or PhD curriculum,
learning about the principles and standards and applying the corresponding
procedures and necessary tools was based on self-initiative and self-learning.
%
Over the course of the present project, affected stages were
% data-related
a) collection, description and storage of data,
% processing-related
c) processing and analyzing data via code, and
% publishing shit
d) publication of data, materials and results.


\subsection{Data collection; data analysis; publishing}
%
In the context of open data, data are not collected for mere internal purposes
but re-use by third parties needs to be considered.
%
Hence, dataset creators need to anticipate which people might use the data for
which purpose, collect the data according to best practices, convert data into a
standardized format (considering, e.g., naming conventions, folder structure,
separating raw from analyzed data), and create metadata.

% data analysis & automatization
The data, materials and code need to be documented more rigorously, and coverage
of procedures need to exceed the coverage given in method sections of regular
articles.
%
Every change made to data, materials or code, and command line invocations need
to be tracked via a version control system \citep[e.g.,][]{halchenko2021datalad}
to allow other researchers to inspect a study's full history.
%
In order to allow reproducibility from input data, changes made to the data, to
computation and visualization of results, every processing step needs to be
designed and tested to be run reliably and automatically.

% publication: findable, accessible, interoperable, reusable
In the stage of preparing a publication, a researcher needs to facilitate
discovery by humans and web bots (e.g., via extensive description and
machine-readable metadata), ensure long-term curation [e.g., maintenance],
availability [e.g., accessibility], and choose an appropriate data host.
% legal issues
Finally, a researcher need to resolve legal issues that raise during due to the
publication of data, materials and code (e.g. statement of agreement / consent,
anonymization, intellectual property rights, use license).


\subsection{Cons from perspective of creators}



% jeez! it's annoying!
In general, creating open data, materials and code requires a considerable about
of time and effort with little incentives and little, immediate rewards.
%
Publishing data (or merely using open data) is associated with the risk that
other working groups, possibly having more funding and ``people and brain
power'' at disposal, are using the same data for a similar research question at
the same time.
%
Hence, there is the concern that someone else might ``claim priority, usually
through publishing, to a research idea or result you yourself have been working
on'' \citep{laine2017afraid}.
%
On the one hand, being second in the ``competitive race in the sciences'' leads
to diminished opportunities to publish own results because high impact journals
favour novel findings.
%
The risk, and therefore stress, is aggravated in case of researchers that are
early in their scientific career \citep[cf.][]{toribio2021early}, created and
curate the published dataset, pre-registered studies based on open data, or have
to stick to inflexible project plans.
%
On the other hand one advantage of publishing a dataset with an assigned DOI is
that it might get re-used and cited in another study.
%
In case of a dataset creator's fear of being superseded / outrun, ``concerns can
be alleviated by delaying the sharing or using a data-sharing repository with an
embargo period'' \citep{nichols2017best}.


\subsection{Pros from perspective of creators}

% better organized, better documented
An immediate benefit of following the principles of a reproducible study is that
a researcher is forced to work more organized and document every step from a
study's start to finish more rigorously.
%
First, documenting every step and justifying every step by weighting pros and
cons of alternative [mutual exclusive] procedural paths leads to better
understanding, avoids tricking oneself into performing unnecessary statistical
tests, HARKing hypothesizing after the results are known; Kerr, 1998,
HARKing...), and therefore supports following general good scientific practices.
%
Similarly, the extra time and effort spend on inspecting data and testing code
leads to higher confidence in one's own work and the reliability of results.
%
Second, a researcher who records all changes to data and code from the start to
the final results can restore a particular state of data and code and trace,
identify and correct errors more easily [similar: Klein, 2018. A practical guide
for transparency in psychological science].
%
Third, tracking and documenting every step increases reliability of results and
can be seen as a "lab protocol" containing information / templates for writing
scientific articles.
%
Last, automating recurring tasks gives yourself the possibility of reusing
certain data, code, documents, etc. in the future.


\subsection{Pros from perspective of consumers}

\todo[inline]{following needs revision but general train of thought should be
clear}

%
From the perspective of reader, open access journals provides low-cost access to
information.
% readers get a better understanding
A ``transparent and complete reporting of all facets of a study, allowing a
critical reader to evaluate the work and fully understand its strengths and
limitations'' \citep{nichols2017best}.
%
It helps readers to take a look at the methods in more detail it is conveyed in
a often limited method section of a regular study.
%
``This also facilitates subsequent research efforts by other investigators, who
can exactly follow (or carefully manipulate) each aspect of a study''
\citep{nichols2017best}.
%
Open materials facilitate tracking (and understanding!) the process (esp.
analyzes) in detail (pipelines are often far easier to understand by reading the
code step-by-step than just reading the method section).
%
Interesting to students how can not just read a method section but also take a
look at the code and follow step by step every command.
%
Fully transparent studies that also include the input data can serve as an
``education playground'' by enabling (undergraduate) student to trace the
progress of real world project, to learn coding and data analysis.
%
Open data provide groups that enjoy minor funding low-threshold access to
datasets.
%
While open data helps researchers to shift time and resources from data
collection to subsequent stages of a study, share code helps researchers adjust
and extent the analysis pipeline as part of an exploratory data analysis.


\subsection{Cons from perspective of consumers}

% avoid "the bigger, the better" and "garbage in, garbage out"
An mostly not explicitly stated issue in the context of open science is that
standards (quality, formats, parameters) and open sciences practices (e.g.,
documenting) might vary across scientific field, or within scientific fields
depending on a working group's knowledge and rigor.
% check the data
Dataset consumers need to assume that everything that is not explicitly stated
in the description of a dataset has not been considered and done by a dataset's
creator.
% laugh with many, don't trust any
Even if the data a from a renowned source, researchers should consider
themselves to be obliged to test and validate a dataset's quality according to
their standards and specific use cases.


\subsubsection{Tell me about opportunity costs without saying opportunity costs}

% problem of preprocessed data
Moreover, choices made during collection and preprocessing of data, despite
being state-of-the-art at the time of being published, might not be optimal for
every use case or made obsolete by more advanced methods.
% tell me about opportunity costs without saying opportunity costs
Hence, researchers choose the path with the greater net return by weighting
costs and benefits of one path (e.g., preprocessing the same data differently
than the preprocessing performed as part of an open dataset) relative to an
alternative path (e.g., using the preprocessed data and performing new
analyses).


\subsection{Short personal assessment}

\todo[inline]{This whole (sub)section might be too personal; nobody (me
included) cares about the my life's journey}

\todo[inline]{Following are some ideas to connect the dry text above to
experiences during the present thesis (that actually made me come up with the
text above)}

% intro
The additional time and effort feel like a burden and do not contribute to the
immediate benefits of the PhD project, how-fucking-ever:


\subsubsection{I used open-source software packages}

Often forgotten, but open-source packages were prerequisite.


\subsubsection{I used existing data}

\todo[inline]{Why was is good for you that the data were already existing?}

\todo[inline]{imo, irrelevant in my case: participants' consent, anonymization,
most legal issues}

\todo[inline]{Caveat of data re-use: know your data! I could write that I pulled
my hair out while searching for inconsistencies in the timings, and found that
the audio track of the audio-description is essentially unsystematically
shifted; similar cases stay probably undiscovered in non-open datasets; vice
versa, a point for less error-prone open science}

\todo[inline]{Covid-Pandemic has impressively shown that collecting data based
on human subjects" can go dormant for almost 1.5 years; plan of the project was
adjusted accordingly; and I am so lucky that I could fall back on open data}

\todo[inline]{it is/was somewhere between "stupid" and "brave" to mess with
someone like Haxby's group; also, cf. "my precious" of Susanne and Lisa}

\todo[inline]{Opportunity costs: analyzes in \citet{sengupta2016extension} were
performed and \acp{roi} published in voxel-space (i.e., not surface-space)}


\subsubsection{Opportunity costs: missing ROIS of RSC \& OPA}

\todo[inline]{does it make sense to talk about "missing" \acp{roi} of RSC and
OPA at all?}

\todo[inline]{a short version could simply state "it would be fucking awesome to
perform a follow-up study that investigates \ac{rsc}, \ac{opa}; most text is
simply a template to be summarized}

%
``Apart from the PPA, results show significantly increased activity in the
ventral precuneus and posterior cingulate region (referred to as ``retrosplenial
complex'', RSC) of the medial parietal cortex, and in the superior lateral
occipital cortex (referred to as ``occipital place area'', OPA) for both
naturalistic stimuli.
% RSC intro
Like the PPA, the RSC and OPA have repeatedly shown increased hemodynamic
activity in studies investigating visual spatial perception and navigation
\citep{chrastil2018heterogeneity, bettencourt2013role, dilks2013occipital,
epstein2019scene}'' \citep{haeusler2022processing}.

%
In the progress of conducting study 2, ``we assumed that results (at least of
the audio-visual stimulus) could also yield significant clusters in the
retrosplenial complex (RSC) and superior lateral occipital cortex (i.e.
“occipital place area”, OPA) but did not explicitly hypothesize that fact''.
%
``We (and Sengupta) chose the PPA among three possible candidates
because it was the first area to be discovered as a visual ``scene-selective''
region, is the most reliably activated region across studies that investigate
visual scene perception'' [response to reviewer \#2].

%
``Whereas the PPA is assumed to be more involved in landmark recognition by
processing basal perceptual features that constitute a scene, the RSC (i.e.
ventral precuneus and posterior cingulate region) exhibits stronger responses
when the scenes are familiar to the participants suggesting the RSC might be
more concerned with localizing, i.e. orienting, the observer in space [e.g.
Epstein \& Vass, 2016]'' [response to reviewer \#2].
% medial parietal cortex: anterior-posterior gradient
``Similarly to the parahippocampal cortex \citep{aminoff2013role}, the medial
parietal cortex exhibits a posterior-anterior gradient from being more involved
in perceptual processes to being more involved in memory related processes
\citep{chrastil2018heterogeneity, hassabis2009construction, silson2019posterior,
steel2021network}'' \citep{haeusler2022processing}.

%
''We believe that a detailed discussion of the RSC Activation in RSC and OPA was
out of scope of the current study, results are an incentive for further
studies''.
% future studies
``Future, complementary studies using specifically designed paradigms could
investigate where in the posterior-anterior axis of the parahippocampal and
medial parietal cortex auditory semantic information is correlated with
increased hemodynamic activity:
% we hypothesize
we hypothesize that the auditory perception of spatial information (compared to
non-spatial information) is correlating with clusters in the middle of possibly
overlapping clusters correlating with visual perception (peak activity more
posterior) and scene construction from memory (peak activity more anterior)''
\citep{haeusler2022processing}.

%
In the context of open science, this was kind of an limitation / aftereffect of
coverage of mask for functional areas \citet{sengupta2016extension}, and the
non-algorithmic procedure of \citet{sengupta2016extension} based on subjective
decision that is hard to replicate \& to amply to the other functional areas
\citep[cf. algorithmic procedure in, e.g.,][]{julian2012algorithmic}.
%
In summary: everything that is not 100\% automatized is not 100\% reproducible,
sadly because automatization takes a long time for some stuff or is not possible
for some stuff.


\paragraph{Code: documenting, version-controlled, automatize pipeline}

\todo[inline]{Retrospectively, I cannot remember one clear case which made me
glad having documented, version-controlled, and automatized stuff; maybe,
materials are a little better organized or my code was a little better to grasp
after not having taken a look at it for a longer time...}

%
This can get super fucking annoying in case of automatized creation of (complex)
figures from numeric results without final editing of the figure by a human.
Jeez! Fuck me sideways!



\paragraph{Publishing stuff}

\todo[inline]{choosing the data host was easy because filtered based on DataLad
support (and Michael knows everything anyway)}

\todo[inline]{mention "open paper"? speech anno paper is on github;
ppa paper is not a public github repository (yet)}

\todo[inline]{My super interesting papers might get cited more often because,
yeah, open access that gets, on average, cited more often [REFERENCE?]}
%
``greater potential impact of a work when it may be cited not just for its
scientific findings but also when its data is reused in other works''
\citep{nichols2017best}.
%
But this is no immediate benefit, and gambling (for the 20\% of PhD's that stay
in science).


\paragraph{Conclusion of my life's story}

% no incentive
``Current incentives do not justify spending large amounts of time preparing
data for sharing, as institutional promotion panels or grant reviewers currently
do not adequately reward such efforts'' \citep{nichols2017best}.

Grateful, that open source neuro-software as well as the data were available,
which allowed me to go far beyond what would have been possible without open
data [kind of not true because I used "in-house data"].
%
Lastly, I feel more confident about my work now compared to the dilettantish
procedures experienced (and followed) in, uhm, another lab in Magdeburg.



\subsection{Conclusions on open science}


\todo[inline]{following could be heavily summarized as a conclusion or phrased
like an (summarizing) personal opinion}

%
Open sciences makes researchers accountable to collect, document, process and
store data and materials according to best practices.
%
Published data and analysis pipelines allow external persons to check the data
and analyses for undiscovered errors, and replicate the results step by step.
%
Varying parameters and running different statistics on the data allows
inspection of robustness of results.

%
``As scientists, we are supposed to be objective arbiters of evidence and
theory, but we are not infallible and must be ready to accept criticism and
revise our claims when errors are discovered'' \citep{nichols2017best}.
%
``However, we need to develop a culture of constructive criticism, which
recognizes that errors are an inevitable part of scientific progress and
protects individual researchers from inappropriately harsh consequences when
honest mistakes are discovered'' \citep{nichols2017best}.

%
``We see no better way to advance understanding on a contested finding than to
have as many researchers as possible puzzling over the data at hand''
\citep{nichols2017best}

%
Benefits are increased robustness and reliability of science when all steps are
openly documented and data are openly available.
%
Multiple datasets can be combined to perform unanticipated use cases, and
extensively and openly documented results of multiple studies facilitate
performing meta-analyses to strengthen the claims of individual studies.
%
Thus, open transparent science is the way to make knowledge and technologies
widely accessible, and increase reproducibility of study results and
replicability of scientific findings while increasing trust of the public into
scientific process and its results.
%
Open science promises increased efficiency (time and financial expense) of
making scientific progress, make advance, and promotes innovation.


\subsection{Call to action: create an incentive or imperative to do it}

\paragraph{Questionable, immediate benefits}
%
``The weight that OS currently has for researchers’ career advancement is rather
small, despite'' \citep{toribio2021early}.
%
At the same time, they might feel that investing additional effort in making
research open (i.e., transparent and reproducible) is unrewarded [Nicholas et
al., 2017], such as conducting replication studies that might not be considered
for publication in high-impact journals'' \citep{toribio2021early}.
%
``As long as scientists are being evaluated on traditional journal metrics,
there are few incentives from a career perspective to fully commit to OS''
\citep{toribio2021early}.


\paragraph{Gambling on long-term benefits}

\todo[inline]{80\% of PhD students leave science anyway}
%
``Thus, while conducting OS may initially require more work in the short term,
it can greatly benefit one’s career in the long term''. \citep{toribio2021early}
%
``OA publications have been found to receive more citations than paywalled
publications [Piwowar et al., 2018] and can therefore aid ECRs’ career
advancement'' \citep{toribio2021early}.
%
``Similarly, open data can be highly beneficial to promote new collaborations
and increase the number of citations and the confidence that the field has in
the findings [Popkin, 2019]'' \citep{toribio2021early}.
%
Currently, playing be the old rules is the ``smarter way'' than gambling getting
cited when data or code get re-used.


\paragraph{Guidelines}
%
``OS still requires the establishment of clear guidelines for transparency and
openness of research at the international level.
%
Examples for guidelines for OA publishing [Nosek et al., 2015; Schiltz, 2018],
and collaborations [Gold et al., 2019] are already existing, and their use
should be promoted by governments and funding agencies, as well as integrated in
the training of ECRs by academic institutions.
%
Organizations and/or regulators in charge of overviewing the open scholarly
system need to be established [cf. Nicholas et al., 2019, 2020]''
\citep{toribio2021early}.


\paragraph{Education}

\todo[inline]{Breed open science enthusiast in (under)graduate curriculi}

%
``It is necessity to promote further training on the benefits and risks of OS
practices [cf. Schönbrodt, 2019].
%
Promoting training would not only increase the knowledge of ECRs about specific
OS practices but also could foster its implementation among ECRs.
%
It would be highly beneficial to introduce these training schemes in the
curriculum of undergraduates programs.
%
Courses should cover the benefits and risks of OS practices, together with a
guideline on how to implement them [cf. Farnham et al., 2017]''
\citep{toribio2021early}.


\paragraph{Carrots...}

\todo[inline]{Start with incentives to foster self-learning and applying the
principles "voluntarily"}

%
More incentives to conduct open science project needs to be established.
%
``Extra efforts may not be valued appropriately by a scientific community who
assesses research based on journal impact metrics and number of publications
[Moher et al., 2018]'' \citep{toribio2021early}.
%
``Individual incentives for researchers should be introduced through, for
example, professional recognition or the allocation of extra funding [Kidwell et
al., 2016; Fecher et al., 2015; Ali-Khan et al., 2018].


\paragraph{...and sticks}

\todo[inline]{later, convince the remaining insurgents by force}
%
``Funding agencies already require publication of findings in OA
schemes and data-sharing plans [Neylon, 2017]'' \citep{toribio2021early}.
%
``Compulsory requirements from funders, which might only lead researchers to
show minimal compliance [Neylon, 2017]'' \citep{toribio2021early}.


\section{Conclusion: Naturalistic stimuli as functional localizer}

Summary of PPA paper:
%#13 natural stimulation
``In summary, natural stimuli like movies \citep{eickhoff2020towards,
hasson2008neurocinematics, sonkusare2019naturalistic} or narratives
\citep{hamilton2018revolution, honey2012not, lerner2011topographic,
silbert2014coupled, wilson2008beyond} can be used as a continuous, complex,
immersive, task-free paradigm that more closely resembles our natural dynamic
environment than traditional experimental paradigms.
% method
We took advantage of three fMRI acquisitions and two stimulus annotations that
are part of the open-data resource
\href{http://www.studyforrest.org}{studyforrest.org} to operationalize the
perception of spatial information embedded in an audio-visual movie and an
auditory narrative, and compare current results to a previous report of a
conventional, block-design localizer.
% results
The present study offers evidence that a model-driven GLM analysis based on
annotations can be applied to a naturalistic paradigm to localize concise
functional areas and networks correlating with specific perceptual processes --
an analysis approach that can be facilitated by the neuroscout.org platform
\citep{delavega2021neuroscout}.
% interpretation
More specifically, our results demonstrate that increased activation in the PPA
during the perception of static pictures generalizes to the perception of
spatial information embedded in a movie and an exclusively auditory stimulus.
% interpretation: aPPA vs. pPPA
Our results provide further evidence that the PPA can be divided into functional
subregions that coactivate during the perception of visual scenes.
% interpretation
Finally, the presented evidence on the in-principle suitability of a naturally
engaging, purely auditory paradigm for localizing the PPA may offer a path to
the development of diagnostic procedures more suitable for individuals with
visual impairments or conditions like nystagmus''
\citep{haeusler2022processing}.



\subsubsection{Caveats of naturalistic stimuli}
%
Challenging data data analysis, but create \& share annotation.
%
Hence, data analysis pipelines should be implemented in common and
well-documented packages and code, and custom code shared along the paper.

%
Just an approximation of real life.
%
Setting is still the scanner.
%
Passive watching \& listening.
%
Executive functions?


\subsubsection{Clinical application}

\todo[inline]{most on this topic is in SRM part}

\todo[inline]{I abandoned the idea to come up with language area (asymmetry);
the topic is clinically more relevant, but problem in case of prediction (esp.
using ROI): most interesting is atypical language lateralization, and there is
usually no lateralization in naturalistic stimuli (but operationalization is
different from localizer paradigms), assumption of (strict) lateralization
probably wrong anyway; templates from papers using \ac{fmri} to localize
language areas are outsourced to separate file}

%
``The ability to non-invasively and automatically delineate cortical areas in
living subjects may have clinical implications, for example by providing
neurosurgeons with detailed, individualized maps of the brains on which they
operate'' \citep{glasser2016multi}.

%
A full feature film might substitute traditional localizer paradigms dedicated
localizer by mapping a variety of brain functions beyond category-specific
visual areas.

%
Therefore, a new independent dataset should be collected that employs
naturalistic stimulation.
%
Additional measures of a variety of localizers would enable comparison of
results from analyzing the naturalistic stimuli and results from localizer
contrasts.

%
For example \ac{fmri} could be used as an noninvasive alternative to map
language areas and potentially assess lateralization (or hemispheric asymmetry)
of functional brain topography related to language (sub)functions, in order to
guide pre- and perioperative assessment of neurosurgery, e.g., in case of
epilepsy.


\section{Conclusion: naturalistic stimuli in general}
%
Naturalistic stimuli are not a panacea but traditional paradigms and
naturalistic paradigms should be used in tandem / reciprocally to generate new
hypotheses and progress our understanding of the brain.

%
In summary, naturalistic stimuli ``impose a meaningful timecourse across
subjects while still allowing for individual variation in brain activity and
behavioral responses, and lend themselves to a broader set of analyses than
either pure rest or pure event-related task designs'' \citep{finn2017can}.
%
``Naturalistic paradigms do not aim to replace the classic, controlled
neuroimaging paradigms (Sonkusare et al., 2019). Due to their complexity and
current limitations in understanding the statistical properties of different
features in naturalistic conditions, naturalistic stimuli are not optimal for
model development [see, e.g., Rust and Movshon, 2005]. Controlled experiments
are still needed for hypothesis testing and developing models, while
naturalistic stimuli are best employed to test models in ecologically valid
settings and to expand them to situations where context matters
more'' \citep{saarimaki2021naturalistic}.





%% References
% \bibliographystyle{unsrtnat}
\bibliographystyle{apacite}
\bibliography{references}


% SRM
%\begin{comment}


\chapter*{Appendix}
%% The appendix contains:\\
%% a) any existing survey materials, e.g. questionnaires\\
%% b) detailed description of formulas and derivations\\
%% Please do not include signed documents
%% (e.g. the vote of the Ethics Committee)}
\addcontentsline{toc}{chapter}{Appendix}
% \setheader{Appendix}
\setcounter{figure}{0}                       % <---------------
\renewcommand\thefigure{A.\arabic{figure}}   % <---------------

% take input from external file
\todo[inline]{Update with plots (and text) showing mean correlations across 1000
models}

\todo[inline]{add representation of time / model (cf. plots in SRM part)}

\todo[inline]{finalize text (cf. text in SRM part}

\begin{figure*}[tbp]
\centering
\includegraphics[width=\linewidth]{figures/corr_vis-regressors-vs-cfs_sub-01_srm-ao-av-vis-shuffled_feat10-iter30_7123-7747.pdf}
    \caption{
    %
    \textbf{Pearson correlation coefficients between regressors of the visual
    localizer and shared features of a Shared Response Model that fitted to
    shuffled time series data.}
    %
    The time series of the shared features within the multi-paradigm \ac{cfs}
    %
    (as calculated for subject 01 in the first fold of the cross-validation)
    %
    were trimmed to match the corresponding \acp{tr} of the visual localizer
    paradigm \citep{sengupta2016extension}.
    %
    The six regressors of the visual localizer model hemodynamic responses to
    the six categories of pictures that were presented in blocks.
    }
\label{fig:corr-vis-reg-srm-shuffled}
\end{figure*}


\begin{figure*}[tbp]
\centering
    \includegraphics[width=\linewidth]{figures/corr_av-regressors-vs-cfs_sub-01_srm-ao-av-vis-shuffled_feat10-iter30_3524-7123.pdf}
    \caption{
    %
    \textbf{Pearson correlation coefficients between regressors of the movie
    and shared features of a Shared Response Model that was fitted to shuffled
    time series data.}
    %
    The time series of the shared features within the multi-paradigm \ac{cfs}
    %
    (as calculated for subject 01 in the first fold of the cross-validation)
    %
    were trimmed to match the corresponding \acp{tr} of the movie paradigm
    \citep{hanke2016simultaneous}
    %
    The regressors \texttt{vse\_new} to \texttt{vno\_cut} are based on
    annotations of the content of movie frames, whereas the regressors
    \texttt{fg\_av\_ger\_lr} to \texttt{fg\_av\_ger\_ud} represent low-level
    visual or auditory confounds
    \citep[cf. Table 3 in][]{haeusler2022processing}.
    %
    \texttt{vse\_new}: change of the camera position to a setting not depicted
    before;
    \texttt{vse\_old}: change of the camera position to a recurring setting;
    %
    \texttt{vlo\_ch}: change of the camera position to another locale within
    the same setting;
    %
    \texttt{vpe\_new}: change of the camera position within a locale not
    depicted before;
    %
    \texttt{vpe\_old}: change of the camera position within a recurring locale;
    %
    \texttt{vno\_cut}: frames within a continuous movie shot;
    %
    \texttt{fg\_av\_ger\_lr}: left-right luminance difference;
    %
    \texttt{fg\_av\_ger\_lrdiff}: left-right volume difference;
    %
    \texttt{fg\_av\_ger\_ml}: mean luminance;
    %
    \texttt{fg\_av\_ger\_pd}: perceptual difference;
    %
    \texttt{fg\_av\_ger\_rms}: root mean square volume;
    %
    \texttt{fg\_av\_ger\_ud}: upper-lower luminance difference.
    }
\label{fig:corr-av-reg-srm-shuffled}
\end{figure*}



\begin{figure*}[tbp]
\centering
    \includegraphics[width=\linewidth]{figures/corr_ao-regressors-vs-cfs_sub-01_srm-ao-av-vis-shuffled_feat10-iter30_0-3524.pdf}
    \caption{
    %
    \textbf{Pearson correlation coefficients between regressors of the
    audio-description and shared features of Shared Response Model that was
    fitted to shuffled time series data.}
    %
    The time series of the shared features within the multi-paradigm \ac{cfs}
    %
    (as calculated for subject 01 in the first fold of the cross-validation)
    %
    were trimmed to match the corresponding \acp{tr} of the
    audio-description paradigm \citep{hanke2014audiomovie}.
    %
    The regressors \texttt{body} to \texttt{sex\_m} are based on
    annotations of nouns spoken by the audio-description's narrator,
    whereas the regressors \texttt{fg\_ad\_ger\_lrdiff} and
    \texttt{fg\_ad\_ger\_rms} represent low-level auditory confounds
    \citep[cf. Table 3 in][]{haeusler2022processing}.
    %
    \texttt{body}: trunk of the body; overlaid clothes;
    %
    \texttt{bpart}: limbs and trousers;
    %
    \texttt{fahead}: (parts) of the face or head;
    %
    \texttt{furn}: moveable furniture (insides \& outsides);
    %
    \texttt{geo}: immobile landmarks;
    %
    \texttt{groom}: rooms \& locales or geometry-defining elements;
    %
    \texttt{object}: moveable and countable entities with firm boundaries;
    %
    \texttt{se\_new}: a setting occurring for the first time;
    %
    \texttt{se\_old}: a recurring setting;
    %
    \texttt{sex\_f}: female name, female person(s);
    %
    \texttt{sex\_m}: male name, male person(s);
    %
    \texttt{fg\_ad\_lrdiff}: left-right volume difference;
    %
    \texttt{fg\_ad\_rms}: root mean square volume.
    %
    \texttt{geo\&groom} is a combination of regressors as used on the positive
    side of the primary contrasts aimed to localize the \ac{ppa}
    \citep[cf. Table 5 in][]{haeusler2022processing}.
    }
\label{fig:corr-ao-reg-srm-shuffled}
\end{figure*}




%\end{comment}


%% Acknowledments
\begin{comment}

\chapter*{Acknowledgments}

% don't show page number
\pagenumbering{gobble}

% take input from external file
% text
I want to thank...
% family
Vadda; Gabriele; Nico, Anna, Antonia, Julius; Annelie, Johannes;
%
Niklas Lampe, Ken Waigel, Stefan Mysliwietz, Ilja Rabinovitch, Gavin Theren,
Madhi; Ruwen Jeffe; Brahim Sokoli
% colleagues
Adina Wagner, Alex and Laura Waite, Tobias Kadela, Julia Mahns, Markus Thoma;
% Mitbewohner
Christian Fröhlich, Kyesam Jung, Binny Davis;
% landlord
Mr. Halking; Valeri Kippes
% job
Anika Völkel;
% eating
Sami of Troja Pizzeria;
% bands
Heaven Shall Burn, Black Sun Empire, Hatebreed, The Ghost Inside, Emmure,
Babymetal, Lorna Shore ("...And I Return To Nothingness EP", wtf?);
%
And my thanks do not go to the media publishing incredibly badly written news
articles online but to all the pet guardians who post videos of their adorable
cats: you made me laugh and let me fall asleep with a good mood after long
days.
% bosses
Simon Eickhoff, Gerhard Jocham, foremost Michael Hanke
%




\end{comment}


%\chapter*{Declaration of authorship}
%% The affidavit (declaration of Authorship) is signed in the
%% "Application for authorization to the doctorate" (Zulassungsantrag).
%% According to the PO, the affidavit is no longer inserted in the dissertation


\end{spacing}
\end{document}
