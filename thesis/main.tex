%% LaTex template for a monographic dissertation
%% at the Faculty of Medicine at Heinrich Heine University Düsseldorf

%% author: Christian Olaf Häusler
%% date: January 2022
%% license:
%% Creative Commons Attribution 4.0 International Public License
%% https://creativecommons.org/licenses/by/4.0/

%% This template was created to meet the requirements as provided by:
%% ENG-PhD_Ordnung_vom_05.03.2018.pdf
%% DE-PhD_Ordnung_vom_05.03.2018.pdf
%% 2020-06-19_GZ-Guidelines_for_preparation_of_Dissertations_-_Monograph.pdf
%% 2021-04-01_GZ-Leitfaden_zur_Erstellung_von_klassischen_Dissertationen.pdf


% font size 11-12pt
\documentclass[english,12pt]{report}
\usepackage[utf8]{inputenc}

% use ä, ü, ...
\usepackage[T1]{fontenc}

% use a common font like Times New Roman or Arial:
% keep LaTex' standard ("Computer Modern")

% margins of the pages;
% left: 30-35mm (due to binding)
% right, top, bottom: 20-25mm
\usepackage[
    a4paper,
    left=30mm,
    right=20mm,
    top=20mm,
    bottom=20mm
]
{geometry}


% manage header, footers, and page numbers
\usepackage{fancyhdr}

% add to dos as comments
\usepackage{todonotes}

% manage spacing between lines
\usepackage{setspace}

% format the chapters' header
\usepackage{titlesec}
\titleformat{\chapter}[hang]{\bf\huge}{\thechapter}{2pc}{}
% other way without showing the number before the chapter title:
%\titleformat{\chapter}[display]{\normalfont\bfseries}{}{0pt}{\Huge}


% citation
% numerical citations:
%\usepackage[numbers]{natbib}
% superscript citations:
%\usepackage[super]{natbib}
% author-year citations:
% \usepackage[round]{natbib}
% APA-style citing and bibiliography
\usepackage[natbibapa]{apacite}

% make the bibliography appear in the table of contents
\usepackage[nottoc,numbib]{tocbibind}

% units: provided as a bundle with the nicefrac package for typing fractions.
% Units uses nicefrac in typesetting physical units in a standard-looking way.
\usepackage{units}

% how to handle links in the PDF
\usepackage[
    colorlinks=true,
    citecolor=black,
    urlcolor=black,
    linkcolor=black
]{hyperref}




\begin{document}
% set pagestyle to 'no header' and 'page number at bottom in the middle'
\pagestyle{plain}
% don't show page number (for now)
\pagenumbering{gobble}
% set spacing between lines to 1.5
\begin{spacing}{1.5}




% create the title page
\begin{titlepage}
\begin{center}

From the Institute Institute of Neuroscience and Medicine,\\
Brain \& Behaviour (INM-7),\\
at Research Centre Jülich, Jülich, Germany

\vfill

\textbf{{\large Evaluating common high-dimensional models of cortical
representational spaces for improving the power of neuroimaging-based
clinical diagnostics}}

maybe add subtitle

\vfill

{\large Dissertation}

\vfill

to obtain the academic title of Doctor of Philosophy (PhD) in Medical Sciences\\
from the Faculty of Medicine at Heinrich Heine University Düsseldorf

\vfill

submitted by\\
\textbf{Christian~Olaf~Häusler}\\
(2022)

\end{center}
\end{titlepage}




%\chapter*{Examiner Details}
\newpage
%pagebreak[4]

\todo[inline]{This page 2 will not be included in the examination copies,
    but will only be inserted in the copies for publication after the review
    process and the oral examination.}

\todo[inline]{You will be notified of the exact text for page 2 once print
    approval is granted.}

\vfill
\noindent As an inaugural dissertation printed by permission of the\\
Faculty of Medicine at Heinrich Heine University Düsseldorf

\vspace*{\fill}
\noindent signed:\\
dean:\\
examiner: Name A\\
co-examiner: Name B



%\chapter*{Dedication}
\newpage
%pagebreak[4]

\begin{center}
\null\vspace{\stretch{1}}
    \textit{pectus amico, cuspis hosti}
\vspace{\stretch{2}}\null
\end{center}




%\chapter*{List of Publications}
\newpage
%pagebreak[4]

\todo[inline]{Own publications are listed here. Submitted manuscripts or
    manuscripts under revision are not listed here. These will be
    listed in the “Application for authorization to the doctorate”
    (Zulassungsantrag)}

\todo[inline]{Please note that you must include a reference for all
    information you take from your own publications. The publications mentioned
    on page 4 must therefore also be listed in the bibliography. Direct
    quotations must be marked with quotation marks}

\vspace*{\fill}

\noindent \textbf{Parts of this work have been published:}\\

\todo[inline]{just testing the bibliography}

% just testing the bibliography
\citep{haeusler2021speechanno},
\citep{halchenko2021datalad},
\citep{haeusler2021ppadata}
\chapter*{Zusammenfassung}
% show page number
\pagenumbering{Roman}
\setcounter{page}{1}

\todo[inline]{The German and English summary should be exactly one page long
each. The summary represents a coherent text. Please do not use subheadings. The
summary should include the following content in a shortened form:\\
the scientific background and current state of research\\
research question and objectives\\
methodology\\
results\\
discussion and conclusions.
}

\todo[inline]{The summary is needed twice:\\
    a) Bound into your dissertation.\\
    b) As a pdf-document, submitted together with the application for
    admission to doctoral examination proceedings.
}

% wissenschaftlicher Hintergrund
lore ipsum
% aktuelle5 Forschungsstand

% Fragestellung und Ziele

% Methodik

% Ergebnisse

% Diskussion

% Schlussfolgerungen




\chapter*{Summary}

% scientific background
lore ipsum
% current state of research

% research question and objectives

% methodology

% results

% discussion

% conclusions

% lore ipsum




\chapter*{Abbreviations}

%
\todo[inline]{The list of abbreviations contains and explains all abbreviations
used in the thesis (except for common linguistic abbreviations as defined in the
dictionary, e.g. Duden, Merriam-Webster). In the case of physical or chemical
quantities, it is also necessary to specify the unit. An alphabetical order is
useful, a subdivision (e.g. SI units, own abbreviations) is possible. For
example, you can format the page in two columns, the abbreviations in bold and
the corresponding explanations in non-bold.}




\tableofcontents

\todo[inline]{The entire table of contents should not be longer than 2 pages.}
%
\todo[inline]{A chapter with only one subchapter is not allowed}




\listoftables

\todo[inline]{does not seem to be mandatory?}




\listoffigures

\todo[inline]{does not seem to be mandatory?}




\chapter{Introduction}
% formatting stuff
\pagenumbering{arabic}
\setcounter{page}{1}

% use an external file
\section{Overview}
%% The introduction begins with an overview of the topic

\todo[inline]{some sentences are kinda quotes from other paper; e.g. Dubois?}

\todo[inline]{other contrasts: phonemes, grammatical tags, prosody, sex}

\todo[inline]{sections are supposed to be numbered}

%
Brain imaging with \ac{bold} \ac{fmri} has been used extensively for almost
three decades to investigate perceptual and cognitive brain functions.
%
Typical analysis procedures average (voxel-wise) data of at least 10-15 subjects
to improve the \ac{snr}.
%
Consequently, these studies do not characterize brain function at the level of
an individual.
%
It is plausible to assume that models of brain functions that are based on a
lowest common denominator approach only capture a fraction of individual
functional brain properties.
%
Therefore, those models are an incomplete foundation for an investigation of
inter-individual differences.
%
However, characterization of individual brain function is by far the most
important application of BOLD fMRI in a clinical context.
%
For example, for pre-surgical screening, or diagnosis of brain function in
health and disease.
%
The goal of the proposed project is to pave the way towards adopting an approach
to the investigation of individual brain function that is based on individual
differences with respect to large normative samples --- a proven strategy that
has been standard in psychological diagnostics and other clinical research for a
long time.\todo{give examples}


\section{Introductory remarks}
%% introductory remarks describe the scientific background of the work as
%% precisely as possible. Cite the most important publications and avoid
%% extensive literature reviews.


\subsection{State of research}


\subsubsection{Functional localization}

\todo[inline]{we don't do speech lateralization anymore; imo the neurosurgery
thing should not be mentioned in the intro anymore; better come up with it in
the general discussion}

%
The most frequently employed paradigm for characterizing individual brains with
BOLD fMRI are so called functional localizers.
%
Functional localizers aim at isolating and localizing brain activity correlated
with specific perceptual processes (e.g. different object categories;
\citet{kanwisher1997ffa}) or cognitive processes (e.g. theory of mind;
\citet{spunt2014validating}).
%
Typically, localizers are dedicated measurements that are used to define
individual \acp{roi}.
%
For example to improve the statistical power of the main experiment's analysis
or to locate brain functions prior to neurosurgery.
% side-effects of neurosurgery
Surgical procedures might impact the post-operative quality of life so much
(e.g. concerning cognitive control or speech production) that it potentially
outweighs the therapeutic benefits.
%
The challenge is to precisely localize relevant brain areas with limited
resources (time, availability and applicability of diagnostic measures for an
individual patient) in order to correctly predict the impact of the planned
procedure.
%
Importantly, functional localizers, despite being tuned for detection power,
quickly become inefficient if one wants to map many different processes in a
limited amount of time.
%
One example of a time-efficient multi-functional localizer for reading, language
comprehension, calculation, motor response, and basic retinotopy was developed
by \citep{pinel2007fast}).\todo{Thirion's work?}
%
It employs a range of dedicated stimuli and specific tasks participants have to
perform in a 5-minute routine.
%
The diagnostic quality of such paradigms relies heavily on participants'
compliance and comprehension of the task instructions, a criterion that can be
difficult to meet in certain target populations, such as patients with dementia.

\todo[inline]{s. \citep{vanderwal2015inscapes}}

\subsubsection{Anatomical alignment (in order to predict)}

The currently dominating approach in neuroimaging group analyses relies on
topological constraints defined by an three-dimensional, anatomical reference
space (e.g. the MNI152 template brain).
%
Surface-based: e.g. \citep{weiner2018defining}.


\subsubsection{Functional alignment (in order to predict)}

\paragraph{Functional alignment in general}
%
An alternative approach to individual localization has been proposed by
\citet{haxby2011common}.
%
They (and e.g. \citet{jiahui2019predicting}) also predicted the location, form,
and size of  target brain areas in ventral temporal cortex from dedicated
localizer scans of other individuals.
%
The key difference of this approach is to rely on similarity of representational
geometry of brain activity patterns and aligning individual brains into a
multi-dimensional a group space.

\todo[inline]{shift from focussing on (connectivity) hyperaligment more to
shared response model}


\paragraph{Hyperalignment}

\todo[inline]{why didn't we used connectivity alignment in the first place?}
%
\citet{haxby2011common} used BOLD response patterns evoked by a 2h action movie
to derive a common representational space.
%
The algorithm, namely hyperalignment, derives this representation using a
variant of Procrustes analysis and computes invertible (orthonormal)
transformations from each individual brain’s voxel-space into this common
reference space.
%
Importantly, the study also showed that an individual's \ac{ffa} or the
\ac{ppa}, can be localized precisely based on data from a reference group.
\todo{explain FFA, PPA}
%
The same authors later showed that this approach can be extended to predict
functional organization across large proportions of the cortical surface, for
example to predict the represented visual field coordinate in visual cortex
based on retinotopic mapping scans of other individuals
\citep{guntupalli2016model}.

\todo[inline]{we do not do connectivity hyperalignment but SRM now!}
%
In his doctoral thesis, recently submitted to the Faculty of Natural Sciences in
Magdeburg, Falko Kaule showed that congruent time-locked BOLD responses across
subjects (i.e. all subjects watching the exact same full-length movie) as used
by Haxby and colleagues are not required to derive a valid alignment of
individuals with a common representational space \citep{kaule2017examination}.
%
Comparable prediction performance can be achieved by using \textbf{functional
connectivity patterns} (correlation of a voxel's time series with reference
regions in the same brain).
%
This finding enables, in principle, the use of different ``calibration'' scans
to determine an alignment with a common representational space, for example with
an age-appropriate stimulus, or a shortened scan time to fit into a particular
clinical schedule.
%
Once a valid alignment is established, known functional properties of a
(normative) reference, derived from extensive scans and analysis of other
subjects, can then be projected into the respective individual voxel space (s.
Fig. 1 in \citep{nishimoto2016lining}).


\paragraph{Shared response model}

\citep{chen2015reduced}




\subsubsection{Naturalistic stimuli in neuroscience}

\todo[inline]{cf. PuG talk}


\paragraph{Intro}

One major goal of cognitive neuroscience is to reveal how the brain processes
information during everyday perception.
%
Traditionally, human brain mapping studies used carefully controlled experiments
and rudimentary stimuli that lack ecological validity.
%
For example, previous studies presented isolated higher-level visual features
such as scenes, faces, human bodies or tools.
%
Results suggest that domain-specic modules like the \ac{ppa}
\citep{epstein1998ppa}, \ac{ffa} \citep{kanwisher1997ffa}), the occipital face
area \ac{ofa} \citep{pitcher2011occipitalfacearea}, the \ac{eba}
\citep{downing2001bodyarea}), and the \ac{loc} \citet{malach1995loc} exist in
the human brain.
%
As a consequence of investigating perceptual and cognitive functions by
utilizing isolated stimuli, studies subdivided the cerebral cortex into
distinctive functional areas whose \ac{bold} activity is specifically correlated
with one particular simplied stimulus type.
%
However, the question remains how those functional areas behave in lifelike
situations and how they might interact.
%
After all, we do not experience the
world around us as separated into small unidimensional stimuli, but perceive ---
through different senses --- a seemingly continuous and unified world.
%
To address
this open question, usage of so called naturalistic stimuli have gained
popularity.


 This


\paragraph{Definition}

\todo[inline]{time-locked events}

\todo[inline]{shorten but add narrative studies; synchronization is also true
for narratives}
%
Naturalistic stimuli are ``a class of stimuli that aim to evoke more
naturalistic patterns of neural responses than traditional controlled artificial
stimuli. Naturalistic paradigms are typically complex and dynamic, and longer
in duration than many conventional stimuli.'' \citep{vanderwal2019movies}.

%
They ``impose a meaningful timecourse across subjects while still allowing for
individual variation in brain activity and behavioral responses, and lend
themselves to a broader set of analyses than either pure rest or pure
event-related task designs.'' \citep{finn2017can}

%
Numerous studies have shown that watching a movie leads to correlated
time-locked brain responses across subjects in many brain regions, and to
synchronized eye movements \citep{hasson2010reliability, lankinen2014isc-meg}.
%
This can be attributed to the way professional movies are shot and edited in
order to intentionally manipulate the viewers' attentional focus and mental
states \citep{brown2012cinematography, dancyger2011film-technique}.

Movies have been used during \ac{fmri} \citep{bartels2004mapping,
hasson2004intersubject}, \ac{eeg} \citep{dmochowski2014audience,
krause2000relative}, simultaneous \ac{eeg}-\ac{frmi}
\citep{whittingstall2010integration}, or \ac{meg} \citep{lankinen2014isc-meg,
luo2010auditory}.
%
The underlying assumption is that movies watched in a laboratory setting offer a
more complex and continuous stimulation that better mimics our natural dynamic
environment.
%
Indeed, studies have shown that freely watching a movie leads to synchronized
spatiotemporal responses across multiple subjects in a large part of the brain
\citep{hasson2010reliability, lankinen2014isc-meg}.\todo{but still different}
%
Professionally produced movies evoke inter-subject correlations (ISC) in more
parts of the brain than, for example, an unedited video of a concert, taken from
a fixed viewpoint \citep{hasson2010reliabilitiy}.
%
This finding could be attributed to a film director's goal to not only direct a
movie, but also to capture and direct the audience's attention
\citep{brown2012cinematography, dancyger2011film-technique}.


\todo[inline]{reviews on narratives: \citep{hamilton2018revolution}, more M/EEG
\citep{alday2019meg}}


\paragraph{Higher validity}
%
Findings suggest that naturalistic stimuli offer a higher ecological validity
\citep{hasson2004intersubject} because they better mimic statistics in our
natural dynamic environment.
%
Further they offer a higher external validity \citep{westfall2016fixing})
because they sample the stimulus space ``better''.\todo{iykwim}
%
''a RSM is only indicated when the stimuli used in the study do not fully
exhaust the theoretical population of stimuli that might have been used \citep{westfall2016fixing}.
%
because selectively sample from the stimulus population leading to a
stimulus-as-fixed-effect fallacy \citep{westfall2016fixing}.(Clarc, The
language-as-fixed-effect fallacy: A critique of language statistics in
psychological research).
%
The conclusions cannot be generalized to a broader population of stimuli without
risking inflated Type I error  (cf. Donnet S, Lavielle
M, Poline JB: Are fMRI event-related response constant in time? A model
selection answer\citep{westfall2016fixing)}.
%
While the functional alignment can also be applied to fMRI data from stimulation
paradigms with simplified stimuli, the transformations for functional alignment
have greatly diminished general validity \citep{haxby2011common}, presumably
because such experiments sample a sparser range of brain states
\citep{guntupalli2016model}.

%
Lastly, increased validity of derived transformation for functional alignment by
sampling a more diverse set of mental states that reflect (confound) statistics
of the natural environment, and enable investigation of the acquired data for a
variety of research questions (e.g. visual or auditory perception, spatial
cognition; emotion; music, speech or social perception)


\paragraph{Better compliance}

\todo{check also \citep{eickhoff2020towards}}
%
The to be diagnosed individuals were scanned while they were watching a movie
without any explicit task or.
%
Improved subject compliance and compatibility due to minimal instruction
requirements (e.g., no fixation of eye gaze) and task demands (no task except
enjoying the movie or audiobook).
%
These minimal instructions make naturalistic paradigms appropriate especially
for elderly or visually impaired persons.
%
Lastly, naturalistic stimuli offer improved data quality, as an interesting and
easy-to-follow stimulus is more capable of putting a participant at ease in the
otherwise claustrophobic, uncomfortable and noisy fMRI scanner.

``Relative to traditional fMRI experiments that typically use highly controlled
stimuli, naturalistic stimuli are more ecologically valid (Zaki and Ochsner,
2009; Hasson and Honey, 2012; Adolphs et al., 2016; Hamilton and Huth, 2018),
convey rich perceptual and semantic information (Bartels and Zeki, 2004; Huth et
al., 2012, 2016) and more fully sample neural representational space (Haxby et
al., 2011, 2014)''\citep{nastase2019measuring}.

``Recent work (Vanderwal et al., 2015) also suggests that naturalistic stimuli
may improve subject compliance (in terms of wakefulness and head motion relative
to, e.g. rest), which is particularly important when scanning patient
populations and children. As mentioned previously, different stimuli will
variably synchronize different brain systems; for example, engaging,
Hollywood-style movies may yield greater, more widespread ISCs than real-life,
unedited videos (Hasson et al., 2010; Cohen et al., 2017)''
\citep{nastase2019measuring}.


\todo[inline]{imo following part can heavily be shortened}

\paragraph{studyforrest}
%
The studyforrest project is an open science project that aims at providing a
versatile resource for investigating human brain function under quasi-natural
conditions.
%
The core of this dataset are two hour long BOLD fMRI scans of participants
watching the movie Forrest Gump (and also listening to a version for the blind
in another scan of equal length).
%
Since its first publication in 2014 \citep{hanke2014audiomovie}, this resources
has led to eight independent studies of international research groups outside
Magdeburg that were published in peer-reviewed journals
(http://studyforrest.org).
%
Based on my knowledge about cinematographic editing techniques, I annotated the
~870 movie cuts with respect to depicted major locations, scene settings,
within-scene rooms and perspectives \citep{haeusler2016cutanno}.
%
The annotation served as prerequisite to investigate cognitive functions, such
as spatial reorientation, perspective taking, and memory retrieval for known
spatial layouts in terms of their occurrence in the movie stimulus.

For 15 participants in the studyforrest datasets two different full-length movie
scans are readily available: one with the normal audio-visual movie
\citep{hanke2016simultaneous}, and a second one with an audio-only movie variant
(originally produced for a visually impaired audience) that is time-locked to
the audio-visual version \citep{hanke2014audiomovie}.

``A case study for successfully sharing naturalistic stimuli is studyforres t.org
(Hanke et al., 2014, 2016), a dataset which includes audio-only and audio-visual
viewings of the movie Forrest Gump during fMRI acquisi- tion. Study Forrest is a
data collection and curation effort designed to serve as a community resource
for new discoveries, in the tradition of distributed science collaborations such
as the International Genetically Engineered Machine competitions. As of October
2019, 29 unique studies had been published using the studyforrest.org dataset,
17 of which were published without any of the original authors of the data
release. This is possible in large part thanks to the richness of naturalistic
stimuli, where the same movie can be used for both task-free as well as
stimulus-driven analyses, with the original stimulus re-annotated for particular
features of interest. For example, studyforrest.org has been used to test
cerebro- vascular biomarkers (Voss et al., 2017) but, among other features, was
also annotated for expressed emotion (Labs et al., 2015) which later informed a
study on emotion encoding gradients in the brain (Lettieri et al., 2019).A case
study for successfully sharing naturalistic stimuli is studyforres t.org (Hanke
et al., 2014, 2016), a dataset which includes audio-only and audio-visual
viewings of the movie Forrest Gump during fMRI acquisi- tion. Study Forrest is a
data collection and curation effort designed to serve as a community resource
for new discoveries, in the tradition of distributed science collaborations such
as the International Genetically Engineered Machine competitions. As of October
2019, 29 unique studies had been published using the studyforrest.org dataset,
17 of which were published without any of the original authors of the data
release. This is possible in large part thanks to the richness of naturalistic
stimuli, where the same movie can be used for both task-free as well as
stimulus-driven analyses, with the original stimulus re-annotated for particular
features of interest. For example, studyforrest.org has been used to test
cerebro- vascular biomarkers (Voss et al., 2017) but, among other features, was
also annotated for expressed emotion (Labs et al., 2015) which later informed a
study on emotion encoding gradients in the brain (Lettieri et al., 2019)''
\citep{dupre2020nature}.


\section{Aims of thesis}
%% At the end of the introduction, add a subchapter on the Aims of Thesis in
%% which you describe the research question and the objectives of your work on
%% a maximum of two pages


\subsection{Overview of aims}
%
Previous work has shown that it is, in principle, possible to combine BOLD fMRI
with a rich naturalistic stimulus for the purpose of localizing areas associated
with particular brain functions \citep{bartels2004mapping}.
%
This can be approached by either modeling specific stimulus features, or by
using the high-dimensional nature of such a stimulus to derive a common
representational space for aligning functional properties of cortex.
%
However, there has been no study aimed at replacing an established localizer
paradigm with a naturalistic stimulus as a short diagnostic routine.
%
A part of a movie of the same length as a dedicated localizer experiment would
have substantial advantages (e.g. task demands compliance, and data quality)
over simplified stimuli presently used in localizer paradigms.
%
Given that naturalistic stimuli offer a rich stimulation correlating with a
variety of different brain functions, they could replace multiple dedicated
localizers and thus offering a more comprehensive and efficient diagnostic in a
similar or even less amount of time.



\subsubsection{Auditory stimulus to localize visual area}
%
This investigation will also provide insight if it is feasible to produce an
audio-only localizer paradigm as an alternative for studies where no visual
stimulation is feasible or desired.
%
Previous results show that there are significant voxel-wise BOLD response time
series correlations between the two datasets in brain areas associated with
speech and story processing (s. Figure 3 in \citep{hanke2016simultaneous}).

%
\todo[inline]{following is actually not true anymore}

For this purpose, I want to evaluate two strategies:

\begin{itemize}
    \item direct modeling of a natural stimulus and
    \item prediction via functional alignment to a reference population
\end{itemize}


\subsubsection{Reproducibility, transparency, openly shared}

\todo{transparency, reproducibility, sharing, check \citep{halchenko2021datalad}}
%
All required algorithms and analyses will be implemented in a way that enables
automated processing.
%
This will facilitate documentation and efficient processing of large number of
datasets.
%
Implementations will be based on open-source software tools to guarantee a
maximum level of reproducibility, and relative ease of long-term maintenance
\citep{eglen2017toward}.
%
The goal is to provide researchers with efficient, validated, ready to use
solutions (stimulation, MR sequence configuration, analysis software) for
functional localization that can be incorporated into their study protocols.
%
We share results and code in standardized file and data formats und use free and
open-source software (FSL, Python packages like ...). Only use publicly
available input data (. \citep{glen2017toward}).


``Naturalistic approaches have a strong potential to further transparent and
openly shared neuroscientific research.  The richness of the data sets collected
inherently favours the analysis of data to address multiple questions or their
reanalysis to address questions other than those exam- ined in the initial
analysis. For example, if one researcher conducts an experiment using stories
and is mainly inter- ested in syntactic processes, the nature of the stimuli
opens up the possibility for other researchers to model phonological processes,
for example, if the data is openly accessible and annotated appropriately. A
stan- dardised way of sharing data from naturalistic exper- imental paradigms is
needed, in order to ensure an easy navigation through the “maze” of openly
shared data and an informed decision regarding which datasets are suitable for
answering specific hypotheses. Ideally, researchers would not only share the
neuroimaging/ electrophysiological data, but also the details of the para- digm
including specific time indications, task descrip- tions and the stimulus files as
presented to the participant. This proposal is different from previous task
ontologies (Poldrack & Gorgolewski, 2014; Turner & Laird, 2012) in that it
captures the specifics of a more eco- logically valid approach and the use of a
natural task, a design which has not yet been incorporated into existing task
ontologies'' \citep{kandylaki2019story}.

%
``A visionary aim of openly sharing data and meta-data of more ecologically
valid designs is the holistic under- standing of human brain function.
Researchers would be able to choose carefully from correctly tagged data- sets
and model brain responses using big data. Then, assisted by current methods in
computer science such as machine learning and artificial neural networks,
researchers could attempt to re-construct brain function in an ecologically
valid manner (see also Hasson et al., 2018)'' \citep{kandylaki2019story}.


\subsubsection{Study 1: Annotation of audio-description}


\todo[inline]{why we need an annotation of speech}

%
Movies are designed to entertain the audience and not to conduct research.
%
Variables (i.e. the ``features'' embedded in the naturalistic stimuli) might be
confounded.
%
Moreover, features of interest might be highly correlated, making a it
impossible to controlling them statistically.
%
Hence, researchers often rely on data-driven methods to analyze the data
``because they do not require an explicit model of the task or stimulus'' and/or
``constructing such a model may be prohibitively difficult.''
\citep{nastase2019measuring}.
%
Which is true but the wrong mindset that lead to a lack of annotation in most
datasets derived from naturalistic paradigms.
%
From data-driven approaches, we gained a lot of knowledge, but data-driven
approaches essentially fall short in case you want to correlate discovered brain
activation patterns with psychological processes.
%
``annotation bottleneck'' \citep{aliko2020naturalistic}i.

``Nevertheless, naturalistic stimuli add additional layers of complexity to the
technical difficulty of data sharing. Some of these – such as recording
acquisition and presentation timings as well as annotating stimuli for features
of interest – are well-recognized from the task-based neuroimaging literature
and implemented in existing standards such as BIDS'' \citep{dupre2020nature}.


\subsubsection{Study 2: task-based vs. visual vs. auditory}
%
It is likely that a rich natural stimulation evokes more widespread
network activity, compared to a simplified stimulus that additionally requires
the subject to perform a task.
%
Hence: compare task-free localizer to dedicated task-based speech localizer
%
compare diagnostic performance of the task-free movie / audio-description fMRI
recordings will be compared to the results of the localizer paradigm.
%
a specific diagnostic contrasts can be selected.

\subsubsection{Study 3: Functional alignment \& stimulus length}
%
All analyses up to this point will have been performed on 2h-long
scans, but any clinical application must aim to minimize the scan time/cost.
%
We test wether reliable functional alignment can be achieved with a task-free,
natural stimulation ``calibration” scan that requires no more acquisition time
than a conventional localizer paradigm.
%
We will estimate the trade-off between diagnostic quality
and required effective scan time by progressively reducing the duration of input
BOLD fMRI data and comparing the results of the reduced model to the reference
computed from the full length scan.

\paragraph{Predict ROI from a reference group}
%
fMRI data acquired from an individual will not be analyzed directly regarding a
specific cognitive function, but will be used to align that individual brain's
voxel space with a common high-dimensional representational reference space.
%
Each axis in this common space can be seen as a kind of cortical tuning
function.
%
The orthonormal transformation of an individual voxel space into this common
reference reflects the particular linear combination of voxel response time
series with respect to each common space component.
%
Using a leave-one-subject-out strategy, individual results of the conventional
localizer will then be projected into the common space, aggregated, and
re-projected into the voxel space of the left-out individual for comparison with
the localizer results for that individual (see Figure 2).
%
Using movie-evoked brain activity, hyperalignment procedure learns subject-wise
optimal transformations of brain activity into a common representational space.
%
Once aligned, the brain activity of the reference group can be used to predict
the activity of another subject.
%
The localization of speech areas will also be performed in the common reference
space directly (the stimulation time axis is preserved in the common space), and
the results will be projected into the voxel space of the left-out individual,
where they can be compared with the localisation result derived from that
subject’s movie scan.
%
Once a valid alignment is established, the inverse transformation is then used
to project functional properties of the common reference into that individual's
voxel space.


\paragraph{minimum length of stimulus; ``calibration scan''}
%
minimum data requirement for reliable functional alignment (within-subject
test-retest???
%
estimate the minimum scan time requirement for deriving a valid functional
alignment by further reducing the amount of input data of the left-out
individual.
\paragraph{just uses an intersecting time-series between stimulus sets}
%
The joint dataset will enable me to study whether a functional alignment to the
common reference space can be performed based on a short scan.





\chapter{Material and Methods}

\todo[inline]{This chapter contains information on patients, examination
materials and methods, depending on the research project. With the help of this
chapter, readers must be able to understand and, if necessary, repeat all
examinations and experiments.}

\todo[inline]{Information on Patients: The statistical information to be
mentioned or described includes number of patients, gender distribution, and age
of patients (mean and median).}

\todo[inline]{Reference Number issued by Ethics Committee: When results from
clinical studies are part of the research project, the reference number assigned
by the Ethics Committee of Heinrich Heine University upon approval must be
stated in the chapter Material and Methods.}

\todo[inline]{Reference Number issued by LANUV or ZETT: For work comprising
results from animal testing, the file reference assigned by LANUV during the
approval procedure must be given in the chapter Materials and Methods, or, in
the case of organ removal, the file reference allocated by ZETT of Heinrich
Heine University. A proof of participation in the laboratory animal science
course must also be given, if applicable.}

\todo[inline]{Methods and Material: This chapter lists: Type and origin of the
material, hardware/equipment with type, manufacturer, location, a detailed
description of the methods used.}

\todo[inline]{Statistical Analysis A short and clear
description of the methods used for material processing, data processing and
statistical analysis.}




\chapter{Results}

\todo[inline]{This chapter contains a clear and concise presentation of your own
results with reference to the research question of the thesis. Results that do
not directly relate to the aim of the thesis should not be included here. Avoid
any interpretation or discussion of the findings in the results chapter. Cite
your own, already published data in the results chapter if you use them in your
work. Further literature citations do not belong in the results chapter.
%
Results of others may not be included in the results chapter. Refer to other
people's results either in the introduction or in the discussion. In order to
use data as a graph, illustration or table, you must obtain written permission
by the author.}




\chapter{Discussion}

% use an external file
\todo[inline]{Phrasing here is still similar to phrasing in general intro}

Human brain mapping studies have traditionally averaged \ac{fmri} data across
participants.
%
However, data need to be assessed on the level of individual persons in order to
advance the field towards a clinical application.
% functional localizer
A promising tool to perform this advancement are/is functional localizers
[because localizers aim to characterize the location size, and shape of
functional areas on the level of individual subject].
% contra localizers
However, traditional localizer paradigms employ selectively sampled, tightly
controlled stimuli, rely heavily on a participant's compliance, and can usually
map just one domain of brain functions
% naturalistic stimuli could replace
Localizer paradigms based on naturalistic stimuli could provide higher
ecological as well as external validity, higher data quality due to increased
compliance, and potentially map a variety of brain functions [ranging from
low-level perception (e.g., luminance) to high-level cognition (e.g., social
cognition)] simultaneously.
% visually impaired
Lastly, an exclusively auditory stimulus like an audiobook or audio drama would
also be appropriate for visually impaired persons[, e.g., suffering from
nystagmus or lack of eyesight].

% PPA as proof of concept
Focussing on the \ac{ppa}, a ``classic'' higher-visual area,
\citep{epstein1998ppa}, the goal of this thesis was to explore whether a movie
and the movie's audio-description could, in principle, substitute a traditional
localizer paradigm.
% paragraph on open science
An additional goal of this dissertation was to perform all studies under the
principles of open, shared, and transparent science.
%
In order to enable independent researchers to validate current results and use
written code to replicate current findings in prospective studies, all created
data, code, analysis steps, and results are published as version-controlled
DataLad \citep[\href{www.datalad.org}{datalad.org};][]{halchenko2021datalad}
datasets.


\section{Recapitulation of work packages}

\todo[inline]{maybe, very short summary of parts here; momentarily it's a draft}

%
First, we extended the studyforrest dataset (study 1).
%
Second and similarly to traditional localizer paradigms, we modeled hemodynamic
activity based on annotated stimulus features embedded in the movie ``Forrest
Gump'' and its audio-description, and created \ac{glm} $t$-contrasts in order to
localize the \ac{ppa} (study 2).
%
Third, we estimated results of the localizer by projecting data through a
\ac{cfs} (study 3).

\todo[inline]{maybe, give an overview of how the remaining part of the thesis is
structured; at the moment: each study as such \& study in light of open science,
general discussion about open science across studies}


\subsection{Speech anno}


\subsubsection{Goal of speech anno}

% what we did in 1 sentence
In study 1 \citep{haeusler2021speechanno}, we created and validated an
annotation of speech occurring in the movie and its audio-description pursuing
two goals.
% aim #1: groundwork for PPA study
The first aim was to build the groundwork that enabled us to conduct study 2.
% aim #2: extend studyforrest
The second aim was to create an exhaustive annotation of speech that
substantially exceeds the groundwork necessary to conduct study 2 in order to
extend the studyforrest dataset as a public resource for independent research.


\subsubsection{Discussion of speech anno}

% validation analysis
We validated the annotation's quality in study 1 and performed a canonical
\ac{glm} analysis by contrasting regressors correlating with speech-related
events to a regressor correlating with events without speech.
% results
As hypothesized, results revealed statistically significant increased
hemodynamic activity in a bilateral cortical network known to be involved in the
perception of speech \citep[e.g.,][]{friederici2011brain, wilson2008beyond}.
% conclusion
These results encouraged us to a) use the annotation as the groundwork for study
2, b) publish the annotation as an extension of the studyforrest project.


\subsubsection{Open science in context of speech anno}

\todo[inline]{how "personal" am I supposed to get?}


\paragraph{Intro}

% additional effort 1
Pursuing the goal of creating a publication-worthy dataset led to additional
work that goes far beyond the work that was necessary to build the \ac{glm} in
study 2.
% additional effort 2
The published annotation provides, among others, time-stamps of phonemes, words
and sentences of all speakers, a grammatical tagging, and an annotation of
syntactic dependencies and semantics.


\paragraph{The self-flagellation retrospectively}
%
Great care was taken during the initial creation and subsequent iterative
corrections in order to provide accurate information to the scientific
community.
%
However, over the course of generating the dataset it became apparent that there
is no such thing as a ``perfect'' annotation:
%
As in human language in general, an annotation of speech will always contain
ambiguities.
%
Additionally, there is a trade-off that needed to be balanced between a) doing
the ``mere minimum'' and putting time and effort in creating additional
information that might not be fruitful, and b) providing a sound/substantial
groundwork for potential use-cases that needed to be anticipated.
%
Any further processing step might be based on a decision that might not match
the requirement of a specific use case.
%
For example, an annotation of semantics might be based on a current state-of-the
art language model that might be superseded by future language models.
%
Therefore, the published annotation does not only comprise the final outcome but
also the raw data and documented code that can automatically be rerun step by
step to reproduce the final outcome of both the annotation and its validation
analysis, all freely accessible in a version-controlled dataset.


\paragraph{Conclusion}
%
In summary, the annotation provides extensive information about the time course
of stimulus features, and therefore a headstart to independent researchers that
wish to ``model hemodynamic brain responses that correlate with a variety of
aspects of spoken language ranging from a speaker's identity, to phonetics,
grammar, syntax, and semantics'' \citep{haeusler2021speechanno} under more
real-life like conditions.
%
Consequently, the outcome of study 1 contributes to the studyforrest project as
a resource for the scientific community by further widening the ``annotation
bottleneck'' \citep{aliko2020naturalistic} of two naturalistic stimuli.


\subsection{PPA paper}

\subsubsection{Recapitulation}

\paragraph{Goal of PPA paper}
% study in one sentence
The goal of study 2 \citep{haeusler2022processing} was to explore whether an
audio-visual and an exclusively auditory naturalistic stimulus could be used in
order to localize the \ac{ppa} as it was previously identified in the same set
of participants by a traditional block-design functional localizer that employed
static pictures \citep{sengupta2016extension}.


\paragraph{Method \& results of PPA paper}
% AV operationalization
For the model-based mass-univariate statistical analysis (i.e.\ac{glm}) of the
movie's data, we operationalized the perception of visual spatial information
based on an annotation of movie cuts and depicted locations
\citep{haeusler2016cutanno}.
% AD operationalization
For the \ac{glm} of the audio-description's data, we extended the annotation of
speech \citep{haeusler2021speechanno} by further annotating nouns that the
narrator uses to describe the movie's absent visual content.

% group results: AV \& AD
On a group-average level, findings demonstrate that increased activation in the
\ac{ppa} generalizes to the perception of spatial information embedded in the
audio-visual movie and the audio-description.
% individual AD
On an individual level, semantic spatial information occurring in the
audio-description is correlated with significant activity in the anterior part
of the \ac{ppa} bilaterally in nine individuals and unilaterally in one
individual.


\paragraph{Discussion \& conclusion of PPA paper}

% conclusion 1
Results add evidence \citep[cf.][]{bartels2004mapping} that a functionally
defined region, such as the \ac{ppa}, can be localized using a model-driven
analysis that is based on a naturalistic stimulus' annotated temporal structure.
% conclusion 2
Further, results suggest that a purely auditory naturalistic stimulus like an
audio-description could potentially substitute a visual localizer as a
diagnostic procedure to assess brain functions in visually impaired individuals
[phrasing pretty similar to \citep{haeusler2022processing}].


\subsubsection{Future studies: PPA}

%
Two aspect revealed by our analyses invite further investigations on the
properties of the \ac{ppa}.
%
First, the responses correlating with an auditory stimulation are spatially
restricted to the anterior \ac{ppa}, and
%
Second, we observed higher intersubject variability of responses of the \ac{ppa}
to a naturalistic auditory stimulation compared to the visual stimulation during
the localizer and movie paradigm.


\paragraph{Just anterior PPA during auditory stimulation}

% our interpretation
Previous studies in the field of visual perception suggest that the \ac{ppa} can
be divided into functionally subregions that might process different stimulus
features.

\todo[inline]{paraphrase stuff from paper here}

%
Hence, we attributed the revealed pattern to different features inherent in the
visual stimuli compared to features inherent in the naturalistic auditory
stimulus [phrasing pretty similar to \citep{haeusler2022processing}].
%
However, our interpretation of the observed pattern ``can only be preliminary,
because the auditory stimulation dataset differs in key acquisition properties
(field-strength, resolution) from the datasets of the movie and visual localizer
representing a confound of undetermined impact'' [still pretty similar to
phrasing in \citep{haeusler2022processing}].
% conclusion
Future studies could employ controlled stimuli, maybe accompanied by a task, to
investigate in detail whether the observed differential activations during
visual and auditory stimulation are replicable.


\paragraph{Interindividual variability in response to auditory stimulation}

% statement
On an individual-level, we observed higher intersubject variability of responses
of the \ac{ppa} to a naturalistic auditory stimulation compared to the
audio-visual movie and visual localizer paradigms \citep[cf. Table 3
in][]{sengupta2016extension}.
% not necessarily noise
However, the divergent pattern from the group mean in four of fourteen
individuals should not necessarily be interpreted as measurement errors,
artefacts, or ``random noise'' \todo{choose just one term} but could also be
attributed to individual differences in responses to the task free, auditory
paradigm.
%
Our naturalistic auditory paradigm differs from block-design localizer paradigms
not just in the exclusively auditory stimulation but also in the accidental,
event-like presentation of spatial information, and the absence of a task which
leaves study participant naive to the investigated cognitive process.


\paragraph{Possibly correlated factors}

\todo[inline]{rephrase in a "readable" (but still preliminary) way}

%
The revealed pattern could correlated with [influenced by] situational factors
like the experimental design (stimulus type, no task), (transient) state of a
participant (e.g., alertness or engagement),  our simply our ``adventurous''
modeling approach[?].
%
\todo{well...}
%
More stable factors might be individual differences in cognitive tendencies or
cognitive abilities like susceptibility / predisposition [?] to attend to, [or]
recognize [or process] auditory spatial information.
% Conclusion
Future studies could employ both controlled and naturalistic stimuli to
investigate whether our results that revealed higher intersubject variability in
response to auditory spatial information are a) replicable across different
experiments and paradigms, and b) reliable within subjects.


\paragraph{Brain \& behavior: intro to "fingerprints"}

\todo[inline]{following is pretty speculative 'cause reliability of differences
is premise for "fingerprints"; hence, just a draft; does it make sense to
discuss it?}

% kanai
In case the pattern is stable within individual subjects / is ``highly
consistent across different sessions [or experiments], then they are
characteristics of the individuals and may reflect differences in their brain
function'' \citep{kanai2011structural} [on structural diffences].
%
``Individual differences in topology (i.e. location, size, shape of functional
areas) and the activity within functional areas can also be considered to be
interesting cases of inter-individual variability to understand the neural basis
of human cognition and behavior, brain-phenotype relationships'', and ``present
useful phenotypes or biomarkers \citep{glasser2016multi,
vanhorn2008individual}''
%
\todo{paraphrase}

\paragraph{Brain \& behavior: example studies}

\todo[inline]{shorten heavily or drop altogether}

%
For example, \citet{kong2019spatial} suggested based on resting-state functional
connectivity measures ``that individual-specific network topography (i.e.,
location and spatial arrangement) might serve as a fingerprint of human behavior
that can predict behavioral phenotypes across cognition, personality, and
emotion'' \citep{kong2019spatial} [with modest accuary, comparable to previous
reports predicting phenotypes based on connectivity strength].

%
\citep{bijsterbosch2018relationship}'s ``results indicate that spatial variation
in the topography of functional regions across individuals is strongly
associated with behaviour'' \citep{bijsterbosch2018relationship}.
%
\citet{bijsterbosch2018relationship} found ``that the spatial arrangement of
functional regions is strongly predictive of non-imaging measures of behavior
and lifestyle'' [however shape \& exact location of brain regions interacted
strongly with  modeling of brain connectivity].
%
\citet{bijsterbosch2018relationship} found ``that individual differences in the
size, shape and exact position of the brain regions [as identified by
resting-state functional connectivity measures] was strongly linked to
individual differences in behavioral tests and questionnaires [including
intelligence, life satisfaction, drug use and aggression problems]''
\citep{bijsterbosch2018relationship}.

%
``The variations in spatial topographical features captured a more direct and
unique representation of subject variability than temporal correlations between
regions defined by group parcellation approaches (coupling).
%
Hence, the cross-subject information represented in commonly adopted
'connectivity fingerprints' could largely reflect spatial variability in the
location of functional regions across individuals, rather than variability in
coupling strength (at least for methods that directly map group-level
parcellations onto individual data)'' \citep{bijsterbosch2018relationship}.

\todo[inline]{drop following paragraph; just here to better understand paragraph
above}
%
``Depending on the employed spatial alignment algorithm and the amount of
removed spatial intersubject variability, the degree to which spatial
information may influence FC estimates possibly varies considerably across
studies.
%
In recent years, significant efforts have gone into the methods that more
accurately estimate the spatial location of functional parcels in individual
subjects [Chong et al., 2017; Glasser et al., 2016; Gordon et al., 2016; Hacker
et al., 2013; Harrison et al., 2015; Varoquaux et al., 2011; Wang et al., 2015],
and into advanced hyperalignment approaches [Chen et al., 2015; Guntupalli et
al., 2016; Guntupalli and Haxby, 2017]'' \citep{bijsterbosch2018relationship}.


\subsubsection{Conclusion on future PPA studies}

\todo[inline]{draft; clean \& rephrase!}
%
In summary, our results invite further studies that investigate the properties
of the parahippocampal area in response to
%
a) in response to an auditory naturalistic stimulation (with or without a
simultaneous task to attend to spatial information, or with subsequent memory
task),
%
b) in response to a controlled paradigm (with or without any task).
%
The respective paradigms could elaborate whether interindividual variability of
responses in the parahippocampal area is related to cognitive processes like
``auditory scene perception'' [is that a valid term?], and to degree auditory
spatial information is utilized for, e.g., spatial orientation, way finding or
[planing, remembering, executing] navigation.


\subsubsection{Future studies: other scene-selective areas}

\todo[inline]{imo, RSC and OPA do not need to be discussed in general
discussion; could be shortly discussed in "open science in general" (->
opportunity costs)}


\subsubsection{Future studies: other functional areas (studyforrest dataset)}

\todo[inline]{imo, a side note; not necessarily needed to be discussed; hence,
draft}

The visual localizer performed by \citet{sengupta2016extension} employed images
from six categories (houses, landscapes, faces, bodies without heads, small
objects, and scrambled images).
%
As a results, the corresponding dataset provides subject-specific \acp{roi}
masks for higher visual areas besides the \ac{ppa}:
%
the fusiform face area (FFA) \citep{kanwisher1997ffa} and the occipital face
area (OFA) \citep{pitcher2011occipitalfacearea},
%
the extrastriate body area (EBA) \citep{downing2001bodyarea},
%
and the lateral occipital complex (LOC) \citep{malach1995loc}.
%
Future studies (e.g., a master's thesis or part of a PhD project) could adjust
our extension of the annotation of speech created in study 2 and the
corresponding analysis pipeline in order to explore hemodynamic responses
correlating with auditory information related to faces, body parts or small
objects.


\subsubsection{Open science in context of PPA paper}

% goal PPA study
Under the perspective of an open science project, the goal of study 2 was to use
three \ac{fmri} datasets \citep{hanke2014audiomovie, hanke2016simultaneous,
sengupta2016extension}, two stimulus annotations \citep{haeusler2021speechanno,
haeusler2016cutanno}, as well as previously published results
\citep{sengupta2016extension} for a new research question.


% skipped work
On the one hand, the availability of the \ac{fmri} data and the subject-specific
\acp{roi} enabled us to shift our focus from acquiring the raw data to
subsequent stages of a research project.
% example of skipped work
For example, the annotations that were created in a "general purpose state"
could be extended immediately to match the needs of study 2, followed by writing
the scripts that preprocessed and statistically analyzed the data.
% additional work
On the other hand, pursuing the goal of an open science project lead to
additional work.
% example of additional work
For example, code needed to be in a state worthy to be published and documented
for other readers, analyses pipelines needed to be in a state to be automatized,
every processing step documented, saved, protocolled, and shared to allow
reproducibility of results and facilitate replicability of finding.

\todo[inline]{shorten results/findings}

% results
Our results have been published in a peer-reviewed, open-access journal
%
``offer evidence that a model-driven GLM analysis based on annotations can be
applied to a naturalistic paradigm to localize concise functional areas and
networks correlating with specific perceptual processes''
\citep{haeusler2022processing},
%
``demonstrate that increased activation in the PPA during the perception of
static pictures generalizes to the perception of spatial information embedded in
a movie and an exclusively auditory stimulus \citep{haeusler2022processing}, and
%
``provide further evidence that the PPA can be divided into functional
subregions that coactivate during the perception of visual scenes''
\citep{haeusler2022processing}

% conclusion
In summary, we re-used existing data as a foundation for a new investigation in
order to generate novel findings encourage further studies, and illustrate the
benefits of publicly and freely available datasets.


\subsection{SRM study}

\todo[inline]{following parts are an old draft}

\subsubsection{Transition from study 2 to study 3}
%
Despite exploratory approach in study 2 (a.k.a. shitty modeling of subjectively
assessed events), results suggest:
%
the response to spatial information must be somewhere in within the response
time series and is detectable.
%
Hence, we might be able to use the response patterns measured during the
presentation of the audio-description in order to generate a \ac{cfs} [needs to
be defined above in overview] in study 3, and align an ``unknown'' test
participant to that \ac{cfs}.

% align left-out subject
Based on our findings in study 2 \citep{haeusler2022processing}, we assumed that
the event structure in both naturalistic stimuli would correlate, among others,
with brain responses that are similar to those correlating with the event
structure in a dedicated functional localizer.

% summary of study 2
Results of study 2 suggest that a naturalistic stimulus might provide an
engaging, task-free paradigm to localize brain functions in individual subjects.


\subsubsection{Goal of SRM study}
% the problem
Considering practical and monetary constraints in a clinical context, a paradigm
lasting 90 to 120 minutes is inappropriate for even an extensive individual
diagnostic procedure.

% goal 1: new procedure
The first goal was to assess a procedure to estimate results of a dedicated
localizer \citep{sengupta2016extension} based on data acquired during
naturalistic stimulation.
%
Following leave-one-subject out cross-validation, we estimated (i.e. predicted)
the results of the visual localizer experiment ($Z$-values of voxels within a
\ac{roi}) of a left-out test participant based on localizer results of a
reference group.

% goal 2: partial alignment
The second goal was to assessed the relationship between length of
naturalistic stimulation used to align the test participant to the fixed
\ac{cfs} and the estimation performance.
%
Lastly, the estimation performance of our new procedure based on \ac{fa} was
compared an estimation performance based on \ac{aa}.


\paragraph{Hypotheses}
%
We hypothesized that increased quantity of data used to calculate the
transformation matrices of the left-out subjects for a \ac{fa} would to increase
prediction performance.
%
Further, we hypothesized that \ac{fa} would eventually perform
``better'' than an estimation based on \ac{aa}.


\subsubsection{Discussion of SRM study}

\todo[inline]{...to be written}


\subsubsection{Future studies on SRM / functional alignment}

\todo[inline]{...to be written}


\subsubsection{Open science in context of SRM study}

\todo[inline]{...to be written}




\section{Open Science}

\todo[inline]{check: anything about open science of each paper missing here?}

\todo[inline]{Following is preliminary phrasing; part has way more captions than
necessary}

\todo[inline]{most importantly:
%
which points make sense to be brought up (and need to be phrased nicely)?
%
which points are missing?
%
which points should be left out?}


\subsection{Intro}
%
Following the practices of open and reproducible science was not mandatory for
submitting the thesis but required additional work and time.
%
``Open and reproducible research means you need to guarantee the accuracy of the
methods used and to explicitly describe and document all stages of the
scientific process to ensure its transparency and traceability''.
%
Standards to follow are not yet fully established and corresponding software
tools are still emerging.
%
Since the best practices are not yet part of a graduate or PhD curriculum,
learning about the principles and standards and applying the corresponding
procedures and necessary tools was based on self-initiative and self-learning.
%
Over the course of the present project, affected stages were
% data-related
a) collection, description and storage of data,
% processing-related
c) processing and analyzing data via code, and
% publishing shit
d) publication of data, materials and results.


\subsection{Data collection; data analysis; publishing}
%
In the context of open data, data are not collected for mere internal purposes
but re-use by third parties needs to be considered.
%
Hence, dataset creators need to anticipate which people might use the data for
which purpose, collect the data according to best practices, convert data into a
standardized format (considering, e.g., naming conventions, folder structure,
separating raw from analyzed data), and create metadata.

% data analysis & automatization
The data, materials and code need to be documented more rigorously, and coverage
of procedures need to exceed the coverage given in method sections of regular
articles.
%
Every change made to data, materials or code, and command line invocations need
to be tracked via a version control system \citep[e.g.,][]{halchenko2021datalad}
to allow other researchers to inspect a study's full history.
%
In order to allow reproducibility from input data, changes made to the data, to
computation and visualization of results, every processing step needs to be
designed and tested to be run reliably and automatically.

% publication: findable, accessible, interoperable, reusable
In the stage of preparing a publication, a researcher needs to facilitate
discovery by humans and web bots (e.g., via extensive description and
machine-readable metadata), ensure long-term curation [e.g., maintenance],
availability [e.g., accessibility], and choose an appropriate data host.
% legal issues
Finally, a researcher need to resolve legal issues that raise during due to the
publication of data, materials and code (e.g. statement of agreement / consent,
anonymization, intellectual property rights, use license).


\subsection{Cons from perspective of creators}



% jeez! it's annoying!
In general, creating open data, materials and code requires a considerable about
of time and effort with little incentives and little, immediate rewards.
%
Publishing data (or merely using open data) is associated with the risk that
other working groups, possibly having more funding and ``people and brain
power'' at disposal, are using the same data for a similar research question at
the same time.
%
Hence, there is the concern that someone else might ``claim priority, usually
through publishing, to a research idea or result you yourself have been working
on'' \citep{laine2017afraid}.
%
On the one hand, being second in the ``competitive race in the sciences'' leads
to diminished opportunities to publish own results because high impact journals
favour novel findings.
%
The risk, and therefore stress, is aggravated in case of researchers that are
early in their scientific career \citep[cf.][]{toribio2021early}, created and
curate the published dataset, pre-registered studies based on open data, or have
to stick to inflexible project plans.
%
On the other hand one advantage of publishing a dataset with an assigned DOI is
that it might get re-used and cited in another study.
%
In case of a dataset creator's fear of being superseded / outrun, ``concerns can
be alleviated by delaying the sharing or using a data-sharing repository with an
embargo period'' \citep{nichols2017best}.


\subsection{Pros from perspective of creators}

% better organized, better documented
An immediate benefit of following the principles of a reproducible study is that
a researcher is forced to work more organized and document every step from a
study's start to finish more rigorously.
%
First, documenting every step and justifying every step by weighting pros and
cons of alternative [mutual exclusive] procedural paths leads to better
understanding, avoids tricking oneself into performing unnecessary statistical
tests, HARKing hypothesizing after the results are known; Kerr, 1998,
HARKing...), and therefore supports following general good scientific practices.
%
Similarly, the extra time and effort spend on inspecting data and testing code
leads to higher confidence in one's own work and the reliability of results.
%
Second, a researcher who records all changes to data and code from the start to
the final results can restore a particular state of data and code and trace,
identify and correct errors more easily [similar: Klein, 2018. A practical guide
for transparency in psychological science].
%
Third, tracking and documenting every step increases reliability of results and
can be seen as a "lab protocol" containing information / templates for writing
scientific articles.
%
Last, automating recurring tasks gives yourself the possibility of reusing
certain data, code, documents, etc. in the future.


\subsection{Pros from perspective of consumers}

\todo[inline]{following needs revision but general train of thought should be
clear}

%
From the perspective of reader, open access journals provides low-cost access to
information.
% readers get a better understanding
A ``transparent and complete reporting of all facets of a study, allowing a
critical reader to evaluate the work and fully understand its strengths and
limitations'' \citep{nichols2017best}.
%
It helps readers to take a look at the methods in more detail it is conveyed in
a often limited method section of a regular study.
%
``This also facilitates subsequent research efforts by other investigators, who
can exactly follow (or carefully manipulate) each aspect of a study''
\citep{nichols2017best}.
%
Open materials facilitate tracking (and understanding!) the process (esp.
analyzes) in detail (pipelines are often far easier to understand by reading the
code step-by-step than just reading the method section).
%
Interesting to students how can not just read a method section but also take a
look at the code and follow step by step every command.
%
Fully transparent studies that also include the input data can serve as an
``education playground'' by enabling (undergraduate) student to trace the
progress of real world project, to learn coding and data analysis.
%
Open data provide groups that enjoy minor funding low-threshold access to
datasets.
%
While open data helps researchers to shift time and resources from data
collection to subsequent stages of a study, share code helps researchers adjust
and extent the analysis pipeline as part of an exploratory data analysis.


\subsection{Cons from perspective of consumers}

% avoid "the bigger, the better" and "garbage in, garbage out"
An mostly not explicitly stated issue in the context of open science is that
standards (quality, formats, parameters) and open sciences practices (e.g.,
documenting) might vary across scientific field, or within scientific fields
depending on a working group's knowledge and rigor.
% check the data
Dataset consumers need to assume that everything that is not explicitly stated
in the description of a dataset has not been considered and done by a dataset's
creator.
% laugh with many, don't trust any
Even if the data a from a renowned source, researchers should consider
themselves to be obliged to test and validate a dataset's quality according to
their standards and specific use cases.


\subsubsection{Tell me about opportunity costs without saying opportunity costs}

% problem of preprocessed data
Moreover, choices made during collection and preprocessing of data, despite
being state-of-the-art at the time of being published, might not be optimal for
every use case or made obsolete by more advanced methods.
% tell me about opportunity costs without saying opportunity costs
Hence, researchers choose the path with the greater net return by weighting
costs and benefits of one path (e.g., preprocessing the same data differently
than the preprocessing performed as part of an open dataset) relative to an
alternative path (e.g., using the preprocessed data and performing new
analyses).


\subsection{Short personal assessment}

\todo[inline]{This whole (sub)section might be too personal; nobody (me
included) cares about the my life's journey}

\todo[inline]{Following are some ideas to connect the dry text above to
experiences during the present thesis (that actually made me come up with the
text above)}

% intro
The additional time and effort feel like a burden and do not contribute to the
immediate benefits of the PhD project, how-fucking-ever:


\subsubsection{I used open-source software packages}

Often forgotten, but open-source packages were prerequisite.


\subsubsection{I used existing data}

\todo[inline]{Why was is good for you that the data were already existing?}

\todo[inline]{imo, irrelevant in my case: participants' consent, anonymization,
most legal issues}

\todo[inline]{Caveat of data re-use: know your data! I could write that I pulled
my hair out while searching for inconsistencies in the timings, and found that
the audio track of the audio-description is essentially unsystematically
shifted; similar cases stay probably undiscovered in non-open datasets; vice
versa, a point for less error-prone open science}

\todo[inline]{Covid-Pandemic has impressively shown that collecting data based
on human subjects" can go dormant for almost 1.5 years; plan of the project was
adjusted accordingly; and I am so lucky that I could fall back on open data}

\todo[inline]{it is/was somewhere between "stupid" and "brave" to mess with
someone like Haxby's group; also, cf. "my precious" of Susanne and Lisa}

\todo[inline]{Opportunity costs: analyzes in \citet{sengupta2016extension} were
performed and \acp{roi} published in voxel-space (i.e., not surface-space)}


\subsubsection{Opportunity costs: missing ROIS of RSC \& OPA}

\todo[inline]{does it make sense to talk about "missing" \acp{roi} of RSC and
OPA at all?}

\todo[inline]{a short version could simply state "it would be fucking awesome to
perform a follow-up study that investigates \ac{rsc}, \ac{opa}; most text is
simply a template to be summarized}

%
``Apart from the PPA, results show significantly increased activity in the
ventral precuneus and posterior cingulate region (referred to as ``retrosplenial
complex'', RSC) of the medial parietal cortex, and in the superior lateral
occipital cortex (referred to as ``occipital place area'', OPA) for both
naturalistic stimuli.
% RSC intro
Like the PPA, the RSC and OPA have repeatedly shown increased hemodynamic
activity in studies investigating visual spatial perception and navigation
\citep{chrastil2018heterogeneity, bettencourt2013role, dilks2013occipital,
epstein2019scene}'' \citep{haeusler2022processing}.

%
In the progress of conducting study 2, ``we assumed that results (at least of
the audio-visual stimulus) could also yield significant clusters in the
retrosplenial complex (RSC) and superior lateral occipital cortex (i.e.
“occipital place area”, OPA) but did not explicitly hypothesize that fact''.
%
``We (and Sengupta) chose the PPA among three possible candidates
because it was the first area to be discovered as a visual ``scene-selective''
region, is the most reliably activated region across studies that investigate
visual scene perception'' [response to reviewer \#2].

%
``Whereas the PPA is assumed to be more involved in landmark recognition by
processing basal perceptual features that constitute a scene, the RSC (i.e.
ventral precuneus and posterior cingulate region) exhibits stronger responses
when the scenes are familiar to the participants suggesting the RSC might be
more concerned with localizing, i.e. orienting, the observer in space [e.g.
Epstein \& Vass, 2016]'' [response to reviewer \#2].
% medial parietal cortex: anterior-posterior gradient
``Similarly to the parahippocampal cortex \citep{aminoff2013role}, the medial
parietal cortex exhibits a posterior-anterior gradient from being more involved
in perceptual processes to being more involved in memory related processes
\citep{chrastil2018heterogeneity, hassabis2009construction, silson2019posterior,
steel2021network}'' \citep{haeusler2022processing}.

%
''We believe that a detailed discussion of the RSC Activation in RSC and OPA was
out of scope of the current study, results are an incentive for further
studies''.
% future studies
``Future, complementary studies using specifically designed paradigms could
investigate where in the posterior-anterior axis of the parahippocampal and
medial parietal cortex auditory semantic information is correlated with
increased hemodynamic activity:
% we hypothesize
we hypothesize that the auditory perception of spatial information (compared to
non-spatial information) is correlating with clusters in the middle of possibly
overlapping clusters correlating with visual perception (peak activity more
posterior) and scene construction from memory (peak activity more anterior)''
\citep{haeusler2022processing}.

%
In the context of open science, this was kind of an limitation / aftereffect of
coverage of mask for functional areas \citet{sengupta2016extension}, and the
non-algorithmic procedure of \citet{sengupta2016extension} based on subjective
decision that is hard to replicate \& to amply to the other functional areas
\citep[cf. algorithmic procedure in, e.g.,][]{julian2012algorithmic}.
%
In summary: everything that is not 100\% automatized is not 100\% reproducible,
sadly because automatization takes a long time for some stuff or is not possible
for some stuff.


\paragraph{Code: documenting, version-controlled, automatize pipeline}

\todo[inline]{Retrospectively, I cannot remember one clear case which made me
glad having documented, version-controlled, and automatized stuff; maybe,
materials are a little better organized or my code was a little better to grasp
after not having taken a look at it for a longer time...}

%
This can get super fucking annoying in case of automatized creation of (complex)
figures from numeric results without final editing of the figure by a human.
Jeez! Fuck me sideways!



\paragraph{Publishing stuff}

\todo[inline]{choosing the data host was easy because filtered based on DataLad
support (and Michael knows everything anyway)}

\todo[inline]{mention "open paper"? speech anno paper is on github;
ppa paper is not a public github repository (yet)}

\todo[inline]{My super interesting papers might get cited more often because,
yeah, open access that gets, on average, cited more often [REFERENCE?]}
%
``greater potential impact of a work when it may be cited not just for its
scientific findings but also when its data is reused in other works''
\citep{nichols2017best}.
%
But this is no immediate benefit, and gambling (for the 20\% of PhD's that stay
in science).


\paragraph{Conclusion of my life's story}

% no incentive
``Current incentives do not justify spending large amounts of time preparing
data for sharing, as institutional promotion panels or grant reviewers currently
do not adequately reward such efforts'' \citep{nichols2017best}.

Grateful, that open source neuro-software as well as the data were available,
which allowed me to go far beyond what would have been possible without open
data [kind of not true because I used "in-house data"].
%
Lastly, I feel more confident about my work now compared to the dilettantish
procedures experienced (and followed) in, uhm, another lab in Magdeburg.



\subsection{Conclusions on open science}


\todo[inline]{following could be heavily summarized as a conclusion or phrased
like an (summarizing) personal opinion}

%
Open sciences makes researchers accountable to collect, document, process and
store data and materials according to best practices.
%
Published data and analysis pipelines allow external persons to check the data
and analyses for undiscovered errors, and replicate the results step by step.
%
Varying parameters and running different statistics on the data allows
inspection of robustness of results.

%
``As scientists, we are supposed to be objective arbiters of evidence and
theory, but we are not infallible and must be ready to accept criticism and
revise our claims when errors are discovered'' \citep{nichols2017best}.
%
``However, we need to develop a culture of constructive criticism, which
recognizes that errors are an inevitable part of scientific progress and
protects individual researchers from inappropriately harsh consequences when
honest mistakes are discovered'' \citep{nichols2017best}.

%
``We see no better way to advance understanding on a contested finding than to
have as many researchers as possible puzzling over the data at hand''
\citep{nichols2017best}

%
Benefits are increased robustness and reliability of science when all steps are
openly documented and data are openly available.
%
Multiple datasets can be combined to perform unanticipated use cases, and
extensively and openly documented results of multiple studies facilitate
performing meta-analyses to strengthen the claims of individual studies.
%
Thus, open transparent science is the way to make knowledge and technologies
widely accessible, and increase reproducibility of study results and
replicability of scientific findings while increasing trust of the public into
scientific process and its results.
%
Open science promises increased efficiency (time and financial expense) of
making scientific progress, make advance, and promotes innovation.


\subsection{Call to action: create an incentive or imperative to do it}

\paragraph{Questionable, immediate benefits}
%
``The weight that OS currently has for researchers’ career advancement is rather
small, despite'' \citep{toribio2021early}.
%
At the same time, they might feel that investing additional effort in making
research open (i.e., transparent and reproducible) is unrewarded [Nicholas et
al., 2017], such as conducting replication studies that might not be considered
for publication in high-impact journals'' \citep{toribio2021early}.
%
``As long as scientists are being evaluated on traditional journal metrics,
there are few incentives from a career perspective to fully commit to OS''
\citep{toribio2021early}.


\paragraph{Gambling on long-term benefits}

\todo[inline]{80\% of PhD students leave science anyway}
%
``Thus, while conducting OS may initially require more work in the short term,
it can greatly benefit one’s career in the long term''. \citep{toribio2021early}
%
``OA publications have been found to receive more citations than paywalled
publications [Piwowar et al., 2018] and can therefore aid ECRs’ career
advancement'' \citep{toribio2021early}.
%
``Similarly, open data can be highly beneficial to promote new collaborations
and increase the number of citations and the confidence that the field has in
the findings [Popkin, 2019]'' \citep{toribio2021early}.
%
Currently, playing be the old rules is the ``smarter way'' than gambling getting
cited when data or code get re-used.


\paragraph{Guidelines}
%
``OS still requires the establishment of clear guidelines for transparency and
openness of research at the international level.
%
Examples for guidelines for OA publishing [Nosek et al., 2015; Schiltz, 2018],
and collaborations [Gold et al., 2019] are already existing, and their use
should be promoted by governments and funding agencies, as well as integrated in
the training of ECRs by academic institutions.
%
Organizations and/or regulators in charge of overviewing the open scholarly
system need to be established [cf. Nicholas et al., 2019, 2020]''
\citep{toribio2021early}.


\paragraph{Education}

\todo[inline]{Breed open science enthusiast in (under)graduate curriculi}

%
``It is necessity to promote further training on the benefits and risks of OS
practices [cf. Schönbrodt, 2019].
%
Promoting training would not only increase the knowledge of ECRs about specific
OS practices but also could foster its implementation among ECRs.
%
It would be highly beneficial to introduce these training schemes in the
curriculum of undergraduates programs.
%
Courses should cover the benefits and risks of OS practices, together with a
guideline on how to implement them [cf. Farnham et al., 2017]''
\citep{toribio2021early}.


\paragraph{Carrots...}

\todo[inline]{Start with incentives to foster self-learning and applying the
principles "voluntarily"}

%
More incentives to conduct open science project needs to be established.
%
``Extra efforts may not be valued appropriately by a scientific community who
assesses research based on journal impact metrics and number of publications
[Moher et al., 2018]'' \citep{toribio2021early}.
%
``Individual incentives for researchers should be introduced through, for
example, professional recognition or the allocation of extra funding [Kidwell et
al., 2016; Fecher et al., 2015; Ali-Khan et al., 2018].


\paragraph{...and sticks}

\todo[inline]{later, convince the remaining insurgents by force}
%
``Funding agencies already require publication of findings in OA
schemes and data-sharing plans [Neylon, 2017]'' \citep{toribio2021early}.
%
``Compulsory requirements from funders, which might only lead researchers to
show minimal compliance [Neylon, 2017]'' \citep{toribio2021early}.


\section{Conclusion: Naturalistic stimuli as functional localizer}

Summary of PPA paper:
%#13 natural stimulation
``In summary, natural stimuli like movies \citep{eickhoff2020towards,
hasson2008neurocinematics, sonkusare2019naturalistic} or narratives
\citep{hamilton2018revolution, honey2012not, lerner2011topographic,
silbert2014coupled, wilson2008beyond} can be used as a continuous, complex,
immersive, task-free paradigm that more closely resembles our natural dynamic
environment than traditional experimental paradigms.
% method
We took advantage of three fMRI acquisitions and two stimulus annotations that
are part of the open-data resource
\href{http://www.studyforrest.org}{studyforrest.org} to operationalize the
perception of spatial information embedded in an audio-visual movie and an
auditory narrative, and compare current results to a previous report of a
conventional, block-design localizer.
% results
The present study offers evidence that a model-driven GLM analysis based on
annotations can be applied to a naturalistic paradigm to localize concise
functional areas and networks correlating with specific perceptual processes --
an analysis approach that can be facilitated by the neuroscout.org platform
\citep{delavega2021neuroscout}.
% interpretation
More specifically, our results demonstrate that increased activation in the PPA
during the perception of static pictures generalizes to the perception of
spatial information embedded in a movie and an exclusively auditory stimulus.
% interpretation: aPPA vs. pPPA
Our results provide further evidence that the PPA can be divided into functional
subregions that coactivate during the perception of visual scenes.
% interpretation
Finally, the presented evidence on the in-principle suitability of a naturally
engaging, purely auditory paradigm for localizing the PPA may offer a path to
the development of diagnostic procedures more suitable for individuals with
visual impairments or conditions like nystagmus''
\citep{haeusler2022processing}.



\subsubsection{Caveats of naturalistic stimuli}
%
Challenging data data analysis, but create \& share annotation.
%
Hence, data analysis pipelines should be implemented in common and
well-documented packages and code, and custom code shared along the paper.

%
Just an approximation of real life.
%
Setting is still the scanner.
%
Passive watching \& listening.
%
Executive functions?


\subsubsection{Clinical application}

\todo[inline]{most on this topic is in SRM part}

\todo[inline]{I abandoned the idea to come up with language area (asymmetry);
the topic is clinically more relevant, but problem in case of prediction (esp.
using ROI): most interesting is atypical language lateralization, and there is
usually no lateralization in naturalistic stimuli (but operationalization is
different from localizer paradigms), assumption of (strict) lateralization
probably wrong anyway; templates from papers using \ac{fmri} to localize
language areas are outsourced to separate file}

%
``The ability to non-invasively and automatically delineate cortical areas in
living subjects may have clinical implications, for example by providing
neurosurgeons with detailed, individualized maps of the brains on which they
operate'' \citep{glasser2016multi}.

%
A full feature film might substitute traditional localizer paradigms dedicated
localizer by mapping a variety of brain functions beyond category-specific
visual areas.

%
Therefore, a new independent dataset should be collected that employs
naturalistic stimulation.
%
Additional measures of a variety of localizers would enable comparison of
results from analyzing the naturalistic stimuli and results from localizer
contrasts.

%
For example \ac{fmri} could be used as an noninvasive alternative to map
language areas and potentially assess lateralization (or hemispheric asymmetry)
of functional brain topography related to language (sub)functions, in order to
guide pre- and perioperative assessment of neurosurgery, e.g., in case of
epilepsy.


\section{Conclusion: naturalistic stimuli in general}
%
Naturalistic stimuli are not a panacea but traditional paradigms and
naturalistic paradigms should be used in tandem / reciprocally to generate new
hypotheses and progress our understanding of the brain.

%
In summary, naturalistic stimuli ``impose a meaningful timecourse across
subjects while still allowing for individual variation in brain activity and
behavioral responses, and lend themselves to a broader set of analyses than
either pure rest or pure event-related task designs'' \citep{finn2017can}.
%
``Naturalistic paradigms do not aim to replace the classic, controlled
neuroimaging paradigms (Sonkusare et al., 2019). Due to their complexity and
current limitations in understanding the statistical properties of different
features in naturalistic conditions, naturalistic stimuli are not optimal for
model development [see, e.g., Rust and Movshon, 2005]. Controlled experiments
are still needed for hypothesis testing and developing models, while
naturalistic stimuli are best employed to test models in ecologically valid
settings and to expand them to situations where context matters
more'' \citep{saarimaki2021naturalistic}.




%% References
% \bibliographystyle{unsrtnat}
\bibliographystyle{apacite}
\bibliography{references}

\todo[inline]{if APA-style is used, show all (i.e. more than 7) authors}



% Appendix
%\appendix

\chapter{Appendix}

\todo[inline]{The appendix contains:\\
a) any existing survey materials, e.g. questionnaires\\
b) detailed description of formulas and derivations\\
Please do not include signed documents (e.g. the vote of the Ethics Committee)}

\todo[inline]{Update with plots (and text) showing mean correlations across 1000
models}

\todo[inline]{add representation of time / model (cf. plots in SRM part)}

\todo[inline]{finalize text (cf. text in SRM part}

\begin{figure*}[tbp]
\centering
\includegraphics[width=\linewidth]{figures/corr_vis-regressors-vs-cfs_sub-01_srm-ao-av-vis-shuffled_feat10-iter30_7123-7747.pdf}
    \caption{
    %
    \textbf{Pearson correlation coefficients between regressors of the visual
    localizer and shared features of a Shared Response Model that fitted to
    shuffled time series data.}
    %
    The time series of the shared features within the multi-paradigm \ac{cfs}
    %
    (as calculated for subject 01 in the first fold of the cross-validation)
    %
    were trimmed to match the corresponding \acp{tr} of the visual localizer
    paradigm \citep{sengupta2016extension}.
    %
    The six regressors of the visual localizer model hemodynamic responses to
    the six categories of pictures that were presented in blocks.
    }
\label{fig:corr-vis-reg-srm-shuffled}
\end{figure*}


\begin{figure*}[tbp]
\centering
    \includegraphics[width=\linewidth]{figures/corr_av-regressors-vs-cfs_sub-01_srm-ao-av-vis-shuffled_feat10-iter30_3524-7123.pdf}
    \caption{
    %
    \textbf{Pearson correlation coefficients between regressors of the movie
    and shared features of a Shared Response Model that was fitted to shuffled
    time series data.}
    %
    The time series of the shared features within the multi-paradigm \ac{cfs}
    %
    (as calculated for subject 01 in the first fold of the cross-validation)
    %
    were trimmed to match the corresponding \acp{tr} of the movie paradigm
    \citep{hanke2016simultaneous}
    %
    The regressors \texttt{vse\_new} to \texttt{vno\_cut} are based on
    annotations of the content of movie frames, whereas the regressors
    \texttt{fg\_av\_ger\_lr} to \texttt{fg\_av\_ger\_ud} represent low-level
    visual or auditory confounds
    \citep[cf. Table 3 in][]{haeusler2022processing}.
    %
    \texttt{vse\_new}: change of the camera position to a setting not depicted
    before;
    \texttt{vse\_old}: change of the camera position to a recurring setting;
    %
    \texttt{vlo\_ch}: change of the camera position to another locale within
    the same setting;
    %
    \texttt{vpe\_new}: change of the camera position within a locale not
    depicted before;
    %
    \texttt{vpe\_old}: change of the camera position within a recurring locale;
    %
    \texttt{vno\_cut}: frames within a continuous movie shot;
    %
    \texttt{fg\_av\_ger\_lr}: left-right luminance difference;
    %
    \texttt{fg\_av\_ger\_lrdiff}: left-right volume difference;
    %
    \texttt{fg\_av\_ger\_ml}: mean luminance;
    %
    \texttt{fg\_av\_ger\_pd}: perceptual difference;
    %
    \texttt{fg\_av\_ger\_rms}: root mean square volume;
    %
    \texttt{fg\_av\_ger\_ud}: upper-lower luminance difference.
    }
\label{fig:corr-av-reg-srm-shuffled}
\end{figure*}



\begin{figure*}[tbp]
\centering
    \includegraphics[width=\linewidth]{figures/corr_ao-regressors-vs-cfs_sub-01_srm-ao-av-vis-shuffled_feat10-iter30_0-3524.pdf}
    \caption{
    %
    \textbf{Pearson correlation coefficients between regressors of the
    audio-description and shared features of Shared Response Model that was
    fitted to shuffled time series data.}
    %
    The time series of the shared features within the multi-paradigm \ac{cfs}
    %
    (as calculated for subject 01 in the first fold of the cross-validation)
    %
    were trimmed to match the corresponding \acp{tr} of the
    audio-description paradigm \citep{hanke2014audiomovie}.
    %
    The regressors \texttt{body} to \texttt{sex\_m} are based on
    annotations of nouns spoken by the audio-description's narrator,
    whereas the regressors \texttt{fg\_ad\_ger\_lrdiff} and
    \texttt{fg\_ad\_ger\_rms} represent low-level auditory confounds
    \citep[cf. Table 3 in][]{haeusler2022processing}.
    %
    \texttt{body}: trunk of the body; overlaid clothes;
    %
    \texttt{bpart}: limbs and trousers;
    %
    \texttt{fahead}: (parts) of the face or head;
    %
    \texttt{furn}: moveable furniture (insides \& outsides);
    %
    \texttt{geo}: immobile landmarks;
    %
    \texttt{groom}: rooms \& locales or geometry-defining elements;
    %
    \texttt{object}: moveable and countable entities with firm boundaries;
    %
    \texttt{se\_new}: a setting occurring for the first time;
    %
    \texttt{se\_old}: a recurring setting;
    %
    \texttt{sex\_f}: female name, female person(s);
    %
    \texttt{sex\_m}: male name, male person(s);
    %
    \texttt{fg\_ad\_lrdiff}: left-right volume difference;
    %
    \texttt{fg\_ad\_rms}: root mean square volume.
    %
    \texttt{geo\&groom} is a combination of regressors as used on the positive
    side of the primary contrasts aimed to localize the \ac{ppa}
    \citep[cf. Table 5 in][]{haeusler2022processing}.
    }
\label{fig:corr-ao-reg-srm-shuffled}
\end{figure*}






\chapter*{Acknowledgments}
% don't show page number
\pagenumbering{gobble}
% text
I want to thank...
% family
Vadda; Gabriele; Nico, Anna, Antonia, Julius; Annelie, Johannes;
%
Niklas Lampe, Ken Waigel, Brahim Sokoli, Stefan Mysliwietz, Ilja Rabinovitch,
Ruwen Jeffe;
% colleagues
Adina Wagner, Alex and Laura Waite, Julia Mahns, Markus Thoma;
% Mitbewohner
Christian Fröhlich, Kyesam Jung, Binny Davis;
% landlord
Mr. Halking;
% job
Anika Völkel;
% eating
Trojan Pizzeria;
% bands
Heaven Shall Burn, Black Sun Empire, Hatebreed, The Ghost Inside, Emmure,
Babymetal, Lorna Shore ("...And I Return To Nothingness EP"!);
% bosses
Simon Eickhoff, Gerhard Jocham, foremost Michael Hanke
%


\chapter*{Declaration of authorship}
%
\todo[inline]{The affidavit (declaration of Authorship) is signed in the
``Application for authorization to the doctorate'' (Zulassungsantrag). According
to the PO, the affidavit is no longer inserted in the dissertation itself.}

\end{spacing}
\end{document}
