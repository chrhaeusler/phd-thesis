%% The introduction begins with an overview of the topic
\todo[inline]{\textit{new terms} in italics}


% introductory quote
``A remarkable feature of the vertebrate brain is the anatomical specialization
of cortical regions for the processing of different types of information. Since
the late 19th century, it has been recognized that restricted lesions of the
human brain result in location-specific sensory, motor or cognitive deficits''
\citep{cohen1994localization}.
% human brain mapping
\textit{Human brain mapping} \citep[e.g.,][]{raichle2009brief} explores the
brain's topographic organization \citep[e.g.,][]{eickhoff2018topographic} by
specifying ``in as much detail as possible the localisation of function in the
human brain'' \citep{savoy2001history}.
% fMRI + BOLD
Brain imaging studies have extensively used \ac{fmri} since the early 1990s to
measure \ac{bold} activity in-vitro.
% higher visual areas
For example, replicated findings in the domain of higher-visual perception
suggest that category-selective brain regions like the parahippocampal place
area
\citep{epstein1998ppa, epstein1999parahippocampal} or the fusiform face area
\citep{kanwisher1997ffa, kanwisher2006fusiform} exhibit significantly increased
\ac{bold} activity correlated with a ``preferred'' stimulus class:
%
the \ac{ppa} responds more strongly when participants are watching pictures of
scenes and landscapes compared to pictures of, e.g., faces;
%
vice versa, the \ac{ffa} responds more strongly when participants are watching
pictures of faces compared to pictures of scenes.

% \citep{talairach1967atlas}
The precise anatomical location of functionally defined regions (measured in
Talairach or MNI coordinates) varies anatomically \citep{friston2006critique,
saxe2006divide}.
%
However, human brain mapping studies usually average data across study
participants for
%
practical (e.g., limited scan time per subject),
%
statistical reasons (e.g., improved of \ac{snr}),
%
or to generalize from study particpants to a broader population.
%
Consequently, studies that average data across study participants may just
``capture the common denominator of each individual cognitive circuit and lose a
large amount of information'' \citep{pinel2007fast}.


\section{Functional localization}
%% Further introductory remarks describe the scientific background of the work
%% as precisely as possible. Cite the most important publications and avoid
%% extensive literature reviews.
% a.k.a. "state of research"

%
In order to assess health and disease and therefore advance human brain mapping
towards a clinical application fMRI data need to be interpreted on the level of
an individual person \citep{dubois2016building, wegrzyn2018thought}.
%
\textit{Functional localizer} experiments \citep[e.g.,][for
reviews]{saxe2006divide, friston2006critique} are \ac{fmri} experiments that
specifically aim to characterize the location size, and shape of functional
areas correlating with perceptual or cognitive processes like the perception of
object categories \citep{kanwisher1997ffa} or speech
\citep{fernandez2001language}, or theory of mind \citep{spunt2014validating}.


\subsection{Pros \& cons}
% ``The ROI approach is advantageous for four reasons:
% 1) it allows hypothesis driven comparisons of signals within independently
% defined ROIs across many different conditions,
% 2) it increases statistical sensitivity in multisubject analyses [Nieto–
% Castañón and Fedorenko 2012],
% 3) it reduces the number of multiple comparisons present in whole-brain
% analyses [Saxe et al.  2006], and
% 4) it identifies ROIs in each participant's native brain space''
% \citep{rosenke2021probabilistic}.


% purpose: ROI improve the statistical power of the main experiment's analysis
Currently, functional localizers are often used as separate experiment to
identify a subject-specific functional \acp{roi} to ``guide, constrain or
interpret results from a main experiment'' \citep{saxe2006divide}.
% clinical application
Functional localizers might also be used as a diagnostic procedure, e.g., prior
to neurosurgery \citep[cf.][]{wegrzyn2018thought}.\todo{kind of a blind quote;
check Wegrzyn's part \& his refs in general discussion}
% cons: design for detection power
However designed to maximize detection power, localizers employ carefully
chosen, tightly-controlled, and simplified stimuli presented in a block-wise
manner often accompanied with a task to keep study participants attentive.
% one localizer = one domain of functions
Consequently, one localizer paradigm can usually map just one domain of brain
functions.
% batteries: intro
In order to map many different processes despite limited resources (time,
availability and applicability of diagnostic measures for an individual
patient), researchers have developed more time-efficient, multi-functional
\textit{localizer batteries} \citep[e.g.,][]{barch2013function,
drobyshevsky2006rapid, pinho2018individual, pinho2020individual, pinel2007fast}.
% batteries: example
For example, \citet{pinel2007fast} employs a range of dedicated stimuli and
specific tasks in a 5-minute routine to map processes of ``auditory and visual
perception, motor actions, reading, language comprehension, and mental
calculation'' \citep{pinel2007fast}.
% but still...
Nevertheless, the currently employed paradigms still struggle to overcome two
challenges.
% validity?
From a theoretical point of view, localizers rely on selectively sampled,
tightly-controlled stimuli presented in blocks, and do not resemble how we
perceive the real world outside of the laboratory during everyday life.
% compliance?
From a practical point of view, localizers depend on the comprehension of the
task instructions and the participants' compliance during the scan session,
criteria that can be difficult to meet in clinical or pediatric populations
\citep{eickhoff2020towards, vanderwal2015inscapes, vanderwal2019movies}.


\section{Naturalistic stimuli}
% intro to "real-life" neuroscience
Since a major goal of neuroscience is not to reveal how the brain
responds to blocks of [tightly-controlled] stimuli presented in a laboratory
setting but how the brain processes information during everyday perception,
\textit{naturalistic stimuli} gained popularity in neuroimaging.

\subsection{Definition}
% definition quote
Naturalistic stimuli are ``a class of stimuli that aim to evoke more
naturalistic patterns of neural responses than traditional controlled artificial
stimuli. Naturalistic paradigms are typically complex and dynamic, and longer in
duration than many conventional stimuli.'' \citep{vanderwal2019movies}.
% movies & narratives
The currently most popular naturalistic stimuli in neuroscience are movies and
auditory narratives \citep[s.][for reviews]{jaaskelainen2021movies,
jaaskelainen2020neural} that, despite providing a time-locked event structure,
``sample a broad range of brain states and engage multiple perceptual and
cognitive systems in parallel'' \citep{haxby2020naturalistic}.


\subsection{Pro (+ early findings)}
%
Naturalistic stimuli have several advantages over traditional paradigms.


\subsubsection{Validity}
% Carefully chosen stimulus sets ``selectively sample from the stimulus
% population leading to a stimulus-as-fixed-effect fallacy [Clarc, The
% language-as-fixed-effect fallacy: A critique of language statistics in
% psychological research]'' \citep{westfall2016fixing}. ``The conclusions cannot
% be generalized to a broader population of stimuli without risking inflated
% Type I error  [cf. Donnet S, Lavielle M, Poline JB: Are fMRI event-related
% response constant in time? A model selection answer]''
% \citep{westfall2016fixing}.

% external validity
From a theoretical point of view, naturalistic stimuli promise a higher external
validity because the \textit{stimulus features} (i.e. the stimulus classes or
the variables) that are embedded in a naturalistic stimulus represent a more
random sample from the ``theoretical population of stimuli that might have been
used'' \citep{westfall2016fixing}.
% ecologically validity
Further, naturalistic stimuli also promise a higher ecological validity
\citep{zaki2009need, hasson2012future, hamilton2018revolution} because they more
closely mimic our rich visual and auditory experiences in real-life outside of
the scanner bore \citep{hasson2008neurocinematics, haxby2020naturalistic}.

\subsubsection{Early findings}
% reviews
Consequently, audio-visual movies and spoken narratives have been used during
\ac{fmri} \citep[s.][for reviews]{hamilton2018revolution,
hasson2008neurocinematics, sonkusare2019naturalistic,
saarimaki2021naturalistic}, and \ac{eeg} or \ac{meg} data acquisition
\citep[s.][for reviews]{alday2019meg, kandylaki2019story}.
% early findings: spatiotemporal synchronization
Early studies have shown that watching a movie \citep{hasson2004intersubject,
hasson2008neurocinematics, hasson2010reliability} or listening to a narrative
\citep{lerner2011topographic, wilson2008beyond} reliably synchronizes
spatiotemporal responses across multiple subjects in a large part of the brain
compared to, for example, an unedited video of a concert taken from a fixed
viewpoint \citep{hasson2004intersubject, hasson2008neurocinematics,
hasson2010reliability, lerner2011topographic, wilson2008beyond}.
% Bartels (2004)
Importantly in context of the current thesis, a pioneering study
\citep{bartels2004mapping} revealed that functional specialization of cortical
areas is preserved during complex, life-like stimulation.
% conclusion: naturalistic stimuli as localizer
Results suggest that naturalistic stimuli could, in principle, be used as a more
life-like paradigm in order to substitute traditional localizer paradigms.


\subsubsection{Compliance}
% a film director's goal is not just to direct a movie, but also to capture and
% direct the audience's attention; professional movies are shot and edited in
% order to intentionally manipulate the viewers' attentional focus and mental
% states \citep{brown2012cinematography, dancyger2011film-technique}.

% engaging & essentially no task
From a practical perspective, naturalistic stimuli require both minimal task
instructions given by the staff and minimal task demands on the study
participant (no task except enjoying the movie or audio story).
%
Moreover, movies and narratives provide an interesting and engaging stimulation
putting participants at ease in the claustrophobic, uncomfortable, and noisy MRI
scanner.
%
Therefore, naturalistic stimuli promise an increased data quality due to
decreased fatigue and head movement, especially in case of children
\citep{vanderwal2015inscapes}, and possibly psychiatric
\citep{eickhoff2020towards} or elderly persons.

\todo[inline]{use of audiobook will be first mentioned in "aims of thesis"}


\subsection{Contra}
% intro
Nevertheless, naturalistic stimuli also have disadvantages over traditional
paradigms.
% produced for commercial use
First, the majority of naturalistic stimuli that have been used in neuroimaging
have been created by professional production companies for commercial purposes
and not in order to conduct research.
% temporal structure not explicitly known
Thus, the temporal structure of stimulus features embedded in a naturalistic
stimulus is fixed, and therefore reproducible, but initially not explicitly
known.
% challenging analysis
Consequently, modeling brain activity correlating with stimulus features
embedded in a stimulus' time course is challenging
\citep{saarimaki2021naturalistic, simony2020analysis} because such models, like
a traditional \ac{glm}, rely on the stimulus features being annotated.
% lack of annotations = bottleneck
The lack of extensive annotations has led to a ``usage bottleneck''
\citep{aliko2020naturalistic} and might be the main reason why explicit models
of task or stimulus are ``notoriously'' \citep{richard2019fast}, if not
``prohibitively'' \citep{nastase2019measuring} difficult to construct.
% 120 minutes are too long
Second, presenting a full feature film usually lasting 90 to 120 minutes is
inappropriate as an individual diagnostic procedure considering practical and
monetary constraints in a clinical context.


\section{Aims of thesis}
%% At the end of the introduction, add a subchapter on the Aims of Thesis in
%% which you describe the research question and the objectives of your work on
%% a maximum of two pages

\todo[inline]{write the following in past, presence or future tense?}


\subsection{Overview}

\todo[inline]{little interim summary here at the beginning}

\todo[inline]{estimation from reference group is not mentioned in theory part
above but is mentioned here for the first time}

% clinical application
Functional localizer paradigms that aim to characterize the location size, and
shape of functional areas on the level of individual subjects are a promising
tool to advance brain mapping towards a clinical application.
% contra localizers
However, traditional localizer paradigms employ selectively sampled,
tightly-controlled stimuli, rely heavily on a participant's compliance, and can
usually map just one domain of brain functions
% naturalistic stimuli could replace
A functional localizer based on a naturalistic stimulus would have a couple of
advantages over a traditional localizer paradigm, and could, in principle,
replace multiple dedicated localizers:
% fixed time-course
a) naturalistic stimuli offer a fixed time-course correlating with a variety of
different brain functions ranging from low-level perception (e.g., luminance) to
high-level cognition (e.g., social cognition);
% richness
b) the richness and multi-dimensionality of naturalistic stimuli might impose a
challenging data analysis put promises a higher ecological and external
validity;
% compliance
c) naturalistic stimuli are an easy-to-follow and immersive paradigm leading to
improved data quality.
% visually impaired
Lastly, an exclusively auditory stimulus like an audiobook or audio drama would
also be appropriate for visually impaired persons, e.g., suffering from
nystagmus or lack of eyesight.
% therefore
Therefore, this dissertation explores --- while following the principles of
open, transparent, and reproducible science --- whether a movie and the movie's
audio-description that was produced for an visually impaired audience could, in
principle, substitute a traditional localizer paradigm.

\subsubsection{focus on PPA}
% focus: ppa
As a proof of concept, this dissertation focuses on the \ac{ppa} that exhibits
increased hemodynamic activity when participants view photos of landscapes,
buildings or landmarks, compared to, e.g., photos of faces or tools
\citep[e.g.,][for reviews]{epstein2014neural, aminoff2013role}.
% auditory semantics unclear
However, reports regarding the correlates of processing spatial information in
verbal stimulation are less clear \citep{aziz2008modulation}.
%
Hence, we will evaluate the potential of both an audio-visual movie and an
auditory narrative to substitute a traditional localizer paradigm in two ways.

\subsubsection{modeling hemodynamic responses}
%
Similarly to traditional localizer paradigms, we first model hemodynamic
activity based on annotated stimulus features embedded in two naturalistic
stimuli (each lasting $\approx$\unit[120]{m}), and create GLM $t$-contrasts in
order to locate the \ac{ppa}.


\subsubsection{Estimating from reference group}

\paragraph{Anatomical alignment}
% intro
In order to address the practical and monetary constrains of a clinical context,
we also estimate a participant's localizer results from other participants'
localizer results.
% previous studies: anatomical alignment
Previous studies have shown that the location of a participant's \ac{ppa} can be
estimated using \ac{aa} \citep{frost2012measuring, rosenke2021probabilistic,
weiner2018defining, zhen2017quantifying}:
%
First, participants functional data are aligned to a \ac{cas}, i.e. individual
functional \acp{roi} are projected from each participant's brain space into an
average brain (e.g., MNI or Talairach brain template).
%
Then, the average of each participants' \ac{roi} is projected from the \ac{cas}
into the brain anatomy of an individual participant, serving as an prediction of
that individual participant's \ac{roi}.
%
However, even cortical surface-based alignment \citep{fischl2012freesurfer}
cannot eliminate the mismatch between brain function and anatomy of
category-selective regions \citep{duncan2009consistency, frost2012measuring,
weiner2018defining, weiner2014mid} [still similar phrasing to
\citep{feilong2018reliable}].


\paragraph{Functional alignment}
% intro to functional alignment
In order to overcome limitations of \ac{aa}, we will employ a new
procedure using \ac{fa} that has been shown to outperform anatomical alignment
\citep{haxby2020hyperalignment, bazeille2021empirical}.
%
\todo{rephrase}
%
Following a leave-one-subject out cross-validation, we first create a \ac{cfs}
that is based on the other participants' response time series of the movie and
audio-description.
%
Then, the localizer results (statistical $Z$-maps) of participants whose
response time series served as training data to create the \ac{cfs} are
projected from their native space into the \ac{cfs}.
%
Further, response time series of the left-out subject (i.e. the test subject) are
aligned to the \ac{cfs} in order to acquire a transformation matrix.
%
Finally, the inverse of a test subject's transformation matrix is used to
project the localizer results from the \ac{cfs} into the test subject's native
brain space.


\paragraph{Partial alignment}
% problem
However, using the complete time series data of a two-hour paradigm in order to
acquire the test subject's transformation matrix is inefficient plus unsuitable
for a clinical population.
%
Therefore, we explore a) whether just a part of a movie or auditory narrative
could serve as a ``diagnostic'' run in order estimate a test
subject's localizer results and b)  which amount of data used for functional
alignment is necessary to outperform an estimation procedure based on anatomical
alignment.


\subsection{Open, transparent, and reproducible science}

% reproducibility crisis
Over the last decade, there has been a growing awareness that results of
scientific publications are not reproducible or general scientific findings are
not replicable letting some authors speak of a ``reproducibility crisis'' or
``replication crisis'' in the sciences \citep{baker2016reproducibility,
plesser2018reproducibility, stupple2019reproducibility, nosek2022replicability}.
% reproducibility: definition
``A study is reproducible if all of the code and data used to generate the
numbers and figures in the paper are available and exactly produce the published
results'' \citep{leek2017most}.
% replicability: definition
A study is replicable if the same analysis of an equivalent experiment's data
leads to consistent results \citep{dubois2016building, leek2017most}.
%
Hence, on a metalevel, this dissertation aims to meet both the requirements of
open, accessible, shared, and transparent science \citep{watson2015will,
fecher2014open} as well as the requirements of a reproducible and replicable
research project:
%
the dissertation follows guidelines and best practices for a) coding and
scientific computing \citep{wilson2014best}, b) procedures and data analyses
\citep{nichols2017best, poldrack2017scanning, poldrack2019establishment}, and c)
sharing code, created data, and results \citep{eglen2017toward, nichols2017best,
pernet2015improving}.


% \paragraph{input data}
% open data \citep{eglen2017toward}
First, my work is built on top of publicly and freely available \ac{fmri} data
that are part of the \textit{studyforrest} project
(\href{www.studyforrest.org}{\url{studyforrest.org}}).
%
The studyforrest project is an open science project that aims to provide a
versatile resource for investigating human brain function under quasi-natural
conditions.
% core stimuli
The core of this dataset are two-hour long \ac{bold} \ac{fmri} scans of
participants watching the movie Forrest Gump \citep{ForrestGumpMovie} and
listening to the movie's audio-description that was created for a visually
impaired audience by adding a narrator to the movie's audio track.
% used by 3rd parties
Since its first publication in 2014 \citep{hanke2014audiomovie}, the
studyforrest project has served as a resource of raw (and preprocessed) data for
international working groups to conduct and publish independent, peer-reviewed
research (s.
\href{www.studyforrest.org/publications.html}{\url{studyforrest.org/publications.html}}).
% annotations
The additional stimulus annotations that have been created over the course of
the dissertation are version-controlled and published in a standardized file
format \citep{haeusler2021speechanno}, and therefore contribute to the
studyforrest project as a resource for the scientific community.

% \paragraph{code, analyses, output} intro
Further, all code is shared to improve reproducibility of current results and to
facilitate replicability of findings on other datasets.
% automatization
Therefore, data analyses pipelines are designed in a way that enables automated
processing.
% FOSS
Analyses pipelines are not implemented in proprietary software but in freely
available and, if possible, open-source software.
% which tools to choose why?
Among potential software packages, we chose the tools that offer the most solid
documentation, and a broad basis of developers and maintainers to ensure
long-term support.
% my code
Custom code written by myself is written in open-source programming languages
(Python and Bash), is version-controlled, documented, and released publicly and
freely accessible.
% DataLad
All input data, custom code, analysis steps and output data are accessible in
standardized \textit{DataLad} (\href{www.datalad.org}{datalad.org}) datasets.
Since DataLad provides a free and open-source software solution that manages
provenience, distribution, and version-control of code and data
\citep{halchenko2021datalad}, all executed steps from downloading the input data
to visualizing the results can be rerun to check and validate the dissertation's
results.

% \paragraph{publications}
% open-access publishing
Last, because ``nature abhors a paywall'' \citep{dupre2020nature}, publications
describing generated data, reasoning of methodological choices, analysis steps,
and results are published in open-access journals.
% neurovault
Unthresholded statistical maps of all computed statistical $t$-contrasts are
additionally published at Neurovault
(\href{https://neurovault.org/}{neurovault.org}).


\subsection{Specific objectives and hypotheses}

\todo[inline]{in case studies shall have their own titles, short text here}

\todo[inline]{parts per individual study might be too long?}

\subsubsection{A studyforrest extension, an annotation of spoken
language in the German dubbed movie ``Forrest Gump'' and its audio-description
(study 1)}

% intro
``Naturalistic stimuli have a fixed but initially unknown temporal structure of
stimulus features of interest, as well as an equally unknown confound
structure'' \citep{haeusler2021speechanno}.
% what we did in 1 sentence
Therefore in study 1 \citep{haeusler2021speechanno}, we created and validated an
annotation of speech occurring in the movie and its audio-description pursuing
two goals:
% aim #1: groundwork for PPA study
The first aim was to built the groundwork that enabled us to create the \ac{glm}
contrasts based on annotated stimulus features in order to localize the \ac{ppa}
in study 2.
% aim #2: extend studyforrest
The second aim was to create an exhaustive annotation of speech that
substantially exceeds the groundwork necessary to conduct study 2 in order to
extend the studyforrest dataset as a public resource for independent research by
complementing formerly published annotations of portrayed emotions
\citep{labs2015portrayed}, perceived emotions \citep{lettieri2019emotionotopy},
as well as cuts and locations depicted in the movie \citep{haeusler2016cutanno}.

% validation analysis
In order to validate the annotation's quality, we performed a canonical \ac{glm}
analysis of modeled hemodynamic activity in study 1 based on information drawn
from the annotation.
% contrast
Regressors correlating with speech-related events were contrasted to a regressor
correlating with events without speech.
% hypothesis
We hypothesized that results would reveal significant clusters in brain regions
that have been shown to be involved in speech processing.

% results
Results revealed statistically significant increased hemodynamic activity in a
bilateral cortical network including temporal, parietal and frontal regions
replicating ``results of studies that employed tightly-controlled stimuli
\citep[s.][for reviews]{friederici2011brain,
hickok2007cortical,price2012twentyyears}, and studies that employed data-driven
methods to analyze \ac{fmri} data from auditory naturalistic stimuli
\citep{honey2012not, lerner2011topographic, silbert2014coupled}.

% conclusion
Results of the validation encouraged us to use the annotation as a groundwork to
be adapted to our specific needs in study 2.


\subsubsection{Processing of visual and non-visual naturalistic spatial
information in the ``parahippocampal place area'' (study 2)}

% study in one sentence
In study 2 \citep{haeusler2022processing}, we investigated whether it is
possible to localize the \ac{ppa} by creating \ac{glm} contrasts
based on annotations of both the two-our lasting movie as well as the movie's
audio-description.

% AV annotation
``For the analysis of the movie stimulus, we took advantage of a previously
published annotation movie cuts and the depicted location after each cut
\citep{haeusler2016cutanno}'' \citep{haeusler2022processing}.
% AD annotation
For the analysis of the audio-description, we extended the annotation of speech
that was created in study 1 \citep{haeusler2021speechanno}:
%
nouns that the narrator uses to describe the movie's absent visual content were
semantically categorized in case the narrator used them to describe locations,
buildings, rooms, faces, bodies, and so on.
% hypo a: group
On a group-average level, we hypothesized that a conventional model-based
mass-univariate statistical analysis would reveal increased hemodynamic activity
in medial temporal regions that were functionally identified as the \ac{ppa} in
the same set of participants by means of a traditional block-design functional
localizer \citep{sengupta2016extension}.
% hypo b: individuals
Moreover, we hypothesized that an exclusively auditory stimulus could, in
principle, localize the \ac{ppa} as an example of a ``visual area'' in
individual persons \citep{haeusler2022processing}.

% group results
On a group-average level, findings demonstrate that increased activation in the
\ac{ppa} during the perception of static pictures generalizes to the perception
of spatial information embedded in an audio-visual movie and exclusively
auditory naturalistic stimulus
\citep{haeusler2022processing}.
% individual AD
``On an individual level, we find significant bilateral activity correlating
with semantic spatial information occurring in the audio-description in the
anterior \ac{ppa} of nine individuals and unilateral activity in one
individual'' \citep{haeusler2022processing}.

% conclusion 1
Results add evidence \citep[cf.][]{bartels2004mapping} that a functionally
defined region, such as the \ac{ppa}, can be localized using a model-driven
\ac{glm} analysis that is based on a naturalistic stimulus' annotated temporal
structure
% conclusion 2
Results also suggest that a naturally engaging, purely auditory paradigm like an
audio-description could, in principle, substitute a visual localizer as a
diagnostic procedure to assess brain functions in visually impaired individuals
\citep{haeusler2022processing}.

\todo[inline]{Following phrasing is bad}

\todo[inline]{but: does it makes sense at all?}

% intro
Lastly, results suggest that we might be able to use the response patterns
measured during the presentation of the audio-description in order to generate a
\ac{cfs} [needs to be defined above in overview] in study 3, and
align an ``unknown'' test participant to that \ac{cfs}.


\subsubsection{Using varying amount of data from naturalistic stimulation for
functional alignment to predict results from a task-based functional localizer
paradigm (study 3)}

\todo[inline]{Following part is preliminary}

\todo[inline]{we agreed on study 3 not having at its own abstract}

\todo[inline]{I might write a little more here (or add an abstract to study 3)}

\todo[inline]{part here is (still) similar to the part written in "overview"
(1.3.1)}


\paragraph{Problem}
% summary of study 2
Results of study 2 suggest that a naturalistic stimulus might provide an
engaging, task-free paradigm to localize brain functions in individual subjects.
%
However considering practical and monetary constraints in a clinical context, a
paradigm lasting 90 to 120 minutes is inappropriate for even an extensive
individual diagnostic procedure.


\paragraph{Solution}
%
Therefore, the first goal of study 3 was to assess a procedure to estimate
results of a dedicated localizer \citep{sengupta2016extension} based on data
acquired during naturalistic stimulation.
%
Following leave-one-subject out cross-validation, we estimated (i.e. predicted)
the results of the visual localizer experiment ($Z$-values of voxels within a
\ac{roi}) of a left-out test participant based on localizer results of a
reference group.\todo{phrasing!}
% partial alignment
The second goal of study 3 was to assessed the relationship between length of
naturalistic stimulation used to align the test participant to the fixed
\ac{cfs} and the estimation performance.
%
Lastly, the estimation performance of our new procedure based on \ac{fa} was
compared an estimation performance based on \ac{aa}.


\paragraph{Hypotheses}
%
We hypothesized that increased quantity of data used to calculate the
transformation matrices of the left-out subjects for a \ac{fa} would to increase
prediction performance.
%
Further, we hypothesized that \ac{fa} would eventually perform
``better'' than an estimation based on \ac{aa}.


\paragraph{Results}


\paragraph{Discussion}
