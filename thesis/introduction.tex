this text is located in introduction.tex

\todo[inline]{erste Zeile eines Absatzes kann eingerückt werden}

\todo[inline]{legenden von Tabellen und Abbildungen sind kleinere Schriftgröße
(10pt-11pt)}

\todo[inline]{Abbildungen: im Text, nicht im Anhang; im Text erwähnt (Abb.1) und
erläutert; mit Titel ("Fig. x: \textbf{Titel.} weitere Informationen") und
Bildunterschrift; Bildunterschrift bzw. Legende: alle verwendeten Maßeinheite
und Abkürzungen aufgeführt und erläutert}

\todo[inline]{Tabellen: Überschrift ("Table 1: \textbf{Überschrift.}
Information, die zu deren Verständnis erforderlich ist."); Legende ist wie bei
Abbildungen unterhalb der Tabelle}

\todo[inline]{Wiedergabe fremder Abbildungen und Tabellen unterliegt
Urheberrecht = entsprechenden Verlag um Erlaubnis bitten; Dies gilt auch für
Abbildungen aus Publikationen, bei denen Sie selbst Autor sind, da die
Vervielfältigungsrechte in der Regel an den Verlag übertragen werden}

\todo[inline]{References: noch nicht akzeptierte / in Revision befindliche
Manuskripte werden in der Arbeit nicht zitiert, sondern im Zulassungsantrag
angegeben}

\todo[inline]{Harvard-Style: d.h. mit Namen in Klammern; Bibliographie
alphabetisch sortiert}

\section{Overview}

\todo[inline]{The introduction begins with an overview of the topic}


\section{Introductory remarks}

\todo[inline]{introductory remarks, which describe the scientific background of
the work as precisely as possible. Cite the most important publications and
avoid extensive literature reviews.}


\section{Aims of thesis}

\todo[inline]{At the end, add a subchapter on the Aims of Thesis in which you
describe the research question and the objectives of your work on a maximum of
two pages.}

