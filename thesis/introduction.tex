%% The introduction begins with an overview of the topic

\todo[inline]{we don't do speech lateralization anymore; imo the neurosurgery
thing should not be mentioned in the intro anymore; better come up with it in
the general discussion}

%
Brain imaging using with \ac{fmri} of \ac{bold} activity has been used
extensively for almost three decades to investigate perceptual and cognitive
brain functions.
%
Human brain mapping (or \textit{topographic brain mapping}) maps brain functions,
perceptual or cognitive processes, onto the anatomy of the brain.
%
Typical analysis procedures average (voxel-wise) data of at least 10-15
subjects to improve the \ac{snr}.
%
Consequently, studies employing an averaging approach do not characterize brain
function at the level of an individual because these models that do not capture
individual brain properties but just a ``common denominator''
\citep{dubois2016building}. \todo{check paper}
%
However, characterization of individual brain function is by far the most
important application of BOLD fMRI in a clinical context.
%
For example, for a diagnosis of brain functions in health and disease, or
pre-surgical screening.
%
This dissertation explores weather naturalistic stimuli, here a Hollywood movie
and and its audio-only variant created for a visually impaired audience, could,
in principle, substitute a traditional, task-based paradigm to identify the
location, size, and shape of a functional area in the brain of individual
subjects.


\section{Functional localization}
%% Further introductory remarks describe the scientific background of the work
%% as precisely as possible. Cite the most important publications and avoid
%% extensive literature reviews.
% a.k.a. "state of research"


\todo[inline]{maybe define ecological \& external validity here already;
not in section about naturalistic stimuli later}

\todo[inline]{get a better definition of functional areas \& localization}

% definition
Studies subdivided the cerebral cortex into distinctive functional areas whose
statistically increased \ac{bold} activity is correlated with one particular
stimulus type.
% group level
On a group average level, previous studies presented isolated higher-level
visual categories such as pictures of scenes, faces, human bodies or tools.
%results from group average studies
Results suggest that category-specific brain regions like the \ac{ppa}
\citep{epstein1998ppa}, \ac{ffa} \citep{kanwisher1997ffa}), the occipital face
area \ac{ofa} \citep{pitcher2011occipitalfacearea}, the \ac{eba}
\citep{downing2001bodyarea}), and the \ac{loc} \citet{malach1995loc} exist in
the human brain.
% individual level: localizers
On the level of individual subjects, the most frequently employed paradigm to
characterize location, size, and shape of function areas with BOLD fMRI are
[grammar?] \textit{functional localizers}.
% localizers: definition
Functional localizers are dedicated measurement that aim at isolating and
localizing brain activity correlated with specific perceptual processes (e.g.
different object categories; \citet{kanwisher1997ffa}) or cognitive processes
(e.g. theory of mind; \citet{spunt2014validating}).
% purpose: ROI
Functional localizers can be used to define an subject-specific \acp{roi} to
improve the statistical power of the main experiment's analysis, or to locate
brain functions prior to neurosurgery.
% purpose: neurosurgery
Surgical procedures might impact the post-operative quality of life so much
(e.g. concerning cognitive control or speech production) that it potentially
outweighs the therapeutic benefits.
% efficiency
The challenge is to precisely localize relevant brain areas with limited
resources (time, availability and applicability of diagnostic measures for an
individual patient).
% one localizer = one domain of functions
Importantly, one dedicated localizer paradigm can only map one domain of brain
functions (e.g. retinotopic mapping, higher-visual perception
\citet{kanwisher1997ffa}, or speech perception
\citet{fernandez2001language}).\todo{other localizer papers?}
% fucking inefficient
Hence, functional localizers, despite [or because] being tuned for detection
power by employing carefully chosen and tightly-controlled, simplified stimuli,
quickly become inefficient if one wants to map many different processes in a
limited amount of time.
% localizer batteries: intro
Researchers have tried to tackle this issue by creating time-efficient
multi-functional \textit{localizer batteries} \citep{pinel2007fast,
pinho2018individual, pinho2020individual}. \todo{others? H.B.Project?}
% localizer batteries: example
For example, \citet{pinel2007fast} employs a range of dedicated stimuli and
specific tasks participants have to perform in a 5-minute routine.\todo{which
domains?}
% task based = shit
Nevertheless, the diagnostic quality of localizer batteries relies heavily on
participants' comprehension of the task instruction and their compliance, a
criterion that can be difficult to meet in clinical or pediatric populations
\citep{eickhoff2020towards, vanderwal2015inscapes, vanderwal2019movies}.
% validity?
Additionally, localizer batteries also rely on carefully chosen and
tightly-controlled, simplified stimuli that are usually presented in blocks do
not resemble how we perceive the real-world during every-day life leading to
questionable external and ecologically validity.\todo{add references}

% interim summary
In summary, we have two issues: a) validity and compliance, and b) efficiency.
% foreshadowing to next sections & transition to naturalistic stimuli
This dissertation will explore how we can both increase validity by using
\textit{naturalistic stimulus} paradigms as well as increase efficiency by
predicting individual functional topography from data collected in a reference
group.


\section{Naturalistic stimuli}

Ultimately, a major goal of cognitive neuroscience is to reveal how the brain
processes information during everyday perception.
%
Given the questionable/debatable ecological validity of traditional functional
localizer paradigms, it is often unclear how functional areas behave in
life-like situations and how they might interact.
%
After all, we do not experience the world around us as separated into small
unidimensional stimuli, but perceive --- through different senses --- a
seemingly continuous and unified world.
%
To address this open question, \textit{naturalistic stimuli} gained popularity
in neuroscience to investigate brain activity under more life-like conditions.


\paragraph{Definition}

% definition quote
Naturalistic stimuli are ``a class of stimuli that aim to evoke more
naturalistic patterns of neural responses than traditional controlled artificial
stimuli. Naturalistic paradigms are typically complex and dynamic, and longer in
duration than many conventional stimuli.'' \citep{vanderwal2019movies}.
%
They ``[Movies] sample a broad range of brain states and engage multiple
perceptual and cognitive systems in parallel'' \citep{haxby2020naturalistic}.
% movies & narratives
The most popular naturalistic stimuli in neuroscience are movies and audio-only
stories that provide time-locked event structure during a continuous,
complex/rich, dynamic, and often prolonged stimulation.


\paragraph{Ecological validity}
% definition
<Definition here>.
% claim; is KINDA QUOTE OF Haxby 2020?
Naturalistic stimuli promise a higher ecological validity [Zaki and Ochsner,
2009; Hasson and Honey, 2012; Adolphs et al., 2016; Hamilton and Huth, 2018]
because they, despite being presented in a laboratory setting, still more
closely mimic the richness real-life visual and auditory experiences
\citep{hasson2004intersubject, haxby2020naturalistic} \todo{check Hasson}
%
and ``convey rich perceptual and semantic information [Bartels and Zeki, 2004;
Huth et al., 2012, 2016]'' \citep{nastase2019measuring}.
%
``We argue here that studies of brain activity evoked by viewing a naturalistic,
dynamic movie better reflect the statistics of natural viewing in a complex,
cluttered, changing, and continuous visual environment''
\citep{haxby2020naturalistic}.

\paragraph{External validity}
% definition
<Definition here>.
% less selection bias
Carefully chosen stimulus sets ``selectively sample from the stimulus population
leading to a stimulus-as-fixed-effect fallacy [Clarc, The
language-as-fixed-effect fallacy: A critique of language statistics in
psychological research]'' \citep{westfall2016fixing}.
%
``The conclusions cannot be generalized to a broader population of stimuli
without risking inflated Type I error  [cf. Donnet S, Lavielle M, Poline JB: Are
fMRI event-related response constant in time? A model selection
answer]'' \citep{westfall2016fixing}.
%
Naturalistic stimuli promise a higher ecological validity because they draw a
more representative sample from the ``theoretical population of stimuli that
might have been used'' \citep{westfall2016fixing}.


\paragraph{Early findings}
% reviews
Movies and narratives have been used during \ac{fmri}
(s.\citet{hamilton2018revolution, hasson2008neurocinematics,
jaaskelainen2021movies, sonkusare2019naturalistic, saarimaki2021naturalistic}
for reviews), and \ac{eeg} or \ac{meg} data acquisition (s. \citet{alday2019meg,
kandylaki2019story} for reviews).
%
Early studies have shown that watching a movie or listening to an auditory
narrative synchronized spatiotemporal responses across multiple subjects in a
large part of the brain compared to, for example, an unedited video of a concert
taken from a fixed viewpoint \citep{hasson2004intersubject,
hasson2008neurocinematics, lerner2011topographic, wilson2008beyond}.
% This finding could be attributed to a film director's goal to not only direct
% a movie, but also to capture and direct the audience's attention; this can be
% attributed to the way professional movies are shot and edited in order to
% intentionally manipulate the viewers' attentional focus and mental states
% \citep{brown2012cinematography, dancyger2011film-technique}.
%
Further, a pioneering study \citep{bartels2004mapping} suggests that functional
specialization of cortical areas preserved during complex, life-like stimulation.


\paragraph{Better compliance}

%
From a practical perspective, naturalistic stimuli promise improved subject
compliance regarding wakefulness and head motion due to minimal instruction
requirements (e.g. no fixation of eye gaze), task demands (no task except
enjoying the movie or audiobook).
%
This is especially the case for young children \citep{vanderwal2015inscapes},
and possibly psychiatric \citep{eickhoff2020towards} or elderly persons
resulting in increased data quality.
%
Since movies and audiobooks are interesting and easy-to-follow stimuli that are
produced to be engaging and immersive, they also promise to put participants at
ease in the otherwise claustrophobic, uncomfortable and noisy fMRI scanner.
%
Lastly, auditory narratives are also appropriate for visually impaired persons
suffering from diminished or lacking eye-sight.


\section{Prediction individual topography from a reference group}

% general idea
Given that one localizer = one domain of brain function is unsuitable to
characterize a variety of perceptual and cognitive processes in a time-efficient
manner, you could just predict the location in a ``new or  unknown'' subjects
from data of a reference group.


\subsection{Anatomical alignment}

\todo[inline]{read closely \citep{weiner2018defining}, check references and
papers that cite it}

\todo[inline]{shortly explain volume alignment (and surface-based alignment?)}

% current approach
One approach is based on anatomical alignment in a three-dimensional, anatomical
reference space (hence, \textit{anatomical alignment}).
%
Anatomical alignment aligns functional data of an individual to an anatomical
brain template (e.g. the MNI152 template brain).
%
In order to predict, the functional data in the subject-specific functional
space, the data of N-1 subjects are mapped from the anatomical template space
into the functional space of the left-out / new / unknown subject.
%
Last step: e.g. mean of projected $Z$-maps.


\subsubsection{but functional-anatomical correspondence}
% problem
But problematic is the shitty \textit{functional-anatomical correspondence}
% definition of functional-anatomical correspondence
<definition of functional-anatomical correspondence here>
% illustration
Hence, even assuming two persons had a identically shaped cortex anatomy, still
location, shape size of functional areas would possibly/probably be different.

\todo[inline]{check \citep{feilong2018reliable}}
%
\todo[inline]{check new papers: \citep{feilong2021neural, busch2021hybrid}}


\subsection{Functional alignment}
%
An alternative approach, to individual localization has been proposed by
\citet{haxby2011common}.
%
They (and e.g. \citet{jiahui2019predicting}) also predicted the location, form,
and size of target brain areas in ventral temporal cortex from dedicated
localizer scans of other individuals.
%
The key difference of this approach is to rely on similarity of representational
geometry of brain activity patterns and aligning individual brains into a
multi-dimensional functional group space, termed \ac{cms}.

\subsubsection{why naturalistic stimuli are better}
%
While the functional alignment can also be applied to fMRI data from stimulation
paradigms with simplified stimuli, the transformations for functional alignment
have greatly diminished general validity \citep{haxby2011common}, presumably
because such experiments sample a sparser range of brain states
\citep{guntupalli2016model}.
%
Increased validity of derived transformation for functional alignment by
sampling a more diverse set of mental states that reflect (confound) statistics
of the natural environment, and enable investigation of the acquired data for a
variety of research questions (e.g. visual or auditory perception, spatial
cognition; emotion; music, speech or social perception)
%
``Even within a sensory modality, such as vision, different types of information
are layered and simultaneously present in natural movies. The rich, layered
information in movies has allowed concurrent modeling of multiple stages of
perceptual processing (Nishimoto et al., 2011; Huth et al., 2012; Güclü and van
Gerven, 2017). Broader sampling and efficient engagement of multiple systems
motivated the use of movie-viewing data as the basis for developing algorithms
for aligning functional anatomy (Conroy et al., 2009, 2013; Sabuncu et al.,
2010; Haxby et al., 2011; Chen et al., 2015; Guntupalli et al., 2016, 2018)''
\citep{haxby2020naturalistic}.


\subsubsection{Hyperalignment}

\todo[inline]{write text in SRM chapter; put a shorter version in here}

\todo[inline]{explain what a ``left-out subject'' is or ``aims of thesis'' below
does not make sense}

\subsubsection{Shared response model}

\todo[inline]{write text in SRM chapter; put a shorter version in here}


\subsubsection{interim summary}
% interim summary
In summary, naturalistic stimuli ``impose a meaningful timecourse across
subjects while still allowing for individual variation in brain activity and
behavioral responses, and lend themselves to a broader set of analyses than
either pure rest or pure event-related task designs'' \citep{finn2017can}.


\section{Aims of thesis}
%% At the end of the introduction, add a subchapter on the Aims of Thesis in
%% which you describe the research question and the objectives of your work on
%% a maximum of two pages

\todo[inline]{imo, section should include roadmap/overview of work packages}
\todo[inline]{titles of subsections are too long}

%
Previous work has shown that it is possible to combine \ac{bold} \ac{fmri} with
a naturalistic stimulation in order to localize areas correlating with
particular brain functions \citep{bartels2004mapping}.
%
This can be approached by either modeling specific stimulus features, or by
using the high-dimensional nature of such a stimulus to derive a \ac{cms} to
align functional properties of cortex across study participants.
%
This dissertation explores whether a movie and the movie's audio-description
that was produced for an visually impaired audience can substitute an
established localizer paradigm.
%
Naturalistic stimuli would have substantial advantages (e.g. task demands
compliance, and data quality) over simplified stimuli presently used in
localizer paradigms.
%
The time-courses of rich, full-length naturalistic stimuli are correlating with a
variety of different brain functions ranging from low-level perception (e.g.
luminance) to high-level cognition (social cognition).
%
Thus a naturalistic stimulus could, in principle, replace multiple dedicated
localizers to provide a more comprehensive diagnostic routine.
%
Nevertheless, a 2-hour paradigm would be unsuitable for a clinical population
and uneconomically.
%
For that reason, this dissertation will also explore weather just a part of a
naturalistic stimulus can be used to align an ``unknown'', left-out subject to a
\ac{cms} in order to a) map results from a functional localizer in a reference
group onto the unknown subject's anatomy and thus b) predict the results from
the functional localizer in the left-out subject.
%


\subsection{Specific objectives and hypotheses}

\subsubsection{A studyforrest extension, an annotation of spoken language in the
German dubbed movie ``Forrest Gump'' and its audio-description}

%
Movies are designed to entertain the audience and not to conduct research.
%
Variables (i.e. the ``features'' embedded in the naturalistic stimuli) might be
confounded.
%
Moreover, features of interest might be highly correlated, making a it
impossible to controlling them statistically.
%
Hence, researchers often rely on data-driven methods to analyze the data
``because they do not require an explicit model of the task or stimulus'' and/or
``constructing such a model may be prohibitively difficult.''
\citep{nastase2019measuring}.
%
Which is true but the wrong mindset that lead to a lack of annotation in most
datasets derived from naturalistic paradigms.
%
From data-driven approaches, we gained a lot of knowledge, but data-driven
approaches essentially fall short in case you want to correlate discovered brain
activation patterns with psychological processes.
%
``annotation bottleneck'' \citep{aliko2020naturalistic}i.

``Nevertheless, naturalistic stimuli add additional layers of complexity to the
technical difficulty of data sharing. Some of these – such as recording
acquisition and presentation timings as well as annotating stimuli for features
of interest – are well-recognized from the task-based neuroimaging literature
and implemented in existing standards such as BIDS'' \citep{dupre2020nature}.

``Modeling the stimulus and task properties is a central challenge with
naturalistic paradigms and one that has great potential if solved (Simony and
Chang, 2020) \citep{saarimaki2021naturalistic}''.
%
``However, modeling the naturalistic stimulus requires a way to extract
time-varying, emotion-related processes. The interpretations based on combining
stimulus-driven features with brain imaging data are based on two critical
assumptions: first, that the researcher knows which features of the stimulus are
driving the brain activity, and second, that these features have been modeled
accurately (Chang et al., 2020). Following the terminology in naturalistic
studies focusing on visual, object, and semantic features (e.g., Huth et al.,
2012, 2016), the term emotion feature refers to any emotion-related variable
that we can continuously extract from either the stimulus or the observer
(Figure 1)'' \citep{saarimaki2021naturalistic}.


\paragraph{Maybe: what was tried...}

...and did not work: IBM Watson \& Google Cloud text-to-speech





%
Why does the dataset contain those annotations? Annotations are of utterly
importance in case you want to perform model-driven analyses
%
For example for a GLM: you want to build contrasts of interest; or you want to
build nuisance regressors just to explain variance
%
both movie and the audio-description provide spatial information.
%
let's try to use semantic spatial information for a proof of concept
%
The goal of this study is to create stimulus annotations to model hemodynamic
responses during the auditory naturalistic stimulation, and validate its quality
for the following studies.
%
Previous results show that there are significant voxel-wise BOLD response time
series correlations between the two datasets in brain areas associated with
speech and story processing (s. Figure 3 in \citep{hanke2016simultaneous}).
%
This investigation will also provide insight if it is feasible to produce an
audio-only localizer paradigm as an alternative for studies where no visual
stimulation is feasible or desired.

Hence:
\begin{itemize}
    \item direct modeling of a natural stimulus and
    \item prediction via functional alignment to a reference population
\end{itemize}


\subsubsection{Processing of visual and non-visual naturalistic spatial
information in the ``parahippocampal place area''}

%
Comparison of results from a task-based functional localizer to
results from naturalistic stimulus paradigms
%
It is likely that a rich natural stimulation evokes more widespread network
activity, compared to a simplified stimulus that additionally requires the
subject to perform a task.
%
Hence: compare task-free localizer to dedicated task-based speech localizer
%
compare diagnostic performance of the task-free movie / audio-description fMRI
recordings will be compared to the results of the localizer paradigm.
%
a specific diagnostic contrasts can be selected.

From TeaP talk:
%
What is still kind of unclear is if these results generalize from static
pictures to results from a more ecologically valid stimuli like a movie.
%
What is still unclear is if these results generalize to the auditory domain.  To
frame it as a question: Does increased hemodynamic activity in the PPA correlate
with auditory spatial information?
%
The goals of your study were the following: We asked ourselves:
%
1) How can we operationalize the perception of spatial information that is
embedded in our naturalistic stimuli?
%
2) Further, we wanted to compare our results to a classic visual localizer
experiment that used blocks of pictures.
%
And lastly, we wanted to explore if an auditory narrative could substitute a
visual experiment to do individual diagnostics
%
Which means: can we localize the PPA in individual persons using a more engaging
\& entertaining, but also exclusively auditory stimulus
%
To operationalize the perception of spatial information, we looked at time
points in the stimuli that should correlate with the perception of spatial
information.
%
And we looked for time points correlating with other perceptual processes that
we needed to build general linear model contrasts.
%
For the movie, we used the movie cuts that basically re-orient the observer in
space.
%
In the context of naturalistic stimuli - not just data-driven but also
model-driven analyses can be applied to data from naturalistic stimuli.
%
- but: to model event-related hemodynamic activity you need to know the temporal
structure of the stimulus
%
- and because naturalistic stimuli are so versatile and trigger so many
perceptual and cognitive processes you can re-use already existing data to
answer new research questions.


results suggest that you can ``localize a auditory PPA'' but not a the ``visual
PPA'' using an auditory narrative; It's similar but different; SRM study will
quantify the prediction performance

\subsubsection{DataLad: distributed system for joint management of code, data,
and their relationship}

\todo[inline]{well, does not really fit in}


\subsubsection{Using varying amount of data from naturalistic stimulation for
functional alignment to predict results from a task-based functional localizer
paradigm}
%
All analyses up to this point will have been performed on 2h-long scans, but any
clinical application must aim to minimize the scan time/cost.
%
We test whether reliable functional alignment can be achieved with a task-free,
natural stimulation ``calibration” scan that requires no more acquisition time
than a conventional localizer paradigm.
%
We will estimate the trade-off between diagnostic quality and required effective
scan time by progressively reducing the duration of input BOLD fMRI data and
comparing the results of the reduced model to the reference computed from the
full length scan.


\paragraph{Predict ROI from a reference group}
%
fMRI data acquired from an individual will not be analyzed directly regarding a
specific cognitive function, but will be used to align that individual brain's
voxel space with a common high-dimensional representational reference space.
%
Each axis in this common space can be seen as a kind of cortical tuning
function.
%
The orthonormal transformation of an individual voxel space into this common
reference reflects the particular linear combination of voxel response time
series with respect to each common space component.
%
Using a leave-one-subject-out strategy, individual results of the conventional
localizer will then be projected into the common space, aggregated, and
re-projected into the voxel space of the left-out individual for comparison with
the localizer results for that individual (see Figure 2).
%
Using movie-evoked brain activity, hyperalignment procedure learns subject-wise
optimal transformations of brain activity into a common representational space.
%
Once aligned, the brain activity of the reference group can be used to predict
the activity of another subject.
%
The localization of speech areas will also be performed in the common reference
space directly (the stimulation time axis is preserved in the common space), and
the results will be projected into the voxel space of the left-out individual,
where they can be compared with the localization result derived from that
subject’s movie scan.
%
Once a valid alignment is established, the inverse transformation is then used
to project functional properties of the common reference into that individual's
voxel space.


\paragraph{minimum length of stimulus; ``calibration scan''}
%
minimum data requirement for reliable functional alignment (within-subject
test-retest???
%
estimate the minimum scan time requirement for deriving a valid functional
alignment by further reducing the amount of input data of the left-out
individual.
\paragraph{just uses an intersecting time-series between stimulus sets}
%
The joint dataset will enable me to study whether a functional alignment to the
common reference space can be performed based on a short scan.


\subsection{Open, transparent, and reproducible science}
%
\todo[inline]{write introductory sentences to ``reproducibility crisis''}
\todo[inline]{quote a ``famous'' open science paper}
%
On a metalevel, this dissertation thrives to follow the principle of open
science [quote?], and provide
transparent, reproducible procedures and results \citep{poldrack2017scanning}
that can easily rerun and validated by third-party researchers.
% aim
Thus [wherever possible and as good as possible], I will follow and implement
best practices for a) coding and scientific computing \citep{wilson2014best}, b)
data analyses \citep{nichols2017best}, and c) sharing code, created data and
results \citep{eglen2017toward, nichols2017best}.


\subsubsection{input = open science \& studyforrest}

\todo[inline]{future (``will be created'') vs. perfect (``has been created'')?}

% open data \citep{eglen2017toward}
As input data, my work will be/is build on top of publicly and freely available,
\ac{fmri} data that is part of the \textit{studyforrest} project.
%
The studyforrest project is an open science project that aims at providing a
versatile resource for investigating human brain function under quasi-natural
conditions.
%
The core of this dataset are two hour long \ac{bold} \ac{fmri} scans of
participants watching the movie Forrest Gump and listening to the movie's
audio-description of equal length.
%
Since its first publication in 2014 \citep{hanke2014audiomovie}, this resource
has led to independent studies of international research groups that were
published in peer-reviewed journals (s. http://studyforrest.org).

\begin{comment} \citep{dupre2020nature}:
%
Study Forrest is a data collection and curation effort designed to serve as a
    community resource for new discoveries.
%
As of October 2019, 29 unique studies had been published using the
    studyforrest.org dataset, 17 of which were published without any of the
    original authors of the data release.
%
This is possible in large part thanks to the richness of naturalistic stimuli,
    where the same movie can be used for both task-free as well as
    stimulus-driven analyses, with the original stimulus re-annotated for
    particular features of interest.
%
For example, studyforrest.org has been used to test cerebrovascular biomarkers
    (Voss et al., 2017) but, among other features, was also annotated for
    expressed emotion (Labs et al., 2015) which later informed a study on
    emotion encoding gradients in the brain (Lettieri et al., 2019)''
    \citep{dupre2020nature}.
%
\end{comment}
%
Data, like stimulus annotations, that will/has been created of the course of the
dissertation project will provided in standardized file format,
version-controlled and published, and therefore extended the studyforrest
project.

%
\subsubsection{analyses, code, output}

%
All data analyses pipelines will be implemented in a way that enables automated
processing.
%
First, analyses pipelines will not be implemented using commercial, proprietary
software but freely available and, if possible, open-source software to
guarantee reproducibility (s. \citep{eglen2017toward}).
% which tools to choose why?
Among potential software packages, we/I will choose the tools that offer the
most solid documentation and basis of developers and maintainers to ensure
long-term support.
% my code
Second, any custom code written by myself will implemented in open-source
programming language, will be version-controlled, documented and released
publicly and freely accessible.\todo[inline]{are we actually allowed regarding
FZJ rules?}
%
Third, all input data, custom code and output data will be provided as
standardized and easy-to-access \textit{DataLad} (datalad.org) datasets, a free
and open-source software solution that manages provenience, distribution,
version-control of code and data, and provides a structured record of all
executed steps from downloading the input data to visualizing the results. The information of past command-line invocations can be used to re-execute
commands to rerun every step and validate the results
\citep{halchenko2021datalad}.\todo{is direct quote?}

%
Consequently, all steps from acquiring the raw data, performing the analyses,
and visualizing the results can be check, rerun and validated by independent
badass motherfuckers.
%
Since ``nature abhors a paywall'' \citep{dupre2020nature}, publications
describing generated data or reporting research results will be published in
open-access journals.

``The paper would be written using a literate programming technique in which the
code for figure generation is embedded within the paper and the data depicted in
figures are transparently accessible. The paper would be distributed along with
the full codebase to perform the analyses and the data necessary to reproduce
the analyses, preferably in a container or virtual machine to enable direct
reproducibility'' \citep{poldrack2017scanning}.
%
``Unthresholded statistical maps and the raw data would be shared
via appropriate community repositories, and the shared raw data would be format­
ted according to a community standard, such as the Brain Imaging Data Structure
(BIDS)73, and annotated using an appropriate ontology to enable automated
meta-analysis'' \citep{poldrack2017scanning}.


\paragraph{just some notes}

\todo[inline]{other contrasts that were tested over the course of the
dissertation: phonemes, grammatical tags, prosody, sex}
