%% The introduction begins with an overview of the topic

% introductory quote
``A remarkable feature of the vertebrate brain is the anatomical specialization
of cortical regions for the processing of different types of information. Since
the late 19th century, it has been recognized that restricted lesions of the
human brain result in location-specific sensory, motor or cognitive deficits''
\citep[][p. 268]{cohen1994localization}.
% human brain mapping
Contemporary \textit{human brain mapping} \citep[e.g.,][]{raichle2009brief}
investigates in vivo the brain's topographic organization
\citep[e.g.,][]{eickhoff2018topographic} by specifying ``in as much detail as
possible the localisation of function in the human brain'' \citep[][p.
10]{savoy2001history}.
% fMRI + BOLD
Since the early 1990s, brain imaging studies have used \ac{fmri} to
measure \ac{bold} activity as a proxy for neural activation.
% higher visual areas
Replicated findings in the domain of higher-visual perception, for example,
suggest that category-selective brain regions, such as the parahippocampal
place area \citep{epstein1998ppa, epstein1999parahippocampal} or the fusiform
face area \citep{kanwisher1997ffa, kanwisher2006fusiform}, exhibit significantly
increased \ac{bold} activity correlated with a ``preferred'' stimulus class.
%
The \ac{ppa} responds more strongly when participants are watching images of
scenes and landscapes compared to images of, e.g., faces;
%
vice versa, the \ac{ffa} responds more strongly when participants are watching
images of faces compared to images of scenes.

%
However, most studies in functional imaging research have averaged data across
study participants for
%
practical (e.g., limited scan time per subject),
%
statistical reasons (e.g., improved signal-to-noise ratio),  % \ac{snr}
%
or to generalize from study participants to a broader (neurologically healthy)
population.
%
While studies that average data across study participants may draw
population-level inferences, they also risk losing individual-level information
and may capture only the ``common denominator'' \citep[][p.  2]{pinel2007fast}
of functional patterns.
%
Yet, in order advance human brain mapping towards clinical applications that
assess health and disease, fMRI data need to be interpreted on the level of
single individuals \citep{dubois2016building, wegrzyn2018thought}.



\section{Functional localization}

%% Further introductory remarks describe the scientific background of the work
%% as precisely as possible. Cite the most important publications and avoid
%% extensive literature reviews.
% a.k.a. "state of research"

% ``The ROI approach is advantageous for four reasons:
% 1) it allows hypothesis-driven comparisons of signals within independently
% defined ROIs across many different conditions,
% 2) it increases statistical sensitivity in multisubject analyses [Nieto–
% Castañón and Fedorenko 2012],
% 3) it reduces the number of multiple comparisons present in whole-brain
% analyses [Saxe et al.  2006], and
% 4) it identifies ROIs in each participant's native brain space''
% \citep{rosenke2021probabilistic}.

% intro to functional localizer
\textit{Functional localizers} \citep[s.][for reviews]{saxe2006divide,
friston2006critique} are \ac{fmri} experiments that aim to characterize the
\textit{topography} (i.e. the location, size, and shape) of functional areas
whose hemodynamic activity correlates with perceptual processes, such as the
perception of object categories \citep{kanwisher1997ffa}, or cognitive
processes, such as theory of mind \citep{spunt2014validating}.
% purpose: ROI improve the statistical power of the main experiment's analysis
Functional localizers are frequently used as a separate experiment to identify a
subject-specific functional \ac{roi} and reduce the number of voxels
investigated in the main experiment \citep{poldrack2007region, saxe2006divide}.
% clinical application
Additionally, they may be employed as a diagnostic tool before neurosurgery
\citep[cf.][]{silva2018challenges, szaflarski2017practice}.
% cons: designed for detection power
However localizers are designed to maximize detection power, and use carefully
selected, simplified, and tightly controlled stimuli presented in a block-wise
manner that are often accompanied by a task to keep study participants
attentive.
% one localizer = one domain of functions
Consequently, one localizer paradigm can typically map only one domain of brain
functions.
% batteries: intro
In order to map many different processes despite limited resources, researchers
have developed more time-efficient, multi-functional \textit{localizer
batteries} \citep[e.g.,][]{barch2013function, drobyshevsky2006rapid,
pinho2018individual, pinho2020individual, pinel2007fast}.
% batteries: example
For example, \citet{pinel2007fast} uses a range of dedicated stimuli and
specific tasks in a 5-minute routine to map processes of ``auditory and visual
perception, motor actions, reading, language comprehension, and mental
calculation'' \citep[][p. 15]{pinel2007fast}.
% but still...
Nonetheless, current paradigms face two challenges.
% validity?
From a theoretical standpoint, localizers rely on selectively sampled, tightly
controlled stimuli presented in blocks, and do not resemble how we perceive the
real world outside of the laboratory in daily life.
% compliance?
From a practical standpoint, localizers rely on participants' comprehension of
task instructions and compliance during the scanning session, which can be
difficult to achieve in clinical or pediatric populations
\citep{eickhoff2020towards, vanderwal2015inscapes, vanderwal2019movies}.


\section{Naturalistic stimuli}
% intro to "real-life" neuroscience
Because a major goal of neuroscience is not to reveal how the brain responds to
blocks of stimuli presented in a laboratory setting, but how the brain processes
information during everyday perception, \textit{naturalistic stimuli} are
gaining popularity in neuroimaging.
% definition quote
Naturalistic stimuli are ``a class of stimuli that aim to evoke more
naturalistic patterns of neural responses than traditional controlled artificial
stimuli. Naturalistic paradigms are typically complex and dynamic, and longer in
duration than many conventional stimuli.'' \citep[][p. 2]{vanderwal2019movies}.
% movies & narratives
Currently, the most popular naturalistic stimuli in neuroscience are movies and
auditory narratives \citep[cf.][for an introduction]{sonkusare2019naturalistic}
that provide a time-locked event structure, sample a broad range of brain
states, and engage multiple perceptual and cognitive systems in parallel
\citep{haxby2020naturalistic}.

%
Naturalistic stimuli have several advantages over traditional paradigms.
% Carefully chosen stimulus sets ``selectively sample from the stimulus
% population leading to a stimulus-as-fixed-effect fallacy [Clarc, The
% language-as-fixed-effect fallacy: A critique of language statistics in
% psychological research]'' \citep{westfall2016fixing}. ``The conclusions cannot
% be generalized to a broader population of stimuli without risking inflated
% Type I error  [cf. Donnet S, Lavielle M, Poline JB: Are fMRI event-related
% response constant in time? A model selection answer]''
% \citep{westfall2016fixing}.
From a theoretical standpoint, naturalistic stimuli promise an increased extent
of both subtypes of external validity, namely population validity and ecological
validity \citep{bracht1968external}.
%
\textit{Population validity} refers to the extent to which inferences drawn from
an experiment's results may generalize from the experiment's sample of subjects
and stimuli to the total population of potential subjects and stimuli
\citep{bracht1968external, westfall2016fixing}.
%
\textit{Ecological validity} refers to the extent to which inferences drawn from
an experiment's results may generalize from the experiment's setting, stimuli,
and task to nonexperimental situations \citep{bracht1968external,
orne1962social, schmuckler2001ecological}.
% external validity
Naturalistic stimuli promise an increased population validity of stimuli because
the \textit{stimulus features} (i.e. the variables or stimulus classes) that are
embedded in a naturalistic stimulus represent a more random sample from the
total population of stimuli that might have been used
\citep{westfall2016fixing}.
% ecologically validity
Naturalistic stimuli also promise a higher ecological validity because they more
closely resemble how we experience our environment outside of the scanner bore
\citep{hasson2012future}.

% reviews
Audio-visual movies and spoken narratives have been used during \ac{fmri}
\citep[s.][for reviews]{hamilton2018revolution, hasson2008neurocinematics,
jaaskelainen2021movies, saarimaki2021naturalistic}, and \ac{eeg} or \ac{meg}
data acquisition \citep[s.][for reviews]{alday2019meg, kandylaki2019story}.
% early findings: spatiotemporal synchronization
Previous studies have shown that watching a movie \citep{hasson2004intersubject,
hasson2010reliability} or listening to a narrative \citep{lerner2011topographic,
wilson2008beyond} reliably synchronizes spatiotemporal responses across multiple
subjects in a large part of the brain compared to, for example, an unedited
video of a concert taken from a fixed viewpoint
\citep{hasson2008neurocinematics}.
% Bartels (2004)
Importantly, a pioneering study by \citet{bartels2004mapping} demonstrated
results at a group level, indicating that functional specialization of cortical
areas is preserved during naturalistic stimulation.
% conclusion: naturalistic stimuli as a localizer
Hence, on an individual level, a naturalistic stimulus could, in theory, also be
used as a more life-like paradigm to replace traditional localizer paradigms.

% a film director's goal is not just to direct a movie, but also to capture and
% direct the audience's attention; professional movies are shot and edited in
% order to intentionally manipulate the viewers' attentional focus and mental
% states \citep{brown2012cinematography, dancyger2011film-technique}.

% engaging & essentially no task
From a practical standpoint, naturalistic stimuli require minimal instructions
given by the experimenter, and place minimal task demands on a study
participants, who can simply enjoy the movie or audio story.
%
In addition, movies and narratives provide an interesting and engaging
stimulation that can put participants at ease in the claustrophobic,
uncomfortable, and noisy MRI scanner.
%
Consequently, naturalistic stimuli promise improved data quality due to reduced
fatigue and head movement, particularly in children
\citep{vanderwal2015inscapes}, and possibly psychiatric
\citep{eickhoff2020towards} or elderly individuals.

% intro
Nevertheless, naturalistic stimuli also have disadvantages over traditional
paradigms.
% produced for commercial use
First, the majority of naturalistic stimuli used in neuroimaging have been
created by professional production companies for commercial purposes, rather
than for research purposes.
% temporal structure not explicitly known
The temporal structure of stimulus features embedded in professionally produced
naturalistic stimuli is fixed, and thus reproducible, but initially not
explicitly known.
% challenging analysis
Consequently, modeling brain activity correlating with stimulus features
embedded in a stimulus' time course is challenging
\citep{saarimaki2021naturalistic, simony2020analysis} because such models, like
a traditional \textit{\ac{glm}}, rely on the stimulus features being annotated.
% lack of annotations = bottleneck
The lack of extensive annotations has resulted in a ``usage bottleneck''
\citep[][p.  16]{aliko2020naturalistic} and may be the main reason why explicit
models of embedded stimulus features are ``notoriously'' \citep[][p.
1]{richard2019fast}, if not ``prohibitively'' \citep[p.
676]{nastase2019measuring} difficult to construct.
% 120 minutes is too long
% SRM
Second, considering practical and monetary constraints in a clinical context,
presenting a full feature film typically lasting 90 to 120 minutes is
inappropriate as an individual diagnostic procedure.





\section{Aims of thesis}
%% At the end of the introduction, add a subchapter on the Aims of Thesis in
%% which you describe the research question and the objectives of your work on
%% a maximum of two pages

% aim = what you hope to achieve; statements of intent}
% objective = the action(s) you will take in order to achieve aims

% main goals
This dissertation explored---while adhering to the principles of
open, transparent, and reproducible science---whether a movie and the movie's
audio-description that was produced for a visually impaired audience could, in
principle, replace a traditional localizer paradigm.
% focus: ppa
As a proof of concept, the dissertation focuses on the \ac{ppa} that exhibits
increased hemodynamic activity when participants view images of landscapes,
buildings or landmarks, as opposed to, for example, images of faces or tools
\citep[s.][for reviews]{epstein2014neural, aminoff2013role}.
% auditory semantics unclear
To the author's knowledge, only one study \citep{aziz2008modulation}
compared hemodynamic activity levels in the \ac{ppa} that were correlated with
different categories presented in spoken sentences.
%
Despite mixed results, the findings by \citet{aziz2008modulation} suggest that
the \ac{ppa} does not solely respond to visually presented scene-related stimuli.

% SRM
I assessed the feasibility of using the movie and audio-description as
alternatives to a visual localizer in two ways.
% 1. modeling hemodynamic responses
In the first approach, we modeled hemodynamic activity based on annotated
stimulus features that are embedded in the audio-visual movie and the movie's
audio-description (each lasting $\approx$\unit[120]{m}).
% but technically similar to traditional localizers
We then created \ac{glm} $t$-contrasts using the modeled hemodynamic time
courses (i.e. regressors) to locate the \ac{ppa}, which was previously
identified in the same group of participants by a conventional block-design
functional localizer using static images \citep{sengupta2016extension}.
%
However, conducting a two-hour long \ac{fmri} scan session may not be desirable
or feasible due to potential compliance issues or constraints on time and
resources.
%
Therefore, we also explored a second approach to localize the \ac{ppa} in an
individual that leverages data collected from an independent sample of other
individuals (i.e. data from a \textit{reference group}).
% previous studies: anatomical alignment
To address the issue of anatomical variability across persons, previous studies
estimated the location of a participant's functional area by performing an
\textit{anatomical alignment} \citep{frost2012measuring,
rosenke2021probabilistic, weiner2018defining, zhen2017quantifying} that relies
on anatomical scans.
%
However, the anatomical location of functional regions varies between
individuals \citep{coalson2018impact, benson2014correction, natu2021sulcal,
wang2015probabilistic, frost2012measuring, langers2014assessment, weiner2014mid,
rosenke2021probabilistic}.
% intro to functional alignment
To address the issue of functional-anatomical variability across persons, we
therefore performed a \textit{functional alignment} \citep[cf.][for
reviews]{haxby2020hyperalignment, bazeille2021empirical}, and investigated
whether we can estimate the results of $t$-contrasts (i.e. statistical $Z$-maps)
that we created in the previous studies to localize the \ac{ppa}.
%
Considering that functional alignment relies on functional scans, we also
evaluated the relationship between the quantity of functional data used for
functional alignment and its subsequent estimation performance.



\subsection{Open, transparent, and reproducible science}

% reproducibility crisis
In the last decade, there has been a growing awareness that the results of
scientific publications are not reproducible or general scientific findings are
not replicable.
%
This prompted some authors to refer to the sciences as being in the midst of a
``reproducibility crisis'' or ``replication crisis''
\citep{baker2016reproducibility, plesser2018reproducibility,
stupple2019reproducibility, nosek2022replicability}.
% reproducibility: definition
``A study is reproducible if all of the code and data used to generate the
numbers and figures in the paper are available and exactly produce the published
results'' \citep[][p. 111]{leek2017most}.
% replicability: definition
A study is replicable if the same analysis of data from an equivalent experiment
yields consistent results \citep{dubois2016building, leek2017most}.
%
The first goal in the context of open science was to meet the requirements of
open, shared, accessible, and transparent science \citep[cf.][]{watson2015will,
fecher2014open} as well as a reproducible and replicable research project:
%
To achieve this, the dissertation follows the guidelines and best practices for
(a) coding and scientific computing \citep{wilson2014best}, (b) procedures and
data analyses \citep{nichols2017best, poldrack2017scanning,
poldrack2019establishment}, and (c) sharing code, created data, and results
\citep{eglen2017toward, nichols2017best, pernet2015improving}.
% \paragraph{input data} open data \citep{eglen2017toward}
The second goal in the context of open science was to reuse a publicly and
freely available dataset for a new research question that was not anticipated at
the time the data were released, to extend the dataset, and to generate novel
findings to be published in open-access journals.

%
The present work is built upon \ac{fmri} data that are part of the
\textit{studyforrest} project
(\href{www.studyforrest.org}{\url{studyforrest.org}}).
% core stimuli
The core of this project are two-hour long \ac{bold} \ac{fmri} scans of
participants watching the movie ``Forrest Gump'' \citep{ForrestGumpMovie} and
listening to the movie's audio-description that was created for a visually
impaired audience by adding a narrator to the movie's audio track.
% used by 3rd parties
Since its first publication in 2014 \citep{hanke2014audiomovie}, the
studyforrest project has served as a resource of raw (and preprocessed) data for
international working groups to conduct and publish independent, peer-reviewed
research (cf.
\href{www.studyforrest.org/publications.html}{\url{studyforrest.org/publications.html}}).
% DataLad
All data, materials, custom code, analysis steps, and results used or created
over the course of this dissertation are version-controlled---i.e. each change
of data is logged and documented---and accessible in standardized
\textit{DataLad}
\citep[\href{www.datalad.org}{\url{datalad.org}};][]{halchenko2021datalad}
datasets.
% automatization
Data analysis pipelines are designed for automated processing and implemented in
freely available and, where possible, open-source software.
% which tools to choose why?
Among potential software packages, the tools were chosen that offered the most
solid documentation, and a large community of developers and maintainers to
ensure long-term support.
% my code
All custom code is written in open-source programming languages (Python and
Bash), version-controlled, documented, and released publicly.
% inference
Therefore, all executed steps from downloading input data to visualizing the
results can be rerun to improve reproducibility of current results and to
facilitate replication of findings on other datasets.
% \paragraph{publications} open-access publishing
Finally, because ``nature abhors a paywall'' \citep{dupre2020nature},
publications describing generated data, methodological choices, analysis steps,
and results are published in open-access journals.
% neurovault
% Unthresholded statistical maps of all computed statistical $t$-contrasts are
% published at Neurovault (\href{https://neurovault.org/}{\url{neurovault.org}}).



\subsection{Specific objectives and hypotheses}

% \subsubsection{DataLad: distributed system for joint management of code, data,
% and their relationship}

% what I did
During the pre-alpha stage of DataLad, I contributed to its development by
testing its features on a real-world project, evaluating its user interface and
documentation, and providing feedback to the software engineering team.
% results
This work laid the technical foundation for achieving the goal of a transparent
and reproducible research project, and resulted in co-authorship of the
software's accompanying publication \citep[s.][]{halchenko2021datalad}.


%\subsubsection{A studyforrest extension, an annotation of spoken language in the
%German dubbed movie ``Forrest Gump'' and its audio-description}

% intro
In \citet{haeusler2021speechanno}, we created and validated an annotation of
speech spoken in the movie, as well as its audio-description, with two goals in
mind.
% goal #1: groundwork for PPA study
The first goal was to lay the groundwork for creating models of hemodynamic
activity in response to the movie and the audio-description in
\citet{haeusler2022processing}.
% goal #2: extend studyforrest
The second goal was to create an exhaustive annotation of speech in order to
extend the studyforrest project and supplement the previously published
annotations of portrayed emotions \citep{labs2015portrayed}, perceived emotions
\citep{lettieri2019emotionotopy}, and cuts and locations depicted in the movie
\citep{haeusler2016cutanno}.
% validation analysis
In order to validate the annotation's quality, we conducted a canonical \ac{glm}
analysis of modeled hemodynamic activity based on information drawn from the
annotation.
% contrast
Regressors correlating with speech-related events were contrasted to a regressor
correlating with events without speech.
% hypothesis
We hypothesized that results would reveal clusters of significantly increased
activity in brain regions known to be involved in processing spoken language.
% results
The analysis revealed statistically significant increased hemodynamic activity
in a bilateral cortical network replicating results of previous studies that
used tightly controlled stimuli \citep[s.][for reviews]{friederici2011brain,
hickok2007cortical,price2012twentyyears}, and studies that used data-driven
methods to analyze \ac{fmri} data from auditory naturalistic stimulation
\citep{honey2012not, lerner2011topographic, silbert2014coupled}.
% conclusion
The results of the validation encouraged us to publish the annotation as an
extension of the studyforrest dataset, and use it as the foundation to be
adapted to our specific needs in \citet{haeusler2022processing}.

% study in one sentence
In \citet{haeusler2022processing}, our goal was to investigate the possibility
of localizing \ac{ppa} using GLM $t$-contrasts based annotations of the
naturalistic stimuli.
% AV annotation For the analysis of the
For the analysis of the movie, we relied on a previously published annotation of
movie cuts and the depicted location after each cut \citep{haeusler2016cutanno}.
% AD annotation
For the analysis of the audio-description, we extended the annotation of speech
by further annotating nouns that the audio-description's narrator uses to
describe the missing visual content.
% hypo
We hypothesized that a mass-univariate \ac{glm} analysis would reveal increased
hemodynamic activity in medial temporal regions that were functionally
identified as the \ac{ppa} in the same set of participants employing a
traditional block-design functional localizer \citep{sengupta2016extension}.
% group results
On a group-average level, the results demonstrate that increased activation in
the \ac{ppa} during the perception of static images generalizes to the
perception of spatial information embedded in an audio-visual movie and
exclusively auditory naturalistic stimulus.
% conclusion 1
Results add evidence \citep[cf.][]{bartels2004mapping} that functional
specialization of cortical areas is preserved during naturalistic stimulation.
% individual AD
On an individual level, the analysis of the movie yielded bilateral clusters of
increased hemodynamic activity in the \ac{ppa} of five participants and a
unilateral cluster in seven participants.
%
The analysis of the audio-description revealed bilateral clusters in nine
participants and a unilateral cluster in one participant.
%
These results imply that model-driven \ac{glm} analysis based on a naturalistic
stimulus' annotation can be used to localize functional areas in individual
subjects.

%
Conducting a two-hour long \ac{fmri} scan session may not be desirable or
feasible due to potential compliance issues or constraints on time and
resources.
%
In Chapter 5, we therefore aimed to explore an alternative approach to identify
the \ac{ppa} in an individual by leveraging data from a reference group.
%
To address the challenge of functional-anatomical variability across
individuals, we employed a functional alignment approach using the \acf{srm}
\citep{chen2015reduced}.
%
This approach allowed us to predict an individual's results of $t$-contrasts,
created in previous studies using time series from the visual localizer, movie,
and audio-description, by projecting the results of persons in the reference
group from their respective anatomical space through a \ac{cfs} into the
individual's anatomical space.
%
Following a leave-one-subject-out cross-validation, we split the dataset
repeatedly in a set of one \textit{test subject} and $N-1$ \textit{training
subjects} (the reference group).
%
We then calculated a \ac{cfs} and subject-specific transformations based on the
training subjects' concatenated response time series of the visual localizer,
the movie, and the audio-description.
%
We assessed the prediction performance of each paradigm by aligning the test
subject's response time series from each paradigm separately with the
corresponding time points within the \ac{cfs}.
%
In an ideal scenario, only a part of a naturalistic stimulus would serve as a
``diagnostic'' run to align a new individual with a \ac{cfs} and then estimate
the results of other paradigms.
%
Therefore, we also aimed to explore the relationship between the quantity of
data of each paradigm used for aligning a test subject with the \ac{cfs} and the
subsequent estimation performance.
% Results
Results suggest that an auditory narrative can in principle be
used to estimate a visual area such as the \ac{ppa}.
%
Moreover, 15 to 30 minutes of functional scanning during movie watching can
generate a sufficient amount of data to estimate brain patterns with higher
fidelity than a procedure based on anatomical alignment.
