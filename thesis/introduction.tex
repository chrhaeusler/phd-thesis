%% The introduction begins with an overview of the topic

%
``A remarkable feature of the vertebrate brain is the anatomical specialization
of cortical regions for the processing of different types of information. Since
the late 19th century, it has been recognized that restricted lesions of the
human brain result in location-specific sensory, motor or cognitive deficits''
\citep{cohen1994localization}.
%
Thanks to technological advanced, brain imaging studies have extensively used
\ac{fmri} since the early 1990s to measure \ac{bold} activity in-vitro.
%
\textit{Human brain mapping} (e.g., \citep{raichle2009brief}) explores the
brain's topographic organization (e.g., \citep{eickhoff2018topographic}) and
attempts ``to specify in as much detail as possible the localisation of function
in the human brain'' \citep{savoy2001history}.
%
For example in the domain of higher-visual perception, replicated findings
suggest that category-specific brain regions like the \ac{ppa}
\citep{epstein1998ppa}, \ac{ffa} \citep{kanwisher1997ffa}), or the \ac{eba}
\citep{downing2001bodyarea}) exhibit significantly increased \ac{bold} activity
correlated with a ``preferred'' stimulus class.

%
Typical analyses of human brain mapping studies average data across subjects for
practical (e.g., limited scan time per subject) or statistical (e.g., improved
of \ac{snr}) reasons.
%
Nevertheless, the precise anatomical location of functionally defined regions of
the brain vary anatomically (measured in Talairach or MNI coordinates) across
individuals \citep{friston2006critique, saxe2006divide}.
%
Consequently, studies employing an averaging approach may just ``capture the
common denominator of each individual cognitive circuit and lose a large amount
of information'' \citep{pinel2007fast}.
%
However, ``interpretation of fMRI data at the level of individual brains is
essential for characterizing brain function in health and disease''
\citep{dubois2016building}.


\section{Functional localization}
%% Further introductory remarks describe the scientific background of the work
%% as precisely as possible. Cite the most important publications and avoid
%% extensive literature reviews.
% a.k.a. "state of research"

\todo[inline]{add IBC: https://project.inria.fr/IBC/publications}

\todo[inline]{ggf. auch: https://github.com/Parietal-INRIA/DiFuMo}

% individual level: localizers
On the level of individual subjects, \textit{functional localizer} experiments
(s. \citep{saxe2006divide, friston2006critique} for reviews) are conducted
during \ac{fmri} to characterize the location size, and shape of functional
areas correlating with perceptual or cognitive processes (e.g. perception of
object categories \citep{kanwisher1997ffa}, speech perception
\citep{fernandez2001language}, or theory of mind \citep{spunt2014validating}).
% purpose: ROI improve the statistical power of the main experiment's analysis
Functional localizers are often used as separate experiment to identify a
subject-specific functional \acp{roi} to ``guide, constrain or interpret results
from a main experiment \citep{saxe2006divide}.
% clinical application
They also promise to advance neuroimaging towards a clinical application (e.g.,
diagnosis prior to neurosurgery) since it depends on making diagnoses for single
cases \citep{wegrzyn2018thought}.\todo{check}
% purpose: neurosurgery
For example, surgical procedures might impact the post-operative quality of life
so much (e.g. concerning cognitive control or speech production) that a
surgical intervention outweighs the therapeutic benefits.

% one localizer = one domain of functions
However, one localizer paradigm can usually map just one domain of brain
functions.
% detection power based on paradigms' design
Designed to maximize detection power, localizers employ carefully chosen,
tightly-controlled, and simplified stimuli presented in a block-wise manner
often accompanied with a task to keep study participants attentive.
% inefficiency -> localizer batteries
Since this approach becomes inefficient if one wants to map many different
processes with limited resources (time, availability and applicability of
diagnostic measures for an individual patient), researchers have developed more
time-efficient, multi-functional \textit{localizer batteries}
\citep{barch2013function, drobyshevsky2006rapid, pinel2007fast}.
% localizer batteries: example
For example, \citet{pinel2007fast} employs a range of dedicated stimuli and
specific tasks in a 5-minute routine to map processes of ``auditory and visual
perception, motor actions, reading, language comprehension, and mental
calculation at an individual level'' \citep{pinel2007fast}.

% intro to contra localizer
But still, current localizer paradigms struggle to overcome especially two
challenges.
%
First, the localizers depend heavily on a participant's comprehension of the
task instruction and his/her compliance, a criterion that can be difficult to
meet in clinical or pediatric populations \citep{eickhoff2020towards,
vanderwal2015inscapes, vanderwal2019movies}.
% validity?
Second, the paradigms rely on selectively sampled, tightly-controlled stimuli
presented in blocks, and do not resemble how we perceive the real world outside
of the laboratory during everyday life.


\paragraph{excursion to current project}

\todo[inline]{better merge following excursion with ``Aims of thesis'' below??
(tough, a short excursion to current project makes sense before the naturalistic
stimulus inferno breaks loose)}

%
For this reason, this dissertation will explore whether a Hollywood movie and
its audio-only variant created for a visually impaired audience could, in
principle, substitute a traditional, task-based paradigm to localize functional
areas.
%
As a proof of concept, the dissertation focusses on the ``parahippocampal place
area'', a classic example of a higher-visual, functional area
\citep{epstein1998ppa, epstein1999parahippocampal} located in the ventral visual
pathway [check PPA paper for reference].
% what PPA is doing
Increased hemodynamic activity is observed in the \ac{ppa} when participants
view photos of landscapes, buildings or landmarks, compared to, e.g., photos of
faces or tools \citep[see reviews][]{epstein2014neural, aminoff2013role}.
%
These stimulus classes that are correlated with hemodynamic responses during
traditional paradigms might also be embedded as \textit{stimulus features} in
more naturalistic stimuli like a movie or spoken narrative.


\section{Naturalistic stimuli}
%
Since a major goal of cognitive neuroscience is not to reveal how the brain
responds to blocks of [tightly-controlled] stimuli presented in a laboratory
setting but how the brain processes information during everyday perception,
\textit{naturalistic stimuli} gained popularity in neuroscience.


\paragraph{Definition}

% definition quote
Naturalistic stimuli are ``a class of stimuli that aim to evoke more
naturalistic patterns of neural responses than traditional controlled artificial
stimuli. Naturalistic paradigms are typically complex and dynamic, and longer in
duration than many conventional stimuli.'' \citep{vanderwal2019movies}.
%
Therefore, naturalistic stimuli [promise to] ``sample a broad range of brain
states and engage multiple perceptual and cognitive systems in parallel''
\citep{haxby2020naturalistic}.
% movies & narratives
The most popular naturalistic stimuli in neuroscience are movies and auditory
narratives (s. \citep{jaaskelainen2021movies, jaaskelainen2020neural} for
reviews) that provide a time-locked event structure during a continuous, rich
and dynamic stimulation.
%
Consequently, naturalistic stimuli promise a higher ecological validity
\citep{zaki2009need, hasson2012future, hamilton2018revolution} because they more
closely mimic our rich visual and auditory experiences outside the scanner bore
in real-life \citep{hasson2008neurocinematics, haxby2020naturalistic}.
%
Naturalistic stimuli also promise a higher external validity because the
\textit{stimulus features} (i.e. the stimulus classes or variables in the
naturalistic stimulus) that are embedded in the naturalistic stimulus represent
a more random sample from the ``theoretical population of stimuli that might
have been used'' \citep{westfall2016fixing}.
%
% Carefully chosen stimulus sets ``selectively sample from the stimulus
% population leading to a stimulus-as-fixed-effect fallacy [Clarc, The
% language-as-fixed-effect fallacy: A critique of language statistics in
% psychological research]'' \citep{westfall2016fixing}. ``The conclusions cannot
% be generalized to a broader population of stimuli without risking inflated
% Type I error  [cf. Donnet S, Lavielle M, Poline JB: Are fMRI event-related
% response constant in time? A model selection answer]''
% \citep{westfall2016fixing}.


\paragraph{Early findings}
% reviews
Audio-visual movies and spoken narratives have been used during \ac{fmri}
(s.\citep{hamilton2018revolution, hasson2008neurocinematics,
sonkusare2019naturalistic, saarimaki2021naturalistic} for reviews), and \ac{eeg}
or \ac{meg} data acquisition (s. \citep{alday2019meg, kandylaki2019story} for
reviews).
%
Early studies have shown that watching a movie \citep{hasson2004intersubject,
hasson2008neurocinematics, hasson2010reliability} or listening to a narrative
\citep{lerner2011topographic, wilson2008beyond} reliably synchronize
spatiotemporal responses across multiple subjects in a large part of the brain
compared to, for example, an unedited video of a concert taken from a fixed
viewpoint \citep{hasson2004intersubject, hasson2008neurocinematics,
hasson2010reliability, lerner2011topographic, wilson2008beyond}.
% This finding could be attributed to a film director's goal to not only direct
% a movie, but also to capture and direct the audience's attention; this can be
% attributed to the way professional movies are shot and edited in order to
% intentionally manipulate the viewers' attentional focus and mental states
% \citep{brown2012cinematography, dancyger2011film-technique}.
%
Further, a pioneering study \citep{bartels2004mapping} suggests that functional
specialization of cortical areas is preserved during complex, life-like
stimulation.


\paragraph{Better compliance}
%
From a practical perspective, naturalistic stimuli promise improved subject
compliance regarding wakefulness and head motion due to minimal instruction
requirements (e.g. no fixation of eye gaze), task demands (no task except
enjoying the movie or auditory story).
%
This is especially the case for young children \citep{vanderwal2015inscapes},
and possibly psychiatric \citep{eickhoff2020towards} or elderly persons
resulting in increased data quality.
%
Since movies and spoke narratives are interesting and easy-to-follow stimuli
that are produced to be engaging and immersive, they also promise to put
participants at ease in the otherwise claustrophobic, uncomfortable, and noisy
fMRI scanner.
%
Lastly, spoken narratives are also appropriate for visually impaired persons
suffering from diminished or lack of eyesight.


\paragraph{contra: challenging data analysis; annotations needed}

%
However, the majority of naturalistic stimuli that have been used in
neuroscience have originally been designed for commercial purposes and not to
conduct research.
%
The temporal structure of stimulus features embedded in a naturalistic stimulus
are fixed and thus reproducible but initially not explicitly known.
%
Modeling brain activity correlating with the stimulus features embedded in the
time course is challenging \citep{saarimaki2021naturalistic, simony2020analysis}
because such models, like a traditional \ac{glm}, rely on the stimulus features
being annotated.
%
The lack of extensive annotations has led to a ``usage bottleneck''
\citep{aliko2020naturalistic} and might be the main reason why explicit models
of task or stimulus are ``notoriously'' \citep{richard2019fast}, if not
``prohibitively'' \citep{nastase2019measuring} difficult to construct.

%
Additionally if one wants to map only one domain of brain functions, a
full-length movie is an inadequate replacement for a traditional localizer
lasting about 15 minutes.

\todo[inline]{ask Michael: the modeled time course might not ``perfectly''
represent time-course of the correlating brain process}


\section{Predicting subject-specific topography from a reference group}

\todo[inline]{Write text in SRM chapter; put a shorter version in here}

\todo[inline]{necessary terms should be introduced here already}

\todo[inline]{needs to be a preparation for ``Aims of thesis''}

\todo[inline]{check \citep{poldrack2019establishment, yarkoni2017choosing}}

%
For that reason, this dissertation --- following a ``leave-one-out
cross-validation`` procedure --- also explores whether just a part of a movie or
auditory narrative can serve as a ``diagnostic'' run in order to estimate the
location of the \ac{ppa} in an ``unknown'' subject's brain (i.e. a
\textit{left-out subject} to test the prediction performance of our model) from
a reference group (i.e. the study subjects providing the data to train our
model).

\todo[inline]{add transition to alignment inferno}


\subsection{Anatomical alignment}


\subsection{Functional alignment}


\subsubsection{Hyperalignment}


\subsubsection{Shared response model}


\subsubsection{interim summary}
% interim summary
In summary, naturalistic stimuli ``impose a meaningful timecourse across
subjects while still allowing for individual variation in brain activity and
behavioral responses, and lend themselves to a broader set of analyses than
either pure rest or pure event-related task designs'' \citep{finn2017can}.

\todo[inline]{add short sentence repeating challenging annotations / analyses}


\section{Aims of thesis}

\todo[inline]{shift some parts into excursions (``this dissertation'') above}

%% At the end of the introduction, add a subchapter on the Aims of Thesis in
%% which you describe the research question and the objectives of your work on
%% a maximum of two pages

%
Previous work on a group-average level has shown that it is possible to combine
\ac{bold} \ac{fmri} with naturalistic stimulation in order to localize brain
areas whose increased hemodynamic activity is correlated with psychological
processes \citep{bartels2004mapping}.
%
This dissertation explores --- while following the principles of open,
transparent, and reproducible science --- whether a movie and the movie's
audio-description that was produced for an visually impaired audience could, in
principle, substitute an established localizer paradigm.
%
Naturalistic stimuli would have substantial advantages (e.g. task demands
compliance, and data quality) over simplified stimuli presently used in
block-design localizer paradigms.

\todo[inline]{add focus on PPA here}

%
Moreover, the time-courses of rich, full-length naturalistic stimuli are
correlating with a variety of different brain functions ranging from low-level
perception (e.g., luminance) to high-level cognition (e.g., social cognition).
%
Thus, a naturalistic stimulus could eventually replace multiple dedicated
localizers in the future to provide a more comprehensive diagnostic routine.
%
However, a two-hour paradigm would be inefficient plus unsuitable for a clinical
population.
%
For that reason, this dissertation also explores whether just a part of a movie
or auditory narrative can be used to estimate individual topography of a
left-out subject from a data of reference group.


\subsection{Open, transparent, and reproducible science}

% reproducibility crisis
Over the last decade, there has been a growing awareness that results of
scientific publications are not reproducible or general scientific findings are
not replicable letting some authors speak of a ``reproducibility crisis'' or
``replication crisis'' in the sciences \citep{baker2016reproducibility,
plesser2018reproducibility, stupple2019reproducibility, nosek2022replicability}.
% reproducibility: definition
``A study is reproducible if all of the code and data used to generate the
numbers and figures in the paper are available and exactly produce the published
results'' \citep{leek2017most}.
% replicability: definition
A study is replicable if the same analysis of an equivalent experiment's data
leads to consistent results \citep{dubois2016building, leek2017most}.
%
Hence, on a metalevel, this dissertation aims to meet both the requirements of
open, accessible, shared, and transparent science \citep{watson2015will,
fecher2014open} as well as the requirements of a reproducible and replicable
research project:
%
the dissertation follows guidelines and best practices for a) coding and
scientific computing \citep{wilson2014best}, b) procedures and data analyses
\citep{nichols2017best, poldrack2017scanning, poldrack2019establishment}, and c)
sharing code, created data, and results \citep{eglen2017toward, nichols2017best,
pernet2015improving}.


\paragraph{input data}

% open data \citep{eglen2017toward}
First, my work is build on top of publicly and freely available \ac{fmri} data
that are part of the \textit{studyforrest} project
(\href{www.studyforrest.org}{studyforrest.org}).
%
The studyforrest project is an open science project that aims to provide a
versatile resource for investigating human brain function under quasi-natural
conditions.
%
The core of this dataset are two-hour long \ac{bold} \ac{fmri} scans of
participants watching the movie Forrest Gump \citep{ForrestGumpMovie} and
listening to the movie's audio-description that was created for a visually
impaired audience by adding a narrator to the movie's audio track.
%
Since its first publication in 2014 \citep{hanke2014audiomovie}, the
studyforrest project has served as a resource of raw (and preprocessed) data for
international working groups to conduct and publish independent, peer-reviewed
research (s.
\href{www.studyforrest.org/publications.html}{studyforrest.org/publications.html}).
%
The stimulus annotations that have been created over the course of the
dissertation are version-controlled and published in a standardized file format
\citep{haeusler2021speechanno}, and therefore contribute to the studyforrest
project as a resource for the scientific community.


\paragraph{code, analyses, output}

Further, all code is shared to improve reproducibility current results and to
facilitate replicability of findings on other datasets.
% automatization
Therefore, data analyses pipelines designed in a way that enables automated
processing.
%
Analyses pipelines are not be implemented in commercial, proprietary software
but in freely available and, if possible, open-source software.
% which tools to choose why?
Among potential software packages, we chose the tools that offer the most solid
documentation, and basis of developers and maintainers to ensure long-term
support.
% my code
Custom code written by myself is written in open-source programming languages
(Python and Bash), is version-controlled, documented, and released publicly and
freely accessible.
%
All input data, custom code, analysis steps and output data are accessible in
standardized \textit{DataLad} (\href{www.datalad.org}{datalad.org}) datasets.
Since DataLad provides a free and open-source software solution that manages
provenience, distribution, and version-control of code and data
\citep{halchenko2021datalad}, all executed steps from downloading the input data
to visualizing the results can be rerun to check and validate the dissertation's
results.


\paragraph{publications}
% open-access publishing
Last, because ``nature abhors a paywall'' \citep{dupre2020nature}, publications
describing generated data, reasoning of methodological choices, analysis steps,
and results are published in open-access journals.
% neurovault
Unthresholded statistical maps of all computed statistical $t$-contrasts are
additionally published at Neurovault
(\href{https://neurovault.org/}{neurovault.org}).


\subsection{Specific objectives and hypotheses}

\todo[inline]{This section is waaaaay longer than in other dissertations; here,
it is not just aims and hypotheses but also (still) an overview}

\todo[inline]{Here should be some random text between section titles}


\subsubsection{A studyforrest extension, an annotation of spoken language in the
German dubbed movie ``Forrest Gump'' and its audio-description}


\paragraph{Intro}

\todo[inline]{revise; cf. cons of naturalistic stimuli above}

% traditionally
``In contrast to stimuli designed to trigger a perceptual process of interest,
while controlling for confounding variables (e.g., color and luminance),
% our approach: annotation and regressors
naturalistic stimuli have a fixed but initially unknown temporal structure of
stimulus features of interest, as well as an equally unknown confound
structure'' \citep{haeusler2021speechanno}.

%
In order to build a model of hemodynamic activity, a researcher needs to a)
inform the model about the time courses of events that are assumed to correlate
with psychological and cognitive processes of interest, and b) model hemodynamic
responses (e.g., as nuisance regressors in a \ac{glm}) that are correlated with
the event structure of potentially confounding variables.

%
Hence, I created an extensive annotation of speech occurring in the movie and
the audio-description.
%
In general, the current annotation of speech extends the studyforrest dataset
as a public resource for independent research, and complements formerly
published annotations of portrayed emotions \citep{labs2015portrayed}, perceived
emotions \citep{lettieri2019emotionotopy}, as well as cuts and locations
depicted in the movie \citep{haeusler2016cutanno}.
%
Specifically for this dissertation, the annotation of speech serves as the basis
for the targeted analyses in study 2.


\paragraph{What I did}
% cloud-based annotations
State of the art cloud-based speech-to-text services (Google Cloud, IBM Watson)
fell short to deliver satisfactory scaffolds of an annotation.
% procedure
Consequently, I revised a preliminary transcription of language spoken by
actors, actresses, and the narrator, and submitted it to a forced aligner
(\href{https://github.com/MontrealCorpusTools/Montreal-Forced-Aligner}{Montreal
Forced Aligner} v1.0.1 \citep{mcauliffe2017montreal}) that identified the exact
onset and offset of each word and phoneme. The forced aligner's output was
extensively cleaned and extended both algorithmically as well as manually.


\paragraph{Annotation content}
%
Consequently as planned, the annotation's content substantially exceeds the
groundwork that was needed to conduct the study reported in Chapter 3 (s.
\citep{haeusler2022processing}), and contributes to the studyforrest project for
public use:
% content
the annotation provides ``information about the exact timing of each of the more
than 2500 spoken sentences, 16000 words (including 202 non-speech
vocalizations), 66000 phonemes, and their corresponding speaker''
\citep{haeusler2021speechanno}.
%
For every word, the annotation additionally provides ``lemmatization, a simple
part-of-speech-tagging (15 grammatical categories), a detailed part-of-speech
tagging (43 grammatical categories), syntactic dependencies, and a semantic
analysis based on word embedding which represents each word in a 300-dimensional
semantic space'' \citep{haeusler2021speechanno}.


\paragraph{Validation}

% what we did
We created a canonical \ac{glm} based on information drawn from the annotation
to model hemodynamic brain activity to validate the dataset's quality.
% results in line with previous studies
As hypothesized, results revealed statistically significant increased
hemodynamic activity in a bilateral cortical network including temporal,
parietal and frontal regions related to processing spoken language.
% results replicate
Our exploratory analysis replicates results of studies that employed
tightly-controlled stimuli (s. \citep{friederici2011brain,
hickok2007cortical,price2012twentyyears} for reviews), and studies that employed
data-driven methods to analyze \ac{fmri} data from auditory naturalistic stimuli
\citep{honey2012not, lerner2011topographic, silbert2014coupled}
% "logic" inference
Consequently, results suggest that the annotation's content and quality enables
independent ``researchers to model hemodynamic brain responses that correlate
with a variety of aspects of spoken language'' \citep{haeusler2021speechanno}
under more ecologically valid conditions.
% transition to study 2
Last, the results encouraged us to use the annotation as a groundwork to be
adapted to our specific needs in study 2.


\subsubsection{Processing of visual and non-visual naturalistic spatial
information in the ``parahippocampal place area''}

\todo[inline]{following is (deliberately) still too long and ``bumpy''}

\todo[inline]{will be cleaned iteratively during next 1000 iterations}


\paragraph{Intro}

% study in one sentence
In this study, we investigated whether it is possible to localize a visual area,
namely the \ac{ppa}, usually identified via contrasting blocks of pictures in a
task-based functional localizer, via two naturalistic stimuli.
% hypo a: group
We hypothesized that a conventional model-based statistical analysis would
reveal increased hemodynamic activity in medial temporal regions that were
functionally identified as the PPA by a previously published study
\citep{sengupta2016extension} that performed a functional localizer in the same
set of participants.
% hypo b: individuals
``We hypothesized further that a purely auditory stimulus could, in principle,
localize the PPA as an example of a ``visual area'' in individual persons,
%
and may offer an alternative paradigm to assess brain functions in visually
impaired individuals'' \citep{haeusler2022processing}.


\paragraph{Literature review}

% #1 previous studies
``Several studies have reported increased hemodynamic activity in the PPA
attributed to the processing of scene-related, spatial information, for example,
when participants were watching static pictures of landscapes compared to
pictures of faces or objects \citep{epstein1998ppa,
epstein1999parahippocampal}'' \citep{haeusler2022processing}.
% auditory semantics unclear
However, reports regarding the correlates of processing spatial information in
verbal stimulation are less clear \citep{aziz2008modulation}.

% rationale
In case the stimulus classes that are correlated with hemodynamic responses
during traditional paradigms are represented by [sampled by?] stimulus features
embedded in the movie ``Forrest Gump'' and its audio-description, it should also
be possible to annotate these features and model feature-related hemodynamic
responses during naturalistic stimulation in order to localize the \ac{ppa}.


\paragraph{Annotation}

% AV stimulus
``The movie stimulus shares the stimulation in the visual domain with classical
localizer stimuli, while featuring real-life-like visual complexity and
naturalistic auditory stimulation'' \citep{haeusler2022processing}.
% AV anno
``For the analysis of the movie stimulus, we took advantage of a previously
published annotation movie cuts and the depicted location after each cut
\citep{haeusler2016cutanno}'' \citep{haeusler2022processing}.

% AD stimulus
``The audio-description maintains the naturalistic nature of the movie stimulus,
but limited to the auditory domain'' \citep{haeusler2022processing}.
% AD annotation
For the analysis of the audio-description stimulus, we extended the published
annotation of speech that was created in study 1 \citep{haeusler2021speechanno}.
%
Nouns that the narrator uses to describe the movie's absent visual content were
semantically categorized in case the narrator used them to describe locations,
buildings, rooms, faces or bodies, and so on.

% statistical analysis
We then performed a canonical model-based, mass-univariate analysis supposed to
closely resemble the previously published data analysis of a traditional
block-design localizer performed with the same set of participants
\citep{sengupta2016extension}.
% events & contrasts
Hemodynamic responses for each stimulus were modeled based on events that should
correlate with the perception of spatial information and contrasted with events
that should correlate with non-spatial perception to a lesser degree, not at all
\citep{haeusler2022processing}.


\paragraph{Results}

% group AV
On a group-average level, significantly increased activation correlating with
visual spatial information occurring in the movie is overlapping with a
traditionally localized \ac{ppa} but also extending into earlier visual
cortices.
% group AD
``Activation correlating with semantic spatial information occurring in the
audio-description is more restricted to the anterior \ac{ppa}''
\citep{haeusler2022processing}.
% individual AD
``On an individual level, we find significant bilateral activity in the PPA of
nine individuals and unilateral activity in one individual''
\citep{haeusler2022processing}.
% individual level
``Bilateral clusters in 9 of 14 participants (of which \texttt{sub-04} shows
only a right-lateralized PPA in the block-design localizer results), and a
unilateral significant cluster in one participant, indicate that the
group-average results are representative for the majority of individual
participants'' \citep{haeusler2022processing}.


\paragraph{Discussion}

% generalization
Findings demonstrate that increased activation in the PPA during the perception
of static pictures generalizes to the perception of spatial information embedded
in a movie or a purely auditory stimulus \citep{haeusler2022processing}.
% model-based
Our results add further evidence that a model-driven GLM analysis based on
annotations can be applied to a naturalistic paradigm to localize functional
areas correlating with pre-defined perceptual processes.
%
Same but different: The present results are evidence that a functionally defined
region, such as the PPA, can be localized using a model-driven \ac{glm} analysis
that is based on a naturalistic stimulus' annotated temporal structure with
respect to a particular hypothesized cognitive or perceptual function.

% auditory localizer
``Finally, the presented evidence on the in-principle suitability of a naturally
engaging, purely auditory paradigm for localizing the PPA may offer a path to
the development of diagnostic procedures more suitable for individuals with
visual impairments or conditions like nystagmus''
\citep{haeusler2022processing}.


\paragraph{Transition to study 3}

\todo[inline]{I am still lost here; depends text in SRM study}


%
``Our results provide further evidence that the PPA can be divided into
functional subregions that coactivate during the perception of visual scenes''
\citep{haeusler2022processing}.

%
Results suggest that you can ``localize an auditory PPA'' but not the ``visual
PPA'' using an auditory narrative.

%
Still, there are responses in the parahippocampal cortex correlating with
spatial information.

%
Still, the AO response might be too different for ``sufficient alignment'' and
estimate the localizer results.

%
Our event structure and model is just an approximation of the ``real'' events
structure that is correlated with the response series in the parahippocampal
cortex (in AO but especially in the AV!).
%
However, exact modeling of that process is not necessary for data-driven
alignment in study 3.

%
Problem: we have a couple of subjects seem to simply not give a fuck about
spatial information in audio-description


\subsubsection{DataLad: distributed system for joint management of code, data,
and their relationship}

\todo[inline]{well, does not really fit in}


\subsubsection{Using varying amount of data from naturalistic stimulation for
functional alignment to predict results from a task-based functional localizer
paradigm}

\todo[inline]{following part is a draft}

\todo[inline]{will probably change a lot after text in study 3 is written}

% intro
\textit{Intro:} In order to map perceptual or cognitive functions onto the brain
anatomy of study participants, researchers usually conduct dedicated experiments
\textit{functional localizers} often accompanied with a task.

% problem
\textit{Current approach \& problem:} Nevertheless, the approach ``one paradigm
to map one domain of brain functions'' becomes impractical if a variety of
domains is supposed to be mapped in a time-efficient manner:
%
a clinical application must aim to minimize the scan time/cost but still provide
valid results

% therefore
\textit{Therefore:} In the current study, we explore a method and the quantity
of data needed to predict individual functional topographies by projecting
results of a localizer experiment (statistical $Z$-maps) from a reference group
into the brain anatomy of individual participants.

%
fMRI data acquired from naturalistic stimulation are used to align a subject's
voxel space with a common representational/functional reference space.
%
We test whether functional alignment can be achieved with a task-free, natural
stimulation ``calibration'' scan that requires no more acquisition time than a
conventional localizer paradigm.
%
Once aligned, the brain activity of the reference group can be used to predict
the activity of another subject.

%
We will estimate the trade-off between diagnostic quality and required effective
scan time by progressively reducing the duration of input BOLD fMRI data and
comparing the results of the reduced model to the reference computed from the
full length scan.


\textit{Method:}
% data
During functional magnetic resonance imaging (fMRI), participants ($N=14$) took
part in a task-based, block-design visual localizer and two naturalistic stimuli
paradigms: an audio-visual movie and the movie's audio-description, both
paradigms free of any task.

% cms creation
Based on these response time series, we first created a common model space
employing a shared response model \citep{chen2015reduced} following
leave-one-subject out cross-validation, a process that also computed
transformation matrices for the subjects that provided the data for the creation
of the model space.

% alignment
Then, we aligned left-out subjects with the common [via Procusted
transformation?] to derive transformation matrices for the left-out-subjects by
also varying the quantity of functional response time series to perform the
alignment.

% prediction
Lastly, the acquired transformation matrices were used to project the functional
topographies from the anatomy of the reference group into the common model
space, and from the common model space into the anatomy of the left-out
subjects.

%
Results of the conventional localizer are projected into the common space.
%
The transpose of subject-specific transformation matrices is used to project
functional properties of the common reference into that individual's voxel
space.
%
Then, data are projected into the voxel space of the left-out individual for
comparison with the localizer results for that individual.


% stimulus length?
[We assessed the relationship between length of naturalistic stimulation used
for a \textit{partial functional alignment} and the performance of predicting
empirical $Z$-maps.] by...

%
\textit{Results} suggest that ``a subject's idiosyncratic functional topography
can be estimated with high fidelity from that subject's fMRI data obtained while
watching a naturalistic movie using hyperalignment to project other subjects’
localizer data into that subject's idiosyncratic cortical anatomy''
\citep{jiahui2020predicting}.

%
\textit{Discussion}: stimulus length: 15-30 min vs. 2h;  results of auditory
stimulus to predict visual localizer are ``modest''.

%
\textit{Conclusion}: ``These findings lay the foundation for developing an
efficient tool for mapping functional topographies for a wide range of
perceptual and cognitive functions in new subjects based only on fMRI data
collected while watching an engaging, naturalistic stimulus and other subjects'
localizer data from a normative sample'' \citep{jiahui2020predicting}.


\paragraph{just some notes}

\todo[inline]{other contrasts that were tested over the course of the
dissertation: phonemes, grammatical tags, prosody, sex}
