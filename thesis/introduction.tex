\section{Overview}
%% The introduction begins with an overview of the topic

\todo[inline]{some sentences are kinda quotes from other paper; e.g. Dubois?}

\todo[inline]{other contrasts: phonemes, grammatical tags, prosody, sex}

\todo[inline]{sections are supposed to be numbered}

%
Brain imaging with \ac{bold} \ac{fmri} has been used extensively for almost
three decades to investigate brain function.
%
Typical analysis procedures average (voxel-wise) data of at least 10-15 subjects
to improve the \ac{snr}.
%
Consequently, these studies do not characterize brain function at the level of
an individual.
%
It is plausible to assume that models of brain functions that are based on a
lowest common denominator approach only capture a fraction of individual
functional brain properties.
%
Therefore, those models are an incomplete foundation for an investigation of
inter-individual differences.
%
However, characterization of individual brain function is by far the most
important application of BOLD fMRI in a clinical context.
%
For example, for pre-surgical screening, or diagnosis of brain function in
health and disease.
%
The goal of the proposed project is to pave the way towards adopting an approach
to the investigation of individual brain function that is based on individual
differences with respect to large normative samples --- a proven strategy that
has been standard in psychological diagnostics and other clinical research for a
long time.\todo{give examples}


\section{Introductory remarks}
%% introductory remarks describe the scientific background of the work as
%% precisely as possible. Cite the most important publications and avoid
%% extensive literature reviews.


\subsection{State of research}


\subsubsection{Functional localization}

\todo[inline]{we don't do speech lateralization anymore; imo the neurosurgery
thing should not be mentioned in the intro anymore; better come up with it in
the general discussion}

%
The most frequently employed paradigm for characterizing individual brains with
BOLD fMRI are so called functional localizers.
%
Functional localizers aim at isolating and localizing brain activity correlated
with specific perceptual processes (e.g. different object categories;
\citet{kanwisher1997ffa}) or cognitive processes (e.g. theory of mind;
\citet{spunt2014validating}).
%
Typically, localizers are dedicated measurements that are used to define
individual \acp{roi}.
%
For example to improve the statistical power of the main experiment's analysis
or to locate brain functions prior to neurosurgery.
% side-effects of neurosurgery
Surgical procedures might impact the post-operative quality of life so much
(e.g. concerning cognitive control or speech production) that it potentially
outweighs the therapeutic benefits.
%
The challenge is to precisely localize relevant brain areas with limited
resources (time, availability and applicability of diagnostic measures for an
individual patient) in order to correctly predict the impact of the planned
procedure.
%
Importantly, functional localizers, despite being tuned for detection power,
quickly become inefficient if one wants to map many different processes in a
limited amount of time.
%
One example of a time-efficient multi-functional localizer for reading, language
comprehension, calculation, motor response, and basic retinotopy was developed
by \citep{pinel2007fast}).\todo{Thirion's work?}
%
It employs a range of dedicated stimuli and specific tasks participants have to
perform in a 5-minute routine.
%
Diagnostic quality of such paradigms relies heavily on participants' compliance
and comprehension of the task instructions, a criterion that can be difficult to
meet in certain target populations, such as patients with dementia.


\subsubsection{Anatomical alignment (in order to predict)}

e.g. \citep{weiner2018defining}


\subsubsection{Functional alignment (in order to predict)}

\paragraph{Functional alignment in general}
%
An alternative approach to individual localization has been proposed by
\citet{haxby2011common}.
%
They (and e.g. \citet{jiahui2019predicting}) showed that target brain areas in
ventral temporal cortex can be reliably predicted from the result of dedicated
localizer scans for other individuals.
%
The key difference of this approach is to rely on similarity of representational
geometry of brain activity patterns for aligning individual brains into a group
space.
%
This is contrary to presently dominating approaches in neuroimaging group
analyses that rely on topological constraints defined by a anatomical reference
space (e.g. the MNI152 template brain).


\paragraph{Hyperalignment}
%
\citet{haxby2011common} used BOLD response patterns evoked by a 2h action movie
to derive a common representational space.
%
The algorithm, namely hyperalignment, derives this representation using a
variant of Procrustes analysis and computes invertible (orthonormal)
transformations from each individual brain’s voxel-space into this common
reference space.
%
Importantly, the study also showed that an individual's \ac{ffa} or the
\ac{ppa}, can be localized precisely based on data from a reference group.
\todo{explain FFA, PPA}

%
The same authors later showed that this approach can be extended to predict
functional organization across large proportions of the cortical surface, for
example to predict the represented visual field coordinate in visual cortex
based on retinotopic mapping scans of other individuals
\citep{guntupalli2016model}.

\todo[inline]{adjust to intersecting stimulus time-series as alternative}
%
In his doctoral thesis, recently submitted to the Faculty of Natural Sciences in
Magdeburg, Falko Kaule showed that congruent time-locked BOLD responses across
subjects (i.e. all subjects watching the exact same full-length movie) as used
by Haxby and colleagues are not required to derive a valid alignment of
individuals with a common representational space \citep{kaule2017examination}.
%
Comparable prediction performance can be achieved by using \textbf{functional
connectivity patterns} (correlation of a voxel's time series with reference
regions in the same brain).
%
This finding enables, in principle, the use of different ``calibration'' scans
to determine an alignment with a common representational space, for example with
an age-appropriate stimulus, or a shortened scan time to fit into a particular
clinical schedule.
%
Once a valid alignment is established, known functional properties of a
(normative) reference, derived from extensive scans and analysis of other
subjects, can then be projected into the respective individual voxel space (s.
Fig. 1 in \citep{nishimoto2016lining}).


\paragraph{Shared response model}

\citep{chen2015reduced}


\subsubsection{Naturalistic stimuli in neuroscience}

\todo[inline]{cf. PuG talk}

\paragraph{Definition}
%
Naturalistic stimuli are ``a class of stimuli that aim to evoke more
naturalistic patterns of neural responses than traditional controlled artificial
stimuli. Naturalistic paradigms are typically complex and dynamic, and longer
in duration than many conventional stimuli.'' \citep{vanderwal2019movies}.

%
They ``impose a meaningful timecourse across subjects while still allowing for
individual variation in brain activity and behavioral responses, and lend
themselves to a broader set of analyses than either pure rest or pure
event-related task designs.'' \citep{finn2017can}

%
Numerous studies have shown that watching a movie leads to correlated
time-locked brain responses across subjects in many brain regions, and to
synchronized eye movements \citep{hasson2010reliability, lankinen2014isc-meg}.
%
This can be attributed to the way professional movies are shot and edited in
order to intentionally manipulate the viewers' attentional focus and mental
states \citep{brown2012cinematography, dancyger2011film-technique}.


\paragraph{Higher validity}
%
Findings suggest that naturalistic stimuli offer a higher
ecological validity \citep{hasson2004intersubject} because they better mimic
statistics in our natural dynamic environment.
%
Further they offer a higher external validity \citep{westfall2016fixing})
because they sample the stimulus space ``better''.\todo{iykwim}
%
While the hyperalignment procedure can also be applied to fMRI data from more
focused stimulation paradigms with simplified stimuli, the transformations for
functional alignment have greatly diminished general validity
\citep{haxby2011common}, presumably because such experiments sample a sparser
range of brain states \citep{guntupalli2016model}.

%
Lastly, increased validity of derived transformation for functional alignment by
sampling a more diverse set of mental states that reflect (confound) statistics
of the natural environment, and enable investigation of the acquired data for a
variety of research questions (e.g. visual or auditory perception, spatial
cognition; emotion; music, speech or social perception)


\paragraph{Better compliance}
%
The to be diagnosed individuals were scanned while they were watching a movie
without any explicit task or instruction other than just to ``enjoy the movie''.
%
Improved subject compliance and compatibility due to minimal instruction
requirements and task demands (e.g., no fixation requirement, hence more
appropriate for elderly or visually impaired persons).
%
Improved data quality, as an interesting and easy-to-follow stimulus is more
capable of putting a participant at ease in the otherwise claustrophobic,
uncomfortable and noisy fMRI scanner.


\section{Aims of thesis}
%% At the end of the introduction, add a subchapter on the Aims of Thesis in
%% which you describe the research question and the objectives of your work on
%% a maximum of two pages


\subsection{Overview of aims}
%
Previous work has shown that it is, in principle, possible to combine BOLD fMRI
with a rich naturalistic stimulus for the purpose of localizing areas associated
with particular brain functions \citep{bartels2004mapping}.
%
This can be approached by either modeling specific stimulus features, or by
using the high-dimensional nature of such a stimulus to derive a common
representational space for aligning functional properties of cortex.
%
To my knowledge, however, there has been no study aimed at replacing one or more
established localizer paradigms for the purpose of improving routine (clinical)
diagnostics by means of natural stimulation.
%
A (part of a) movie would have substantial advantages (e.g.  task demands,
compliance, and data quality) over simplified stimuli presently used in
localizer paradigms.
%
Improved efficiency by replacing multiple dedicated localizer scans with a
single natural stimulation potentially enables more comprehensive diagnostics in
a similar or even less amount of time.

%
For this purpose, I want to evaluate two strategies:

\begin{itemize}
    \item direct modeling of a natural stimulus and
    \item prediction via functional alignment to a reference population
\end{itemize}

\subsubsection{Auditory stimulus to localize visual area}




\subsubsection{Reproducibility}

\todo{transparency, reproducibility, sharing, connection to
\citep{halchenko2021datalad}}
%
All required algorithms and analyses will be implemented in a way that enables
automated processing.
%
This will facilitate documentation and efficient processing of large number of
datasets.
%
Implementations will be based on open-source software tools to guarantee a
maximum level of reproducibility, and relative ease of long-term maintenance
\citep{eglen2017toward}.
%
Once available, the developed stimulation paradigm and analysis implementations
will be integrated in the upcoming CBBS imaging platform that aim at providing
reusable software components for neuroimaging research to all groups in
Magdeburg.
%
The goal is to provide researchers with efficient, validated, ready to use
solutions (stimulation, MR sequence configuration, analysis software) for
functional localization that can be incorporated into their study protocols.


\subsubsection{Study 1: Annotation of audio-description}

\paragraph{studyforrest}
%
The studyforrest project is an open science project that aims at providing a
versatile resource for investigating human brain function under quasi-natural
conditions.
%
The core of this dataset are two hour long BOLD fMRI scans of participants
watching the movie Forrest Gump (and also listening to a version for the blind
in another scan of equal length).
%
Since its first publication in 2014 \citep{hanke2014audiomovie}, this resources
has led to eight independent studies of international research groups outside
Magdeburg that were published in peer-reviewed journals
(http://studyforrest.org).
%
Based on my knowledge about cinematographic editing techniques, I annotated the
~870 movie cuts with respect to depicted major locations, scene settings,
within-scene rooms and perspectives \citep{haeusler2016cutanno}.
%
The annotation served as prerequisite to investigate cognitive functions, such
as spatial reorientation, perspective taking, and memory retrieval for known
spatial layouts in terms of their occurrence in the movie stimulus.

For 15 participants in the studyforrest datasets two different full-length movie
scans are readily available: one with the normal audio-visual movie
\citep{hanke2016simultaneous}, and a second one with an audio-only movie variant
(originally produced for a visually impaired audience) that is time-locked to
the audio-visual version \citep{hanke2014audiomovie}.

\paragraph{actual goal: why we need an annotation of speech}
%
This investigation will also provide insight if it is feasible to produce an
audio-only localizer paradigm as an alternative for studies where no visual
stimulation is feasible or desired.
%
Previous results show that there are significant voxel-wise BOLD response time
series correlations between the two datasets in brain areas associated with
speech and story processing (s. Figure 3 in \citep{hanke2016simultaneous}).

% from speech-anno talk
First, movies are designed to entertain the audience and not to
conduct research.  Variables or let’s call them stimulus features might be
confounded.  You can try to control them statistically. but then you might run
into another problem: Your features of interest might be highly correlated, so
controlling them statistically is impossible.  This is way data-driven methods
are often used to analyze the data.  From data-driven approaches, we gained a
lot of knowledge, but data-driven approaches essentially fall short in case you
want to correlate discovered brain activation patterns with psychological
processes.  Unfortunately, some people - especially researchers that use
data-driven methods – claim that model-driven methods just not suitable.  At the
moment, my favorite quote in this regard is that model-driven methods are
difficult to employ because they are based on annotations that are
"prohibitively difficult" to create.
%
``ISC analyses—because they do not require an explicit model of the task or
stimulus—are particularly useful for naturalistic experimental paradigms, where
constructing such a model may be prohibitively difficult.  Relative to
traditional fMRI experiments that typically use highly controlled stimuli,
naturalistic stimuli are more ecologically valid (Zaki and Ochsner, 2009; Hasson
and Honey, 2012; Adolphs et al., 2016; Hamilton and Huth, 2018), convey rich
perceptual and semantic information (Bartels and Zeki, 2004; Huth et al., 2012,
2016) and more fully sample neu- ral representational space (Haxby et al., 2011,
2014). Recent work (Vanderwal et al., 2015) also suggests that naturalistic
stimuli. \citep{nastase2019measuring}


\subsubsection{Study 2: task-based vs. visual vs. auditory}
%
Compare task-free localizer to dedicated task-based speech localizer:
%
the diagnostic performance of the task-free movie fMRI
recording will be compared to the results of the localizer paradigm
%
1) quantification of the similarity of identified speech-related brain areas
(e.g. Jaccard index or Dice coefficient).
%
However, it is likely that a rich natural stimulation evokes more widespread
network activity, compared to a simplified stimulus that additionally requires
the subject to perform a task.


\subsubsection{Study 3: Functional alignment \& stimulus length}
%
With the results from WP1a (specificity) and WP1b (reliability) a specific
diagnostic contrasts can be selected.
%
However, all analyses up to this point will have been performed on 2h-long
scans, but any clinical application must aim to minimize the scan time/cost.
%
In this work package I will estimate the trade-off between diagnostic quality
and required effective scan time by progressively reducing the duration of input
BOLD fMRI data and comparing the results of the reduced model to the reference
computed from the full length scan.
%
Based on this estimate, and the corresponding number of critical events and
their timing a shortened movie stimulus can be produced (target duration about
15 min).
%
Depending on the nature of the diagnostic contrast and the required number
of events, this can either been a contiguous segment of the original movie, a
different and shortened cut, or a completely different movie stimulus that
matches the desired properties.

\paragraph{Predict ROI from a reference group}
%
fMRI data acquired from an individual will not be analyzed directly regarding a
specific cognitive function, such as speech processing, but will be used to
align that individual brain's voxel space with a common high-dimensional
representational reference space.
%
With the data from WP2a the whole-brain hyperalignment procedure -- using an
implementation of a locally constrained estimation of a sparse whole-brain
transformation (Guntupalli et al., 2016) that is already available in the lab --
will be employed to derive a common functional reference space from the BOLD
response time series for the full movie.
%
Each axis in this common space can be seen as a kind of cortical tuning
function.
%
The orthonormal transformation of an individual voxel space into this common
reference reflects the particular linear combination of voxel response time
series with respect to each common space component.
%
Using a leave-one-subject-out strategy, individual results of the conventional
speech localizer will then be projected into the common space, aggregated, and
re-projected into the voxel space of the left out individual for comparison with
the localizer results for that individual (see Figure 2).
%
Using movie-evoked brain activity, hyperalignment procedure learns subject-wise
optimal transformations of brain activity into a common representational space.
%
Once hyperaligned, the brain activity of one or more subjects can be used to
predict the activity of another subject.
%
Based on the results of WP1a and WP1b localization of speech areas will also be
performed in the common reference space directly (the stimulation time axis is
preserved in the common space), and the results will be projected into the voxel
space of the left-out individual, where they can be compared with the
localisation result derived from that subject’s movie scan.
%
Once a valid alignment is established, the inverse transformation is then used
to project functional properties of the common reference into that individual's
voxel space.


\paragraph{minimum length of stimulus; ``calibration scan''}

%
test minimum data requirement for reliable functional alignment (within-subject
test-retest
%
estimate the minimum scan time requirement for deriving a valid functional
alignment by further reducing the amount of input data for the short movie scan
of the left-out individual.
%
We test wether reliable functional alignment can be achieved with a task-free,
natural stimulation ``calibration” scan that requires no more acquisition time
than a conventional localizer paradigm.
%
The joint dataset will enable me to study whether a functional alignment to the
common reference space can be performed based on a short scan.

\todo[inline]{why didn't we used connectivity alignment in the first place?}
%
Due to the aforementioned constraints of the time series hyperalignment
regarding the number of simultaneously considered voxels, this work will be
performed using the connectivity hyperalignment from WP2c only.
