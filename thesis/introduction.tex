this text is located in introduction.tex

\section{Overview}
%% The introduction begins with an overview of the topic
\todo[inline]{some sentences are kinda quotes from other paper; e.g. ?}

\todo[inline]{andere Kontraste, aber wenig results:
                Phoneme, Audio-Tags, Prosody, Sex}
%
Introduction Brain imaging with blood oxygen level-dependent \ac{bold}
functional magnetic resonance imaging \ac{fmri} has been used extensively for
almost three decades to investigate brain function.
%
Typical analysis procedures average (voxel-wise) data of at least 10-15 subjects
to improve the \ac{snr}.
%
Consequently, these studies do not characterize brain function at the level of
an individual.
%
It is not implausible to assume that models of brain function that are based on
a lowest common denominator approach only capture a fraction of individual
functional brain properties.
%
Therefore, those models are an incomplete foundation for an investigation of
inter-individual differences.
%
However, characterization of individual brain function is by far the most
important application of BOLD fMRI in a clinical context.
%
For example, for pre-surgical screening, or diagnosis of brain function in
health and disease.
%
The goal of the proposed project is to pave the way towards adopting an approach
to the investigation of individual brain function that is based on individual
differences with respect to large normative samples --- a proven strategy that
has been standard in psychological diagnostics and other clinical research for a
long time.


\subsection{State of research}

%
The most frequently employed paradigm for characterizing individual brains with
BOLD fMRI are so called functional localizers.
%
Functional localizers aim at isolating and localizing brain activity correlated
with specific perceptual processes (e.g. faces; \citep{kanwisher1997ffa}) or
cognitive processes (e.g. theory of mind; \citep{spunt2014validating}).
%
Typically, localizers are dedicated measurements that are used to define
individual \acp{roi}, for example to improve the statistical power of the main
experiment's analysis.
% practical use
They are also used to locate brain functions prior to neurosurgery.
%
Surgical procedures might impact the post-operative quality of life so much
(e.g. concerning cognitive control or speech production) that it potentially
outweighs the therapeutic benefits.
%
The challenge is therefore to precisely localize relevant brain areas with
limited resources (time, availability and applicability of diagnostic measures
for an individual patient) in order to correctly predict the impact of the
planned procedure.
%
Importantly, functional localizers, despite being tuned for detection power,
quickly become inefficient if one wants to map many different processes in a
limited amount of time.
%
One example of a time-efficient multi-functional localizer for reading, language
comprehension, calculation, motor response, and basic retinotopy was developed
by \citep{pinel2007fast}).
%
It employs a range of dedicated stimuli and specific tasks participants have to
perform in a 5-minute routine.
%
Diagnostic quality of such paradigms relies heavily on participants' compliance
and comprehension of the task instructions, a criterion that can be difficult to
meet in certain target populations, such as patients with dementia.

\todo[inline]{explain functional alignment in general; Hyperalignment vs. SRM}
%
An alternative approach to individual localization has been proposed by
\citet{haxby2011common}.
%
They (and e.g. \citet{jiahui2019predicting}) showed that target brain areas in
ventral temporal cortex can be reliably predicted from the result of dedicated
localizer scans for other individuals.
%
The key difference of this approach is to rely on similarity of representational
geometry of brain activity patterns for aligning individual brains into a group
space.
%
This is contrary to presently dominating approaches in neuroimaging group
analyses that rely on topological constraints defined by a anatomical reference
space (e.g. the MNI152 template brain).
%
\citet{haxby2011common} used BOLD response patterns evoked by a 2h action movie
to derive a common representational space.
%
The algorithm, namely hyperalignment, derives this representation using a
variant of Procrustes analysis and computes invertible (orthonormal)
transformations from each individual brain’s voxel-space into this common
reference space.
%
Importantly, the study also showed that an individual's \ac{ffa} or the
\ac{ppa}, can be localized precisely based on data from a reference group.
\todo{explain FFA, PPA}
%
The to be diagnosed individuals were scanned while they were watching a movie
without any explicit task or instruction other than just to ``enjoy the movie''.
%
The same authors later showed that this approach can be extended to predict
functional organization across large proportions of the cortical surface, for
example to predict the represented visual field coordinate in visual cortex
based on retinotopic mapping scans of other individuals
\citep{guntupalli2016model}.
%
In his doctoral thesis, recently submitted to the Faculty of Natural Sciences in
Magdeburg, Falko Kaule showed that congruent time-locked BOLD responses across
subjects (i.e. all subjects watching the exact same full-length movie) as used
by Haxby and colleagues are not required to derive a valid alignment of
individuals with a common representational space \citep{kaule2017examination}.
%
Comparable prediction performance can be achieved by using \textbf{functional
connectivity patterns} (correlation of a voxel's time series with reference
regions in the same brain).
%
This finding enables, in principle, the use of different ``calibration'' scans
to determine an alignment with a common representational space, for example with
an age-appropriate stimulus, or a shortened scan time to fit into a particular
clinical schedule.
%
Once a valid alignment is established, known functional properties of a
(normative) reference, derived from extensive scans and analysis of other
subjects, can then be projected into the respective individual voxel space (s.
Fig. 1 in \citep{nishimoto2016lining}).

%
While the hyperalignment procedure can also be applied to fMRI data from more
focused stimulation paradigms with simplified stimuli, the transformations for
functional alignment have greatly diminished general validity
\citep{haxby2011common}, presumably because such experiments sample a sparser
range of brain states \citep{guntupalli2016model}.
%
Taken together, these findings suggest that a movie offers both a higher general
validity and, because it better mimics statistics in our natural dynamic
environment, higher ecological validity. Numerous studies have shown that
watching a movie leads to correlated time-locked brain responses across subjects
in many brain regions, and to synchronized eye movements
\citep{hasson2010reliability, lankinen2014isc-meg}.
%
This can be attributed to the way professional movies are shot and edited in
order to intentionally manipulate the viewers' attentional focus and mental
states \citep{brown2012cinematography, dancyger2011film-technique}.
%
In general, movie stimuli promise several advantages over simplified stimuli:

\begin{itemize}
    \item improved subject compliance and compatibility due to
            minimal instruction requirements and task demands (e.g., no fixation
            requirement, hence more appropriate for elderly or visually impaired
        persons).
    \item improved data quality, as an interesting and
            easy-to-follow stimulus is more capable of putting a participant at
            ease in the otherwise claustrophobic, uncomfortable and noisy fMRI
        scanner.
    \item increased validity of derived transformation for
            functional alignment by sampling a more diverse set of mental states
            that reflect (confound) statistics of the natural environment, and
            enable investigation of the acquired data for a variety of research
            questions (e.g. visual or auditory perception, spatial cognition;
            emotion; music, speech or social perception)
\end{itemize}


\subsection{Own work}

\todo[inline]{following part can probably be deleted; or restructured towards
open science and reproducibility, fuck yeah! -> Aims of study}
%
For the past three years I have been involved in the studyforrest project run by
J.-Prof. Michael Hanke at the Psychoinformatics Laboratory
(www.psychoinformatics.de) at the Institute of Psychology in Magdeburg.
%
The
studyforrest project is an open science project that aims at providing a
versatile resource for investigating human brain function under quasi-natural
conditions.
%
The core of this dataset are two hour long BOLD fMRI scans of
participants watching the movie Forrest Gump (and also listening to a version
for the blind in another scan of equal length).
%
Since its first publication in
2014 \citep{hanke2014audiomovie}, this resources has led to eight independent studies
of international research groups outside Magdeburg that were published in
peer-reviewed journals (http://studyforrest.org).
%
As part of my work as a
student assistant in the psychoinformatics lab I contributed to two major
extensions of the dataset by implementing and executing a number of fMRI data
analyses:
\begin{itemize}
    \item a high-resolution fMRI dataset of participants listening to music
    (Hanke et al., 2015), and
    \item functional localizers for visual brain areas and
        retinotopic mappings for the studyforrest participants
        \citep{sengupta2016extension}. In both cases my analysis
        implementations are published alongside the original data.
\end{itemize}

For my master's thesis (Häusler, 2016) I investigated the hypothesis that cuts
in a movie cause participants to repeatedly re-orient themselves in the virtual
spatial environment depicted by the movie[t][u], and that a movie can be used to
study aspects of spatial cognition without a dedicated task.
%
Based on my knowledge about cinematographic editing techniques, I annotated the
~870 movie cuts with respect to depicted major locations, scene settings,
within-scene rooms and perspectives.
%
The annotation served as prerequisite to investigate cognitive functions, such
as spatial reorientation, perspective taking, and memory retrieval for known
spatial layouts in terms of their occurrence in the movie stimulus.
%
This resulted in a third peer-reviewed publication (Häusler and Hanke, 2016),
and my first individual contribution to the studyforrest project.
%
I also implemented a reproducible fMRI data analysis procedure for automated
parallel processing of the movie dataset on the computer cluster of the
Institute of Psychology.
%
Empirical results of analysis contrasts comparing brain activity patterns
between different types of cuts revealed networks for spatial cognition closely
resembling the results reported by previous studies \citep{epstein1998ppa,
vanassche2016functional} using dedicated task-fMRI paradigms.
%
These results are in line with previous research \citep{bartels2004mapping}
suggesting functional segregation in the brain is preserved during complex,
lifelike stimulation.
%
Another recently completed master's thesis submitted by my co-worker Pierre Ibe
(Ibe, 2017) provides further evidence for this interpretation in the domain of
social cognition, and the processing of pleasant and unpleasant events of
``interpersonal touch'' in the movie ``Forrest Gump''.


\subsection{Work program}
%
Previous work has shown that it is, in principle, possible to combine BOLD fMRI
with a rich naturalistic stimulus for the purpose of localizing areas associated
with particular brain functions \citep{bartels2004mapping}.
%
This can be approached by either modeling specific stimulus features, or by
using the high-dimensional nature of such a stimulus to derive a common
representational space for aligning functional properties of cortex.
%
To my knowledge, however, there has been no study aimed at replacing one or more
established localizer paradigms for the purpose of improving routine (clinical)
diagnostics by means of natural stimulation.
%
As listed above, using a (short) movie would have substantial advantages (e.g.
task demands, compliance, and data quality) over simplified stimuli presently
used in localizer paradigms[ab][ac].
%
Improved efficiency by replacing multiple dedicated localizer scans with a
single natural stimulation potentially enables more comprehensive diagnostics in
a similar or even less amount of time.
%
For this purpose, I want to evaluate two strategies:

\begin{itemize}
    \item direct modeling of a natural stimulus and
    \item prediction via functional alignment to a reference population
\end{itemize}

In order to guarantee validity and cost-effectiveness of
my approach, the project will be embedded in two already ongoing lines of
research:

\begin{itemize}
    \item the aforementioned studyforrest project, and
    \item BOLD fMRI based diagnostic routine for speech lateralization (Bethmann
        et al., 2007) that is performed for all new participants of MRI studies
        at the Leibniz Institute for Neurobiology (LIN), Magdeburg.
\end{itemize}

The proposed work is split into two semi-parallel work packages (s. Figure 1).
%
The first investigates a direct modeling.
%
The second explores the suitability of functional alignment.
%
All required algorithms and analyses will be implemented in a way that enables
automated processing.
%
This will facilitate documentation and efficient processing of large number of
datasets.
%
Implementations will be based on open-source software tools to guarantee a
maximum level of reproducibility, and relative ease of long-term maintenance
\citep{eglen2017toward}.
%
Once available, the developed stimulation paradigm and analysis implementations
will be integrated in the upcoming CBBS imaging platform that aim at providing
reusable software components for neuroimaging research to all groups in
Magdeburg.
%
The goal is to provide researchers with efficient, validated, ready to use
solutions (stimulation, MR sequence configuration, analysis software) for
functional localization that can be incorporated into their study protocols.
%
Moreover, beyond the scope of this project the targeted homogenization of
acquisition procedures will further the goal of large scale data collection for
the purpose of producing a normative reference of brain function as measured by
fMRI.
%
Figure 1 provides an overview of the specific unit in both work packages.
%
Each unit is described in more detail in the following sections.
%
The schedule is designed to align similar tasks across work packages in time and
minimize the number of units that have to be processed in parallel.
%
The schedule leaves sufficient time to compensate for possible delays in the
overall progress and includes dedicated time to prepare and finalize
publications and PhD thesis within the funding period.
%
Figure 1. Gantt chart of the work package timing.
%
Components in gray correspond to work primarily performed as part of routine
data acquisitions at the LIN with very low workload for the PhD student.

\subsubsection{WP1: Speech lateralization diagnosis from
natural stimulation BOLD fMRI}

%
The goal of this work package is to develop a short routine for localizing brain
areas associated with neural processing of human speech using a task-free
natural stimulation paradigm.
%
The package is designed so that work can start immediately, because essential
fMRI data is already available.


\paragraph{a) Evaluate candidates for diagnostic contrasts}
%
Existing BOLD fMRI data from the
studyforrest project will be used to evaluate a number of (non-exclusive)
candidates for contrasting brain activity patterns evoked by different aspects
of speech perception and comprehension throughout the Forrest Gump
movie.
%
Examples include: speech vs. non-speech vocalizations (moans, screams,
etc.; more than 150 events are annotated already); active vs. passive voice;
speech vs. natural sounds (e.g. a dog barking, gun fire, non-vocal music).
%
Candidates will be evaluated regarding their ability to isolate functional areas
known to be associated with processing of speech using standard mass-univariate
GLM analyses.
%
Alternatively, manipulating the movie's speech content or its
temporal structure provide additional experimental conditions for a contrast
candidate \citep{silbert2014coupled}.


\paragraph{b) Test cross-modal reliability}

For 15 participants in the studyforrest datasets two different full-length movie
scans are readily available: one with the normal audio-visual movie
\citep{hanke2016simultaneous}, and a second one with an audio-only movie variant
(originally produced for a visually impaired audience) that is time-locked to
the audio-visual version \citep{hanke2014audiomovie}.
%
Based on these data within-subject retest reliability of localizer contrast
candidates developed in WP1a will be assessed.
%
This investigation will also
provide insight if it is feasible to produce an audio-only localizer paradigm as
an alternative for studies where no visual stimulation is feasible or desired.
%
Previous results show that there are significant voxelwise BOLD response time
series correlations between the two datasets in brain areas associated with
speech and story processing (s. Figure 3 in \citep{hanke2016simultaneous}).


\paragraph{c) Test adequate stimulus length and choosing a short(ened) movie}
%
With the results from WP1a (specificity) and WP1b (reliability) a specific
diagnostic contrasts can be selected.
%
However, all analyses up to this point will have been performed on 2h-long
scans, but any clinical application must aim to minimize the scan time/cost.
%
In this work package I will estimate the trade-off between diagnostic quality
and required effective scan time by progressively reducing the duration of input
BOLD fMRI data and comparing the results of the reduced model to the reference
computed from the full length scan.
%
Based on this estimate, and the corresponding number of critical events and
their timing a shortened movie stimulus can be produced (target duration about
15 min).
%
Depending on the nature of the diagnostic contrast and the required number
of events, this can either been a contiguous segment of the original movie, a
different and shortened cut, or a completely different movie stimulus that
matches the desired properties.

\paragraph{d) fMRI Data acquisition for the short movie}

%
Once a short movie stimulus is available, the stimulation paradigm will be
implemented.
%
In collaboration with the Non-Invasive Brain Imaging Unit at the LIN fMRI data
for this short movie will be acquired.
%
The acquisition will be integrated into an existing protocol and an ongoing
acquisition effort performed by the LIN group that is designed to diagnose the
lateralization of speech processing in the brain, using a standard task-based
localizer paradigm (Fernandez et al., 2001).
%
This integrative approach will yield comparative within-subject BOLD fMRI
datasets for the task-free movie stimulation and a dedicated established
localizer paradigm.
%
A sufficient number of participants will be required to properly assess the
performance of the new short movie stimulus (N greater/equal 50).
%
Over the past three years about 50 individuals per year have been participating
in these data acquisitions, hence 15 months are scheduled for this data
acquisition phase.

\paragraph{e) Compare task-free localizer to dedicated task-based speech
localizer}
%
Using the data from WP1d, the diagnostic performance of the task-free movie fMRI
recording will be compared to the results of the localizer paradigm currently
employed at the LIN. The approach will be two-fold:
%
1) quantification of the similarity of identified speech-related brain areas
(e.g. Jaccard index or Dice coefficient).
%
However, it is likely that a rich natural stimulation evokes more widespread
network activity, compared to a simplified stimulus that additionally requires
the subject to perform a  task as in Fernandez (2001).
%
Therefore, the performance of the two paradigms will also be compared directly
in terms of their diagnostic classification performance (i.e. left vs. right
lateralized speech processing).
%
The ability of performing the latter assessment depends, however, on a
sufficient number of individuals in each category (the majority of people
exhibit left-lateralized speech).

\paragraph{f) Evaluate short movie with respect to localization of other brain
functions}
%
Lastly, the new short movie stimulus will be assessed with respect to its
diagnostic potential for other functionally defined regions in the brain.
%
Again, data from the studyforrest project will be used to perform assessments
analog to WP1e for a range of additional individually determined regions based
on established localizer paradigms (Sengupta, et al., 2016).
%
These regions include: the \ac{ffa}, \ac{ppa}, and the extrastriate body area
(EBA) which are associated with face perception \citep{kanwisher1997ffa,
pitcher2011occipitalfacearea}, scene perception \citep{epstein1998ppa}, and the
perception of human bodies \citep{downing2001bodyarea}, respectively.
%
The actual selection will depend on the specific properties of the chosen short
movie stimulus.


\subsubsection{WP2: Predict speech processing related areas from a reference
group}
%
The goal of this work package is to develop an alternative short routine for
localizing brain areas associated with neural processing of human speech, also
using a task-free natural stimulation paradigm. In contrast to WP1 the short
movie fMRI data acquired from an individual will not be analyzed directly
regarding a specific cognitive function, such as speech processing, but will be
used to align that individual brain’s voxel space with a common high-dimensional
representational reference space. Once a valid alignment is established, the
inverse transformation is then used to project functional properties of the
common reference into that individual’s voxel space. The required fMRI data
measurements are largely identical to those in WP1, and acquisitions are
scheduled in a way that minimizes dependencies between work packages.
Importantly, WP2 does not depend on a complete success of WP1. WP2 does not rely
on the ability to achieve a localization performance based on a short movie that
is on par with that of a dedicated localizer scan. However, the short movie
stimulus developed in WP1 will be used to test whether reliable functional
alignment can be achieved with a task-free, natural stimulation “calibration”
scan that requires no more acquisition time than a conventional localizer
paradigm.


\paragraph{a) Extend the studyforrest dataset with standard speech
localizer[ar]}
%
In order to be able to evaluate the prediction performance of
brain areas associated with speech processing, the studyforrest dataset has to
be extended with BOLD fMRI scans for an established localizer paradigm (the same
as in WP1 that is already in use at the LIN; Fernandez et al., 2001). At the
same time, additional fMRI data acquisitions for the full-length movie might
have to be performed in order to compensate for original studyforrest
participants that may no longer be able or willing to volunteer for additional
measurements (N expected to be less than 10). This work will be performed in
collaboration with the Non-Invasive Brain Imaging Unit at the LIN.


\paragraph{b) Predict dedicated speech localizer from reference group data}

With the data from WP2a
the whole-brain hyperalignment procedure -- using an implementation of a
locally constrained estimation of a sparse whole-brain transformation
(Guntupalli et al., 2016) that is already available in the lab -- will be
employed to derive a common functional reference space from the BOLD response
time series for the full movie.
%
Each axis in this common space can be seen as a
kind of cortical tuning function.
%
The orthonormal transformation of an
individual voxel space into this common reference reflects the particular linear
combination of voxel response time series with respect to each common space
component.
%
Using a leave-one-subject-out strategy, individual results of the
conventional speech localizer will then be projected into the common space,
aggregated, and re-projected into the voxel space of the left out individual for
comparison with the localizer results for that individual (see Figure 2).

%
Figure 2. Schematic of Hyperalignment and an Example of its Use. (Left)
Schematic of hyperalignment.
%
Using movie-evoked brain activity, hyperalignment procedure learns subject-wise
optimal transformations of brain activity into a common representational space.
%
(Right) An example of how hyperalignment might be used.
%
Once hyperaligned, the brain activity of one or more subjects can be used to
predict the activity of another subject.
%
This allows, for example, the inference of a retinotopic map for a subject
without conducting a retinotopic mapping experiment for the subject;
%
adapted from Nishimoto et al. (2016).

%
As noted before, a prolonged complex natural stimulus might reveal additional
speech-related brain areas.
%
Based on the results of WP1a and WP1b localization of speech areas will also be
performed in the common reference space directly (the stimulation time axis is
preserved in the common space), and the results will be projected into the voxel
space of the left-out individual, where they can be compared with the
localisation result derived from that subject’s movie scan.


\paragraph{c) Test functional alignment to a reference space using a short
calibration scan}
%
With the availability of a short movie stimulus from WP1c, the studyforrest
dataset will be extended with fMRI data for this stimulus.
%
The joint dataset will enable me to study whether a functional alignment to the
common reference space produced in WP2b can be performed based on this short
scan. Two principal strategies will be tested:
%
1) Compute an additional common reference space from the fMRI scans of N-1
subjects using the new short movie scans, while keeping the actual voxel space
transformation fixed and identical to WP2b.
%
This will yield a new short common reference with the same axes as the previous
full length reference that can be used in the same way for aligning a left-out
subject as the full-length reference in WP2b.
%
2) The functional alignment on the full-length movie from WP2b will be
recomputed, but this time based on functional connectivity vectors, as opposed
to BOLD response time series.
%
This will be performed using an implementation developed by Falko Kaule as part
of his PhD project.
%
Analog connectivity vectors will then by estimated from the short movie scan of
the left-out subject and will be used for direct hyperalignment.
%
Differences in the transformations between the two approaches will be quantified
in terms of overall similarity, but also in terms of within-subject differences
regarding the weighting of individual brain areas across all common space
components.
%
This comparison is possible, because in both cases the subject’s voxel space is
identical, and the orthonormal transformation into the respective common space
preserves all variance.

%
It should be noted that the number of voxels that can be considered
simultaneously for functional BOLD response time series alignment is limited by
the number of timepoints in the calibration scan (about 300-400 voxels for a
15min scan with a 2s TR, corresponding to a local cortical neighborhood of about
1cm in diameter for a standard resolution).
%
This limitation does not exist in
this form for a functional alignment that is based on connectivity vectors.
%
The
length of these connectivity vectors is determined by the number of reference
(or seed) regions in the brain.
%
The results of this work package will
determine whether it is necessary to include dedicated reference scans for short
calibration stimulation sequences in a future collection of normative data.


\paragraph{d) Test minimum data requirement for reliable functional alignment
(within-subject test-retest)}
%
Analog to WP1c, in this last work unit I will estimate the minimum scan time
requirement for deriving a valid functional alignment by further reducing the
amount of input data for the short movie scan of the left-out individual.
%
Due to the aforementioned constraints of the time series hyperalignment
regarding the number of simultaneously considered voxels, this work will be
performed using the connectivity hyperalignment from WP2c only.
%
Taken together, the results of all work packages detail what performance and
versatility can be expected from an approach to localization that is based on
task-free natural stimulation fMRI.
%
Moreover, it comprehensively informs a future study about scan type, and scan
time requirements for the collection of a normative reference.
%
The availability of such a reference would enable quantitative and qualitative
description of an individual’s brain function with respect to such a norm, and
consequently progress the field towards neuroimaging studies of individual
differences that more closely resemble their psychological counterparts
\citep{dubois2016building}.


\section{Introductory remarks}
%% introductory remarks describe the scientific background of the work as
%% precisely as possible. Cite the most important publications and avoid
%% extensive literature reviews.


\section{Aims of thesis}
%% At the end of the introduction, add a subchapter on the Aims of Thesis in
%% which you describe the research question and the objectives of your work on
%% a maximum of two pages

