%% The introduction begins with an overview of the topic

%
One of the most distinct feature of the human brain in comparison to any other
organ of the body is its (functional) topographic organization (s.
\citep{debeck2008interpreting} for a review).
%
[Definition of topographic organization here].
%
(Classical) neuropsychological studies [cite an early classic] have shown that
hat localized damage to a small part of the brain will result in very specific
disturbances of isolated mental facilities, such as vision, movement or language
[\citep{rorden2004using} \citep{eickhoff2018topographic}.
%
Thanks to technological advanced, brain imaging studies have extensively used
\ac{fmri} since the early 1990s to measure \ac{bold} activity in-vitro.
%
The procedure [other word?] of \textit{human brain mapping} (or
\textit{topographic brain mapping}) maps brain functions, perceptual or
cognitive processes, to locations in the anatomy of the brain.\todo{check
definitions}.

\todo[inline]{revise to travel from broad view to higher visual areas}
%
Numerous functionally distinct regions of in the brain (e.g. V1, MT, the
fusiform face area, the hippocampal place area) have been
idetified.\todo{consult textbook}
% results from group average studies
In the domain of higher-visual perception, replicated findings suggest that
category-specific brain regions like the \ac{ppa} \citep{epstein1998ppa},
\ac{ffa} \citep{kanwisher1997ffa}), or the \ac{eba} \citep{downing2001bodyarea})
exhibit significantly increased \ac{bold} activity correlated with a
``prefered'' stimulus typ.
%
Nevertheless, typical analysis procedures average (voxel-wise) data of at least
10-15 subjects, [e.g.] to improve the \ac{snr}.
%
\todo{other reasons? check e.g. Dubois}.
%
However, the precise positions of functionally distinct regions of the brain
vary anatomically (measured in Talairach or MNI coordinates) across individuals
\citep{saxe2006divide}.\todo{check}
%
Consequently, studies employing an averaging approach may just ``capture the
common denominator of each individual cognitive circuit and lose a large amount
of information'' \citep{pinel2007fast}.
%
However, ``interpretation of fMRI data at the level of individual brains is
essential for characterizing brain function in health and disease''
\citep{dubois2016building}.


\section{Functional localization}
%% Further introductory remarks describe the scientific background of the work
%% as precisely as possible. Cite the most important publications and avoid
%% extensive literature reviews.
% a.k.a. "state of research"

\todo{check localizer reviews for purposes of localizers}

\todo[inline]{IBC sollte vorkommen https://project.inria.fr/IBC/publications;
evtl. auch https://github.com/Parietal-INRIA/DiFuMo}

% individual level: localizers
On the level of individual subjects, \textit{functional localizer} experiments
(s. \citep{saxe2006divide, friston2006critique} for reviews) are conducted using
\ac{fmri} to characterize the location size, and shape of functional areas.
correlating with perceptual or cognitive processes (e.g. perception of object
categories \citet{kanwisher1997ffa}; speech perception
\citet{fernandez2001language} or theory of mind \citep{spunt2014validating})
onto the cortex of individual study participants.
% purpose: ROI
Functional localizers are often used to define an subject-specific \acp{roi} to
improve the statistical power of the main experiment's analysis, or to locate
brain functions prior to neurosurgery.\todo{is this the case already?; }
% clinical application
They also promise advance neuroimaging towards a clinical application since it
depends on making diagnoses for single cases
\citep{wegrzyn2018thought}.\todo{check}
% purpose: neurosurgery
For example, surgical procedures might impact the post-operative quality of life
so much (e.g. concerning cognitive control or speech production) that a the
surgical intervention outweighs the therapeutic benefits.

% one localizer = one domain of functions
However, one localizer paradigm can usually map just one domain of brain
functions.
% detection power based on paradigms' design
Designed to maximize detection power, localizers employ carefully chosen,
tightly-controlled, and simplified stimuli presented in a block-wise manner
often accompanied with a task to keep study participants attentive.
% inefficiency -> localizer batteries
Since this approach becomes inefficient if one wants to map many different
processes with limited resources (time, availability and applicability of
diagnostic measures for an individual patient), researchers have developed more
time-efficient, multi-functional \textit{localizer batteries}
\citep{barch2013function, drobyshevsky2006rapid, pinel2007fast}.
% localizer batteries: example
For example, \citet{pinel2007fast} employs a range of dedicated stimuli and
specific tasks in a 5-minute routine to map processes of ``auditory and visual
perception, motor actions, reading, language comprehension, and mental
calculation at an individual level'' \citet{pinel2007fast}.
% task based = shit
Nevertheless, the diagnostic quality of localizer batteries still relies heavily
on a participant's comprehension of the task instruction and his/her compliance,
a criterion that can be difficult to meet in clinical or pediatric populations
\citep{eickhoff2020towards, vanderwal2015inscapes, vanderwal2019movies}.


\todo[inline]{placement of following paragraph is not optimal}
\todo[inline]{yet, a short excourse to the current project makes sense before
the naturalistic stimulus inferno breaks loose}
%
However, stimulus features (e.g. stimulus classes) driving hemodynamic responses
in classic localizers might also be embedded in prolonged, quasi-naturalistic
stimuli like a movie or spoken narrative.
%
This dissertation will explore whether paradigms using \textit{naturalistic
stimuli}, here a Hollywood movie and and its audio-only variant created for a
visually impaired audience could, in principle, substitute a traditional,
task-based paradigm to localize functional areas in the brain of individual
subjects.
%
As a proof of concept, the dissertation focusses on the ``parahippocampal place
area'', a classic example of a higher-visual, functional area
\citep{epstein1998ppa, epstein1999parahippocampal} located in the ventral visual
pathway [check PPA paper for reference]:
% what PPA is doing
the ``\ac{ppa} exhibits increased hemodynamic activity correlated with the
perception of landscape photos compared to faces or objects''
\citep{haeusler2022processing}
% what PPA responds to
Increased hemodynamic activity is observed in the \ac{ppa} when participants
view photos of landscapes, buildings or landmarks, compared to e.g. pictures of
faces or tools (\citep[see reviews][]{epstein2014neural, aminoff2013role}).


\section{Naturalistic stimuli}

% validity?
Additionally, localizer batteries also rely on carefully chosen and
tightly-controlled, simplified stimuli that are usually presented in blocks do
not resemble how we perceive the real-world during every-day life leading to
questionable external and ecologically validity.\todo{add references}

Ultimately, a major goal of cognitive neuroscience is not to reveal how the
brain responds to blocks of photos presented in an laboratory setting but how
the brain processes information during everyday perception.

%
To address this open question, \textit{naturalistic stimuli} gained popularity
in neuroscience to investigate brain activity under more life-like conditions.

%
Given the questionable/debatable ecological validity of traditional functional
localizer paradigms, it is often unclear how functional areas behave in
life-like situations and how they might interact.
%
After all, we do not experience the world around us as separated into small
unidimensional stimuli, but perceive --- through different senses --- a
seemingly continuous and unified world.


\paragraph{Definition}

% definition quote
Naturalistic stimuli are ``a class of stimuli that aim to evoke more
naturalistic patterns of neural responses than traditional controlled artificial
stimuli. Naturalistic paradigms are typically complex and dynamic, and longer in
duration than many conventional stimuli.'' \citep{vanderwal2019movies}.
%
They ``[Movies] sample a broad range of brain states and engage multiple
perceptual and cognitive systems in parallel'' \citep{haxby2020naturalistic}.
% movies & narratives
The most popular naturalistic stimuli in neuroscience are movies and auditory
narratives (s. \citep{jaaskelainen2021movies, jaaskelainen2020neural} for an
overview) that provide time-locked event structure during a continuous,
complex/rich, dynamic, and often prolonged stimulation.


\paragraph{Ecological validity}
% definition
<Definition here>.
% claim; is KINDA QUOTE OF Haxby 2020?
Naturalistic stimuli promise a higher ecological validity \citep{zaki2009need,
hasson2012future, hamilton2018revolution} because they, despite being presented
in a laboratory setting, still more closely mimic the richness of real-life
visual and auditory experiences \citep{hasson2008neurocinematics,
haxby2020naturalistic}.

\paragraph{External validity}
% definition
<Definition here>.
% less selection bias
Carefully chosen stimulus sets ``selectively sample from the stimulus population
leading to a stimulus-as-fixed-effect fallacy [Clarc, The
language-as-fixed-effect fallacy: A critique of language statistics in
psychological research]'' \citep{westfall2016fixing}.
%
``The conclusions cannot be generalized to a broader population of stimuli
without risking inflated Type I error  [cf. Donnet S, Lavielle M, Poline JB: Are
fMRI event-related response constant in time? A model selection
answer]'' \citep{westfall2016fixing}.
%
Naturalistic stimuli promise a higher ecological validity because they draw a
more representative sample from the ``theoretical population of stimuli that
might have been used'' \citep{westfall2016fixing}.


\paragraph{Early findings}
% reviews
Audio-visual movies and spoken narratives have been used during \ac{fmri}
(s.\citet{hamilton2018revolution, hasson2008neurocinematics,
sonkusare2019naturalistic, saarimaki2021naturalistic}
for reviews), and \ac{eeg} or \ac{meg} data acquisition (s. \citet{alday2019meg,
kandylaki2019story} for reviews).
%
Early studies have shown that watching a movie \citep{hasson2004intersubject,
hasson2008neurocinematics, hasson2010reliability} or listening to a
narrative \citep{lerner2011topographic, wilson2008beyond} reliably synchronize spatiotemporal responses across multiple subjects
in a large part of the brain compared to, for example, an unedited video of a
concert taken from a fixed viewpoint \citep{hasson2004intersubject,
hasson2008neurocinematics, hasson2010reliability, lerner2011topographic,
wilson2008beyond}.
% This finding could be attributed to a film director's goal to not only direct
% a movie, but also to capture and direct the audience's attention; this can be
% attributed to the way professional movies are shot and edited in order to
% intentionally manipulate the viewers' attentional focus and mental states
% \citep{brown2012cinematography, dancyger2011film-technique}.
%
Further, a pioneering study \citep{bartels2004mapping} suggests that functional
specialization of cortical areas preserved during complex, life-like
stimulation.


\paragraph{Better compliance}
%
From a practical perspective, naturalistic stimuli promise improved subject
compliance regarding wakefulness and head motion due to minimal instruction
requirements (e.g. no fixation of eye gaze), task demands (no task except
enjoying the movie or audiobook).
%
This is especially the case for young children \citep{vanderwal2015inscapes},
and possibly psychiatric \citep{eickhoff2020towards} or elderly persons
resulting in increased data quality.
%
Since movies and audiobooks are interesting and easy-to-follow stimuli that are
produced to be engaging and immersive, they also promise to put participants at
ease in the otherwise claustrophobic, uncomfortable and noisy fMRI scanner.
%
Lastly, spoken narratives are also appropriate for visually impaired persons
suffering from diminished or lacking eye-sight.

how we can both increase validity by using
 paradigms, as well as increased diagnostic
efficiency by predicting individual functional topography from data collected in
a reference group.


\paragraph{contra: challenging data analysis; annotations needed}

\todo[inline]{shift some text from "aims of anno-paper" here}

``Standards methods such as the general linear model [1] require the
experimenter to construct a design matrix that models features of the presented
stimuli across time. Such design matrices are notoriously difficult to construct
for naturalistic stimuli as one has to rely on manual annotations (see [2]) or
deep learning techniques (see e.g. [3], [4], [5] or [6]) that are hard to use,
and provide high-dimensional, cumbersome models of the stimulus''
\citep{richard2019fast}.


\section{Prediction individual topography from a reference group}

\todo[inline]{explain ``leave-one-out cross-validation`` and ``left-out
subject''; otherwise ``aims of thesis'' below do not make sense}

\todo[inline]{check \citep{poldrack2019establishment, yarkoni2017choosing}}


\subsection{Anatomical alignment}

\todo[inline]{write text in SRM chapter; put a shorter version in here}


\subsection{Functional alignment}


\subsubsection{Hyperalignment}


\subsubsection{Shared response model}


\subsubsection{interim summary}
% interim summary
In summary, naturalistic stimuli ``impose a meaningful timecourse across
subjects while still allowing for individual variation in brain activity and
behavioral responses, and lend themselves to a broader set of analyses than
either pure rest or pure event-related task designs'' \citep{finn2017can}.

\todo[inline]{add short sentence about challenging \& annotations data analysis}


\section{Aims of thesis}
%% At the end of the introduction, add a subchapter on the Aims of Thesis in
%% which you describe the research question and the objectives of your work on
%% a maximum of two pages
\todo[inline]{following structure is split: general aims, meta-aim
(reproducibility), specific aims per study; tbh, I do not like it but don't know
how to handle it better (better transitional sentences between subsections?)}

%
Previous work on a group average level has shown that it is possible to combine
\ac{bold} \ac{fmri} with naturalistic stimulation in order to localize brain
areas whose increased hemodynamic activity is correlated with particular brain
functions \citep{bartels2004mapping}.
%
This dissertation explores --- while following the principles of open,
transparent, and reproducible science --- whether a movie and the movie's
audio-description that was produced for an visually impaired audience could, in
principle, substitute an established localizer paradigm.
%
Naturalistic stimuli would have substantial advantages (e.g. task demands
compliance, and data quality) over simplified stimuli presently used in
localizer paradigms.
%
Moreover, the time-courses of rich, full-length naturalistic stimuli are
correlating with a variety of different brain functions ranging from low-level
perception (e.g.  luminance) to high-level cognition (social cognition).
%
Thus, a naturalistic stimulus could replace multiple dedicated localizers in the
future to provide a more comprehensive diagnostic routine.
%
However, a two-hour paradigm would be inefficient plus unsuitable for a clinical
population.
%
For that reason, this dissertation also explores whether just a part of a movie
or auditory narrative can be used to estimate the location, size and shape of a
``unknown'' subject (i.e. a 'left-out' subject to test the prediction
performance of our model) from a reference group (i.e. the study subjects
providing the data to train our model).


\subsection{Open, transparent, and reproducible science}

% reproducibility crisis
Over the last decade there has been a growing awareness that results of
scientific publications are not reproducible or general scientific findings are
not replicable letting some authors speak of an ``reproducibility crisis'' or
``replication crisis'' in the sciences \citep{baker2016reproducibility,
plesser2018reproducibility, stupple2019reproducibility, nosek2022replicability}.
% reproducibility: definition
``A study is reproducible if all of the code and data used to generate the
numbers and figures in the paper are available and exactly produce the published
results'' \citep{leek2017most}.
% replicability: definition
A study is replicable if the same analysis of an identical experiment's data
leads to consistent results \citep{dubois2016building, leek2017most}.
%
Hence, on a metalevel, this dissertation aims to meet both the requirements of
open, accessible, shared and transparent science \citep{watson2015will,
fecher2014open} as well as the requirements of a reproducible and replicable
research project:
%
the dissertation follows guidelines and best practices for a) coding and
scientific computing \citep{wilson2014best}, b) procedures and data analyses
\citep{nichols2017best, poldrack2017scanning, poldrack2019establishment}, and c)
sharing code, created data and results \citep{eglen2017toward, nichols2017best,
pernet2015improving}.


\paragraph{input data}

% open data \citep{eglen2017toward}
First, my work will build on top of publicly and freely available \ac{fmri} data
that is part of the \textit{studyforrest} project
(\href{www.studyforrest.org}{studyforrest.org}).
%
The studyforrest project is an open science project that aims to provide a
versatile resource for investigating human brain function under quasi-natural
conditions.
%
The core of this dataset are two hour long \ac{bold} \ac{fmri} scans of
participants watching the movie Forrest Gump and listening to the movie's
audio-description that was created for a visually impaired audience by adding a
narrator to the movie's audio track.
%
Since its first publication in 2014 \citep{hanke2014audiomovie}, the
studyforrest project has served as a resource of raw (and preprocessed) data for
international working groups to conduct and publish independent research (s.
\href{www.studyforrest.org/publications.html}{studyforrest.org/publications.html}).
%
The stimulus annotations that have been created over the course of the
dissertation are version-controlled and published in a standardized file format
\citep{haeusler2021speechanno}, and therefore contribute to the studyforrest
project as a resource for the scientific community.


\paragraph{code, analyses, output}

Further, all code is shared to improve reproducibility of results based on
current results and to facilitate their replicability on other datasets.
% automatization
Therefore, data analyses pipelines are implemented in a way that enable
automated processing.
%
Analyses pipelines are not be implemented in commercial, proprietary software
but in freely available and, if possible, open-source software.
% which tools to choose why?
Among potential software packages, we chose the tools that offer the most solid
documentation, and basis of developers and maintainers to ensure long-term
support.
% my code
Custom code written by myself is written in open-source programming languages
(Python and Bash), is version-controlled, documented, and released publicly and
freely accessible.\todo{but: FZJ rules?}
%
All input data, custom code, analysis steps and output data are accessible in
standardized \textit{DataLad} (\href{www.datalad.org}{datalad.org}) datasets.
Since DataLad provides a free and open-source software solution that manages
provenience, distribution, and version-control of code and data
\citep{halchenko2021datalad}, all executed steps from downloading the input data
to visualizing the results can be rerun to check and validate the dissertation's
results.


\paragraph{publications}
% open-access publishing
Last, because ``nature abhors a paywall'' \citep{dupre2020nature}, publications
describing generated data, reasoning of methodological choices, analysis steps,
and results are published in open-access journals.
% neurovault
Unthresholded statistical maps of all computed statistical $t$-contrasts are
additionally published at Neurovault
(\href{https://neurovault.org/}{neurovault.org}).


\subsection{Specific objectives and hypotheses}

[Here should be some random text between section titles]

\subsubsection{A studyforrest extension, an annotation of spoken language in the
German dubbed movie ``Forrest Gump'' and its audio-description}

\todo[inline]{maybe shift some text from here into literature review
(annotations as ``challenge'' / ``bottleneck'') and just shortly be mentioned
here again}


\paragraph{Intro}

\todo[inline]{example of a ``real'' naturalistic stimulus that was designed by
researchers?}

%
The majority of naturalistic stimuli that have been used in neuroscience have
originally been designed for commercial purposes and not to conduct research.
%
The time-courses and the \textit{stimulus features} (i.e., the variables)
embedded in a naturalistic stimulus are fixed and thus reproducible but not
explicitly known.

\paragraph{why we need stimulus annotations}

% traditionally
``In contrast to stimuli designed to trigger a perceptual process of interest,
while controlling for confounding variables (e.g., color and luminance),
% our approach: annotation and regressors
naturalistic stimuli have a fixed but initially unknown temporal structure of
stimulus features of interest, as well as an equally unknown confound structure.
% annotation ftw
In order to evaluate the suitability of the stimulus for the targeted analyses,
and to inform the required hemodynamic models, we annotated the temporal
structure of a range of stimulus features''.

%
Thus, modeling brain activity correlating with the stimulus features embedded in
the time course is challenging \citep{saarimaki2021naturalistic,
simony2020analysis} because such models rely on stimulus features being
annotated.

``Model-driven methods, like the general linear model (GLM), which are based on
stimulus annotations can be useful to test hypotheses on specific brain
functions under more ecologically valid conditions, to statistically control
confounding stimulus features, and to explain not just ``how'' the brain is
responding to a stimulus but also ``why'' \citep{hamilton2018revolution}''
\citep{haeusler2022processing}.\todo{Wegrzyn did not like that sentence}
%
However, the lack of extensive annotations has led to a ``usage bottleneck''
\citep{aliko2020naturalistic} and might be the main reason why explicit models
of task or stimulus are often ``prohibitively difficult'' to construct
\citep{nastase2019measuring}.
%
The stimulus annotations need to be extensive for primarily two reasons:
%
A researcher needs to a) inform a model about the time courses of events that
are assumed to correlate with a psychological and cognitive processes of
interest, and b) model hemodynamic responses (e.g., as nuisance regressors in a
\ac{glm}) that are correlated with the event structure of potentially
confounding variables.


\paragraph{what I did}
% yeah, fuck it; I'll do it myself
Hence, I created an annotation of speech occurring in the movie and the
audio-description.
% cloud-based annotations
State of the art cloud-based speech-to-text services (Google Cloud, IBM Watson)
fell short to deliver satisfactory scaffolds of an annotation.
% procedure
Consequently, I revised a preliminary transcription of language spoken by
actors, actresses, and the narrator, and submitted it to a forced aligner
(\href{https://github.com/MontrealCorpusTools/Montreal-Forced-Aligner}{Montreal
Forced Aligner} v1.0.1 \citep{mcauliffe2017montreal}) that identified the exact
onset and offset of each word and phoneme. The forced aligner's output was
extensively cleaned and extended both algorithmically as well as manually.


\paragraph{annotation content}
%
Consequently as planned, the annotation's content substantially exceeds the
groundwork that was needed to conduct the study reported in Chapter 3 (s.
\citep{haeusler2022processing}), and serves as an extension of the studyforrest
project for public use:
% content
the annotation provides ``information about the exact timing of each of the more than 2500
spoken sentences, 16000 words (including 202 non-speech vocalizations), 66000
phonemes, and their corresponding speaker'' \citep{haeusler2021speechanno}.
%
For every word, the annotation additionally provides ``lemmatization, a simple
part-of-speech-tagging (15 grammatical categories), a detailed part-of-speech
tagging (43 grammatical categories), syntactic dependencies, and a semantic
analysis based on word embedding which represents each word in a 300-dimensional
semantic space'' \citep{haeusler2021speechanno}.


\paragraph{Validation}

\todo[inline]{sentences are still pretty similar to paper even if not marked as
direct quote}

% what we did
We created a canonical \ac{glm} based on information drawn from the annotation
to model hemodynamic brain activity to validate the dataset's quality.
% results in line with previous studies
Results revealed statistically significant increased hemodynamic activity in a
bilateral cortical network including temporal, parietal and frontal regions
related to processing spoken language.
% results replicate
Our exploratory analysis replicates results of studies that employed
tightly-controlled stimuli (s. \citep{friederici2011brain,
hickok2007cortical,price2012twentyyears} for reviews), and studies that employed
data-driven methods to analyze \ac{fMRI} data from naturalistic stimulus
paradigms \citep{honey2012not, lerner2011topographic, silbert2014coupled}
% "logic" inference
Consequently, results suggest that the annotation's content and quality enables
independent ``researchers to model hemodynamic brain responses that correlate
with a variety of aspects of spoken language'' \citep{haeusler2021speechanno}
under more ecologically valid conditions.
% transition to study 2
Last, the results encouraged us to use the annotation as a groundwork to be
adapted to our specific needs in study 2.


\subsubsection{Processing of visual and non-visual naturalistic spatial
information in the ``parahippocampal place area''}

\todo[inline]{revise intro regarding naturalistic stimulus as potential
substitute of traditional paradigms}

%

%
If stimulus features driving localization in dedicated localizer experiments are
also embedded in naturalistic stimuli, localization of the same / similar
functional areas should also be possible by averaging those (heavily
confounded?) events in a naturalistic stimulus that assumingly trigger the same
/ similar brain process.

%
\paragraph{Aim}

%
The aim of this study was to compare how the medial temporal cortex of
individuals respond to “geographically relevant events” in the  three different
experimental paradigms.
%
Previous research suggest the existence of a ``parahippocampal place area'' [PPA;
epstein1998ppa] in the human cortex.
%
The PPA exhibits increased BOLD activation
when subjects passively view pictures of environmental stimuli like landscapes
or rooms compared to other stimuli like tools or faces [Epstein et al., 1999].
PPA and FFA in auditory speech \citep{aziz2008modulation}

%
A previous analysis \citep{hanke2016simultaneous} of the movie ``Forrest Gump''
and its audio-description revealed significant correlations between \ac{bold}
response time series of the two stimuli in spatially corresponding voxels of the
ventral medial cortices (s. Figure 3 in \citep{hanke2016simultaneous}).
%
\todo[inline]{looking at Fig. 3 in \citep{hanke2016simultaneous} this statement
is not true! So, what was the incentive to do it? Well, it's Florian Schulz'
fault... ;-)}


We investigated if/whether it is possible to localize a visual area, usually
identified via a task-based functional localizer, via naturalistic stimuli.


\paragraph{State of research}

Previous work on a group average level has shown that it is possible to combine
\ac{bold} \ac{fmri} with naturalistic stimulation in order to localize brain
areas whose increased hemodynamic activity is correlated with particular brain
functions \citep{bartels2004mapping} [same sentence as above].
%
\citep{bartels2004mapping} annotated the occurrence and perceived intensity of
colors, faces, language, and human bodies but, surprisingly, not scenes, in the
James Bond movie ``Tomorrow never dies''.
%
Hence, it is still ``kind of unclear'' is results from traditional paradigms
that aim to localize the \ac{ppa} [needs (still?) to be defined above]
generalize from pictures presented in blocks to results from a more ecologically
valid stimuli like a movie.
%
It is still unclear is if these results generalize to the auditory domain
\citep{aziz2008modulation}.

``Increased hemodynamic activity in the PPA generalizes from pictures to mental
imagery of landscapes \citep{ocraven2000mental}, haptic exploration of
scenes constructed from LEGO blocks \citep{wolbers2011modality}, and
scene-related sounds \citep{van2017development}.
% % O'Craven: watching pictures
In a study conducted by \citet{ocraven2000mental} participants viewed
alternating blocks of pictures showing familiar places and famous faces during
an initial experimental paradigm.
% O'Craven: mental imagery
In a subsequent paradigm, participants were instructed to ``form a vivid mental
image'' of the previously viewed pictures.
% O'Craven results
The PPA showed increased activation during imagination of places compared to
faces but the imagination task showed a smaller activation level compared to the
perceptual task.
% Wolbers: haptic exploration
In a block design study conducted by \citet{wolbers2011modality} the PPA of
sighted as well as blind participants showed increased activation during a
delayed match-to-sample task of haptically explored scenes constructed from LEGO
bricks compared to abstract geometric objects''.
% Wolbers: connectivity analysis

% Aziz (2008): place related sentences
``To our knowledge only one study \citep{aziz2008modulation} compared hemodynamic
activity levels in the PPA that were correlated with different semantic
categories occurring in \textit{speech}.
% Aziz' stimuli; ``The Taj Mahal faces a long thin reflecting pool'', ``Marilyn
% Monroe has a large square jaw'', ``The television has a long antenna''
\citet{aziz2008modulation} used sentences that described famous or generic
places, faces, or objects.
% Aziz' tasks:
Participants were instructed to press a button whenever the sentence described
an inaccurate or improbable fact (e.g. ``Marilyn Monroe has a large square
jaw'').
% Aziz' results:
Activation in the left, but not right, PPA was significantly reduced when
participants listened to place-related sentences compared to listening to
face-related sentences. Moreover, this effect was only observed in sentences
involving famous places.

% summary if literature review
Taken together, the literature suggests that the PPA does not exclusively
respond to visually presented scene-related, spatial information''.

% open question
``In this study, we investigate whether increased hemodynamic activity in the PPA
that is usually detected by contrasting blocks of pictures is also present under
more natural conditions''.

% #1 previous studies
``Several studies have reported increased hemodynamic activity in the PPA
attributed to the processing of scene-related, spatial information, for example,
when participants were watching static pictures of landscapes compared to
pictures of faces or objects \citep{epstein1998ppa, epstein1999parahippocampal}.
% auditory semantics unclear
However, reports regarding the correlates of processing spatial information in
verbal stimulation are less clear \citep{aziz2008modulation}.
% current study: similar
In line with previous studies, we investigated the hemodynamic response to
spatial information using a model-based, mass-univariate approach.
% current study: dissimilar
However, instead of using a conventional set of stimuli, assembled to
specifically and predominantly evoke the processing of spatial information, we
employed naturalistic stimuli, a movie and its audio-description, that were
designed for entertainment''.

%#3 hypo
``We hypothesized that due to the complex nature of the stimuli, unrelated factors
would be balanced across a large number of events, and make the bias of spatial
information accessible to a conventional model-based statistical analysis of BOLD
fMRI data.
% we need a detailed description -> annotation
This model-driven approach required a detailed annotation of the occurrence of
relevant stimulus features.
% event counts ftw
The annotation of both stimuli revealed the respective number of incidentally
occurring events to be similar to those of a conventional experimental paradigm
used to localize functional regions of interest''.

%#4
We modeled hemodynamic responses correlating with spatial information embedded
in the naturalistic stimuli, capitalizing on conceptually similar, but
perceptually different stimulation events.


% we use 2 naturalistic stimuli
``To answer this question, we operationalized the perception of both
\textit{visual and auditory} spatial information using two naturalistic stimuli
(\citep[see reviews][]{hamilton2018revolution, hasson2008neurocinematics,
sonkusare2019naturalistic}).
% AV stimulus
The current operationalization of visual spatial perception is based on an
annotation of cuts and depicted locations in the audio-visual movie ``Forrest
Gump'' \citep{haeusler2016cutanno}, while
% AD stimulus
the operationalization of non-visual spatial perception is based on an
annotation of speech occurring in the movie's audio-description
\citep{haeusler2021studyforrest}.
% pictures -> movie -> audio-description
The movie stimulus shares the stimulation in the visual domain with classical
localizer stimuli, while featuring real-life-like visual complexity and
naturalistic auditory stimulation. The audio-description maintains the
naturalistic nature of the movie stimulus, but limited to the auditory domain''.

``% current GLM contrasts
We applied model-based, mass-univariate analyses to BOLD fMRI data from both
naturalistic stimuli \citep{hanke2016simultaneous, hanke2014audiomovie},
available from the open-data resource
\href{http://www.studyforrest.org}{studyforrest.org}.
% comparison to previous GLM
We compare current results to results of a previously performed model-based,
mass-univariate analysis that was applied to data from a conventional functional
localizer performed with the same set of participants
\citep{sengupta2016extension}.
% intro to hypo a
Similarly to the functional localizer, we currently also capitalize on events
that ought to evoke the cognitive processing of spatial information.
% hypo a
Thus, we hypothesized that our whole-brain analyses would reveal increased
hemodynamic activity in medial temporal regions that were functionally
identified as the PPA by the analysis of the localizer data.
%hypo b: individuals
We hypothesized further that a purely auditory stimulus could, in principle,
localize the PPA as an example of a ``visual area'' in individual persons,
% application in individuals
and may offer an alternative paradigm to assess brain functions in visually
impaired individuals.

%
``Based on an annotation of cuts in the movie and an annotation of speech spoken
by the audio-description's narrator, we selected events in both stimuli that
should correlate with the perception of spatial information and contrasted them
with events that should correlate with non-spatial perception to a lesser
degree, or not at all''.


\paragraph{Hypotheses}

% hypo a
Thus, we hypothesized that our whole-brain analyses would reveal increased
hemodynamic activity in medial temporal regions that were functionally
identified as the PPA by the analysis of the localizer data.
%hypo b: individuals
We hypothesized further that a purely auditory stimulus could, in principle,
localize the PPA as an example of a ``visual area'' in individual persons,
% application in individuals
and may offer an alternative paradigm to assess brain functions in visually
impaired individuals.


%
If stimulus features of dedicated localizers that drive responses in de PPA are
embedded in movie / audio drama, too -> PPA can be localized with dynamic
stimulus via standard GLM
%
Similar to the results of the dedicated localizer using static pictures
[sengupta2016extension], we expect, using the audio drama and movie, spatially
constrained, spatially congruent [deckungsgleich = steile These] increased BOLD
activation in the medial temporal cortex on an individual level using. We expect
this increased activation - modelled by a standard voxel-wise general linear
model - to happen at time points / events of the movie and audio drama that
[might] trigger processes of “spatial scene perception” compared to time points
/ events that [probably] do not trigger spatial perception [or trigger to a less
amount].

\paragraph{What we did}

``We model hemodynamic activity based on annotations of selected stimulus
features,
% VIS
and compare results to a block-design visual localizer''.

Similar to static pictures of dedicated localizers, we annotated “geographically
relevant events” in the movie Forrest Gump and its corresponding audio drama
that might trigger increased hemodynamic activity in the medial temporal cortex.
%
For the movie, we used a published annotation of cuts (that virtually relocate
an observer in the movie’s environment) and the location depicted immediately
after the cut [haeusler2016annotation].
%
For the audio drama, we annotated a transcript of every word spoken by the
narrator. Whenever the narrator cued the observer about a scene change, about a
scene’s or room’s spatial layout, about an object or a person, her/his face, we
annotated that auditory cue as such.

\todo[inline]{purpose of the AV (secondary) vs. AO (primary)}

%
Hence, we we operationalize the perception of spatial information that is
embedded two naturalistic stimuli?
%
Compare our results to a classic visual localizer experiment that used blocks of
pictures to results from naturalistic stimulus paradigms.
%
We operationalized the assumed perceptual process not just in a
audio-visual movie but also in an exclusively auditory stimulus.
% group
Does increased hemodynamic activity in the PPA correlate with auditory spatial
information (group-level)?
% individuals
Can we localize the PPA in individual persons using a more engaging \&
entertaining, but also exclusively auditory stimulus

\paragraph{How we did it}

\todo[inline]{mention existing annotation}

\todo[inline]{we did standard analysis and as similar as possible to Sengupta,
esp. in study 2; we want to replicate results with naturalistic stimuli}

To operationalize the perception of spatial information, we looked at time
points in the stimuli that should correlate with the perception of spatial
information

%
For the movie, we used the movie cuts that basically re-orient the observer in
space.

% AV anno
``For the analysis of the movie stimulus, we took advantage of a previously
published annotation of 869 movie cuts and the depicted location after each cut
\citep{haeusler2016cutanno}.
% we focus on cuts
Contrary to manually annotating stimulus features of movie frames (for example,
as performed by \citet{bartels2004mapping} for color, faces, language, and human
bodies), we categorized movie cuts that - in general - realign the viewer within
the movie environment by switching to another perspective within the same
setting, or to a position in an entirely different setting.
% this this should work
More specifically, we sought to exploit a cinematographic bias as
% establishing shots & shots within setting early in the movie
at a setting's first occurrence in a movie, shots tend to broadly establish the
setting and the spatial layout within the setting.
% cut to recurrent scene
On revisiting an already established setting, the shot sizes tend to decrease
and more often depict people talking to each other or objects that are relevant
to the evolved plot
\citep{brown2012cinematography, katz1991film, mascelli1998five}''.

% AV regressors/events
Based on this cinematographic bias, we assigned each cut to one of five
categories (see Table [X] in \citep{haeusler2022processing}):
%




\todo[inline]{we extended the annotation of speech with the annotation of nouns}
%
For the audio-descriptions, we semantically categorized nouns spoken by the
narrator in case he used them to describe faces or bodies, locations, buildings,
rooms, and so on.
%
In both stimuli, we looked for time points correlating with other perceptual
processes that we needed to build general linear model contrasts.
%
create events from stimulus annotations;

% AD annotation
For the analysis of the audio-description stimulus, we extended a publicly
available annotation of its speech content \citep{haeusler2021studyforrest} by
classifying concrete and countable nouns that the narrator uses to describe the
movie's absent visual content.
% annotation procedure
An initial annotation was performed by one individual,
% corrections
and minor corrections were applied after comparing with a second categorization
done by the author.
% reference to table with rules and examples
A complete overview of all 18 noun categories, their inclusion criteria, and
examples can be seen in Table [X] in \citep{haeusler2022processing}.
% why and how of categories 1
Some categories reflect the verbal counterpart of the stimulus categories that
were used in the visual localizer experiment (e.g. \texttt{body}, \texttt{face},
\texttt{head}, \texttt{object}, \texttt{setting\_new}, and
\texttt{setting\_rec}).
% why and how of categories 2
Other categories were created to semantically cluster remaining nouns into
categories that had no counterpart in the visual localizer experiment (e.g.
\texttt{bodypart}, \texttt{female}, \texttt{fname}, \texttt{furniture},
\texttt{geo}, \texttt{groom}, \texttt{male}, and \texttt{persons}).


%
model hemodynamic responses;
%
create contrast(s).




\paragraph{Results: where}

%
Results suggest that you can ``localize a auditory PPA'' but not the ``visual
PPA'' using an auditory narrative; It's similar but different; SRM study will
quantify the prediction performance
%
The audio-description must somehow at some time points trigger responses in the
ventromedial cortex.

%
There are responses in the parahippocampal cortex correlating with
spatial information.

``On a group level, increased activation correlating with visual spatial information
occurring in the movie is overlapping with a traditionally localized PPA.
% results: group AD
Activation correlating with semantic spatial information occurring in the
audio-description is more restricted to the anterior PPA.
% results: individual AD
On an individual level, we find significant bilateral activity in the PPA
of nine individuals and unilateral activity in one individual''.

% group AV
``On a group-average level, results for the movie show significantly
increased hemodynamic activity spatially overlapping with a conventionally
localized PPA but also extending into earlier visual cortices.
% group AD
Likewise, results for the audio-description identify significant activation in
the PPA but restricted to its anterior part.
% individual level
Bilateral clusters in 9 of 14
participants (of which \texttt{sub-04} shows only a right-lateralized PPA
in the block-design localizer results), and a unilateral
significant cluster in one participant, indicate that the group average results
are representative for the majority of individual participants.
% meaning: generalization to naturalistic stimuli
These findings suggest that increased activation in the PPA during the
perception of static pictures generalizes to the perception of spatial
information embedded in a movie or a purely auditory narrative.
%
Current results may partially deviate from \citet{sengupta2016extension}, due to
the uniform cluster forming threshold employed in this study versus their
adaptive procedure (bilateral clusters in 12 of 14 participants and a unilateral
right cluster in two participants (\texttt{sub-04}, \texttt{sub-20})''.


\paragraph{Results: interpretation}

% conclusion: generalizability
``Results suggest that activation in the PPA generalizes to spatial information
embedded in a movie and an auditory narrative''.


\todo[inline]{natase 2016 is just a poster}

%
Localized functional regions of interest can be recovered from neural responses
to dynamic naturalistic stimuli [nastase2016localizing].
%
However, highly unbalanced class frequencies result in relatively low true
positive rates and many false positives—overall classification accuracy is not a
very useful evaluation metric in this context [nastase2016localizing].
%
False positives (i.e., voxels with similar response profiles to the target ROI)
are localized to potentially meaningful structures
[nastase2016localizing].

\paragraph{Conclusion}


\paragraph{Transition to study 3}

%
But probably our event structure and model is not just an approximation of the
``real'' events structure that is actually correlated with the response series
in the parahippocampal cortex.
%
But still incentive to build shared response model and use it for prediction
%
2h movie/audio-description is too long anyway


\subsubsection{DataLad: distributed system for joint management of code, data,
and their relationship}

\todo[inline]{well, does not really fit in}


\subsubsection{Using varying amount of data from naturalistic stimulation for
functional alignment to predict results from a task-based functional localizer
paradigm}

\paragraph{from (previous) abstract in chapter 5}
% intro
\textit{Intro:} In order to map perceptual or cognitive functions onto the brain
anatomy of study participants, researchers usually conduct dedicated experiments
\textit{functional localizers} often accompanied with a task.

% problem
\textit{Current approach \& problem:} Nevertheless, the approach ``one paradigm
to map one domain of brain functions'' becomes impractical if a variety of
domains is supposed to be mapped in a time-efficient manner.
%
A clinical application must aim to minimize the scan time/cost but still provide
valid results

% therefore
\textit{Therefore:} In the current study, we explore a method and quantity of
data needed to (reliably) predict individual and idiosyncratic functional
topographies by projecting results of a localizer experiment (statistical
$Z$-maps) from a reference group into the brain anatomy of individual
participants.

%
We test whether functional alignment can be achieved with a task-free, natural
stimulation ``calibration'' scan that requires no more acquisition time than a
conventional localizer paradigm.
%
fMRI data acquired from naturalistic stimulation are to align a subject's voxel
space with a common representational/functional reference space.
%
Once aligned, the brain activity of the reference group can be used to predict
the activity of another subject.


\textit{Method:}
% data
During functional magnetic resonance imaging (fMRI), participants ($N=14$) took
part in a task-based, block-design visual localizer and two naturalistic stimuli
paradigms: an audio-visual movie and the movie's audio-description, both
paradigms free of any task.
% cms creation
Based on these response time series, we first created a common model space
employing a shared response model \citep{chen2015reduced} following a $N-1$ fold
procedure [correct term??], a process that also computed transformation matrices
for the subjects that provided the data for the creation of the model space.

%
Using a leave-one-subject-out cross-validation, individual results of the
conventional localizer are projected into the common space.

% alignment
Then, we aligned left-out subjects with the common [via Procusted
transformation?] to derive transformation matrices for the left-out-subjects by
also varying the quantity of functional response time series to perform the
alignment.

%
Once a valid alignment is established, the inverse transformation [i.e. the
transpose of the matrix] is then used to project functional properties of the
common reference into that individual's voxel space.

% prediction
Lastly, the acquired transformation matrices were used to project the functional
topographies from the anatomy of the reference group into the common model
space, and from the common model space into the anatomy of the left-out
subjects.

%
Then, data are projected into the voxel space of the left-out individual for
comparison with the localizer results for that individual.

%
We will estimate the trade-off between diagnostic quality and required effective
scan time by progressively reducing the duration of input BOLD fMRI data and
comparing the results of the reduced model to the reference computed from the
full length scan.

% stimulus length?
[We assessed the relationship between length of naturalistic stimulation used
for a \textit{partial functional alignment} and the performance of predicting
empirical $Z$-maps.]

%
estimate the minimum scan time requirement for deriving a valid functional
alignment by further reducing the amount of input data of the left-out
individual.

%
\textit{Results} suggest that ``a subject's idiosyncratic functional topography
can be estimated with high fidelity from that subject's fMRI data obtained while
watching a naturalistic movie using hyperalignment to project other subjects’
localizer data into that subject's idiosyncratic cortical anatomy''
\citep{jiahui2020predicting}.

%
\textit{Discussion}: stimulus length: 15-30 min vs. 2h;  results of auditory
stimulus to predict visual localizer are ``modest''.

%
\textit{Conclusion}: ``These findings lay the foundation for developing an
efficient tool for mapping functional topographies for a wide range of
perceptual and cognitive functions in new subjects based only on fMRI data
collected while watching an engaging, naturalistic stimulus and other subjects'
localizer data from a normative sample'' \citep{jiahui2020predicting}.


\paragraph{just some notes}

\todo[inline]{other contrasts that were tested over the course of the
dissertation: phonemes, grammatical tags, prosody, sex}
