%% The introduction begins with an overview of the topic

\todo[inline]{we don't do speech lateralization anymore; imo the neurosurgery
thing should not be mentioned in the intro anymore; better come up with it in
the general discussion}

%
Brain imaging using with \ac{fmri} of \ac{bold} activity has been used for
almost three decades to investigate perceptual and cognitive brain functions.
%
Human brain mapping (or \textit{topographic brain mapping}) maps brain
functions, perceptual or cognitive processes, onto the anatomy of the brain.
%
Nevertheless, typical analysis procedures average (voxel-wise) data of at least
10-15 subjects to improve the \ac{snr}.
%
Consequently, studies employing an averaging approach do not characterize brain
function at the level of an individual because these models that do not capture
individual brain properties but just a ``common denominator''
\citep{dubois2016building}. \todo{check paper}

and the ``interpretation of fMRI data at the level of individual brains is
essential for characterizing brain function in health and disease''
\citep{dubois2016building}

%
However, characterization of individual brain function is by far the most
important application of BOLD fMRI in a clinical context.
%
For example, for a diagnosis of brain functions in health and disease, or
pre-surgical screening.
%
This dissertation explores weather naturalistic stimuli, here a Hollywood movie
and and its audio-only variant created for a visually impaired audience, could,
in principle, substitute a traditional, task-based paradigm to identify the
location, size, and shape of a functional area in the brain of individual
subjects.


\section{Functional localization}
%% Further introductory remarks describe the scientific background of the work
%% as precisely as possible. Cite the most important publications and avoid
%% extensive literature reviews.
% a.k.a. "state of research"


\todo[inline]{maybe define ecological \& external validity here already;
not in section about naturalistic stimuli later}

\todo[inline]{get a better definition of functional areas \& localization}

% definition
Studies subdivided the cerebral cortex into distinctive functional areas whose
statistically increased \ac{bold} activity is correlated with one particular
stimulus type.
% group level
On a group average level, previous studies presented isolated higher-level
visual categories such as pictures of scenes, faces, human bodies or tools.
%results from group average studies
Results suggest that category-specific brain regions like the \ac{ppa}
\citep{epstein1998ppa}, \ac{ffa} \citep{kanwisher1997ffa}), the occipital face
area \ac{ofa} \citep{pitcher2011occipitalfacearea}, the \ac{eba}
\citep{downing2001bodyarea}), and the \ac{loc} \citet{malach1995loc} exist in
the human brain.
% individual level: localizers
On the level of individual subjects, the most frequently employed paradigm to
characterize location, size, and shape of function areas with BOLD fMRI are
[grammar?] \textit{functional localizers}.
% localizers: definition
Functional localizers are dedicated measurement that aim at isolating and
localizing brain activity correlated with specific perceptual processes (e.g.
different object categories; \citet{kanwisher1997ffa}) or cognitive processes
(like speech perception \citet{fernandez2001language} or theory of mind
\citet{spunt2014validating}).
% purpose: ROI
Functional localizers can be used to define an subject-specific \acp{roi} to
improve the statistical power of the main experiment's analysis, or to locate
brain functions prior to neurosurgery.
% purpose: neurosurgery
Surgical procedures might impact the post-operative quality of life so much
(e.g. concerning cognitive control or speech production) that it potentially
outweighs the therapeutic benefits.
% efficiency
The challenge is to precisely localize relevant brain areas with limited
resources (time, availability and applicability of diagnostic measures for an
individual patient).
% one localizer = one domain of functions
Importantly, one dedicated localizer paradigm can only map one domain of brain
functions.
% fucking inefficient
Hence, functional localizers, despite [or because] being tuned for detection
power by employing carefully chosen and tightly-controlled, simplified stimuli,
quickly become inefficient if one wants to map many different processes in a
limited amount of time.
% localizer batteries: intro
Researchers have tried to tackle this issue by creating time-efficient
multi-functional \textit{localizer batteries} \citep{barch2013function,
drobyshevsky2006rapid, pinel2007fast}.
% localizer batteries: example
For example, \citet{pinel2007fast} employs a range of dedicated stimuli and
specific tasks in a 5-minute routine to map processes of ``auditory and visual
perception, motor actions, reading, language comprehension, and mental
calculation at an individual level'' \citet{pinel2007fast}.
% task based = shit
Nevertheless, the diagnostic quality of localizer batteries relies heavily on
participants' comprehension of the task instruction and their compliance, a
criterion that can be difficult to meet in clinical or pediatric populations
\citep{eickhoff2020towards, vanderwal2015inscapes, vanderwal2019movies}.
% validity?
Additionally, localizer batteries also rely on carefully chosen and
tightly-controlled, simplified stimuli that are usually presented in blocks do
not resemble how we perceive the real-world during every-day life leading to
questionable external and ecologically validity.\todo{add references}

% interim summary
In summary, we have two issues: a) validity and compliance, and b) efficiency.
% foreshadowing to next sections & transition to naturalistic stimuli
This dissertation will explore how we can both increase validity by using
\textit{naturalistic stimulus} paradigms as well as increase efficiency by
predicting individual functional topography from data collected in a reference
group.


\section{Naturalistic stimuli}

Ultimately, a major goal of cognitive neuroscience is to reveal how the brain
processes information during everyday perception.
%
Given the questionable/debatable ecological validity of traditional functional
localizer paradigms, it is often unclear how functional areas behave in
life-like situations and how they might interact.
%
After all, we do not experience the world around us as separated into small
unidimensional stimuli, but perceive --- through different senses --- a
seemingly continuous and unified world.
%
To address this open question, \textit{naturalistic stimuli} gained popularity
in neuroscience to investigate brain activity under more life-like conditions.


\paragraph{Definition}

% definition quote
Naturalistic stimuli are ``a class of stimuli that aim to evoke more
naturalistic patterns of neural responses than traditional controlled artificial
stimuli. Naturalistic paradigms are typically complex and dynamic, and longer in
duration than many conventional stimuli.'' \citep{vanderwal2019movies}.
%
They ``[Movies] sample a broad range of brain states and engage multiple
perceptual and cognitive systems in parallel'' \citep{haxby2020naturalistic}.
% movies & narratives
The most popular naturalistic stimuli in neuroscience are movies and audio-only
stories that provide time-locked event structure during a continuous,
complex/rich, dynamic, and often prolonged stimulation.


\paragraph{Ecological validity}
% definition
<Definition here>.
% claim; is KINDA QUOTE OF Haxby 2020?
Naturalistic stimuli promise a higher ecological validity \citep{zaki2009need,
hasson2012future, hamilton2018revolution} because they, despite being presented
in a laboratory setting, still more closely mimic the richness of real-life
visual and auditory experiences \citep{hasson2008neurocinematics,
haxby2020naturalistic}.

\paragraph{External validity}
% definition
<Definition here>.
% less selection bias
Carefully chosen stimulus sets ``selectively sample from the stimulus population
leading to a stimulus-as-fixed-effect fallacy [Clarc, The
language-as-fixed-effect fallacy: A critique of language statistics in
psychological research]'' \citep{westfall2016fixing}.
%
``The conclusions cannot be generalized to a broader population of stimuli
without risking inflated Type I error  [cf. Donnet S, Lavielle M, Poline JB: Are
fMRI event-related response constant in time? A model selection
answer]'' \citep{westfall2016fixing}.
%
Naturalistic stimuli promise a higher ecological validity because they draw a
more representative sample from the ``theoretical population of stimuli that
might have been used'' \citep{westfall2016fixing}.


\paragraph{Early findings}
% reviews
Audio-visual movies and spoken narratives have been used during \ac{fmri}
(s.\citet{hamilton2018revolution, hasson2008neurocinematics,
jaaskelainen2021movies, sonkusare2019naturalistic, saarimaki2021naturalistic}
for reviews), and \ac{eeg} or \ac{meg} data acquisition (s. \citet{alday2019meg,
kandylaki2019story} for reviews).
%
Early studies have shown that watching a movie \citep{hasson2004intersubject,
hasson2008neurocinematics, hasson2010reliability} or listening to a
narrative \citep{lerner2011topographic, wilson2008beyond} reliably synchronize spatiotemporal responses across multiple subjects
in a large part of the brain compared to, for example, an unedited video of a
concert taken from a fixed viewpoint \citep{hasson2004intersubject,
hasson2008neurocinematics, hasson2010reliability, lerner2011topographic,
wilson2008beyond}.
% This finding could be attributed to a film director's goal to not only direct
% a movie, but also to capture and direct the audience's attention; this can be
% attributed to the way professional movies are shot and edited in order to
% intentionally manipulate the viewers' attentional focus and mental states
% \citep{brown2012cinematography, dancyger2011film-technique}.
%
Further, a pioneering study \citep{bartels2004mapping} suggests that functional
specialization of cortical areas preserved during complex, life-like
stimulation.


\paragraph{Better compliance}

%
From a practical perspective, naturalistic stimuli promise improved subject
compliance regarding wakefulness and head motion due to minimal instruction
requirements (e.g. no fixation of eye gaze), task demands (no task except
enjoying the movie or audiobook).
%
This is especially the case for young children \citep{vanderwal2015inscapes},
and possibly psychiatric \citep{eickhoff2020towards} or elderly persons
resulting in increased data quality.
%
Since movies and audiobooks are interesting and easy-to-follow stimuli that are
produced to be engaging and immersive, they also promise to put participants at
ease in the otherwise claustrophobic, uncomfortable and noisy fMRI scanner.
%
Lastly, spoken narratives are also appropriate for visually impaired persons
suffering from diminished or lacking eye-sight.


\section{Prediction individual topography from a reference group}

% general idea
Given that one localizer = one domain of brain function is unsuitable to
characterize a variety of perceptual and cognitive processes in a time-efficient
manner, you could just predict the location in a ``new or  unknown'' subjects
from data of a reference group.

``How can we tell whether individual differences in metrics such as BOLD
activations or functional connectivity are actually related to differences in
the underlying neural activity or communication, respectively? One ubiquitous
problem arises in matching different brains such that functionally meaningful
comparisons across subjects are possible in the first place. In a typical fMRI
analysis pipeline, both structural and functional data from individual brains
are spatially warped to a common anatomical space. The most widely used common
space [18] is the MNI152 atlas, to which subjects’ brains are anatomically
warped via a volumetric transform ([19] for review). Such registration is
appropriate for subcortical structures which are inherently volumetric; by
contrast, the cortex is a 2D structure and volumetric alignment does not
properly align folding patterns across subjects. Although switching to cortical
folding-based inter-subject alignment [20,21] has been shown to somewhat reduce
functional mismatch [22,23] (but see [24]), this has not yet become common
practice'' \citep{dubois2016building}.


\subsection{Anatomical alignment}

\todo[inline]{write text in SRM chapter first; put a shorter version in here}


\subsection{Functional alignment}

\todo[inline]{write text in SRM chapter first; put a shorter version in here}


\subsubsection{Hyperalignment}

\todo[inline]{write text in SRM chapter; put a shorter version in here}

\todo[inline]{explain ``left-out subject'' is or ``aims of thesis'' below
does not make sense}


\subsubsection{Shared response model}

\todo[inline]{write text in SRM chapter; put a shorter version in here}


\subsubsection{interim summary}
% interim summary
In summary, naturalistic stimuli ``impose a meaningful timecourse across
subjects while still allowing for individual variation in brain activity and
behavioral responses, and lend themselves to a broader set of analyses than
either pure rest or pure event-related task designs'' \citep{finn2017can}.


\section{Aims of thesis}
%% At the end of the introduction, add a subchapter on the Aims of Thesis in
%% which you describe the research question and the objectives of your work on
%% a maximum of two pages
\todo[inline]{following structure is split: general aims, meta-aim
(reproducibility), specific aims per study; tbh, I do not like it but don't know
how to handle it better (better transitional sentences between subsections?)}

%
Previous work on a group average level has shown that it is possible to combine
\ac{bold} \ac{fmri} with naturalistic stimulation in order to localize areas
correlating with particular brain functions \citep{bartels2004mapping}.
%
This dissertation explores --- while following the principles of open,
transparent, and reproducible science --- whether a movie and the movie's
audio-description that was produced for an visually impaired audience could, in
principle, substitute an established localizer paradigm.
%
Naturalistic stimuli would have substantial advantages (e.g. task demands
compliance, and data quality) over simplified stimuli presently used in
localizer paradigms.
%
Moreover, the time-courses of rich, full-length naturalistic stimuli are
correlating with a variety of different brain functions ranging from low-level
perception (e.g.  luminance) to high-level cognition (social cognition).
%
Thus, a naturalistic stimulus could replace multiple dedicated localizers in the
future to provide a more comprehensive diagnostic routine.
%
However, a two-hour paradigm would be generally inefficient plus unsuitable for
a clinical population.
%
For that reason, this dissertation also explores whether just a part of a movie
or auditory narrative can be used to estimate the location, size and shape of a
``unknown'' subject (i.e. a 'left-out' subject to test the prediction
performance of our model) from a reference group (i.e. the study subjects
providing the data to train our model).


\subsection{Open, transparent, and reproducible science}

% reproducibility crisis
Over the last decade there has been a growing awareness that results of
scientific publications are not reproducible or general scientific findings are
not replicable leading to a ``reproducibility crisis'' or ``replication crisis''
in the sciences \citep{baker2016reproducibility, plesser2018reproducibility,
stupple2019reproducibility, nosek2022replicability}.
% reproducibility: definition
``A study is reproducible if all of the code and data used to generate the
numbers and figures in the paper are available and exactly produce the published
results'' \citep{leek2017most}.
% replicability: definition
A study is replicable if the same analysis of an identical experiment's data
lead to consistent results \citep{dubois2016building, leek2017most}.
%
Hence, on a metalevel, this dissertation aims to meet both the requirements of
open, accessible, shared and transparent science \citep{watson2015will,
fecher2014open} as well as the requirements of a reproducible and replicable
research project:
%
the dissertation follows guidelines and best practices for a) coding and
scientific computing \citep{wilson2014best}, b) procedures and data analyses
\citep{nichols2017best, poldrack2017scanning, poldrack2019establishment}, and c)
sharing code, created data and results \citep{eglen2017toward, nichols2017best,
pernet2015improving}.



\paragraph{input data}

% open data \citep{eglen2017toward}
First, my work will build on top of publicly and freely available \ac{fmri} data that
is part of the \textit{studyforrest} project
(\href{www.studyforrest.org}{studyforrest.org})
%
The studyforrest project is an open science project that aims at providing a
versatile resource for investigating human brain function under quasi-natural
conditions.
%
The core of this dataset are two hour long \ac{bold} \ac{fmri} scans of
participants watching the movie Forrest Gump and listening to the movie's
audio-description of equal length.
%
Since its first publication in 2014 \citep{hanke2014audiomovie}, the
studyforrest project has served as a resource of raw (and preprocessed) data for
international working groups to conduct and publish independent research (s.
\href{www.studyforrest.org/publications.html}{studyforrest.org/publications.html}).
%
The stimulus annotations that have been created over the course of the
dissertation are version-controlled and published in a standardized file format
\citep{haeusler2021speechanno}, and therefore contribute to the studyforrest
project as an resource for the scientific community.


\paragraph{code, analyses, output}

Further, all code is shared to improve reproducibility of results based on
current results and to facilitate their replicability on other datasets.
% automatization
Therefore, data analyses pipelines are implemented in a way that enable
automated processing.
%
Analyses pipelines are not be implemented in commercial, proprietary software
but in freely available and, if possible, open-source software.
% which tools to choose why?
Among potential software packages, we chose the tools that offer the most solid
documentation, and basis of developers and maintainers to ensure long-term
support.
% my code
Custom code written by myself is written in open-source programming languages
(Python and Bash), is version-controlled, documented, and released publicly and
freely accessible.\todo{but: FZJ rules?}
%
All input data, custom code, analysis steps and output data are accessible in
standardized \textit{DataLad} (\href{www.datalad.org}{datalad.org}) datasets.
Since DataLad provides a free and open-source software solution that manages
provenience, distribution, version-control of code and data
\citep{halchenko2021datalad}, all executed steps from downloading input data to
visualizing the results can be rerun to check and validate the dissertation's
results.


\paragraph{publications}
% open-access publishing
Last, because ``nature abhors a paywall'' \citep{dupre2020nature}, publications
describing generated data or reporting results are published in open-access
journals.
% neurovault
Unthresholded statistical maps of all computed statistical $t$-contrasts are
additionally published at Neurovault
(\href{https://neurovault.org/}{neurovault.org}).


\subsection{Specific objectives and hypotheses}

\todo[inline]{imo, this section is supposed to give a roadmap and overview, too}

\subsubsection{A studyforrest extension, an annotation of spoken language in the
German dubbed movie ``Forrest Gump'' and its audio-description}

%
The majority of naturalistic stimuli that have been used in neuroscience have
originally been designed to entertain an audience and not to conduct research.
%
The time-courses of variables (i.e. the \textit{features}) embedded in a
naturalistic stimulus are fixed and thus reproducible [makes sense?].
%
Nevertheless, modeling brain activity correlating with stimulus features in
naturalistic stimuli is a challenging \citep{saarimaki2021naturalistic,
simony2020analysis} because models rely on  extensive stimulus annotations:
%
stimulus annotations need to comprise annotations of features of interest as
well as annotations of potentially confounding features.
%
The lack of extensive annotations of naturalistic stimuli has led to an ``usage
bottleneck'' \citep{aliko2020naturalistic} and might be the main reason why
explicit models of task or stimulus are often ``prohibitively difficult'' to
construct \citep{nastase2019measuring}.

% yeah, fuck it; I'll do it myself
Hence, I extended the studyforrest project with an annotation of spoken
language.
% content
``The annotation provides information about the exact timing of each of the more
than 2500 spoken sentences, 16000 words (including 202 non-speech
vocalizations), 66000 phonemes, and their corresponding speaker''
\citep{haeusler2021speechanno}.
%
Further, we explored the stimulus' confound structure and validated the
annotation's quality by building a model of hemodynamic activity based
information drawn from it.
%
Results suggested that the annotation could be used to model hemodynamic
responses correlating with auditory, semantic information in study 2.


\paragraph{Maybe: what was tried...}

IBM Watson \& Google Cloud text-to-speech


\subsubsection{Processing of visual and non-visual naturalistic spatial
information in the ``parahippocampal place area''}

%
\todo[inline]{text is draft adapted from TeaP talk}
\todo[inline]{print PPA paper and write text here}
\todo[inline]{this is why we do standard analysis, esp. in study2;
\citep{carp2012plurality}; AIM: we want to replicate results with naturalistic
stimuli}

%
A previous analysis \citep{hanke2016simultaneous} revealed significant
correlations between \ac{bold} response time series of audio-visual movie and
its audio-description in spatially corresponding voxels of the ventral medial
cortices (s. Figure 3 in \citep{hanke2016simultaneous}).
%
we investigated if it is possible to localize a visual area, usually
identified via a task-based functional localizer, via naturalistic stimuli.
%
\citep{bartels2004mapping} annotated the occurrence (and perceived intensity)
regarding color, faces, language, and human bodies but, surprisingly, not
scenes.
%
Hence, it is still ``kind of unclear'' is if results generalize from
static pictures to results from a more ecologically valid stimuli
like a movie.
%
It is still unclear is if these results generalize to the auditory domain
\citep{aziz2008modulation}.
%
How can we operationalize the perception of spatial information that is
embedded in our naturalistic stimuli?
%
Notably, we operationalized the assumed perceptual process not just in a
audio-visual movie but also in an exclusively auditory stimulus.
%
Does increased hemodynamic activity in the PPA correlate with auditory spatial
information (group-level)?
%
Can we localize the PPA in individual persons using a more engaging
\& entertaining, but also exclusively auditory stimulus

Compare our results to a classic visual localizer experiment that used blocks of
pictures to results from naturalistic stimulus paradigms.
%
To operationalize the perception of spatial information, we looked at time
points in the stimuli that should correlate with the perception of spatial
information
%
For the movie, we used the movie cuts that basically re-orient the observer in
space.
%
For the audio-descriptions, we semantically categorized nouns spoken by the
narrator in case he used them to describe faces or bodies, locations, buildings,
rooms, and so on.

%
In both stimuli, we looked for time points correlating with other perceptual
processes that we needed to build general linear model contrasts.
%
create events from stimulus annotations;
%
model hemodynamic responses;
%
create contrast(s).

%
Results suggest that you can ``localize a auditory PPA'' but not the ``visual
PPA'' using an auditory narrative; It's similar but different; SRM study will
quantify the prediction performance
%
The audio-description must somehow at some time points trigger responses in the
ventromedial cortex.

%
There are responses in the parahippocampal cortex correlating with
spatial information.
%
But probably our event structure and model is not just an approximation of the
``real'' events structure that is actually correlated with the response series
in the parahippocampal cortex.
%
Hence: build shared response model and use it for prediction
%
2h movie/audio-description is to long anyway


\subsubsection{DataLad: distributed system for joint management of code, data,
and their relationship}

\todo[inline]{well, does not really fit in}


\subsubsection{Using varying amount of data from naturalistic stimulation for
functional alignment to predict results from a task-based functional localizer
paradigm}

\todo[inline]{following is a text adopted/adjusted from the project proposal}

%
All analyses up so far have been performed on 2h-long scans, but any
clinical application must aim to minimize the scan time/cost.
%
We tested whether reliable functional alignment can be achieved with a task-free,
natural stimulation ``calibration'' scan that requires no more acquisition time
than a conventional localizer paradigm.
%
fMRI data acquired from an individual will not be analyzed directly regarding a
specific cognitive function, but will be used to align that individual brain's
voxel space with a common high-dimensional representational reference space.
%
Using movie-evoked brain activity, hyperalignment procedure learns subject-wise
optimal transformations of brain activity into a common representational space.

%
Once aligned, the brain activity of the reference group can be used to predict
the activity of another subject.
%
Using a leave-one-subject-out strategy, individual results of the conventional
localizer will then be projected into the common space, aggregated, and
re-projected into the voxel space of the left-out individual for comparison with
the localizer results for that individual (see Figure 2).
%
Once a valid alignment is established, the inverse transformation is then used
to project functional properties of the common reference into that individual's
voxel space.
%
estimate the minimum scan time requirement for deriving a valid functional
alignment by further reducing the amount of input data of the left-out
individual.
%
We will estimate the trade-off between diagnostic quality and required effective
scan time by progressively reducing the duration of input BOLD fMRI data and
comparing the results of the reduced model to the reference computed from the
full length scan.


\paragraph{just some notes}

\todo[inline]{other contrasts that were tested over the course of the
dissertation: phonemes, grammatical tags, prosody, sex}
