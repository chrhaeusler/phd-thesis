%% The introduction begins with an overview of the topic



\todo[inline]{MH: wenn kumulativ, dann ganze Geschichte in Einleitung; sonst ist
unklar, wie einzelnen Teile zusammenpassen}

%
Brain imaging using with \ac{fmri} of \ac{bold} activity has been used
extensively for almost three decades to investigate perceptual and cognitive
brain functions.
%
Human brain mapping maps functions onto the anatomy of the brain.
%
But typical analysis procedures average (voxel-wise) data of at least 10-15
subjects to improve the \ac{snr}.
%
Consequently, studies emplying an averagign approach do not characterize brain
function at the level of an individual because these models that do not capture
individual brain properties but just a ``common denominator''
\citep{dubois2016building}. \todo{check paper}
%
However, characterization of individual brain function is by far the most
important application of BOLD fMRI in a clinical context.
%
For example, for a diagnosis of brain functions in health and disease, or
pre-surgical screening.
%
This dissertation explores weather naturalistic stimuli, here a Hollywood movie
and and its audio-only variant created for a visually impaired audience, could,
in principle, substitute a traditional, task-based paradigm to identify
the location, size, and shape of a functional area in the brain of individual
subjects.

\todo[inline]{we don't do speech lateralization anymore; imo the neurosurgery
thing should not be mentioned in the intro anymore; better come up with it in
the general discussion}


\section{Functional localization}
%% Further introductory remarks describe the scientific background of the work
%% as precisely as possible. Cite the most important publications and avoid
%% extensive literature reviews.
% a.k.a. "state of research"


\todo[inline]{you need to define ecological and external validity already here
not in section about naturalistic stimuli later}

% definition
Functional areas (or networks) are brain regions in the brain that are assumed
to do shit I need still to define here.
%
Studies subdivided the cerebral cortex into distinctive functional areas whose
\ac{bold} activity is specifically correlated with one particular simplied
stimulus type.
% group level
On a group average level, previous studies presented isolated higher-level
visual features such as scenes, faces, human bodies or tools.
% results from group average studies
Results suggest that domain-specic modules like the \ac{ppa}
\citep{epstein1998ppa}, \ac{ffa} \citep{kanwisher1997ffa}), the occipital face
area \ac{ofa} \citep{pitcher2011occipitalfacearea}, the \ac{eba}
\citep{downing2001bodyarea}), and the \ac{loc} \citet{malach1995loc} exist in
the human brain.
% individual level: localizers
On the level of individual subjects, the most frequently employed paradigm to
characterize location, size, and shape of function areas with BOLD fMRI are
[grammar?] \textit{functional localizers}.
% localizers: definition
Functional localizers are dedicated measurement that aim at isolating and
localizing brain activity correlated with specific perceptual processes (e.g.
different object categories; \citet{kanwisher1997ffa}) or cognitive processes
(e.g. theory of mind; \citet{spunt2014validating}).\todo{is Spunt
individual-level?}
% purpose: ROI
Functional localizers can be used to define an \acp{roi} on the level of an
individual subject to improve the statistical power of the main experiment's
analysis, or to locate brain functions prior to neurosurgery.
% purpose: neurosurgery
Surgical procedures might impact the post-operative quality of life so much
(e.g. concerning cognitive control or speech production) that it potentially
outweighs the therapeutic benefits.
% efficiency
The challenge ist to precisely localize relevant brain areas with limited
resources (time, availability and applicability of diagnostic measures for an
individual patient).
% one localizer = one domain of functions
Importantly, one dedicated localizer paradigm can only map one domain of brain
functions (e.g. retinotopic mapping, higher-visual perception
\citet{kanwisher1997ffa}, or speech perception
\citet{fernandez2001language}).\todo{other localizer papers?}
% fucking inefficient
Hence, functional localizers, despite [or because] being tuned for detection
power by emplying carefully chosen and tightly-controlled, simplified stimuli,
quickly become inefficient if one wants to map many different processes in a
limited amount of time.
% localizer batteries: intro
Researchers have tried to circumvent that fucking problem by creating
time-efficient multi-functional \texit{localizer batteries} for reading,
language comprehension, calculation, motor response, and basic retinotopy was
developed by \citep{pinel2007fast, pinho2018individual,
pinho2020individual}.\todo{other localizer batteries? H.Brain Project?}
% localizer batteries: example
For example, \citet{pinel2007fast} employs a range of dedicated stimuli and
specific tasks participants have to perform in a 5-minute routine.
% task based = shit
Nevertheless, the diagnostic quality of localizer batteries relies heavily on
participants' compliance and comprehension of the task instructions, a criterion
that can be difficult to meet in clinical or pedriatric populations
\citep{eickhoff2020towards, vanderwal2015inscapes, vanderwal2019movies}.
% validity?
Additionally, localizer batteries also rely on carefully chosen and
tightly-controlled, simplified stimuli that are usually presented in blocks do
not resemble how we perceive the real-world during every-day life leading to
questionable external and ecologically validity.\todo{add references}

% iterim summary
In summary, we have two issues: a) validity and compliance, and b) efficiency.
% foreshadowing to next sections & transition to naturalistic stimuli
This dissertation will explore how we can both increase validity by using
\textit{naturalistic stimulus} paradigms as well as increase efficiency by
predicting individual functional topography from data collected in a reference
group.


\section{Naturalistic stimuli}

Ultimately, a major goal of cognitive neuroscience is to reveal how the brain
processes information during everyday perception.
%
Given the questionable ecological validity of traditional functional localizer
paradigms, it is often unclear how functional areas behave in life-like
situations and how they might interact.
%
After all, we do not experience the world around us as separated into small
unidimensional stimuli, but perceive --- through different senses --- a
seemingly continuous and unified world.
%
To address this open question, the usage of \textit{naturalistic
stimuli} have gained popularity.


\paragraph{Definition}

\todo[inline]{add narrative studies because synchronization is also true there}

% definition quote
Naturalistic stimuli are ``a class of stimuli that aim to evoke more
naturalistic patterns of neural responses than traditional controlled artificial
stimuli. Naturalistic paradigms are typically complex and dynamic, and longer in
duration than many conventional stimuli.'' \citep{vanderwal2019movies}.
% movies & narratives
The most popular naturalistic stimuli in neuroscience are movies and audio-only
stories that provide time-locked event structure during a continuous,
complex/rich, dynamic, and often prolonged stimulation.


\paragraph{Ecological validity}
% definition
<Definition here>.
% claim
Naturalistic stimuli promise a higher ecological validity because they, despite
being presented in a laboratory setting, still more closely mimic real-life
experiences \citep{hasson2004intersubject}.\todo{check}


\paragraph{External validity}
% definition
<Definition here>.
% less selection bias
Carefully chosen stimulus sets ``selectively sample from the stimulus population
leading to a stimulus-as-fixed-effect fallacy (Clarc, The
language-as-fixed-effect fallacy: A critique of language statistics in
psychological research)'' \citep{westfall2016fixing}.
%
``The conclusions cannot be generalized to a broader population of stimuli
without risking inflated Type I error  (cf. Donnet S, Lavielle M, Poline JB: Are
fMRI event-related response constant in time? A model selection
answer'' \citep{westfall2016fixing}.
%
Naturalistic stimuli promise a higher ecological validity because they draw a
more representative sample from the ``theoretical population of stimuli that
might have been used'' \citep{westfall2016fixing}.



\paragraph{Early findings}
%
\todo[inline]{add an early audio-paper}

% reviews
Movies and narratives have been used during \ac{fmri}
(s.\citet{hamilton2018revolution, hasson2008neurocinematics,
jaaskelainen2021movies, sonkusare2019naturalistic, saarimaki2021naturalistic}
for reviews), or \ac{eeg} and \ac{meg} (s. \citet{alday2019meg,
kandylaki2019story} for reviews).

%
Early studies have shown that watching a movie or listening to an auditory
narrative synchronized spatiotemporal responses across multiple subjects in a
large part of the brain [give percentages] \citep{hasson2010reliability,
lankinen2014isc-meg} compared to, for example, an unedited video of a
concert/park scene, taken from a fixed viewpoint
\citep{hasson2010reliability}.\todo{but still different}
% This finding could be attributed to a film director's goal to not only direct
% a movie, but also to capture and direct the audience's attention; this can be
% attributed to the way professional movies are shot and edited in order to
% intentionally manipulate the viewers' attentional focus and mental states
% \citep{brown2012cinematography, dancyger2011film-technique}.

%
Functional specialization seems to be preserved
\citep{bartels2004mapping}.\todo{check master thesis, project proposal}



%
``The advent of digital video and programs for controlling presentation of
digital video made it feasible to present dynamic, naturalistic movies in
functional brain imaging experiments. Pioneering studies in 2004 (Bartels and
Zeki, 2004; Hasson et al., 2004) have led to widespread use of such stimuli,
producing clear evidence of their utility for evoking reliable, information-rich
patterns of brain activity over a larger extent of cortex than is activated by
more controlled experiments. Movies also provide a rich context with narrative
structure (Hasson et al., 2008a, 2015; Chen et al., 2017), better hold attention
(Hasson et al., 2008b), and enhance subject compliance (Vanderwal et al.,
2015)'' \citep{haxby2020naturalistic}.

``Whereas controlled experiments minimize extraneous information in stimuli and
tasks, experiments with naturalistic stimuli better simulate the full richness
of natural visual and auditory experience. Movies sample a broad range of brain
states and engage multiple perceptual and cognitive systems in parallel. Even
within a sensory modality, such as vision, different types of information are
layered and simultaneously present in natural movies. The rich, layered
information in movies has allowed concurrent modeling of multiple stages of
perceptual processing (Nishimoto et al., 2011; Huth et al., 2012; Güclü and van
Gerven, 2017). Broader sampling and efficient engagement of multiple systems
moti- vated the use of movie-viewing data as the basis for developing algo-
rithms for aligning functional anatomy (Conroy et al., 2009, 2013; Sabuncu et
al., 2010; Haxby et al., 2011; Chen et al., 2015; Guntupalli et al., 2016,
2018). Researchers have begun to dissociate overlappingneural representations in
naturalistic paradigms using analytic methods such as multivariate pattern
classification (MVPC; Haxby et al., 2001, 2014), representational similarity
analysis (Kriegeskorte et al., 2008; Kriegeskorte and Kievit, 2013), and forward
encoding models (Naselaris et al., 2010; Nunez-Elizalde et al., 2019).
Critically, the relative contri- butions of different types of information are
more faithfully represented in naturalistic stimuli than in controlled
experimental manipulations'' \citep{haxby2020naturalistic}.

``We argue here that studies of brain activity evoked by viewing a naturalistic,
dynamic movie better reflect the statistics of natural viewing in a complex,
cluttered, changing, and continuous visual environment. Using naturalistic,
dynamic stimuli has the potential to lead to surprising new insights about how
neural resources for visual perception are structured and allocated. We refer
here to naturalistic, dynamic stimuli as those that present visual episodes,
sometimes accompanied by a sound- track, with the complexity of natural scenes.
We examine the use of naturalistic audiovisual clips or movies and distinguish
these from still images and from highly-controlled or schematic videos (e.g.
point-light displays or videos of isolated body parts performing simple actions
with no context), which have an intermediate status between non- naturalistic
stimuli and naturalistic, complex videos'' \citep{haxby2020naturalistic}.


\paragraph{Better compliance}

\todo{check also \citep{eickhoff2020towards}}
%
The to be diagnosed individuals were scanned while they were watching a movie
without any explicit task or.
%
Improved subject compliance and compatibility due to minimal instruction
requirements (e.g., no fixation of eye gaze) and task demands (no task except
enjoying the movie or audiobook).
%
These minimal instructions make naturalistic paradigms appropriate especially
for elderly or visually impaired persons.
%
Lastly, naturalistic stimuli offer improved data quality, as an interesting and
easy-to-follow stimulus is more capable of putting a participant at ease in the
otherwise claustrophobic, uncomfortable and noisy fMRI scanner.

``Relative to traditional fMRI experiments that typically use highly controlled
stimuli, naturalistic stimuli are more ecologically valid (Zaki and Ochsner,
2009; Hasson and Honey, 2012; Adolphs et al., 2016; Hamilton and Huth, 2018),
convey rich perceptual and semantic information (Bartels and Zeki, 2004; Huth et
al., 2012, 2016) and more fully sample neural representational space (Haxby et
al., 2011, 2014)''\citep{nastase2019measuring}.

``Recent work (Vanderwal et al., 2015) also suggests that naturalistic stimuli
may improve subject compliance (in terms of wakefulness and head motion relative
to, e.g. rest), which is particularly important when scanning patient
populations and children. As mentioned previously, different stimuli will
variably synchronize different brain systems; for example, engaging,
Hollywood-style movies may yield greater, more widespread ISCs than real-life,
unedited videos (Hasson et al., 2010; Cohen et al., 2017)''
\citep{nastase2019measuring}.

``Naturalistic paradigms do not aim to replace the classic, controlled
neuroimaging paradigms (Sonkusare et al., 2019). Due to their complexity and
current limitations in understanding the statistical properties of different
features in naturalistic conditions, naturalistic stimuli are not optimal for
model development (see, e.g., Rust and Movshon, 2005). Controlled experiments
are still needed for hypothesis testing and developing models, while
naturalistic stimuli are best employed to test models in ecologically valid
settings and to expand them to situations where context matters
more'' \citep{saarimaki2021naturalistic}.


\paragraph{studyforrest}
%
\todo[inline]{following part can heavily be shortened}
\todo[inline]{maybe shift it into open science section}
%
The studyforrest project is an open science project that aims at providing a
versatile resource for investigating human brain function under quasi-natural
conditions.
%
The core of this dataset are two hour long BOLD fMRI scans of participants
watching the movie Forrest Gump (and also listening to a version for the blind
in another scan of equal length).
%
Since its first publication in 2014 \citep{hanke2014audiomovie}, this resources
has led to eight independent studies of international research groups outside
Magdeburg that were published in peer-reviewed journals
(http://studyforrest.org).
%
Based on my knowledge about cinematographic editing techniques, I annotated the
~870 movie cuts with respect to depicted major locations, scene settings,
within-scene rooms and perspectives \citep{haeusler2016cutanno}.
%
The annotation served as prerequisite to investigate cognitive functions, such
as spatial reorientation, perspective taking, and memory retrieval for known
spatial layouts in terms of their occurrence in the movie stimulus.

For 15 participants in the studyforrest datasets two different full-length movie
scans are readily available: one with the normal audio-visual movie
\citep{hanke2016simultaneous}, and a second one with an audio-only movie variant
(originally produced for a visually impaired audience) that is time-locked to
the audio-visual version \citep{hanke2014audiomovie}.

``A case study for successfully sharing naturalistic stimuli is studyforres t.org
(Hanke et al., 2014, 2016), a dataset which includes audio-only and audio-visual
viewings of the movie Forrest Gump during fMRI acquisi- tion. Study Forrest is a
data collection and curation effort designed to serve as a community resource
for new discoveries, in the tradition of distributed science collaborations such
as the International Genetically Engineered Machine competitions. As of October
2019, 29 unique studies had been published using the studyforrest.org dataset,
17 of which were published without any of the original authors of the data
release. This is possible in large part thanks to the richness of naturalistic
stimuli, where the same movie can be used for both task-free as well as
stimulus-driven analyses, with the original stimulus re-annotated for particular
features of interest. For example, studyforrest.org has been used to test
cerebro- vascular biomarkers (Voss et al., 2017) but, among other features, was
also annotated for expressed emotion (Labs et al., 2015) which later informed a
study on emotion encoding gradients in the brain (Lettieri et al., 2019).A case
study for successfully sharing naturalistic stimuli is studyforres t.org (Hanke
et al., 2014, 2016), a dataset which includes audio-only and audio-visual
viewings of the movie Forrest Gump during fMRI acquisi- tion. Study Forrest is a
data collection and curation effort designed to serve as a community resource
for new discoveries, in the tradition of distributed science collaborations such
as the International Genetically Engineered Machine competitions. As of October
2019, 29 unique studies had been published using the studyforrest.org dataset,
17 of which were published without any of the original authors of the data
release. This is possible in large part thanks to the richness of naturalistic
stimuli, where the same movie can be used for both task-free as well as
stimulus-driven analyses, with the original stimulus re-annotated for particular
features of interest. For example, studyforrest.org has been used to test
cerebro- vascular biomarkers (Voss et al., 2017) but, among other features, was
also annotated for expressed emotion (Labs et al., 2015) which later informed a
study on emotion encoding gradients in the brain (Lettieri et al., 2019)''
\citep{dupre2020nature}.


\section{Functional alignment (in order to predict)}

%
Lastly, increased validity of derived transformation for functional alignment by
sampling a more diverse set of mental states that reflect (confound) statistics
of the natural environment, and enable investigation of the acquired data for a
variety of research questions (e.g. visual or auditory perception, spatial
cognition; emotion; music, speech or social perception)

%
While the functional alignment can also be applied to fMRI data from stimulation
paradigms with simplified stimuli, the transformations for functional alignment
have greatly diminished general validity \citep{haxby2011common}, presumably
because such experiments sample a sparser range of brain states
\citep{guntupalli2016model}.

\subsubsection{Anatomical alignment (in order to predict)}

The currently dominating approach in neuroimaging group analyses relies on
topological constraints defined by an three-dimensional, anatomical reference
space (e.g. the MNI152 template brain).
%
Surface-based: e.g. \citep{weiner2018defining}.

\paragraph{Functional alignment in general}
%
An alternative approach to individual localization has been proposed by
\citet{haxby2011common}.
%
They (and e.g. \citet{jiahui2019predicting}) also predicted the location, form,
and size of  target brain areas in ventral temporal cortex from dedicated
localizer scans of other individuals.
%
The key difference of this approach is to rely on similarity of representational
geometry of brain activity patterns and aligning individual brains into a
multi-dimensional a group space.

\todo[inline]{shift from focussing on (connectivity) hyperaligment more to
shared response model}


\paragraph{Hyperalignment}

\todo[inline]{why didn't we used connectivity alignment in the first place?}
%
\citet{haxby2011common} used BOLD response patterns evoked by a 2h action movie
to derive a common representational space.
%
The algorithm, namely hyperalignment, derives this representation using a
variant of Procrustes analysis and computes invertible (orthonormal)
transformations from each individual brain’s voxel-space into this common
reference space.
%
Importantly, the study also showed that an individual's \ac{ffa} or the
\ac{ppa}, can be localized precisely based on data from a reference group.
\todo{explain FFA, PPA}
%
The same authors later showed that this approach can be extended to predict
functional organization across large proportions of the cortical surface, for
example to predict the represented visual field coordinate in visual cortex
based on retinotopic mapping scans of other individuals
\citep{guntupalli2016model}.

\todo[inline]{we do not do connectivity hyperalignment but SRM now!}
%
In his doctoral thesis, recently submitted to the Faculty of Natural Sciences in
Magdeburg, Falko Kaule showed that congruent time-locked BOLD responses across
subjects (i.e. all subjects watching the exact same full-length movie) as used
by Haxby and colleagues are not required to derive a valid alignment of
individuals with a common representational space \citep{kaule2017examination}.
%
Comparable prediction performance can be achieved by using \textbf{functional
connectivity patterns} (correlation of a voxel's time series with reference
regions in the same brain).
%
This finding enables, in principle, the use of different ``calibration'' scans
to determine an alignment with a common representational space, for example with
an age-appropriate stimulus, or a shortened scan time to fit into a particular
clinical schedule.
%
Once a valid alignment is established, known functional properties of a
(normative) reference, derived from extensive scans and analysis of other
subjects, can then be projected into the respective individual voxel space (s.
Fig. 1 in \citep{nishimoto2016lining}).


\paragraph{Shared response model}

\todo[inline]{Put following stuff into study3 and then a shorter version here}

Problem: ``One of the main obstacles in leveraging brain activity across
subjects is the considerable heterogene- ity of functional topographies from
individual to individual. Variability in functional–anatomical correspondence
across individuals means that even high-performing anatomical alignment does not
ensure fine-grained functional alignment [e.g., 36]. As an example, multi-voxel
pattern analysis models that perform well within subjects often degrade in
performance when evaluated across subjects [e.g., 37, 38]''
\citep{kumar2020brainiak}.

``SRM [13], alongside other methods of hyperalignment [39–41], aims to resolve
this alignment prob- lem by aligning based on functional data. SRM estimation is
driven by the commonality in func- tional responses induced by a shared stimulus
(e.g. watching a movie). Unlike ISC analysis, which presupposes (often very
coarse) functional correspondence, SRM isolates the shared response while
accommodating misalignment across subjects. SRM decomposes multi-subject fMRI
data into a lower-dimensional shared space and subject-specific transformation
matrices for projecting from each subject’s idiosyncratic voxel space into the
shared space (Figure 2). Each of these topo- graphic transformations effectively
rotates and reduces each subject’s voxel space to find a subspace of shared
features where the multivariate trajectory of responses to the stimulus is best
aligned. These shared features do not correspond to individual voxels; rather,
they are distributed across the full voxel space of each subject; each shared
feature can be understood as a weighted sum of many voxels''
\citep{kumar2020brainiak}.

``Transformations estimated from one subset of data can be used to project
unseen data into the shared space. Projecting data into shared space increases
both temporal and spatial ISC (by design), and in many cases improves
between-subject model performance to the level of within-subject performance.
Between-subject models with SRM can, in some cases, exceed the performance of
within-subject models because (a) the reduced-dimension shared space can
highlight stimulus- related variance by filtering out noisy or
non-stimulus-related features, and (b) the between-subject model can effectively
leverage a larger volume of data after functional alignment than is available
for any single subject. De-noised individual-subject data can be reconstructed
by projecting data from the reduced-dimension shared space back into any given
subject’s brain space.  Furthermore, in cases where each subject’s unique
response is of more interest than the shared signal, SRM can be used to factor
out the shared component thereby isolating the idiosyncratic response for each
subject [13]'' \citep{kumar2020brainiak}.

``Building on the initial probabilistic SRM formulation [13, 42], several
variants of SRM have been developed to address related challenges. For example,
a fast SRM implementation has been introduced for rapidly analyzing large
datasets with reduced memory demands [43]. The robust SRM algorithm tolerates
subject specific outlying response elements [44], and the semi-supervised SRM
capitalizes on categorical stimulus labels when available [45]. Finally,
estimating the SRM from functional connectivity data rather than response time
series circumvents the need for a single shared stimulus across subjects;
connectivity SRM allows us to derive a single shared response space across
different stimuli with a shared connectivity profile [46]''
\citep{kumar2020brainiak}.

``Figure 2: Schematic of the shared response model (SRM). (a) Data are typically
split into a training set (light gray) used to estimate the SRM and a test set
(dark gray) used for evaluation. The SRM is estimated from response time series
from the training set for multiple subjects (top left; transposed here for
visualization). The multi-subject response time series are decomposed into a set
of subject- specific orthogonal topographic transformation matrices and a
reduced-dimension shared response space. The learned subject-specific
topographic bases can be used to project test data (bottom left) into the shared
space. This projection functionally aligns the test data''
\citep{kumar2020brainiak}.

\citep{chen2015reduced}


% iterim summary
In summary, naturalistic stimuli ``impose a meaningful timecourse across
subjects while still allowing for individual variation in brain activity and
behavioral responses, and lend themselves to a broader set of analyses than
either pure rest or pure event-related task designs'' \citep{finn2017can}.


\section{Aims of thesis}
%% At the end of the introduction, add a subchapter on the Aims of Thesis in
%% which you describe the research question and the objectives of your work on
%% a maximum of two pages

%
\todo[inline]{titles of subsections are too long}

%
Previous work has shown that it is, in principle, possible to combine BOLD fMRI
with a rich naturalistic stimulus for the purpose of localizing areas associated
with particular brain functions \citep{bartels2004mapping}.
%
This can be approached by either modeling specific stimulus features, or by
using the high-dimensional nature of such a stimulus to derive a common
representational space for aligning functional properties of cortex.
%
However, there has been no study aimed at replacing an established localizer
paradigm with a naturalistic stimulus as a short diagnostic routine.
%
A part of a movie of the same length as a dedicated localizer experiment would
have substantial advantages (e.g. task demands compliance, and data quality)
over simplified stimuli presently used in localizer paradigms.
%
Given that naturalistic stimuli offer a rich stimulation correlating with a
variety of different brain functions, they could replace multiple dedicated
localizers and thus offering a more comprehensive and efficient diagnostic in a
similar or even less amount of time.


\paragraph{Auditory stimulus to localize visual area}
%
This investigation will also provide insight if it is feasible to produce an
audio-only localizer paradigm as an alternative for studies where no visual
stimulation is feasible or desired.
%
Previous results show that there are significant voxel-wise BOLD response time
series correlations between the two datasets in brain areas associated with
speech and story processing (s. Figure 3 in \citep{hanke2016simultaneous}).

%
\todo[inline]{following is actually not true anymore}

For this purpose, I want to evaluate two strategies:

\begin{itemize}
    \item direct modeling of a natural stimulus and
    \item prediction via functional alignment to a reference population
\end{itemize}



\subsection{Open, transparent, and reproducible science}

\todo[inline]{this could also be the last subsection in the overview (s. above)
as a (meta thing regardin) state of research}

\todo[inline]{to increase Reproducibility, automated, transparency, openly
shared; all steps from accquiring the dataset to visualization of results can be
reproduced and validated by independent researcher and validated by independent
researchers; documented code, output etc.; DataLad for distribution; provenance
tracking, repository with version-control system}

\todo{no commercial tools and close, proprietary code, but all free and - if
possible - open-source tools/packages, all data used as input from publicly
available, transparency, all code will be released publicly and freely,
reproducibility, all generated code, analyses, results will be published and
freely accessible, sharing, check}

\todo{following best practices for coding and scientific computing
\citep{wilson2014best}, data analysis \citep{nichols2017best}, and sharing code
\citep{eglen2017toward}, created data and results \citep{nichols2017best}}

\todo[inline]{best practices for transparent and reproducible research
\citep{poldrack2017scanning}}

%
All required algorithms and analyses will be implemented in a way that enables
automated processing.
%
This will facilitate documentation and efficient processing of large number of
datasets.
%
Implementations will be based on open-source software tools to guarantee a
maximum level of reproducibility, and relative ease of long-term maintenance
\citep{eglen2017toward}.

% datalad
All input data, code and output data will be provided as standardized and
easy-to-access \textit{DataLad} (datalad.org) datasets, a free and open-source software
solution that manages provenience, distribution, version-control of code and
data, and provides as structured record of all executed steps from downloading
the input data to visualizing the results.
\citep{halchenko2021datalad}. The information of past command-line invocations
can be used to re-execute commands to rerun every step and validate the results.
%
The goal is to provide researchers with efficient, validated, ready to use
solutions (stimulation, MR sequence configuration, analysis software) for
functional localization that can be incorporated into their study protocols.
%
We share results and code in standardized file and data formats und use free and
open-source software (FSL, Python packages like ...). Only use publicly
available input data \citep{eglen2017toward}.

``The entire analysis workflow (including both successful and failed analyses)
would be completely automated in a workflow engine and packaged in a software
container or virtual machine to ensure computational reproducibility. All data
sets and results would be assigned version numbers to enable explicit tracking
of provenance. Automated quality control would assess the analysis at each stage
to detect potential errors'' \citep{poldrack2017scanning}.

``The paper would be written using a literate programming technique in which the
code for figure generation is embedded within the paper and the data depicted in
figures are transparently accessible. The paper would be distributed along with
the full codebase to perform the analyses and the data necessary to reproduce
the analyses, prefera­ bly in a container or virtual machine to enable direct
reproducibility. Unthresholded statistical maps and the raw data would be shared
via appropriate community repositories, and the shared raw data would be format­
ted according to a community standard, such as the Brain Imaging Data Structure
(BIDS)73, and annotated using an appropriate ontology to enable automated
meta-analysis'' \citep{poldrack2017scanning}.

``Naturalistic approaches have a strong potential to further transparent and
openly shared neuroscientific research.  The richness of the data sets collected
inherently favours the analysis of data to address multiple questions or their
reanalysis to address questions other than those exam- ined in the initial
analysis. For example, if one researcher conducts an experiment using stories
and is mainly inter- ested in syntactic processes, the nature of the stimuli
opens up the possibility for other researchers to model phonological processes,
for example, if the data is openly accessible and annotated appropriately. A
stan- dardised way of sharing data from naturalistic exper- imental paradigms is
needed, in order to ensure an easy navigation through the “maze” of openly
shared data and an informed decision regarding which datasets are suitable for
answering specific hypotheses. Ideally, researchers would not only share the
neuroimaging/ electrophysiological data, but also the details of the para- digm
including specific time indications, task descrip- tions and the stimulus files as
presented to the participant. This proposal is different from previous task
ontologies (Poldrack \& Gorgolewski, 2014; Turner \& Laird, 2012) in that it
captures the specifics of a more eco- logically valid approach and the use of a
natural task, a design which has not yet been incorporated into existing task
ontologies'' \citep{kandylaki2019story}.

%
``A visionary aim of openly sharing data and meta-data of more ecologically
valid designs is the holistic under- standing of human brain function.
Researchers would be able to choose carefully from correctly tagged data- sets
and model brain responses using big data. Then, assisted by current methods in
computer science such as machine learning and artificial neural networks,
researchers could attempt to re-construct brain function in an ecologically
valid manner (see also Hasson et al., 2018)'' \citep{kandylaki2019story}.


\subsection{Creation of naturalistic stimulus annotations}

%
Movies are designed to entertain the audience and not to conduct research.
%
Variables (i.e. the ``features'' embedded in the naturalistic stimuli) might be
confounded.
%
Moreover, features of interest might be highly correlated, making a it
impossible to controlling them statistically.
%
Hence, researchers often rely on data-driven methods to analyze the data
``because they do not require an explicit model of the task or stimulus'' and/or
``constructing such a model may be prohibitively difficult.''
\citep{nastase2019measuring}.
%
Which is true but the wrong mindset that lead to a lack of annotation in most
datasets derived from naturalistic paradigms.
%
From data-driven approaches, we gained a lot of knowledge, but data-driven
approaches essentially fall short in case you want to correlate discovered brain
activation patterns with psychological processes.
%
``annotation bottleneck'' \citep{aliko2020naturalistic}i.

``Nevertheless, naturalistic stimuli add additional layers of complexity to the
technical difficulty of data sharing. Some of these – such as recording
acquisition and presentation timings as well as annotating stimuli for features
of interest – are well-recognized from the task-based neuroimaging literature
and implemented in existing standards such as BIDS'' \citep{dupre2020nature}.


``Modeling the stimulus and task properties is a central challenge with
naturalistic paradigms and one that has great potential if solved (Simony and
Chang, 2020) \citep{saarimaki2021naturalistic}''.
%
``However, modeling the naturalistic stimulus requires a way to extract
time-varying, emotion-related processes. The interpretations based on combining
stimulus-driven features with brain imaging data are based on two critical
assumptions: first, that the researcher knows which features of the stimulus are
driving the brain activity, and second, that these features have been modeled
accurately (Chang et al., 2020). Following the terminology in naturalistic
studies focusing on visual, object, and semantic features (e.g., Huth et al.,
2012, 2016), the term emotion feature refers to any emotion-related variable
that we can continuously extract from either the stimulus or the observer
(Figure 1)'' \citep{saarimaki2021naturalistic}.


\subsection{Comparison of results from a task-based functional localizer to results from naturalistic stimulus paradigms}
%
It is likely that a rich natural stimulation evokes more widespread network
activity, compared to a simplified stimulus that additionally requires the
subject to perform a task.
%
Hence: compare task-free localizer to dedicated task-based speech localizer
%
compare diagnostic performance of the task-free movie / audio-description fMRI
recordings will be compared to the results of the localizer paradigm.
%
a specific diagnostic contrasts can be selected.


\subsection{Using varying amount of data from naturalistic stimulation for
functional alignment  to predict results from a task-baed functional localizer
paradigm}
%
All analyses up to this point will have been performed on 2h-long scans, but any
clinical application must aim to minimize the scan time/cost.
%
We test wether reliable functional alignment can be achieved with a task-free,
natural stimulation ``calibration” scan that requires no more acquisition time
than a conventional localizer paradigm.
%
We will estimate the trade-off between diagnostic quality and required effective
scan time by progressively reducing the duration of input BOLD fMRI data and
comparing the results of the reduced model to the reference computed from the
full length scan.

\paragraph{Predict ROI from a reference group}
%
fMRI data acquired from an individual will not be analyzed directly regarding a
specific cognitive function, but will be used to align that individual brain's
voxel space with a common high-dimensional representational reference space.
%
Each axis in this common space can be seen as a kind of cortical tuning
function.
%
The orthonormal transformation of an individual voxel space into this common
reference reflects the particular linear combination of voxel response time
series with respect to each common space component.
%
Using a leave-one-subject-out strategy, individual results of the conventional
localizer will then be projected into the common space, aggregated, and
re-projected into the voxel space of the left-out individual for comparison with
the localizer results for that individual (see Figure 2).
%
Using movie-evoked brain activity, hyperalignment procedure learns subject-wise
optimal transformations of brain activity into a common representational space.
%
Once aligned, the brain activity of the reference group can be used to predict
the activity of another subject.
%
The localization of speech areas will also be performed in the common reference
space directly (the stimulation time axis is preserved in the common space), and
the results will be projected into the voxel space of the left-out individual,
where they can be compared with the localisation result derived from that
subject’s movie scan.
%
Once a valid alignment is established, the inverse transformation is then used
to project functional properties of the common reference into that individual's
voxel space.


\paragraph{minimum length of stimulus; ``calibration scan''}
%
minimum data requirement for reliable functional alignment (within-subject
test-retest???
%
estimate the minimum scan time requirement for deriving a valid functional
alignment by further reducing the amount of input data of the left-out
individual.
\paragraph{just uses an intersecting time-series between stimulus sets}
%
The joint dataset will enable me to study whether a functional alignment to the
common reference space can be performed based on a short scan.


\paragraph{just some notes}

\todo[inline]{other contrasts that were tested over the course of the
dissertation: phonemes, grammatical tags, prosody, sex}
