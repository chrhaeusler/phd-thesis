%% The introduction begins with an overview of the topic
\todo[inline]{reduce number of literal quotes; give page number in literal
quotes?}

\todo[inline]{delete unnecessary sub(sub)sections / captions to turn them into
paragraphs}

\todo[inline]{write \textit{new terms} in italics}

\todo[inline]{give more hints in advanced about auditory naturalistic stimulus /
auditory PPA?}


% introductory quote
``A remarkable feature of the vertebrate brain is the anatomical specialization
of cortical regions for the processing of different types of information. Since
the late 19th century, it has been recognized that restricted lesions of the
human brain result in location-specific sensory, motor or cognitive deficits''
\citep{cohen1994localization}.
% human brain mapping
Contemporary \textit{Human brain mapping} \citep[e.g.,][]{raichle2009brief}
explores in vivo the brain's topographic organization
\citep[e.g.,][]{eickhoff2018topographic} by specifying ``in as much detail as
possible the localisation of function in the human brain''
\citep{savoy2001history}.
% fMRI + BOLD
Brain imaging studies have extensively used \ac{fmri} since the early 1990s to
measure \ac{bold} activity as a proxy for neural activation.
% higher visual areas
For example, replicated findings in the domain of higher-visual perception
suggest that category-selective brain regions like the parahippocampal place
area \citep{epstein1998ppa, epstein1999parahippocampal} or the fusiform face
area \citep{kanwisher1997ffa, kanwisher2006fusiform} exhibit significantly
increased \ac{bold} activity correlated with a ``preferred'' stimulus class:
%
the \ac{ppa} responds more strongly when participants are watching pictures of
scenes and landscapes compared to pictures of, e.g., faces;
%
vice versa, the \ac{ffa} responds more strongly when participants are watching
pictures of faces compared to pictures of scenes.
%
%\citep{talairach1967atlas}
The precise anatomical location of functionally defined regions [measured in
Talairach or MNI coordinates] varies anatomically from person to person
\citep{friston2006critique, saxe2006divide}.

%
Nevertheless, studies in functional imaging research have usually averaged data
across study participants for
%
practical (e.g., limited scan time per subject),
%
statistical reasons (e.g., improved \ac{snr}),
%
or to generalize from study participants to a broader (neurologically healthy)
population.
%
Consequently, studies that average data across study participants may draw
population-level inferences but, on the flip side, lose individual-level
information and may capture just the ``common denominator''
\citep{pinel2007fast} of functional patterns.
%
However, in order advance human brain mapping towards a clinical application
that assesses health and disease, fMRI data need to be interpreted on the level
of an individual person \citep{dubois2016building, wegrzyn2018thought}.



\section{Functional localization}

%% Further introductory remarks describe the scientific background of the work
%% as precisely as possible. Cite the most important publications and avoid
%% extensive literature reviews.
% a.k.a. "state of research"



% intro to functional localizer
\textit{Functional localizers} \citep[cf.,][for reviews]{saxe2006divide,
friston2006critique} are \ac{fmri} experiments that specifically aim to
characterize the \textit{topography} (i.e. the location, size, and shape) of
functional areas correlating with perceptual or cognitive processes like the
perception of object categories \citep{kanwisher1997ffa} or theory of mind
\citep{spunt2014validating}.

% ``The ROI approach is advantageous for four reasons:
% 1) it allows hypothesis-driven comparisons of signals within independently
% defined ROIs across many different conditions,
% 2) it increases statistical sensitivity in multisubject analyses [Nieto–
% Castañón and Fedorenko 2012],
% 3) it reduces the number of multiple comparisons present in whole-brain
% analyses [Saxe et al.  2006], and
% 4) it identifies ROIs in each participant's native brain space''
% \citep{rosenke2021probabilistic}.

% constrain the number of voxels in voxel-wise analyses or

% purpose: ROI improve the statistical power of the main experiment's analysis
Currently, functional localizers are often used as a separate experiment to
identify a subject-specific functional \acp{roi} that constrain the number of
voxels under investigation in the main experiment \citep{poldrack2007region,
saxe2006divide}.
% clinical application
Functional localizers might also be used as a diagnostic procedure, e.g., prior
to neurosurgery \citep[cf.][]{silva2018challenges, szaflarski2017practice}.
% cons: designed for detection power
However designed to maximize detection power, localizers employ carefully
chosen, tightly controlled, and simplified stimuli presented in a block-wise
manner often accompanied by a task to keep study participants attentive.
% one localizer = one domain of functions
Consequently, one localizer paradigm can usually map just one domain of brain
functions.
% batteries: intro
In order to map many different processes despite limited resources, researchers
have developed more time-efficient, multi-functional \textit{localizer
batteries} \citep[e.g.,][]{barch2013function, drobyshevsky2006rapid,
pinho2018individual, pinho2020individual, pinel2007fast}.
% batteries: example
For example, \citet{pinel2007fast} employ a range of dedicated stimuli and
specific tasks in a 5-minute routine to map processes of ``auditory and visual
perception, motor actions, reading, language comprehension, and mental
calculation'' \citep{pinel2007fast}.
% but still...
Nevertheless, the current paradigms face two challenges.
% validity?
From a theoretical point of view, localizers rely on selectively sampled,
tightly controlled stimuli presented in blocks, and do not resemble how we
perceive the real world outside of the laboratory during everyday life.
% compliance?
From a practical point of view, localizers depend on the comprehension of the
task instructions and the participants' compliance during the scan session,
criteria that can be difficult to meet in clinical or pediatric populations
\citep{eickhoff2020towards, vanderwal2015inscapes, vanderwal2019movies}.


\section{Naturalistic stimuli}
% intro to "real-life" neuroscience
Since a major goal of neuroscience is not to reveal how the brain responds to
blocks of stimuli presented in a laboratory setting but how the brain processes
information during everyday perception, \textit{naturalistic stimuli} are
gaining popularity in neuroimaging.

% definition quote
Naturalistic stimuli are ``a class of stimuli that aim to evoke more
naturalistic patterns of neural responses than traditional controlled artificial
stimuli. Naturalistic paradigms are typically complex and dynamic, and longer in
duration than many conventional stimuli.'' \citep{vanderwal2019movies}.
% movies & narratives
The currently most popular naturalistic stimuli in neuroscience are movies and
auditory narratives \citep[cf.][for reviews]{jaaskelainen2021movies,
jaaskelainen2020neural} that provide a time-locked event structure that samples
a broad range of brain states, and engage multiple perceptual and cognitive
systems in parallel \citep{haxby2020naturalistic}.


\subsection{Pro (+ early findings)}
%
Therefore, naturalistic stimuli have several advantages over traditional
paradigms.


\subsubsection{Validity}
% Carefully chosen stimulus sets ``selectively sample from the stimulus
% population leading to a stimulus-as-fixed-effect fallacy [Clarc, The
% language-as-fixed-effect fallacy: A critique of language statistics in
% psychological research]'' \citep{westfall2016fixing}. ``The conclusions cannot
% be generalized to a broader population of stimuli without risking inflated
% Type I error  [cf. Donnet S, Lavielle M, Poline JB: Are fMRI event-related
% response constant in time? A model selection answer]''
% \citep{westfall2016fixing}.

From a theoretical point of view, naturalistic stimuli promise an increased
extent of both subtypes of external validity, namely population validity and
ecological validity \citep{bracht1968external}.
%
\textit{Population validity} refers to the extent to which inferences drawn from
an experiment's results may generalize from the experiment's sample of subjects
and stimuli to the total population of potential subjects and stimuli
\citep{bracht1968external, westfall2016fixing}.
%
\textit{Ecological validity} refers to the extent to which inferences drawn from
an experiment's results may generalize from the experiment's setting, stimuli,
and task to nonexperimental situations \citep{bracht1968external,
orne1962social, schmuckler2001ecological}.
% external validity
Naturalistic stimuli promise an increased population validity of stimuli because
the \textit{stimulus features} (i.e. the stimulus classes or the variables) that
are embedded in a naturalistic stimulus represent a more random sample from the
total population of stimuli that might have been used
\citep{westfall2016fixing}.
% ecologically validity
Naturalistic stimuli also promise a higher ecological validity
\citep{hasson2012future} because they more closely resemble how we visually and
auditorily experience our environment outside of the scanner bore.


\subsubsection{Early findings}
% reviews
Consequently, audio-visual movies and spoken narratives have been used during
\ac{fmri} \citep[cf.][for reviews]{hamilton2018revolution,
hasson2008neurocinematics, sonkusare2019naturalistic,
saarimaki2021naturalistic}, and \ac{eeg} or \ac{meg} data acquisition
\citep[s.][for reviews]{alday2019meg, kandylaki2019story}.
% early findings: spatiotemporal synchronization
Early studies have shown that watching a movie \citep{hasson2004intersubject,
hasson2008neurocinematics, hasson2010reliability} or listening to a narrative
\citep{lerner2011topographic, wilson2008beyond} reliably synchronizes
spatiotemporal responses across multiple subjects in a large part of the brain
compared to, for example, an unedited video of a concert taken from a fixed
viewpoint \citep{hasson2004intersubject, hasson2008neurocinematics,
hasson2010reliability, lerner2011topographic, wilson2008beyond}.
% Bartels (2004)
Importantly in the context of the present thesis, a pioneering study
\citep{bartels2004mapping} revealed that functional specialization of cortical
areas is preserved during complex, life-like stimulation.
% conclusion: naturalistic stimuli as a localizer
Hence, results of \citet{bartels2004mapping} suggest that naturalistic stimuli
could, in principle, be used as a more life-like paradigm in order to substitute
traditional localizer paradigms.


\subsubsection{Compliance}
% a film director's goal is not just to direct a movie, but also to capture and
% direct the audience's attention; professional movies are shot and edited in
% order to intentionally manipulate the viewers' attentional focus and mental
% states \citep{brown2012cinematography, dancyger2011film-technique}.

% engaging & essentially no task
From a practical perspective, naturalistic stimuli require minimal instructions
given by the experimenter, and place minimal task demands on a study participant
(i.e.  no task except enjoying the movie or audio story).
%
Further, movies and narratives provide an interesting and engaging stimulation
that can put participants at ease in the claustrophobic, uncomfortable, and
noisy MRI scanner.
%
Therefore, naturalistic stimuli promise increased data quality due to decreased
fatigue and head movement, especially in the case of children
\citep{vanderwal2015inscapes}, and possibly psychiatric
\citep{eickhoff2020towards} or elderly persons.


\subsection{Contra}
% intro
Nevertheless, naturalistic stimuli also have disadvantages over traditional
paradigms.
% produced for commercial use
First, the majority of naturalistic stimuli that have been used in neuroimaging
have been created by professional production companies for commercial purposes
and not in order to conduct research.
% temporal structure not explicitly known
Thus, the temporal structure of stimulus features embedded in professionally
produced naturalistic stimuli is fixed, and therefore reproducible, but
initially not explicitly known.
% challenging analysis
Consequently, modeling brain activity correlating with stimulus features
embedded in a stimulus' time course is challenging
\citep{saarimaki2021naturalistic, simony2020analysis} because such models, like
a traditional \textit{\ac{glm}}, rely on the stimulus features being annotated.
% lack of annotations = bottleneck
The lack of extensive annotations has led to a ``usage bottleneck''
\citep{aliko2020naturalistic} and might be the main reason why explicit models
of embedded stimulus features are ``notoriously'' \citep{richard2019fast}, if
not ``prohibitively'' \citep{nastase2019measuring} difficult to construct.
% 120 minutes is too long
Second, presenting a full feature film usually lasting 90 to 120 minutes is
inappropriate as an individual diagnostic procedure considering practical and
monetary constraints in a clinical context.


\section{Aims of thesis}
%% At the end of the introduction, add a subchapter on the Aims of Thesis in
%% which you describe the research question and the objectives of your work on
%% a maximum of two pages

\todo[inline]{write the following in past}

% clinical application
In summary, functional localizer paradigms that aim to characterize the
topography of functional areas on the level of individual subjects are a
promising tool to advance brain mapping towards a clinical application.
% contra localizers
However, traditional localizer paradigms employ selectively sampled, tightly
controlled stimuli, rely heavily on a participant's compliance, and can usually
map just one domain of brain functions
% naturalistic stimuli could replace
A functional localizer based on a naturalistic stimulus could, in principle,
replace a traditional localizer paradigm while also providing a couple of
advantages:
% fixed time course
a) naturalistic stimuli offer a fixed time course correlating with a variety of
different brain functions ranging from low-level perception (e.g., luminance) to
high-level cognition (e.g., social cognition);
% richness
b) the richness and multi-dimensionality of naturalistic stimuli might impose a
challenging data analysis put promises a higher external validity;
% compliance
c) naturalistic stimuli are an easy-to-follow and immersive paradigm leading to
improved data quality.
% visually impaired
Lastly, an exclusively auditory stimulus like an audiobook or audio drama would
also be appropriate for visually impaired persons, e.g., suffering from
nystagmus or lack of eyesight.


\subsubsection{Focus on PPA}

\todo[inline]{mih: audio stimulus to localize a visual area comes unexpected;
might be enough to first declare that as an actual challenge (along the lines of
"if valid, how far can we take this?")}

\todo[inline]{"We" vs. "I"?}

% therefore
Therefore, this dissertation explores---while following the principles of
open, transparent, and reproducible science---whether a movie and the movie's
audio-description that was produced for a visually impaired audience could, in
principle, substitute a traditional localizer paradigm.
% focus: ppa
As a proof of concept, I will focus on the \ac{ppa} that exhibits increased
hemodynamic activity when participants view photos of landscapes, buildings or
landmarks, compared to, e.g., photos of faces or tools \citep[e.g.,][for
reviews]{epstein2014neural, aminoff2013role}.
% auditory semantics unclear
To our knowledge only one study \citep[cf.][]{aziz2008modulation} compared
hemodynamic activity levels in the \ac{ppa} that were correlated with different
categories presented in spoken sentences.
%
Despite mixed results, the findings by \citet{aziz2008modulation} suggest that
\ac{ppa} does not exclusively respond to visually presented scene-related
stimuli.
%
I will evaluate the movie's and audio-description's potential to substitute a
visual localizer in two ways.

\subsubsection{Annotation as groundwork}

\subsubsection{Application 1: \ac{glm} $t$-contrasts}



% 1. modeling hemodynamic responses
First, similarly to traditional localizer paradigms, we model hemodynamic
activity correlating with the perception spatial information.
% based on stimulus annotations
Differently from traditional localizers, we model hemodynamic activity based on
annotated stimulus features that are embedded in the audio-visual movie and the
movie's audio-description (each lasting $\approx$\unit[120]{m}).
% but technically similar to traditional localizers
Using these modeled hemodynamic time courses (i.e. regressors), we then create
\ac{glm} $t$-contrasts in order to locate the \ac{ppa}.


\subsubsection{Application 2: Estimation from ref}

\todo[inline]{Estimation from reference mentioned here for the first time}

\todo[inline]{When SRM study is written, revise accordingly}

% 2. estimation from reference
Second, in order to address the practical and monetary constraints of a clinical
context, we estimate localizer results of participants from results of
participants in a \textit{reference group}.


\paragraph{Anatomical alignment}
% previous studies: anatomical alignment
Previous studies have shown that the location of a participant's \ac{ppa} can be
estimated using \textit{anatomical alignment} \citep{frost2012measuring,
rosenke2021probabilistic, weiner2018defining, zhen2017quantifying}.
%
First, in order to addresses the issue of anatomical variability, functional
\acp{roi} of persons in the reference group are anatomically aligned to a
\textit{common anatomical space}, i.e. functional data are projected from each
person's brain space into an average brain anatomy like the Montreal
Neurological Institute brain atlas \citep[MNI152][]{fonov2011unbiased}.
%
Then, the average of \acp{roi} is projected from the common anatomical space
into the individual person's brain anatomy serving as an estimation of a
functional region's most probable location.
%
However, even cortical surface-based alignment \citep{fischl1999cortical,
yeo2009spherical} cannot eliminate the functional-anatomical variability of
category-selective regions across persons \citep{frost2012measuring,
weiner2018defining, weiner2014mid}.


\paragraph{Functional alignment}

\todo[inline]{this is very high-level overview; SRM is not even mentioned}

% intro to functional alignment
In order to address the issue of functional-anatomical variability across
subjects, we will use \textit{functional alignment} \citep[cf.][for
reviews]{haxby2020hyperalignment, bazeille2021empirical} to estimate a
participant's localizer results from a reference group.
%
Whereas anatomical alignment aligns voxels (or vertices) that share the same
anatomical location to a common anatomical space, functional alignment aligns
voxels (or vertices) that share similar functional properties to a
\textit{\ac{cfs}}.


\paragraph{Cross-validation}

\todo[inline]{I really tried to simplify the next paragraph; problem is it gets
so vague that it gets almost wrong; at least I would be puzzled more if the
text would give less details}

%
Following an exhaustive leave-one-out cross-validation, we split the dataset
(N$=$14 subjects) repeatedly in a set of $N-1$ training subjects (i.e. the
reference group) and one test subject (i.e. the left-out subject).
%
First, for every fold of the cross-validation, we train a model of a \ac{cfs}
based on the training subjects' response time series to a movie, an auditory
narrative and a visual localizer.
%
Second, we project the training subject's localizer results from their
respective brain space into the \ac{cfs}.
%
Third, the we align the test subject's response time series to the \ac{cfs} in
order to acquire a \textit{transformation matrix} for the test subject.
%
Last, we use the test subject's transformation matrix to project the training
subjects' localizer results from the \ac{cfs} into the test subject's brain
space that serve as an estimation of the test subject's localizer result.

\paragraph{Partial alignment}
% problem
However, using the complete time series data of a two-hour paradigm in order to
acquire a test subject's transformation matrix is inefficient plus unsuitable
for a clinical population.
%
Preferably, just a part of a movie or an auditory narrative could serve as a
``diagnostic'' run in order estimate the localizer results.
%
Therefore, we also assess the relationship between the quantity of functional
data used to functionally align a test subject to the \ac{cfs} and the
estimation performance.


\subsection{Open, transparent, and reproducible science}

% reproducibility crisis
Over the last decade, there has been a growing awareness that results of
scientific publications are not reproducible or general scientific findings are
not replicable letting some authors speak of a ``reproducibility crisis'' or
``replication crisis'' in the sciences \citep{baker2016reproducibility,
plesser2018reproducibility, stupple2019reproducibility, nosek2022replicability}.
% reproducibility: definition
``A study is reproducible if all of the code and data used to generate the
numbers and figures in the paper are available and exactly produce the published
results'' \citep{leek2017most}.
% replicability: definition
A study is replicable if the same analysis of an equivalent experiment's data
leads to consistent results \citep{dubois2016building, leek2017most}.
%
Hence, on a metalevel, this dissertation aims to meet both the requirements of
open, shared, accessible, and transparent science \citep[cf.][]{watson2015will,
fecher2014open} as well as the requirements of a reproducible and replicable
research project:
%
the dissertation follows guidelines and best practices for a) coding and
scientific computing \citep{wilson2014best}, b) procedures and data analyses
\citep{nichols2017best, poldrack2017scanning, poldrack2019establishment}, and c)
sharing code, created data, and results \citep{eglen2017toward, nichols2017best,
pernet2015improving}.

% \paragraph{input data}
% open data \citep{eglen2017toward}
First, my work is built on top of publicly and freely available \ac{fmri} data
that are part of the \textit{studyforrest} project
(\href{www.studyforrest.org}{\url{studyforrest.org}}).
%
The studyforrest project is an open science project that aims to provide a
versatile resource for investigating human brain function under quasi-natural
conditions.
% core stimuli
The core of this dataset are two-hour long \ac{bold} \ac{fmri} scans of
participants watching the movie Forrest Gump \citep{ForrestGumpMovie} and
listening to the movie's audio-description that was created for a visually
impaired audience by adding a narrator to the movie's audio track.
% used by 3rd parties
Since its first publication in 2014 \citep{hanke2014audiomovie}, the
studyforrest project has served as a resource of raw (and preprocessed) data for
international working groups to conduct and publish independent, peer-reviewed
research (s.
\href{www.studyforrest.org/publications.html}{\url{studyforrest.org/publications.html}}).

\todo[inline]{mih: hint at why they were created to make sure that they had a
purpose here AND they have been published: "to enable the type of analysis
underlying this thesis"}

\todo[inline]{coh: this is done under "specific objectives"; imo, it is better
to keep the open science part self-contained; same for Datalad: I made a
subsection for the DataLad paper in "specific objectives" (which was strangely
missing)}

% annotations
The additional stimulus annotations that have been created over the course of
the dissertation are version-controlled---i.e. each change of data is logged and
documented---to provide a historical record, and published in a standardized
file format \citep{haeusler2021speechanno} to contribute to the studyforrest
project.

% \paragraph{code, analyses, output} intro
Further, all code is version-controlled and shared to improve the
reproducibility of current results and to facilitate the replicability of
findings on other datasets.
% automatization
Therefore, data analysis pipelines are designed in a way that enables automated
processing.
% FOSS
Analyses pipelines are not implemented in proprietary software but in freely
available and, if possible, open-source software.
% which tools to choose why?
Among potential software packages, we chose the tools that offer the most solid
documentation, and a broad basis of developers and maintainers to ensure
long-term support.
% my code
Custom code written by myself is written in open-source programming languages
(Python and Bash), is version-controlled, documented, and released publicly and
freely accessible.

% DataLad
All input data, custom code, analysis steps, and output data are accessible in
standardized \textit{DataLad} (\href{www.datalad.org}{datalad.org}) datasets.
%
Since DataLad provides a free and open-source software solution that manages
provenance, distribution, and version-control of code and data
\citep{halchenko2021datalad}, all executed steps from downloading the input data
to visualizing the results can be rerun to validate the dissertation's results.

% \paragraph{publications} open-access publishing
Last, because ``nature abhors a paywall'' \citep{dupre2020nature}, publications
describing generated data, the reasoning of methodological choices, analysis
steps, and results are published in open-access journals.
% neurovault
Unthresholded statistical maps of all computed statistical $t$-contrasts are
additionally published at Neurovault
(\href{https://neurovault.org/}{neurovault.org}).


\subsection{Specific objectives and hypotheses}

\todo[inline]{mih: this section goes beyond hypotheses -> comment out results}

\todo[inline]{if studies have their on caption, 2-3 sentences here here}


\subsubsection{DataLad: distributed system for joint management of code, data,
and their relationship}

\todo[inline]{This is the technical basis / groundwork}

\todo[inline]{This should state that you contributed to the development and
evaluation of that software in order to enable it to become you workhorse}


In order implement the open science procedures, I chose DataLad as the technical
foundation.
%
DataLad build on git (annex) \citep[cf.][for an git(hub) tutorial from an open
science perspective]{gilroy2019furthering}.
%
DataLad is fucking awesome because it does...
%
I contributed to the development of DataLad during starting from its early
stages in YEAR [???] by using alpha versions, evaluating usability of user
interface and documentation, and giving feedback to the team of software
engineers to improve the software's usability and documentation.
%
This resulted in a co-authorship of the software's accompanying publication.
\citep[cf.][]{halchenko2021datalad}.



\subsubsection{A studyforrest extension, an annotation of spoken language in the
German dubbed movie ``Forrest Gump'' and its audio-description}

\todo[inline]{This is the non-technical basis / groundwork}

\todo[inline]{Reduces literal quotes from my own studies}

% intro
``Naturalistic stimuli have a fixed but initially unknown temporal structure of
stimulus features of interest, as well as an equally unknown confound
structure'' \citep{haeusler2021speechanno}.
% what we did in 1 sentence
Therefore in \citep{haeusler2021speechanno}, we created and validated an
annotation of speech occurring in the movie and its audio-description pursuing
two goals.
% goal #1: groundwork for PPA study
The first goal was to build the groundwork that enabled us to model hemodynamic
time courses (i.e. regressors), and create \ac{glm} contrasts in order to
localize the \ac{ppa} in \citet{haeusler2022processing}.

Using these modeled hemodynamic time courses (i.e. regressors), we then create
\ac{glm} $t$-contrasts in order to locate the \ac{ppa}.

% goal #2: extend studyforrest
The second goal to extend the studyforrest dataset as a public resource for
independent research by creating an exhaustive annotation of speech that
complements the formerly published annotations of portrayed emotions
\citep{labs2015portrayed}, perceived emotions \citep{lettieri2019emotionotopy},
as well as cuts and locations depicted in the movie \citep{haeusler2016cutanno}.

% validation analysis
In order to validate the annotation's quality, we performed a canonical \ac{glm}
analysis of modeled hemodynamic activity based on information drawn from the
annotation.
% contrast
Regressors correlating with speech-related events were contrasted to a regressor
correlating with events without speech.
% hypothesis
We hypothesized that results would reveal significant clusters in brain regions
that have been shown to be involved in processing spoken language.
% results
Indeed, our analysis revealed statistically significant increased hemodynamic
activity in a bilateral cortical network replicating ``results of studies that
employed tightly controlled stimuli \citep[s.][for reviews]{friederici2011brain,
hickok2007cortical,price2012twentyyears}, and studies that employed data-driven
methods to analyze \ac{fmri} data from auditory naturalistic stimuli
\citep{honey2012not, lerner2011topographic, silbert2014coupled}.
% conclusion
Hence, the results of the validation encouraged us to use the annotation as a
groundwork to be adapted to our specific needs in \citet{haeusler2022processing}.


\subsubsection{Processing of visual and non-visual naturalistic spatial
information in the ``parahippocampal place area''}

\todo[inline]{First application of groundwork}

\todo[inline]{proof of concept without "innovative" annalysis / method}

% study in one sentence
In \citet{haeusler2022processing}, we investigated whether it is possible to
localize the \ac{ppa} by creating \ac{glm} contrasts based on annotations of
both the two-our lasting movie as well as the movie's audio-description.
% AV annotation
``For the analysis of the movie stimulus, we took advantage of a previously
published annotation movie cuts and the depicted location after each cut
\citep{haeusler2016cutanno}'' \citep{haeusler2022processing}.
% AD annotation
For the analysis of the audio-description, we extended the annotation of speech
that was created in \citet{haeusler2021speechanno}:
%
nouns that the narrator uses to describe the movie's absent visual content were
semantically categorized in case the narrator used them to describe locations,
buildings, rooms, faces, bodies, and so on.
% hypo a: group
On a group-average level, we hypothesized that a conventional model-based
mass-univariate statistical analysis would reveal increased hemodynamic activity
in medial temporal regions that were functionally identified as the \ac{ppa} in
the same set of participants employing a traditional block-design functional
localizer \citep{sengupta2016extension}.
% hypo b: individuals
Moreover, we hypothesized that an exclusively auditory stimulus could, in
principle, localize the \ac{ppa} as an example of a ``visual area'' in
individual persons.

% group results
On a group-average level, findings demonstrate that increased activation in the
\ac{ppa} during the perception of static pictures generalizes to the perception
of spatial information embedded in an audio-visual movie and exclusively
auditory naturalistic stimulus \citep{haeusler2022processing}.
% individual AD
On an individual level, results revealed that semantic spatial information
occurring in the audio-description is correlated with increased bilateral
activity in the anterior \ac{ppa} of nine individuals and unilateral activity in
the anterior \ac{ppa} of one individual.
% conclusion 1
Results add evidence \citep[cf.][]{bartels2004mapping} that a functionally
defined region, such as the \ac{ppa}, can be localized using a model-driven
\ac{glm} analysis that is based on a naturalistic stimulus' annotated temporal
structure.
% conclusion 2
Results also suggest that an exclusively auditory naturalistic stimulus could
potentially substitute a visual localizer as a diagnostic procedure.



\subsubsection{Using a varying amount of data from naturalistic stimulation for
functional alignment to predict results from a task-based functional localizer
paradigm}

\todo[inline]{Second application of groundwork}

\todo[inline]{proof of concept with "innovative" annalysis / method}

\todo[inline]{We agreed on SRM study not having at its own abstract; I might
write a little more here (or add an abstract to Chapter 5)}

\todo[inline]{When 1.3.0  \& SRM study is written, revise accordingly}


\paragraph{Problem}
% summary of haeusler2022processing
Results of \citet{haeusler2022processing} suggest that a naturalistic stimulus
might provide an engaging, task-free paradigm to localize brain functions in
individual subjects.
%
However considering practical and monetary constraints in a clinical context, a
paradigm lasting 90 to 120 minutes is inappropriate for even an extensive
individual diagnostic procedure.
%
Results from \citet{haeusler2022processing} also suggest that we might be able
to use the response patterns measured during the presentation of the
audio-description in order to generate a \ac{cfs} [needs to be defined above in
overview] in chapter 5's study, and align an ``unknown'' test participant to
that \ac{cfs}.


\paragraph{Solution}
%
Therefore, the first goal of the study in chapter 5 was to assess a procedure to
estimate results of a dedicated localizer \citep{sengupta2016extension} based on
data acquired during naturalistic stimulation.
%
Following leave-one-out cross-validation, we estimated (i.e. predicted) the
results of the visual localizer experiment ($Z$-values of voxels within a
\ac{roi}) of a left-out test participant based on localizer results of a
reference group.\todo{phrasing!}
% partial alignment
The second goal was to assess the relationship between the length of
naturalistic stimulation used to align the test participant to the fixed
\ac{cfs} and the estimation performance.
%
Lastly, the estimation performance of our new procedure based on \ac{fa} was
compared an estimation performance based on \ac{aa}.


\paragraph{Hypotheses}


\paragraph{Results}


\paragraph{Discussion}
