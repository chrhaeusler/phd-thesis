%% The introduction begins with an overview of the topic

%
``A remarkable feature of the vertebrate brain is the anatomical specialization
of cortical regions for the processing of different types of information. Since
the late 19th century, it has been recognized that restricted lesions of the
human brain result in location-specific sensory, motor or cognitive deficits''
\citep{cohen1994localization}.




%
Thanks to technological advanced, brain imaging studies have extensively used
\ac{fmri} since the early 1990s to measure \ac{bold} activity in-vitro.
%
\textit{Human brain mapping} \citep[e.g.,][]{raichle2009brief} explores the
brain's topographic organization \citep[e.g.,][]{eickhoff2018topographic} and
attempts ``to specify in as much detail as possible the localisation of function
in the human brain'' \citep{savoy2001history}.
% higher visual areas
For example in the domain of higher-visual perception, replicated findings
suggest that category-selective brain regions like the \ac{ppa}
\citep{epstein1998ppa, epstein1999parahippocampal} or the \ac{ffa}
\citep{kanwisher1997ffa, kanwisher2006fusiform} exhibit significantly increased
\ac{bold} activity correlated with a ``preferred'' stimulus class: the \ac{ppa}
responds more strongly when participants are watching pictures of scenes and
landscapes compared to pictures of, e.g., faces, and the \ac{ffa} responds more
strongly when participants are watching pictures of faces compared to pictures
of scenes.

%
Typical analyses of human brain mapping studies average data across subjects for
practical (e.g., limited scan time per subject), statistical (e.g., improved
of \ac{snr}) reasons, or to generalize from study subjects to a broader
population.
\todo[inline]{Practical constains: "not enough data practically obtainable from
one subject"}
%
Nevertheless, the precise anatomical location of functionally defined regions of
the brain vary anatomically (measured in Talairach or MNI coordinates) across
individuals \citep{friston2006critique, saxe2006divide}.
%
Consequently, studies employing an averaging approach may just ``capture the
common denominator of each individual cognitive circuit and lose a large amount
of information'' \citep{pinel2007fast}.
%
However, ``interpretation of fMRI data at the level of individual brains is
essential for characterizing brain function in health and disease''
\citep{dubois2016building}.


\section{Functional localization}
%% Further introductory remarks describe the scientific background of the work
%% as precisely as possible. Cite the most important publications and avoid
%% extensive literature reviews.
% a.k.a. "state of research"

\todo[inline]{merge following stuff in text beneath}

``Functional localizers are inefficient because they only estimate one or a few
topographies for each localizer'' \citep{jiahui2020predicting}.

``There are several limitations to the independent localizer approach.
%
1) it is not always possible to obtain an independent localizer scan.
%
This is especially the case in patient populations, for example in the
congenitally blind [Mahon et al. 2009; Bedny et al. 2011; Striem-Amit, Cohen, et
al. 2012a; van den Hurk et al. 2017] or individuals with visual
agnosia/prosopagnosia [Schiltz and Rossion 2006; Steeves et al. 2006; Sorger et
al.  2007; Barton 2008; Gilaie-Dotan et al. 2009; Susilo et al. 2015].
%
2) performing a localizer scan before each experiment is costly in terms of
scanning time, as well as mental effort and attention resources of the
participant.
%
The latter can result in fatigue during the main experiment of interest, leading
to lower quality data.
%
3) as localizer scans are typically conducted in a subject-specific manner and
researchers vary in the manner they define the ROIs (e.g., whether smoothing was
employed, if they use anatomical constraints, what thresholding methods were
employed), it is hard to assess variability between participants and across
studies \citep{rosenke2021probabilistic}.



% individual level: localizers
On the level of individual subjects, \textit{functional localizer} experiments
\citep[e.g.,][for reviews]{saxe2006divide, friston2006critique} are conducted
during \ac{fmri} to characterize the location size, and shape of functional
areas correlating with perceptual or cognitive processes like the perception of
object categories \citep{kanwisher1997ffa} or speech
\citep{fernandez2001language}, or theory of mind \citep{spunt2014validating}.
% purpose: ROI improve the statistical power of the main experiment's analysis
Functional localizers are often used as separate experiment to identify a
subject-specific functional \acp{roi} to ``guide, constrain or interpret results
from a main experiment \citep{saxe2006divide}.

% from \citep{rosenke2021probabilistic}
``The ROI approach is advantageous for four reasons:
%
1) it allows hypothesis driven comparisons of signals within independently
defined ROIs across many different conditions,
%
2) it increases statistical sensitivity in multisubject analyses [Nieto–
Castañón and Fedorenko 2012],
%
3) it reduces the number of multiple comparisons present in whole-brain analyses
[Saxe et al.  2006], and
%
4) it identifies ROIs in each participant's native brain space''
\citep{rosenke2021probabilistic}.

% clinical application
They also promise to advance neuroimaging towards a clinical application (e.g.,
diagnosis prior to neurosurgery) because ``in the clinical setting making
diagnoses for single cases is imperative'' \citep{wegrzyn2018thought}.
% purpose: neurosurgery
For example, surgical procedures might impact the post-operative quality of life
so much (e.g. concerning cognitive control or speech production) that a surgical
intervention outweighs the therapeutic benefits.

% one localizer = one domain of functions
However, one localizer paradigm can usually map just one domain of brain
functions.
% detection power based on paradigms' design
Designed to maximize detection power, localizers employ carefully chosen,
tightly-controlled, and simplified stimuli presented in a block-wise manner
often accompanied with a task to keep study participants attentive.
% inefficiency -> localizer batteries
Since this approach becomes inefficient if one wants to map many different
processes with limited resources (time, availability and applicability of
diagnostic measures for an individual patient), researchers have developed more
time-efficient, multi-functional \textit{localizer batteries}
\citep{barch2013function, drobyshevsky2006rapid, pinho2018individual,
pinho2020individual, pinel2007fast}.
% localizer batteries: example
For example, \citet{pinel2007fast} employs a range of dedicated stimuli and
specific tasks in a 5-minute routine to map processes of ``auditory and visual
perception, motor actions, reading, language comprehension, and mental
calculation at an individual level'' \citep{pinel2007fast}.

% intro to contra localizer
But still, current localizer paradigms struggle to overcome especially two
challenges.
%
First, the localizers depend heavily on the participants' comprehension of the
task instructions and their compliance, a criterion that can be difficult to
meet in clinical or pediatric populations \citep{eickhoff2020towards,
vanderwal2015inscapes, vanderwal2019movies}.
% validity?
Second, the paradigms rely on selectively sampled, tightly-controlled stimuli
presented in blocks, and do not resemble how we perceive the real world outside
of the laboratory during everyday life.


\section{Naturalistic stimuli}
%
Since a major goal of neuroscience is not to reveal how the brain
responds to blocks of [tightly-controlled] stimuli presented in a laboratory
setting but how the brain processes information during everyday perception,
\textit{naturalistic stimuli} gained popularity in neuroimaging.


\paragraph{Definition}

% definition quote
Naturalistic stimuli are ``a class of stimuli that aim to evoke more
naturalistic patterns of neural responses than traditional controlled artificial
stimuli. Naturalistic paradigms are typically complex and dynamic, and longer in
duration than many conventional stimuli.'' \citep{vanderwal2019movies}.
%
Therefore, naturalistic stimuli [promise to] ``sample a broad range of brain
states and engage multiple perceptual and cognitive systems in parallel''
\citep{haxby2020naturalistic}.
% movies & narratives
The most popular naturalistic stimuli in neuroscience are movies and auditory
narratives \citep[s.][for reviews]{jaaskelainen2021movies,
jaaskelainen2020neural} that provide a time-locked event structure during a
continuous, rich and dynamic stimulation.
%
Consequently, naturalistic stimuli promise a higher ecological validity
\citep{zaki2009need, hasson2012future, hamilton2018revolution} because they more
closely mimic our rich visual and auditory experiences outside the scanner bore
in real-life \citep{hasson2008neurocinematics, haxby2020naturalistic}.
%
Naturalistic stimuli also promise a higher external validity because the
\textit{stimulus features} (i.e. the stimulus classes or variables) that are
embedded in the naturalistic stimulus represent a more random sample from the
``theoretical population of stimuli that might have been used''
\citep{westfall2016fixing}.
%
% Carefully chosen stimulus sets ``selectively sample from the stimulus
% population leading to a stimulus-as-fixed-effect fallacy [Clarc, The
% language-as-fixed-effect fallacy: A critique of language statistics in
% psychological research]'' \citep{westfall2016fixing}. ``The conclusions cannot
% be generalized to a broader population of stimuli without risking inflated
% Type I error  [cf. Donnet S, Lavielle M, Poline JB: Are fMRI event-related
% response constant in time? A model selection answer]''
% \citep{westfall2016fixing}.


\paragraph{Early findings}
% reviews
Audio-visual movies and spoken narratives have been used during \ac{fmri}
\citep[s.][for reviews]{hamilton2018revolution, hasson2008neurocinematics,
sonkusare2019naturalistic, saarimaki2021naturalistic}, and \ac{eeg} or \ac{meg}
data acquisition \citep[s.][for reviews]{alday2019meg, kandylaki2019story}.
%
Early studies have shown that watching a movie \citep{hasson2004intersubject,
hasson2008neurocinematics, hasson2010reliability} or listening to a narrative
\citep{lerner2011topographic, wilson2008beyond} reliably synchronize
spatiotemporal responses across multiple subjects in a large part of the brain
compared to, for example, an unedited video of a concert taken from a fixed
viewpoint \citep{hasson2004intersubject, hasson2008neurocinematics,
hasson2010reliability, lerner2011topographic, wilson2008beyond}.
% This finding could be attributed to a film director's goal to not only direct
% a movie, but also to capture and direct the audience's attention; this can be
% attributed to the way professional movies are shot and edited in order to
% intentionally manipulate the viewers' attentional focus and mental states
% \citep{brown2012cinematography, dancyger2011film-technique}.
%
Furthermore, a pioneering study \citep{bartels2004mapping} suggests that
functional specialization of cortical areas is preserved during complex,
life-like stimulation.


\paragraph{Better compliance}
%
From a practical perspective, naturalistic stimuli promise improved subject
compliance regarding wakefulness and head motion due to minimal instruction
requirements (e.g. no fixation of eye gaze), task demands (no task except
enjoying the movie or auditory story).
%
This is especially the case for young children \citep{vanderwal2015inscapes},
and possibly psychiatric \citep{eickhoff2020towards} or elderly persons
resulting in increased data quality.
%
Since movies and spoke narratives are interesting and easy-to-follow stimuli
that are produced to be engaging and immersive, they also promise to put
participants at ease in the otherwise claustrophobic, uncomfortable, and noisy
MRI scanner.
%
Lastly, spoken narratives are also appropriate for visually impaired persons
suffering from diminished or lack of eyesight.


\paragraph{contra: challenging data analysis; annotations needed}

%
However, the majority of naturalistic stimuli that have been used in
neuroimaging have originally been designed for commercial purposes and not to
conduct research.
%
The temporal structure of stimulus features embedded in a naturalistic stimulus
is fixed and thus reproducible but initially not explicitly known.
%
Modeling brain activity correlating with the stimulus features embedded in the
time course is challenging \citep{saarimaki2021naturalistic, simony2020analysis}
because such models, like a traditional \ac{glm}, rely on the stimulus features
being annotated.
%
The lack of extensive annotations has led to a ``usage bottleneck''
\citep{aliko2020naturalistic} and might be the main reason why explicit models
of task or stimulus are ``notoriously'' \citep{richard2019fast}, if not
``prohibitively'' \citep{nastase2019measuring} difficult to construct.



\paragraph{interim summary}

% Quote from Finn
In summary, naturalistic stimuli ``impose a meaningful timecourse across
subjects while still allowing for individual variation in brain activity and
behavioral responses, and lend themselves to a broader set of analyses than
either pure rest or pure event-related task designs'' \citep{finn2017can}.


\section{Aims of thesis}
%% At the end of the introduction, add a subchapter on the Aims of Thesis in
%% which you describe the research question and the objectives of your work on
%% a maximum of two pages


\subsection{short summary of introduction so far}

\todo[inline]{still too long and (of course) bumpy}

% clinical application
An application of neuroimaging in a clinical setting requires the
``interpretation of fMRI data at the level of individual brains is essential for
characterizing brain function in health and disease''
\citep{dubois2016building}.
% functional localizers
Functional localizers paradigms that aim to characterize the location size, and
shape of functional areas and promise to advance neuroimaging towards a clinical
application.  (e.g., diagnosis prior to neurosurgery) because ``in the clinical
setting making diagnoses for single cases is imperative''
\citep{wegrzyn2018thought}.
% contra localizers
However, localizers a) rely on selectively sampled, tightly-controlled stimuli
presented in blocks to maximize detection power, b) can usually map just one
domain of brain functions, and c) heavily on the participants' general
compliance and their comprehension of the task instruction.
% naturalistic stimuli
Alternatively, naturalistic stimuli might be used as an interesting,
easy-to-follow, engaging, immersive paradigm that puts participants at ease in
the otherwise claustrophobic, uncomfortable, and noisy MRI scanner leading to
improved data quality \citep{eickhoff2020towards}.
%
Moreover, spoken narratives are also appropriate for visually impaired persons
suffering from diminished or lack of eyesight.
%
The richness, multi-dimensionality of naturalistic stimuli might impose a
challenging data analysis put promises a higher ecological and external
validity.
%
The time-courses of rich, full-length naturalistic stimuli are correlating with
a variety of different brain functions ranging from low-level perception (e.g.,
luminance) to high-level cognition (e.g., social cognition).
%
A naturalistic stimulus could eventually replace multiple dedicated localizers
in the future to provide a more comprehensive diagnostic routine.


\subsection{Therefore: this dissertation}

%
Therefore, this dissertation explores --- while following the principles of
open, transparent, and reproducible science --- whether a movie and the movie's
audio-description that was produced for an visually impaired audience could, in
principle, substitute a traditional localizer paradigm.

\paragraph{focus on PPA}
% focus: ppa
As a proof of concept, this dissertation focuses on the \ac{ppa} that exhibits
increased hemodynamic activity when participants view photos of landscapes,
buildings or landmarks, compared to, e.g., photos of faces or tools
\citep[e.g.,][for reviews]{epstein2014neural, aminoff2013role}.

``Several studies have reported increased hemodynamic activity in the PPA
attributed to the processing of scene-related, spatial information, for example,
when participants were watching static pictures of landscapes compared to
pictures of faces or objects \citep{epstein1998ppa,
epstein1999parahippocampal}'' \citep{haeusler2022processing}.
% auditory semantics unclear
However, reports regarding the correlates of processing spatial information in
verbal stimulation are less clear \citep{aziz2008modulation}.

%
We will evaluate the potential of an audio-visual movie and an auditory
narrative to substitute a traditional localizer paradigm in two ways.


\paragraph{modeling hemodynamic responses}
%
Similarly to traditional localizer paradigms, we first model hemodynamic
activity based on annotated stimulus features embedded in two-hour-lasting
naturalistic stimuli, and create build GLM $t$-contrasts in order to locate the
\ac{ppa}.


\paragraph{estimating from reference group}

\todo[inline]{streamline with SRM intro}

\todo[inline]{do not introduce too many technical terms (e.g. LOOCV)}

\todo[inline]{training and \& testing data is okay, imo}

\todo[inline]{make clear that we do not just estimate parameters, i.e.
statistical inference but ``prediction''}

%
However, 2h is too fucking much.
%
Hence, we employ a new approach to estimate the results of the functional
localizer from data of the movie and audio-description.
%
Previous studies have shown that localizer can be estimated using anatomical
alignment [e.g. CITATION] by projecting ...
%

\paragraph{Problem: surface-based antomical alignment is not enough}
%
Surface-based inter-subject alignments / strategies for registering individual
brains together that respect cortical foldings \citep{fischl1999cortical,
yeo2009spherical} have been shown to reduce the variability in
\textit{functional--anatomical correspondence} \citep{feilong2018reliable,
kumar2020brainiak} across persons \citep{klein2010evaluation,
frost2012measuring} [but see \citep{langers2014assessment}].
%
Cortical surface-based alignment \citep{fischl2012freesurfer} that respects
sulcal locations can reduce the mismatch between brain function and anatomy of
category-selective regions \citep{duncan2009consistency, frost2012measuring,
weiner2018defining, weiner2014mid} [still similar phrasing to
\citep{feilong2018reliable}].
%
Nevertheless, surface-based alignment [\citep{fischl2012freesurfer}] does not
eliminate the mismatch between brain function and anatomy of category-selective
regions \citep{duncan2009consistency, frost2012measuring, weiner2018defining,
weiner2014mid}[still similar phrasing to \citep{feilong2018reliable}].

\paragraph{Solution: functional alignment}
%
A more recent approach, uses functional alignment \citep{haxby2020hyperalignment,
bazeille2021empirical}.
%
Previous results has shown that prediction using functional alignment
outperforms functional alignment.

% align left-out subject
Based on our previous findings \citep{haeusler2022processing}, we assumed that
the naturalistic stimuli would trigger, among others, brain responses that are
similar to those triggered by the functional localizer.


\paragraph{Partial alignment}
%
However, even if one wants to map a variety of brain functions, a two-hour
paradigm would be inefficient plus unsuitable for a clinical population.

%
Therefore, we assess the quantity of data needed
to predict individual functional topographies by projecting results of a
localizer experiment (statistical $Z$-maps) from a reference group into the
brain anatomy of individual participants.
%
We test which amount of data is needed to perform an alignment
to the \ac{cfs} that provides transformation matrices that
outperform a prediction based on an anatomical alignment.

%
We explore whether just a part of a movie or auditory narrative can serve as a
``diagnostic'' run in order to estimate the location of the \ac{ppa} in an
``unknown'' subject's brain.


\subsection{Open, transparent, and reproducible science}

% reproducibility crisis
Over the last decade, there has been a growing awareness that results of
scientific publications are not reproducible or general scientific findings are
not replicable letting some authors speak of a ``reproducibility crisis'' or
``replication crisis'' in the sciences \citep{baker2016reproducibility,
plesser2018reproducibility, stupple2019reproducibility, nosek2022replicability}.
% reproducibility: definition
``A study is reproducible if all of the code and data used to generate the
numbers and figures in the paper are available and exactly produce the published
results'' \citep{leek2017most}.
% replicability: definition
A study is replicable if the same analysis of an equivalent experiment's data
leads to consistent results \citep{dubois2016building, leek2017most}.
%
Hence, on a metalevel, this dissertation aims to meet both the requirements of
open, accessible, shared, and transparent science \citep{watson2015will,
fecher2014open} as well as the requirements of a reproducible and replicable
research project:
%
the dissertation follows guidelines and best practices for a) coding and
scientific computing \citep{wilson2014best}, b) procedures and data analyses
\citep{nichols2017best, poldrack2017scanning, poldrack2019establishment}, and c)
sharing code, created data, and results \citep{eglen2017toward, nichols2017best,
pernet2015improving}.


\paragraph{input data}

% open data \citep{eglen2017toward}
First, my work is built on top of publicly and freely available \ac{fmri} data
that are part of the \textit{studyforrest} project
(\href{www.studyforrest.org}{\url{studyforrest.org}}).
%
The studyforrest project is an open science project that aims to provide a
versatile resource for investigating human brain function under quasi-natural
conditions.
%
The core of this dataset are two-hour long \ac{bold} \ac{fmri} scans of
participants watching the movie Forrest Gump \citep{ForrestGumpMovie} and
listening to the movie's audio-description that was created for a visually
impaired audience by adding a narrator to the movie's audio track.
%
Since its first publication in 2014 \citep{hanke2014audiomovie}, the
studyforrest project has served as a resource of raw (and preprocessed) data for
international working groups to conduct and publish independent, peer-reviewed
research (s.
\href{www.studyforrest.org/publications.html}{\url{studyforrest.org/publications.html}}).
%
The additional stimulus annotations that have been created over the course of
the dissertation are version-controlled and published in a standardized file
format \citep{haeusler2021speechanno}, and therefore contribute to the
studyforrest project as a resource for the scientific community.


\paragraph{code, analyses, output}

Further, all code is shared to improve reproducibility current results and to
facilitate replicability of findings on other datasets.
% automatization
Therefore, data analyses pipelines designed in a way that enables automated
processing.
%
Analyses pipelines are not implemented in proprietary software but in freely
available and, if possible, open-source software.
% which tools to choose why?
Among potential software packages, we chose the tools that offer the most solid
documentation, and basis of developers and maintainers to ensure long-term
support.
% my code
Custom code written by myself is written in open-source programming languages
(Python and Bash), is version-controlled, documented, and released publicly and
freely accessible.
%
All input data, custom code, analysis steps and output data are accessible in
standardized \textit{DataLad} (\href{www.datalad.org}{datalad.org}) datasets.
Since DataLad provides a free and open-source software solution that manages
provenience, distribution, and version-control of code and data
\citep{halchenko2021datalad}, all executed steps from downloading the input data
to visualizing the results can be rerun to check and validate the dissertation's
results.


\paragraph{publications}
% open-access publishing
Last, because ``nature abhors a paywall'' \citep{dupre2020nature}, publications
describing generated data, reasoning of methodological choices, analysis steps,
and results are published in open-access journals.
% neurovault
Unthresholded statistical maps of all computed statistical $t$-contrasts are
additionally published at Neurovault
(\href{https://neurovault.org/}{neurovault.org}).


\subsection{Specific objectives and hypotheses}

\todo[inline]{in case studies shall have their own titles, short text here}

\subsubsection{A studyforrest extension, an annotation of spoken
language in the German dubbed movie ``Forrest Gump'' and its audio-description
(study 1)}

\todo[inline]{this part is still bumpy but general layout should be clear}

\todo[inline]{rephrase the first aim (to better match study 2)}

%
``Naturalistic stimuli have a fixed but initially unknown temporal structure of
stimulus features of interest, as well as an equally unknown confound
structure'' \citep{haeusler2021speechanno}.
%
Therefore in study 1 \citep{haeusler2021speechanno}, we created an annotation of
speech occurring in both the movie as well as the audio-description.
%
Furthermore, we validated the annotation's quality by performing a canonical
\ac{glm} analysis of modeled hemodynamic activity based on information drawn
from the annotation.

We pursued to goals:
% aim #1
the first aim was to create the groundwork necessary for study 2 [Chapter X] in
which we created \ac{glm} contrasts aimed to localize the \ac{ppa} based on
annotated stimulus features occurring in a movie as well as the movie's
audio-description.
% aim #2
The second aim was to create an exhaustive annotation of speech that
substantially exceeds the groundwork that was needed to conduct study 2 in order
to [complement formerly published annotations of portrayed emotions
\citep{labs2015portrayed}, perceived emotions \citep{lettieri2019emotionotopy},
as well as cuts and locations depicted in the movie \citep{haeusler2016cutanno}
and] extend the studyforrest dataset as a public resource for independent
research.

%
In order to validate the annotation's quality, we contrasted regressors based on
speech-related events to a regressor based on events without speech, and
hypothesized results would reveal significant clusters in brain regions that
have been shown to be involved in speech processing.
%
Indeed, results revealed statistically significant increased hemodynamic
activity in a bilateral cortical network including temporal, parietal and
frontal regions replicating ``results of studies that employed
tightly-controlled stimuli \citep[s.][for reviews]{friederici2011brain,
hickok2007cortical,price2012twentyyears}, and studies that employed data-driven
methods to analyze \ac{fmri} data from auditory naturalistic stimuli
\citep{honey2012not, lerner2011topographic, silbert2014coupled}.
%
Results encouraged us to use the annotation as a groundwork to be adapted to our
specific needs in study 2.


\subsubsection{Processing of visual and non-visual naturalistic spatial
information in the ``parahippocampal place area'' (study 2)}

% study in one sentence
In study 2 \citep{haeusler2022processing}, we investigated whether it is
possible to localize the \ac{ppa}, usually identified via contrasting blocks of
pictures in a task-based functional localizer, by creating \ac{glm} contrasts
based on annotations of both an two-our lasting audio-visual as well as an
exclusively-auditory naturalistic stimulus.
% AV anno
``For the analysis of the movie stimulus, we took advantage of a previously
published annotation movie cuts and the depicted location after each cut
\citep{haeusler2016cutanno}'' \citep{haeusler2022processing}.
% AD annotation
For the analysis of the audio-description, we extended the published annotation
of speech that was created in study 1 \citep{haeusler2021speechanno}.
%
Nouns that the narrator uses to describe the movie's absent visual content were
semantically categorized in case the narrator used them to describe locations,
buildings, rooms, faces, bodies, and so on.

% hypo a: group
On a group-average level, we hypothesized that a conventional model-based
mass-univariate statistical analysis would reveal increased hemodynamic activity
in medial temporal regions that were functionally identified as the \ac{ppa} in
the same set of participants by means of traditional block-design functional
localizer \citep{sengupta2016extension}.
% hypo b: individuals
Moreover, we hypothesized that a exclusively auditory stimulus could, in
principle, localize the \ac{ppa} as an example of a ``visual area'' in
individual persons \citep{haeusler2022processing}.
% group results
On a group-average level, findings demonstrate that increased activation in the
\ac{ppa} during the perception of static pictures generalizes to the perception
of spatial information embedded in an audio-visual movie and audio-only stimulus
\citep{haeusler2022processing}.
% individual AD
``On an individual level, we find significant bilateral activity correlating
with semantic spatial information occurring in the audio-description in the
anterior \ac{ppa} of nine individuals and unilateral activity in one
individual'' \citep{haeusler2022processing}.
%
Results add evidence that a functionally defined region, such as the \ac{ppa},
can be localized using a model-driven \ac{glm} analysis that is based on a
naturalistic stimulus' annotated temporal structure with respect to a particular
hypothesized cognitive or perceptual function.
%
Results also suggest that a naturally engaging, purely auditory paradigm like an
audio-description could, in principle, substitute a visual localizer as a
diagnostic procedure to assess brain functions in visually
impaired individuals \citep{haeusler2022processing}.

\paragraph{Transition to study 3}

\todo[inline]{the are responses correlating with auditory spatial information;
hence we might be able to use the response patterns measured during the
presentation of the audio-description in order to align subjects to a common
functional space in study 3}

\todo[inline]{the exact modeling of that process is not necessary for data-driven
alignment in study 3}


\subsubsection{Study 3: Using varying amount of data from naturalistic
stimulation for functional alignment to predict results from a task-based
functional localizer paradigm (study 3)}

\todo[inline]{title is still preliminary}


\paragraph{Intro}


\todo[inline]{we agreed on study 3 not having an abstract; hence, I might write a
little more here (or add an abstract to study 3, of course}

\todo[inline]{too long, bumpy, phrasing/terminology far from optimal}

% summary of study 2
As suggested by the results of study 2, a naturalistic stimulus might provide an
engaging, task-free paradigm to localize brain functions in individual subjects.
%
Employing a rich, multi-dimensional stimulus correlating with a variety of brain
functions might be used overcome the limitations of traditional, dedicated
localizer that become time-inefficient when one wants to map a variety of brain
functions (a.k.a. ``one paradigm to map one domain of brain functions'').
% problem
However, a clinical application of a naturalistic stimulus paradigm must both
provide valid results as well as minimize the scan time/costs.


\todo[inline]{following is a draft; revise while revising the SRM intro}

\textit{Therefore} in study 3 (s. Chapter X), we use data of the movie and
audio-description to predict localizer results \citep{sengupta2016extension}.
%intro
We test whether we can estimate statistical results (values $Z$-maps of
localizer contrast with in a \ac{roi}) from a test subject from data in a
reference group that was used.
%
We compare \textit{empirical $Z$-maps} acquired from the visual localizer to
\textit{predicted $Z$-maps} acquired via an anatomical or functional alignment
procedure.

\paragraph{partial alignment}

We assessed the relationship between length of naturalistic stimulation used
for a \textit{partial functional alignment} and the performance of predicting
empirical $Z$-maps by...
%
We test whether functional alignment can be achieved with a task-free, natural
stimulation ``calibration'' scan that requires no more acquisition time than a
conventional localizer paradigm.
%
We will estimate the trade-off between diagnostic quality and required effective
scan time by progressively reducing the duration of input BOLD fMRI data and
comparing the results of the reduced model to the reference computed from the
full length scan.


\paragraph{Hypotheses}

%
We hypothesized that increased quantity of data used to calculate the
transformation matrices of left-out subjects for a functional alignment would to
increase prediction performance.
%
Further, we hypothesized that functional alignment would eventually perform
``better'' than an estimation based on anatomical alignment.


\paragraph{Results}

\todo[inline]{Probably, unnecessary here?}


\paragraph{Discussion}

\todo[inline]{Probably, unnecessary here?}


\paragraph{backup of procedure}

\todo[inline]{can probably be deleted; will be explained shortly in the overview
at the beginning of "Aims \& hypotheses"}

% cms creation
Based on these response time series, we first created a common model space
employing a shared response model \citep{chen2015reduced} following
leave-one-subject out cross-validation, a process that also computed
transformation matrices for the subjects that provided the data for the creation
of the model space.

% alignment
We aligned left-out subjects with the \ac{cfs} to derive
transformation matrices for the left-out-subjects to perform the alignment.

%
Once aligned, the brain activity of the reference group can be used to predict
the activity of another subject.
% prediction
Lastly, the acquired transformation matrices were used to project the functional
topographies from the anatomy of the reference group into the common model
space, and from the common model space into the anatomy of the left-out
subjects.
