\section{Overview}
%% The introduction begins with an overview of the topic

\todo[inline]{some sentences are kinda quotes from other paper; e.g. Dubois?}

\todo[inline]{other contrasts: phonemes, grammatical tags, prosody, sex}

\todo[inline]{sections are supposed to be numbered}

%
Brain imaging with \ac{bold} \ac{fmri} has been used extensively for almost
three decades to investigate perceptual and cognitive brain functions.
%
Typical analysis procedures average (voxel-wise) data of at least 10-15 subjects
to improve the \ac{snr}.
%
Consequently, these studies do not characterize brain function at the level of
an individual.
%
It is plausible to assume that models of brain functions that are based on a
lowest common denominator approach only capture a fraction of individual
functional brain properties.
%
Therefore, those models are an incomplete foundation for an investigation of
inter-individual differences.
%
However, characterization of individual brain function is by far the most
important application of BOLD fMRI in a clinical context.
%
For example, for pre-surgical screening, or diagnosis of brain function in
health and disease.
%
The goal of the proposed project is to pave the way towards adopting an approach
to the investigation of individual brain function that is based on individual
differences with respect to large normative samples --- a proven strategy that
has been standard in psychological diagnostics and other clinical research for a
long time.\todo{give examples}


\section{Introductory remarks}
%% introductory remarks describe the scientific background of the work as
%% precisely as possible. Cite the most important publications and avoid
%% extensive literature reviews.


\subsection{State of research}


\subsubsection{Functional localization}

\todo[inline]{we don't do speech lateralization anymore; imo the neurosurgery
thing should not be mentioned in the intro anymore; better come up with it in
the general discussion}

%
The most frequently employed paradigm for characterizing individual brains with
BOLD fMRI are so called functional localizers.
%
Functional localizers aim at isolating and localizing brain activity correlated
with specific perceptual processes (e.g. different object categories;
\citet{kanwisher1997ffa}) or cognitive processes (e.g. theory of mind;
\citet{spunt2014validating}).
%
Typically, localizers are dedicated measurements that are used to define
individual \acp{roi}.
%
For example to improve the statistical power of the main experiment's analysis
or to locate brain functions prior to neurosurgery.
% side-effects of neurosurgery
Surgical procedures might impact the post-operative quality of life so much
(e.g. concerning cognitive control or speech production) that it potentially
outweighs the therapeutic benefits.
%
The challenge is to precisely localize relevant brain areas with limited
resources (time, availability and applicability of diagnostic measures for an
individual patient) in order to correctly predict the impact of the planned
procedure.
%
Importantly, functional localizers, despite being tuned for detection power,
quickly become inefficient if one wants to map many different processes in a
limited amount of time.
%
One example of a time-efficient multi-functional localizer for reading, language
comprehension, calculation, motor response, and basic retinotopy was developed
by \citep{pinel2007fast}).\todo{Thirion's work?}
%
It employs a range of dedicated stimuli and specific tasks participants have to
perform in a 5-minute routine.
%
The diagnostic quality of such paradigms relies heavily on participants'
compliance and comprehension of the task instructions, a criterion that can be
difficult to meet in certain target populations, such as patients with dementia.

\todo[inline]{s. \citep{vanderwal2015inscapes}}

\subsubsection{Anatomical alignment (in order to predict)}

The currently dominating approach in neuroimaging group analyses relies on
topological constraints defined by an three-dimensional, anatomical reference
space (e.g. the MNI152 template brain).
%
Surface-based: e.g. \citep{weiner2018defining}.


\subsubsection{Functional alignment (in order to predict)}

\paragraph{Functional alignment in general}
%
An alternative approach to individual localization has been proposed by
\citet{haxby2011common}.
%
They (and e.g. \citet{jiahui2019predicting}) also predicted the location, form,
and size of  target brain areas in ventral temporal cortex from dedicated
localizer scans of other individuals.
%
The key difference of this approach is to rely on similarity of representational
geometry of brain activity patterns and aligning individual brains into a
multi-dimensional a group space.

\todo[inline]{shift from focussing on (connectivity) hyperaligment more to
shared response model}


\paragraph{Hyperalignment}

\todo[inline]{why didn't we used connectivity alignment in the first place?}
%
\citet{haxby2011common} used BOLD response patterns evoked by a 2h action movie
to derive a common representational space.
%
The algorithm, namely hyperalignment, derives this representation using a
variant of Procrustes analysis and computes invertible (orthonormal)
transformations from each individual brain’s voxel-space into this common
reference space.
%
Importantly, the study also showed that an individual's \ac{ffa} or the
\ac{ppa}, can be localized precisely based on data from a reference group.
\todo{explain FFA, PPA}
%
The same authors later showed that this approach can be extended to predict
functional organization across large proportions of the cortical surface, for
example to predict the represented visual field coordinate in visual cortex
based on retinotopic mapping scans of other individuals
\citep{guntupalli2016model}.

\todo[inline]{we do not do connectivity hyperalignment but SRM now!}
%
In his doctoral thesis, recently submitted to the Faculty of Natural Sciences in
Magdeburg, Falko Kaule showed that congruent time-locked BOLD responses across
subjects (i.e. all subjects watching the exact same full-length movie) as used
by Haxby and colleagues are not required to derive a valid alignment of
individuals with a common representational space \citep{kaule2017examination}.
%
Comparable prediction performance can be achieved by using \textbf{functional
connectivity patterns} (correlation of a voxel's time series with reference
regions in the same brain).
%
This finding enables, in principle, the use of different ``calibration'' scans
to determine an alignment with a common representational space, for example with
an age-appropriate stimulus, or a shortened scan time to fit into a particular
clinical schedule.
%
Once a valid alignment is established, known functional properties of a
(normative) reference, derived from extensive scans and analysis of other
subjects, can then be projected into the respective individual voxel space (s.
Fig. 1 in \citep{nishimoto2016lining}).


\paragraph{Shared response model}

\citep{chen2015reduced}




\subsubsection{Naturalistic stimuli in neuroscience}

\todo[inline]{cf. PuG talk}


\paragraph{Intro}

One major goal of cognitive neuroscience is to reveal how the brain processes
information during everyday perception.
%
Traditionally, human brain mapping studies used carefully controlled experiments
and rudimentary stimuli that lack ecological validity.
%
For example, previous studies presented isolated higher-level visual features
such as scenes, faces, human bodies or tools.
%
Results suggest that domain-specic modules like the \ac{ppa}
\citep{epstein1998ppa}, \ac{ffa} \citep{kanwisher1997ffa}), the occipital face
area \ac{ofa} \citep{pitcher2011occipitalfacearea}, the \ac{eba}
\citep{downing2001bodyarea}), and the \ac{loc} \citet{malach1995loc} exist in
the human brain.
%
As a consequence of investigating perceptual and cognitive functions by
utilizing isolated stimuli, studies subdivided the cerebral cortex into
distinctive functional areas whose \ac{bold} activity is specifically correlated
with one particular simplied stimulus type.
%
However, the question remains how those functional areas behave in lifelike
situations and how they might interact.
%
After all, we do not experience the
world around us as separated into small unidimensional stimuli, but perceive ---
through different senses --- a seemingly continuous and unified world.
%
To address
this open question, usage of so called naturalistic stimuli have gained
popularity.


 This


\paragraph{Definition}

\todo[inline]{time-locked events}

\todo[inline]{shorten but add narrative studies; synchronization is also true
for narratives}
%
Naturalistic stimuli are ``a class of stimuli that aim to evoke more
naturalistic patterns of neural responses than traditional controlled artificial
stimuli. Naturalistic paradigms are typically complex and dynamic, and longer
in duration than many conventional stimuli.'' \citep{vanderwal2019movies}.

%
They ``impose a meaningful timecourse across subjects while still allowing for
individual variation in brain activity and behavioral responses, and lend
themselves to a broader set of analyses than either pure rest or pure
event-related task designs.'' \citep{finn2017can}

%
Numerous studies have shown that watching a movie leads to correlated
time-locked brain responses across subjects in many brain regions, and to
synchronized eye movements \citep{hasson2010reliability, lankinen2014isc-meg}.
%
This can be attributed to the way professional movies are shot and edited in
order to intentionally manipulate the viewers' attentional focus and mental
states \citep{brown2012cinematography, dancyger2011film-technique}.

Movies have been used during \ac{fmri} \citep{bartels2004mapping,
hasson2004intersubject}, \ac{eeg} \citep{dmochowski2014audience,
krause2000relative}, simultaneous \ac{eeg}-\ac{frmi}
\citep{whittingstall2010integration}, or \ac{meg} \citep{lankinen2014isc-meg,
luo2010auditory}.
%
The underlying assumption is that movies watched in a laboratory setting offer a
more complex and continuous stimulation that better mimics our natural dynamic
environment.
%
Indeed, studies have shown that freely watching a movie leads to synchronized
spatiotemporal responses across multiple subjects in a large part of the brain
\citep{hasson2010reliability, lankinen2014isc-meg}.\todo{but still different}
%
Professionally produced movies evoke inter-subject correlations (ISC) in more
parts of the brain than, for example, an unedited video of a concert, taken from
a fixed viewpoint \citep{hasson2010reliabilitiy}.
%
This finding could be attributed to a film director's goal to not only direct a
movie, but also to capture and direct the audience's attention
\citep{brown2012cinematography, dancyger2011film-technique}.


\todo[inline]{reviews on narratives: \citep{hamilton2018revolution}, more M/EEG
\citep{alday2019meg}}


\paragraph{Higher validity}
%
Findings suggest that naturalistic stimuli offer a higher ecological validity
\citep{hasson2004intersubject} because they better mimic statistics in our
natural dynamic environment.
%
Further they offer a higher external validity \citep{westfall2016fixing})
because they sample the stimulus space ``better''.\todo{iykwim}
%
''a RSM is only indicated when the stimuli used in the study do not fully
exhaust the theoretical population of stimuli that might have been used \citep{westfall2016fixing}.
%
because selectively sample from the stimulus population leading to a
stimulus-as-fixed-effect fallacy \citep{westfall2016fixing}.(Clarc, The
language-as-fixed-effect fallacy: A critique of language statistics in
psychological research).
%
The conclusions cannot be generalized to a broader population of stimuli without
risking inflated Type I error  (cf. Donnet S, Lavielle
M, Poline JB: Are fMRI event-related response constant in time? A model
selection answer\citep{westfall2016fixing)}.
%
While the functional alignment can also be applied to fMRI data from stimulation
paradigms with simplified stimuli, the transformations for functional alignment
have greatly diminished general validity \citep{haxby2011common}, presumably
because such experiments sample a sparser range of brain states
\citep{guntupalli2016model}.

%
Lastly, increased validity of derived transformation for functional alignment by
sampling a more diverse set of mental states that reflect (confound) statistics
of the natural environment, and enable investigation of the acquired data for a
variety of research questions (e.g. visual or auditory perception, spatial
cognition; emotion; music, speech or social perception)


\paragraph{Better compliance}

\todo{check also \citep{eickhoff2020towards}}
%
The to be diagnosed individuals were scanned while they were watching a movie
without any explicit task or.
%
Improved subject compliance and compatibility due to minimal instruction
requirements (e.g., no fixation of eye gaze) and task demands (no task except
enjoying the movie or audiobook).
%
These minimal instructions make naturalistic paradigms appropriate especially
for elderly or visually impaired persons.
%
Lastly, naturalistic stimuli offer improved data quality, as an interesting and
easy-to-follow stimulus is more capable of putting a participant at ease in the
otherwise claustrophobic, uncomfortable and noisy fMRI scanner.

``Relative to traditional fMRI experiments that typically use highly controlled
stimuli, naturalistic stimuli are more ecologically valid (Zaki and Ochsner,
2009; Hasson and Honey, 2012; Adolphs et al., 2016; Hamilton and Huth, 2018),
convey rich perceptual and semantic information (Bartels and Zeki, 2004; Huth et
al., 2012, 2016) and more fully sample neural representational space (Haxby et
al., 2011, 2014)''\citep{nastase2019measuring}.

``Recent work (Vanderwal et al., 2015) also suggests that naturalistic stimuli
may improve subject compliance (in terms of wakefulness and head motion relative
to, e.g. rest), which is particularly important when scanning patient
populations and children. As mentioned previously, different stimuli will
variably synchronize different brain systems; for example, engaging,
Hollywood-style movies may yield greater, more widespread ISCs than real-life,
unedited videos (Hasson et al., 2010; Cohen et al., 2017)''
\citep{nastase2019measuring}.


\todo[inline]{imo following part can heavily be shortened}

\paragraph{studyforrest}
%
The studyforrest project is an open science project that aims at providing a
versatile resource for investigating human brain function under quasi-natural
conditions.
%
The core of this dataset are two hour long BOLD fMRI scans of participants
watching the movie Forrest Gump (and also listening to a version for the blind
in another scan of equal length).
%
Since its first publication in 2014 \citep{hanke2014audiomovie}, this resources
has led to eight independent studies of international research groups outside
Magdeburg that were published in peer-reviewed journals
(http://studyforrest.org).
%
Based on my knowledge about cinematographic editing techniques, I annotated the
~870 movie cuts with respect to depicted major locations, scene settings,
within-scene rooms and perspectives \citep{haeusler2016cutanno}.
%
The annotation served as prerequisite to investigate cognitive functions, such
as spatial reorientation, perspective taking, and memory retrieval for known
spatial layouts in terms of their occurrence in the movie stimulus.

For 15 participants in the studyforrest datasets two different full-length movie
scans are readily available: one with the normal audio-visual movie
\citep{hanke2016simultaneous}, and a second one with an audio-only movie variant
(originally produced for a visually impaired audience) that is time-locked to
the audio-visual version \citep{hanke2014audiomovie}.

``A case study for successfully sharing naturalistic stimuli is studyforres t.org
(Hanke et al., 2014, 2016), a dataset which includes audio-only and audio-visual
viewings of the movie Forrest Gump during fMRI acquisi- tion. Study Forrest is a
data collection and curation effort designed to serve as a community resource
for new discoveries, in the tradition of distributed science collaborations such
as the International Genetically Engineered Machine competitions. As of October
2019, 29 unique studies had been published using the studyforrest.org dataset,
17 of which were published without any of the original authors of the data
release. This is possible in large part thanks to the richness of naturalistic
stimuli, where the same movie can be used for both task-free as well as
stimulus-driven analyses, with the original stimulus re-annotated for particular
features of interest. For example, studyforrest.org has been used to test
cerebro- vascular biomarkers (Voss et al., 2017) but, among other features, was
also annotated for expressed emotion (Labs et al., 2015) which later informed a
study on emotion encoding gradients in the brain (Lettieri et al., 2019).A case
study for successfully sharing naturalistic stimuli is studyforres t.org (Hanke
et al., 2014, 2016), a dataset which includes audio-only and audio-visual
viewings of the movie Forrest Gump during fMRI acquisi- tion. Study Forrest is a
data collection and curation effort designed to serve as a community resource
for new discoveries, in the tradition of distributed science collaborations such
as the International Genetically Engineered Machine competitions. As of October
2019, 29 unique studies had been published using the studyforrest.org dataset,
17 of which were published without any of the original authors of the data
release. This is possible in large part thanks to the richness of naturalistic
stimuli, where the same movie can be used for both task-free as well as
stimulus-driven analyses, with the original stimulus re-annotated for particular
features of interest. For example, studyforrest.org has been used to test
cerebro- vascular biomarkers (Voss et al., 2017) but, among other features, was
also annotated for expressed emotion (Labs et al., 2015) which later informed a
study on emotion encoding gradients in the brain (Lettieri et al., 2019)''
\citep{dupre2020nature}.


\section{Aims of thesis}
%% At the end of the introduction, add a subchapter on the Aims of Thesis in
%% which you describe the research question and the objectives of your work on
%% a maximum of two pages


\subsection{Overview of aims}
%
Previous work has shown that it is, in principle, possible to combine BOLD fMRI
with a rich naturalistic stimulus for the purpose of localizing areas associated
with particular brain functions \citep{bartels2004mapping}.
%
This can be approached by either modeling specific stimulus features, or by
using the high-dimensional nature of such a stimulus to derive a common
representational space for aligning functional properties of cortex.
%
However, there has been no study aimed at replacing an established localizer
paradigm with a naturalistic stimulus as a short diagnostic routine.
%
A part of a movie of the same length as a dedicated localizer experiment would
have substantial advantages (e.g. task demands compliance, and data quality)
over simplified stimuli presently used in localizer paradigms.
%
Given that naturalistic stimuli offer a rich stimulation correlating with a
variety of different brain functions, they could replace multiple dedicated
localizers and thus offering a more comprehensive and efficient diagnostic in a
similar or even less amount of time.



\subsubsection{Auditory stimulus to localize visual area}
%
This investigation will also provide insight if it is feasible to produce an
audio-only localizer paradigm as an alternative for studies where no visual
stimulation is feasible or desired.
%
Previous results show that there are significant voxel-wise BOLD response time
series correlations between the two datasets in brain areas associated with
speech and story processing (s. Figure 3 in \citep{hanke2016simultaneous}).

%
\todo[inline]{following is actually not true anymore}

For this purpose, I want to evaluate two strategies:

\begin{itemize}
    \item direct modeling of a natural stimulus and
    \item prediction via functional alignment to a reference population
\end{itemize}


\subsubsection{Reproducibility, transparency, openly shared}

\todo{transparency, reproducibility, sharing, check \citep{halchenko2021datalad}}
%
All required algorithms and analyses will be implemented in a way that enables
automated processing.
%
This will facilitate documentation and efficient processing of large number of
datasets.
%
Implementations will be based on open-source software tools to guarantee a
maximum level of reproducibility, and relative ease of long-term maintenance
\citep{eglen2017toward}.
%
The goal is to provide researchers with efficient, validated, ready to use
solutions (stimulation, MR sequence configuration, analysis software) for
functional localization that can be incorporated into their study protocols.
%
We share results and code in standardized file and data formats und use free and
open-source software (FSL, Python packages like ...). Only use publicly
available input data (. \citep{glen2017toward}).


``Naturalistic approaches have a strong potential to further transparent and
openly shared neuroscientific research.  The richness of the data sets collected
inherently favours the analysis of data to address multiple questions or their
reanalysis to address questions other than those exam- ined in the initial
analysis. For example, if one researcher conducts an experiment using stories
and is mainly inter- ested in syntactic processes, the nature of the stimuli
opens up the possibility for other researchers to model phonological processes,
for example, if the data is openly accessible and annotated appropriately. A
stan- dardised way of sharing data from naturalistic exper- imental paradigms is
needed, in order to ensure an easy navigation through the “maze” of openly
shared data and an informed decision regarding which datasets are suitable for
answering specific hypotheses. Ideally, researchers would not only share the
neuroimaging/ electrophysiological data, but also the details of the para- digm
including specific time indications, task descrip- tions and the stimulus files as
presented to the participant. This proposal is different from previous task
ontologies (Poldrack & Gorgolewski, 2014; Turner & Laird, 2012) in that it
captures the specifics of a more eco- logically valid approach and the use of a
natural task, a design which has not yet been incorporated into existing task
ontologies'' \citep{kandylaki2019story}.

%
``A visionary aim of openly sharing data and meta-data of more ecologically
valid designs is the holistic under- standing of human brain function.
Researchers would be able to choose carefully from correctly tagged data- sets
and model brain responses using big data. Then, assisted by current methods in
computer science such as machine learning and artificial neural networks,
researchers could attempt to re-construct brain function in an ecologically
valid manner (see also Hasson et al., 2018)'' \citep{kandylaki2019story}.


\subsubsection{Study 1: Annotation of audio-description}


\todo[inline]{why we need an annotation of speech}

%
Movies are designed to entertain the audience and not to conduct research.
%
Variables (i.e. the ``features'' embedded in the naturalistic stimuli) might be
confounded.
%
Moreover, features of interest might be highly correlated, making a it
impossible to controlling them statistically.
%
Hence, researchers often rely on data-driven methods to analyze the data
``because they do not require an explicit model of the task or stimulus'' and/or
``constructing such a model may be prohibitively difficult.''
\citep{nastase2019measuring}.
%
Which is true but the wrong mindset that lead to a lack of annotation in most
datasets derived from naturalistic paradigms.
%
From data-driven approaches, we gained a lot of knowledge, but data-driven
approaches essentially fall short in case you want to correlate discovered brain
activation patterns with psychological processes.
%
``annotation bottleneck'' \citep{aliko2020naturalistic}i.

``Nevertheless, naturalistic stimuli add additional layers of complexity to the
technical difficulty of data sharing. Some of these – such as recording
acquisition and presentation timings as well as annotating stimuli for features
of interest – are well-recognized from the task-based neuroimaging literature
and implemented in existing standards such as BIDS'' \citep{dupre2020nature}.


\subsubsection{Study 2: task-based vs. visual vs. auditory}
%
It is likely that a rich natural stimulation evokes more widespread
network activity, compared to a simplified stimulus that additionally requires
the subject to perform a task.
%
Hence: compare task-free localizer to dedicated task-based speech localizer
%
compare diagnostic performance of the task-free movie / audio-description fMRI
recordings will be compared to the results of the localizer paradigm.
%
a specific diagnostic contrasts can be selected.

\subsubsection{Study 3: Functional alignment \& stimulus length}
%
All analyses up to this point will have been performed on 2h-long
scans, but any clinical application must aim to minimize the scan time/cost.
%
We test wether reliable functional alignment can be achieved with a task-free,
natural stimulation ``calibration” scan that requires no more acquisition time
than a conventional localizer paradigm.
%
We will estimate the trade-off between diagnostic quality
and required effective scan time by progressively reducing the duration of input
BOLD fMRI data and comparing the results of the reduced model to the reference
computed from the full length scan.

\paragraph{Predict ROI from a reference group}
%
fMRI data acquired from an individual will not be analyzed directly regarding a
specific cognitive function, but will be used to align that individual brain's
voxel space with a common high-dimensional representational reference space.
%
Each axis in this common space can be seen as a kind of cortical tuning
function.
%
The orthonormal transformation of an individual voxel space into this common
reference reflects the particular linear combination of voxel response time
series with respect to each common space component.
%
Using a leave-one-subject-out strategy, individual results of the conventional
localizer will then be projected into the common space, aggregated, and
re-projected into the voxel space of the left-out individual for comparison with
the localizer results for that individual (see Figure 2).
%
Using movie-evoked brain activity, hyperalignment procedure learns subject-wise
optimal transformations of brain activity into a common representational space.
%
Once aligned, the brain activity of the reference group can be used to predict
the activity of another subject.
%
The localization of speech areas will also be performed in the common reference
space directly (the stimulation time axis is preserved in the common space), and
the results will be projected into the voxel space of the left-out individual,
where they can be compared with the localisation result derived from that
subject’s movie scan.
%
Once a valid alignment is established, the inverse transformation is then used
to project functional properties of the common reference into that individual's
voxel space.


\paragraph{minimum length of stimulus; ``calibration scan''}
%
minimum data requirement for reliable functional alignment (within-subject
test-retest???
%
estimate the minimum scan time requirement for deriving a valid functional
alignment by further reducing the amount of input data of the left-out
individual.
\paragraph{just uses an intersecting time-series between stimulus sets}
%
The joint dataset will enable me to study whether a functional alignment to the
common reference space can be performed based on a short scan.
