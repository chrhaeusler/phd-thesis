\todo[inline]{speaking of ``topography'' is little overstatement}

\section{Introduction}

% higher visual areas higher visual areas
In the domain of higher-visual perception, functionally defined,
category-selective brain regions like the \ac{ppa} \citep{epstein1998ppa}, the
\ac{ffa} \citep{kanwisher1997ffa}, or \ac{eba} \citep{downing2001bodyarea}
exhibit significantly increased \ac{bold} activity correlated with a
``preferred'' \citep{debeck2008interpreting} stimulus class.

\todo[inline]{rephrase to one sentence that is not plagiarized}

% variability on higher functional areas 1
Previous studies have shown that hemodynamic activity in functionally defined,
category selective vary in ``location, extent and magnitude'' across study
participants \citep{rosenke2021probabilistic, frost2012measuring,
zhen2017quantifying, zhen2015quantifying}.

% variability on higher functional areas 2
The ``topographies of these category-selective areas are mostly distributed
similarly across individuals, but great individual variability exists in the
locus, the size, and the shape of the category-selective areas
[\citep{rosenke2021probabilistic, zhen2017quantifying, zhen2015quantifying,
frost2012measuring}]'' [still similar phrasing to\citep{jiahui2020predicting}].

% definition: functional-anatomical correspondence
However, location, size and shape of category-selective regions can differ
across individuals by millimeters or centimeters [\citep{zhen2017quantifying,
zhen2015quantifying}] \citep{feilong2018reliable}.


\todo[inline]{maybe, give an example; maybe, better a non-PPA study}

For example \citet{zhen2015quantifying} have shown that increased hemodynamic
activity in face-selective regions ``showed dramatic variation in the location,
extent, and strength across individual brains'' \citep{zhen2015quantifying}, and
\citet{zhen2017quantifying} have shown that increased hemodynamic activity in
scene-selective areas like the PPA ``varied dramatically in location, extent,
and magnitude'' across study participants \citep{zhen2017quantifying}.

% localizer
Hence, in order to identify functional areas in individual persons, block-design
\textit{functional localizer} paradigms are traditionally used that contrast
regressors representing modeled hemodynamic responses to presented stimulus
classes (i.e. landscapes, faces or bodies).



\paragraph{Problem: domain-specific and fucking boring}

% problem: one localizer for one domain
However, functional localizers are designed to maximize detection power and thus
dedicated to map just one domain of brain functions, for example, retinotopic
visual areas \citep{wang2015probabilistic}, scene-selective regions
\citep{stigliani2015temporal}, theory of mind \citep{spunt2014validating}, or
semantic processes \citep{fernandez2001language}.
% which gets messy
Consequently, if one wants to map a variety of domains, the approach ``one
paradigm for one domain of functions'' gets time-consuming and inefficient.

\todo[inline]{maybe cut part on localizer batteries 'cause it's in the general
intro already}

% localizer batteries: intro
Researchers have tried to tackle that issue by creating time-efficient
multi-functional \textit{localizer batteries} \citep[e.g.,][]{barch2013function,
drobyshevsky2006rapid, pinel2007fast}.

% task based = shit
Nevertheless, the diagnostic quality of localizer batteries relies heavily on
a participant's comprehension of the task instructions and general compliance,
a criterion that can be difficult to meet in clinical or pediatric populations.



\paragraph{what we have done in study 1 \& 2}

\todo[inline]{streamline with general intro}
%
In study 2 \citep{haeusler2022processing}, we have shown that a functionally
defined region, such as the \ac{ppa}, can be localized using a model-driven
\ac{glm} analysis that is based on the annotated temporal structure of a
two-hour long naturalistic stimulus.
%
Results also suggest that a naturally engaging, purely auditory paradigm like an
audio-description could, in principle, substitute a visual localizer as a
diagnostic procedure to assess brain functions in visually impaired individuals
\citep{haeusler2022processing}.
%
However, a two-hour long paradigm is unsuitable for a clinical application due
to practical and monetary reasons.


\subsection{Solution: estimate it from reference group}

% solution: predict from reference
An approach to reduce time and costs of individual diagnostics is to identify a
functional area in an individual person based on data collected from an
independent sample of different persons (i.e. a \textit{reference group}).


\subsubsection{Anatomical alignment}

\todo[inline]{cite just the best paper that did it}
%
A traditional procedure \citep{frost2012measuring, weiner2018defining,
zhen2017quantifying, zhen2015quantifying, rosenke2021probabilistic,
wang2015probabilistic} to estimate the most probable location of a functional
area in a person's anatomy from a reference group performs --- in order to
resolve anatomical variability across persons --- an \textit{anatomical
alignment}:
%
functional data from persons in the reference group are aligned to (i.e.
projected into) a \textit{common anatomical space}, and then projected from the
common anatomical space into the individual person's brain anatomy.


\paragraph{volume-based}
% volume-based
Volume-based anatomical alignment \citep[s.][for a review]{klein2009evaluation}
aligns voxel-wise data of individuals to a three-dimensional brain template
\citep[e.g., MNI152 template;][]{fonov2011unbiased}.


\paragraph{Problem of volume-based alignent}

However, ''the cortical surface is highly folded and, although there are many
consistencies across subjects'' \citep{frost2012measuring}, ``there is
significant interindividual variability in sulcal and gyral folding patterns
[Fischl et al., 2008; Hill et al., 2010; Van Essen and Dierker, 2007]''
\citep{zhen2017quantifying}.
%
``Volume-based registration techniques are unable to account for individual
differences in cortical folding'' \citep{frost2012measuring}.
%
As a result, ``the same coordinate in MNI space can refer to different
anatomical areas across subjects'' \citep{frost2012measuring}.



\paragraph{Surface-based alignment: the method}

% method in one sentence
Surface-based anatomical alignment \citep{fischl1999cortical} aligns vertex-wise
data of individuals to a two-dimensional template \citep[e.g., FreeSurfer
fsaverage template;][]{fischl1999high} and ``explicitly align the folding
pattern of the cortical surface [Fischl et al., 1999b; Goebel et al., 2006]''
\citep{frost2012measuring}.

% alignment of Brodman areas (histologically defined)
Previous studies have shown that surface-based anatomical alignment ``improved
macro-anatomical correspondence across brains, and provided evidence of
improved concomitant alignment of histiologically defined Brodman areas [Fischl
et al., 2008]'' \citep{frost2012measuring}.


\paragraph{Surface-based works better}

\todo[inline]{both compare to linear volume-based registration}

% Frost: what they did
\citet{frost2012measuring} ``localized 13 widely studied functional areas and
found a large variability in the degree to which functional areas respect
macro-anatomical boundaries across the cortex'' \citep{frost2012measuring}.
%
They ``substantially reducing macro-anatomical variability of the cortex through
curvature driven surface alignment'' \citep{frost2012measuring}.
% Frost: results
They compared standard affine/piece- wise linear volume normalization techniques
to surface-based alignment,
%
and found that ``curvature driven cortex based alignment
(CBA)'' reduces variability of functionally localized regions in high-level
visual cortex across study participants''.


\todo[inline]{\citet{rosenke2021probabilistic}}

% high-visual cortex
\citet{rosenke2021probabilistic} have shown cortex-based alignment reduced
between-subject variability ``in the majority of functionally defined ROIs
[which ones] compared to nonlinear volumetric alignment [NVA]''
\citep{rosenke2021probabilistic}.



\todo[inline]{\citet{weiner2018defining}}

``Dice coefficients were lower than CBA, as CBA out-performed Talairach
alignment by nearly a factor of two in the right hemisphere (1.85 ± 0.26) and by
a factor of more than two in the left hemisphere (2.23 ± 0.24).
%
The implementation of CBA significantly improved the alignment and
predictability of place selectivity within medial VTC across participants
compared to Talairach alignment'' \citep{weiner2018defining}.


\paragraph{Remaining variability}
%
However, surface-based alignment strategies that respect cortical foldings
\citep{fischl2012freesurfer, yeo2009spherical} can reduce but not eliminate
\citep[e.g.,][]{coalson2018impact, benson2014correction, natu2021sulcal,
wang2015probabilistic, frost2012measuring, langers2014assessment, weiner2014mid,
rosenke2021probabilistic} the variability across persons.

``Even with an arguably more accurate surface-based alignment, the variability
of functional regions remains prominent  [Frost and Goebel, 2012]''
\citep{zhen2015quantifying}.

% works better with retinotopic areas than category-specific areas
Similar to NVA, similarity across person after CBA is higher ``for
retinotopically defined regions, with character-selective regions showing the
lowest consistency for both alignments, closely followed by mFus- and
IOG-faces'' \citep{rosenke2021probabilistic}.


\paragraph{functional-anatomical correspondence}
%
``Differences still remain in the location of functional regions after
curvature-based alignment indicating that not all functional areas are tightly
bound to anatomical landmarks (Figs. 5, 8 and 9)'' \citep{frost2012measuring}.
%
''There is a strong structural-functional correspondence in some areas whilst in
others the spatial location of the functional area varies greatly across
subjects within a cortical area'' \citep{frost2012measuring}.
%
``there is a surprising amount of variability in that not all
functional areas are tightly bound to anatomical landmarks, i.e. there is no
general rule describing how macroanatomical and functional areas correlate''
\citep{frost2012measuring}.
%
``Remaining variability in the observed spatial location of functional regions,
thus, reflects the "true" functional variability, i.e. the quantified
variability is a good estimator of the underlying structural–functional
correspondence'' \citep{frost2012measuring}.

%
\textit{functional--anatomical correspondence}: ``the mismatch between brain
function and anatomy'' \citep{feilong2018reliable}, across persons.



\subsubsection{Functional alignment}

\todo[inline]{mih: This is too dense.}

\todo[inline]{mih: Mixes "what is functional alignment" with what is a CFS with
how to align (but only linearly, i.e. with a matrix).}

\todo[inline]{It needs to mention at least "rotation" and must provide some
conceptual idea on what is being aligned, if it is not the 3D space of the
anatomy.}

\todo[inline]{Maybe reintroduce "CFS" (comes out of the blue)}

% intro
Since anatomical alignment addresses the issue of anatomical variability but
does not resolve the issue of functional-anatomical variability across subjects,
algorithms --- like \textit{hyperalignment} \citep{haxby2011common,
guntupalli2016model} or the \textit{shared response model}
\citep{chen2015reduced, zhang2016searchlight} --- have been developed that
perform a \textit{functional alignment} to a \textit{\ac{cfs}}.
%
Functional alignment does not align voxels (or surface vertices) that share the
same anatomical location but voxels that share similar functional properties in
order to preserve functional idiosyncrasies across persons.
%
In general, functional alignment algorithms are usually used to construct both a
\ac{cfs} (serving as ``a high-dimensional, functional brain template'') as well
as subject-specific transformation matrices.

\todo[inline]{Here it is conceptual, maybe just "transformation"?}
%
A subject's transformation matrix can be used to project functional data from a
subject's anatomical space into the \ac{cfs}, or to project data from the
\ac{cfs} into a subject's anatomical space \citep{haxby2020hyperalignment}.
%
The construction of the \ac{cfs} and transformation matrix can be created (i.e.
\textit{trained}) based on the maximization of the inter-subject similarity of
\ac{bold} response time series correlating with a time-locked external
stimulation \citep{haxby2011common, chen2015reduced, sabuncu2010function}, or
based on the inter-subject similarity of connectivity profiles
\citep{feilong2018reliable, guntupalli2018computational, nastase2019leveraging}.
%
Functional alignment algorithms can be applied to \ac{fmri} data from paradigms
employing simplified stimuli.
%
However, it has been shown that data from naturalistic stimuli provide increased
validity of the \ac{cfs} and increased generalizability of transformation
matrices to novel stimuli or tasks, presumably because naturalistic stimuli
sample a broader range of brain states \citep{haxby2011common,
guntupalli2016model}.


\subsubsection{Estimation via functional alignment}
%
Hence, a more promising procedure to estimate the most probable location of a
functional area in a person's anatomy from a reference group performs --- in
order to resolve functional-anatomical variability across persons --- an
functional alignment:
%
functional data from persons in the reference group are aligned to (i.e.
projected into) a \ac{cfs}, and then projected from the \ac{cfs} into the
individual person's brain anatomy.


\paragraph{studies that estimated functional areas using functional alignment}

% intro
Previous studies \citep{jiahui2020predicting, guntupalli2016model,
haxby2011common} that employed hyperalignment have shown that a subject's
idiosyncratic retinotopy of occipital areas and functional topography of
category-selective areas in the ventral temporal cortex can be estimated by
projecting data of a reference group through a \ac{cfs} into that subject's
cortical anatomy.
% used stimuli and length
For example, \citep{jiahui2020predicting} constructed a \ac{cfs} and
transformation matrices based on data from a) the movie ``Grand Budapest Hotel''
($\sim$\unit[50]{m}; \ac{tr}=\unit[1]{s}), and b) ``Forrest Gump''
($\sim$\unit[120]{m}; \ac{tr}=\unit[2]{s}).
% summary of results
Results revealed that the empirical results of a statistical contrasts aimed to
localize the \ac{ffa} correlate more highly with maps that were estimated from
other subjects' data based on hyperalignment than with maps that were estimated
based on surface-based, anatomical alignment \citep{jiahui2020predicting}.


\subsection{Here, we}

\todo[inline]{we do way too much, imo emphasis should be on (partial) alignment
via movie segments to predict PPA from functional localizer; alignment via
functional localizer runs and the prediction of "auditory PPA" are a bonus and
should be considered at the end of the introduction more like a "sideshow"}

\paragraph{PPA}
% summary in one sentence
Here, we focus on the \ac{ppa} as an example of a functional area.

\todo[inline]{PPA varies more than, e.g. FFA}

%
\citet{zhen2017quantifying} ``observed that SSRs showed remarkable variability
in their scene selectivity (CV=0.45), which again is nearly twice that seen for
faces in FSRs (CV=0.25) [\citet{zhen2015quantifying}]''.

%
\citet{zhen2017quantifying} ``noticed that the SSRs showed larger
interindividual variability than the FSRs in both spatial topography and
functional selectivity''.

\todo[inline]{Surface-based alignment improves but still...}

``The prediction accuracy of our pROI is significantly higher than
that achieved by volume-based Talairach alignment'' \citep{weiner2018defining}.
%
\citet{weiner2018defining} showed that ``that cortical folding patterns and
probabilistic predictions reliably identify place-selective voxels in medial VTC
across individuals and experiments''.
%
However, ``this structural-functional coupling is not always perfect and there
is inter-subject variability as to how much the place-selective voxels extend
within the PHG, as well as the LG and medial aspects of the fusiform gyrus.
%
Despite this inter-subject variability, place-selective voxels are always
located within the CoS across participants.'' \citep{weiner2018defining}.

%
\citet{frost2012measuring} ``revealed that the PPA demonstrated significant
spatial variability through constructing the probabilistic atlases of the PPA''
after surface-based alignment \citep{zhen2017quantifying}.



\paragraph{cross-validation}

\todo[inline]{mih: create figure defining labels (e.g., "training subjects")}

\todo[inline]{coh: needs time that I do not have}

We estimate the results (statistical $Z$-maps) of a visual localizer's
$t$-contrast in one subject (i.e. the \textit{test subject}) based on data from
other subjects (i.e.  the \textit{training subjects}).

% my intro
We perform an exhaustive leave-one-out cross-validation (LOOCV) analysis using
data from XY subjects.

%
``we performe an exhaustive LOOCV analysis after the volumetric (NVA) as well as
surface (CBA) alignment to establish how well our atlas can predict fROIs in new
subjects'' \citep{rosenke2021probabilistic}.
%
``For each fold of the LOOCV, we generated a group probabilistic fROI (G) and a
left-out subject's individual fROI (I)'' \citep{rosenke2021probabilistic}.

% from Weiner
``In each iteration, the pROI was first generated from 23 participants and then
tested how well it predicted the left out, 24th participant''
\citep{weiner2018defining}.

% Common spaces
For each subject (the test subject), we create create the common spaces based on
data from other subjects (the training subjects).

% both, anatomical & functional alignment
We employed both a procedure using a volume-based, anatomical alignment as well
as a procedure using a volume-based, functional alignment in order to predict
$z$-values within a \ac{roi}.



\subsubsection{our anatomical alignment procedure}

\todo[inline]{keep this to a minimum}

%
In case of anatomical alignment
%
MNI space
%
transformation matrices from the study forrest project.



\subsubsection{our functional alignment procedure}

\todo[inline]{imo, this part should be kept to a minimum}

\todo[inline]{we do way to much: alignment via naturalistic stimuli vs.
functional localizer runs, combined with partial alignment and regular PPA vs.
auditory PPA}

\todo[inline]{hence, details, including all "mathy", stuff into methods}

%
In case of the functional alignment, we apply the \ac{srm} algorithm
\citep{chen2015reduced, richard2019fast} to the training subjects' \ac{bold}
\ac{fmri} time series of the visual localizer and naturalistic stimulus
paradigms in order to both create an \ac{cfs} as well as derive a
subject-specific transformation matrices.


\paragraph{SRM}
%
The \ac{srm} is an unsupervised probabilistic latent-factor model that
decomposes \ac{bold} \ac{fmri} responses time series of participants
experiencing the same stimulus into a lower-dimensional space of shared features
(i.e. the \ac{cfs}) and subject-specific orthogonal topographic transformation
matrices \citep{kumar2020brainiak, cohen2017computational}.
%
The dimensions of the shared feature space do not correspond to individual
voxels but \textit{shared features} that can be understood as a weighted sum of
many voxels distributed across the full voxel space of each subject
\citep{kumar2020brainiak}.
%
In contrast to hyperalignment, the number of dimensions of the shared feature
space is not set by the number of voxels (or surface vertices) but is
pre-specified by the researcher, a procedure that also filters out noise and
reduces overfitting \citep{chen2015reduced}.

\todo[inline]{rephrase to better match our case}
% project into CFS
Second, we used the subject-specific transformation matrices in order to perform
a mapping of the visual localizer's results (i.e. the training subjects'
\textit{empirical $Z$-maps}) from each training subject's voxel space into the
\ac{cfs} (i.e. shared feature space).


\paragraph{alignment via naturalistic stimuli}

% align test subject
Third, we use time series data from the naturalistic stimuli to align the test
subject to the \ac{cfs} (that was derived from the training subjects' data) in
order to acquire the test subject's transformation matrix.
% project from CFS into test subject
Last, the transpose of the test subject's transformation matrix is used to
project the training subjects' functional localizer results from the \ac{cfs}
into the test subject's voxel space serving as an estimation (hence, a
\textit{predicted $Z$-map}) of the test subject's localizer results (i.e. the
test subject's empirical $Z$-map).


\paragraph{alignment via localizer runs}

\todo[inline]{Procedure \& hypotheses might be shifted downwards below the
hypotheses regarding prediction via (partial) alignment using movie to predict
localizer}

%
We also test alignment based on one to four localizer runs (each lasting
\unit[312]{s}; \unit[5.2]{m}) of data from the visual localizer to acquire the
transformation matrices and perform a cross-subject-prediction (vs.
cross-subject-cross-experiment-prediction).
%
Because: ``Between-subject models with SRM can, in some cases, exceed the
performance of within-subject models because (a) the reduced-dimension shared
space can highlight stimulus-related variance by filtering out noisy or
non-stimulus-related features, and (b) the between-subject model can effectively
leverage a larger volume of data after functional alignment than is available
for any single subject'' \citep{kumar2020brainiak}.


\paragraph{partial alignment}

\todo[inline]{previous studies on PPA used anatomical scans}

Given that a two-hour long stimulation is unsuitable for a clinical setting, we
critically also assess the relationship between the length of the naturalistic
stimulus used for alignment of the test subject to the fixed \ac{cfs} and the
subsequent estimation performance.
%
Therefore, we use an increasing number of segments of the naturalistic stimuli
(each lasting $\sim$\unit[15]{m}) to align the test subject to the corresponding
segments' time points within the \ac{cfs}.


\subsubsection{prediction of "auditory PPA"}

\todo[inline]{Procedure \& hypotheses might be shifted downwards below the
hypotheses regarding prediction via (partial) alignment using movie to predict
localizer}

%
Results of our previous study \citep{haeusler2022processing} suggest that not
just the dedicated visual localizer \citep{sengupta2016extension} but also the
audio-visual movie Forrest Gump \citep{hanke2016simultaneous} and the movie's
audio-description \citep{hanke2014audiomovie} sample the response vector space
of hemodynamic responses to ``spatial information'' in a time-locked manner
across subjects.


\subsubsection{assessment via Pearson's}
%
We assess the predictability of a left-out subject's activation pattern using
Pearon's Coefficient


\subsection{Hypotheses}

\todo[inline]{draft; s. also general introduction}
%
We hypothesized that an increased quantity of data used to calculate the
transformation matrices of the test subjects would lead to an increasing
prediction performance.
%
Further, we hypothesized that functional alignment procedure would eventually
outperform an estimation based on anatomical alignment.


\subsection{Summary of results}

\todo[inline]{usually, papers provide a overview of methods, results, and often
also a conclusion in the intro}

% template from Jiahui
``Results show strong correlations of face-selectivity topographic maps derived
from a subject's own localizer data with maps derived from other subjects'
localizer data projected into that subject's cortical anatomy''
\citep{jiahui2020predicting}.


\subsection{Vision}

\todo[inline]{2-3 sentences are enough here; better write more in discussion}

Our results suggest that it is possible to ``scan once, estimate many''...



\section{Methods}

\todo[inline]{it's pretty rigorously shortened now; I hope it's okay and
nothing that should urgently be mentioned is missing}

% we get the data from the naturalistic PPA paper (its subdataset)
% datalad get -n inputs/studyforrest-ppa-analysis/inputs/studyforrest-data-aligned
% datalad get inputs/studyforrest-ppa-analysis/inputs/studyforrest-data-aligned/sub-??/in\_bold3Tp2/sub-??\_task-a?movie\_run-?\_bold*.*

% reference to PPA-Paper
For the current study, we used the same subset of the studyforrest dataset that
was used in study 2 \citep{haeusler2022processing}:
%
the same subjects ($N=14$) were
% VIS
a) participating in a dedicated six-category block-design visual localizer
\citep{sengupta2016extension},
% AV
b) watching the audio-visual movie ``Forrest Gump''
\citep{hanke2016simultaneous}, and
% AD
c) listening to the movie's audio-description \citep{hanke2014audiomovie}.
% see corresponding papers for details
An exhaustive description of participants, stimulus creation, procedure,
stimulation setup, and fMRI acquisition can be found in the corresponding
publications, whereas a summary is provided in \citet{haeusler2022processing}.



\subsection{Preprocessing}

% data sources
The current analyses were carried out on the same preprocessed fMRI data (s.
\href{https://github.com/psychoinformatics-de/studyforrest-data-aligned
}{\url{github.com/psychoinformatics-de/studyforrest-data-aligned}}) that were
used for
%
a) the technical validation of the dataset \citep{hanke2016simultaneous},
%
b) the localization of higher-visual areas \citep{sengupta2016extension}, and
%
c) the investigation of responses of the \ac{ppa} correlating with naturalistic
spatial information in study 2 \citep{haeusler2022processing}.

%
We reran the preprocessing and analyses steps performed in
\citet{sengupta2016extension} and \citet{haeusler2022processing} using FEAT
v6.00 \citep[FMRI Expert Analysis Tool;][]{woolrich2001autocorr} as shipped with
FSL v5.0.9 \citep[\href{https://www.fmrib.ox.ac.uk/fsl}{FMRIB's Software
Library;}][]{smith2004fsl} in order to reproduce both the time series that
served as final input for the statistical analyses in the two previous studies
as well as their results (i.e. the statistical $Z$-maps).
% temporal filtering
In summary, high-pass temporal filtering was applied using a Gaussian-weighted
least-squares straight line to every run of the visual localizer (cutoff period
of \unit[100]{s}; sigma= \unit[100]{s}/2)[???], and every segment of the movie
and audio-description (\unit[150]{s}; sigma=\unit[75.0]{s}).
% brain extraction
Brains were extracted from surrounding tissues using BET \citep{smith2002bet}.
% spatial smoothing
As in the previous studies, data from all three paradigms were spatially
smoothed (Gaussian kernel with full width at half maximum of \unit[4.0]{mm}).
% grand mean normalization
A grand-mean intensity normalization was applied to each run of the functional
localizer and each segment of the naturalistic stimuli.


\subsection{Modeling of the \ac{cfs}}

On these reproduced time series data, we then performed further analyses steps
via Python scripts that relied on
%
NiBabel v3.2.1 (\href{https://nipy.org}{\url{nipy.org}}),
%
NumPy v1.20.2 (\href{https://numpy.org}{\url{numpy.org}}),
%
Pandas v1.2.3 (\href{https://pandas.pydata.org}{\url{pandas.pydata.org}}),
%
Scipy v1.6.2 (\href{https://scipy.org}{\url{scipy.org}}),
%
scikit-learn v1.0 (\href{https://scikit-learn.org}{\url{scikit-learn.org}}),
%
BrainIAK v0.11 (\href{https://brainiak.org}{\url{brainiak.org}}),
%
Matplotlib v3.4.0 (\href{https://matplotlib.org}{\url{matplotlib.org}}),
%
seaborn v0.11.2 (\href{https://seaborn.pydata.org}{\url{seaborn.pydata.org}}),
%
and calling command line functions of FSL.

%\paragraph{Fixing FSL output}

% grand_mean_for_4d.py (formerly: data_normalize_4d.py):
% is not necessary anymore: FSL has applied grand mean scaling to
% 'filtered_func_data.nii.gz'

% input: 'sub-*/run-?.feat/filtered_func_data.nii.gz' (of VIS, AO & AV)
% output: saved to 'sub-??_task-*_run-?_bold_filtered.nii.gz'

% FSL adds back the mean value for each voxel's time course at the end of the
% preprocessing;
% hence, the script substracts that mean again but multiplies it by 10000
% (like FSL does it, too)

% definition of grand mean scaling for 4d data:
% voxel values in every image are divided by the average global mean
% intensity of the whole session. This effectively removes any mean global
% differences in intensity between sessions.

% FSL User Guide:
% filtered_func_data will normally have been temporally high-pass filtered,
% it is not zero mean; the mean value for each voxel's time course has been
% added back in for various practical reasons.
% When FILM begins the linear modeling, it starts by removing this mean.


\paragraph{Getting the data in shape}

% masks-from-mni-to-bold3Tp2.py:
% - merges unilateral ROIs overlaps (already in MNI) to bilateral ROI
% - output: 'masks/in_mni/PPA_overlap_prob.nii.gz'
% - warps union of ROIs from MNI into each subjects space
% output: 'sub-*/masks/in_bold3Tp2/grp_PPA_bin.nii.gz' + audio_fov.nii.gz dilate
% the ROI masks by 1 voxel; output: 'grp_PPA_bin_dil.nii.gz'

% masks-from-mni-to-bold3Tp2.py:
% warp MNI masks into individual bold3Tp2 spaces

% masks-from-t1w-to-bold3Tp2.py:
% transforms 'inputs/tnt/sub-*/t1w/brain_seg*.nii.gz'
% into individual's bold3Tp2
% output: 'sub-*/masks/in_bold3Tp2/brain_seg*.nii.gz'

% mask-builder-voxel-counter.py:
% builds different individual masks by dilating, merging other masks
% creates a FoV of AO stimulus for every subject from 4d time-series of AO run
% output: sub-*/masks/in_bold3Tp2/audio_fov.nii.gz'
% counts the voxels
% long story short: we cannot used all gyri that contain PPA to some degree
% even if the mask by FoV of AO stimulus and individual gray matter mask

% data_mask_concat_runs.py:
% masks are not dilated and not masked with subject-specific gray matter mask
% outputs:
% 'sub-*_task_aomovie-avmovie_run-1-8_bold-filtered.npy
% 'sub-*_task_visloc_run-1-4_bold-filtered.npy'

\todo[inline]{problem 1: grpPPA contains N=14 subject, not N-1 subjects}

\todo[inline]{problem 2: voxels outside of PPA-mask; probably, because of
warping procedures}

\todo[inline]{how to explain in one sentence that we need reduction of voxels?}

% reason why we do it
The \ac{srm} requires that the number of samples (the number of \acp{tr}) exceed
the number of features (the number of voxels) in order to train a reliable
model.
% union of PPA masks
Hence, in order to create bilateral \acp{roi} in each subjects' anatomical voxel
space, we warped the union of individual \acp{ppa}
\citep[s.][]{haeusler2022processing} from MNI space into each subjects' voxel
space using previously computed subject-specific, non-linear transformation
matrices
\citep[][\href{https://github.com/psychoinformatics-de/studyforrest-data-templatetransforms
}{\url{github.com/psychoinformatics-de/studyforrest-data-templatetransforms}}]{hanke2014audiomovie},
and applied it as a mask to each subjects' time series data.
% AO FoV
Then, data were further masked with the subject-specific \ac{fov} of the
audio-description.
%
The number of remaining voxels for each subject can be seen in Table
\ref{tab:ppamaskvoxels}.



\todo[inline]{show table or just report range, mean \& SD?}

\todo[inline]{If there is nothing else for this table, this could be in-text avg
plus range}

\todo[inline]{I z-scored the runs runwise, concatenated them, and z-scored
again}

% normalization
Data of every run were independently normalized ($z$-scored) to a mean of zero
and a standard deviation of one ($\mu=0$, $\sigma=1$).
%
The last 75 \acp{tr} of the audio-description were missing in subject-04 due to
an image reconstruction problem \citep[s.][]{hanke2014audiomovie}.
%
The \ac{srm} allows the number of voxels to be different across subjects but the
number of samples must be the same.
%
Hence, we removed the last 75 \acp{tr} of the audio-description from the other
subjects' time series.
% summary; AO + AV = 7123 TRs (not 7198 TRs anymore); localizer has 4 x 156 TRs
As a result, the data to fit the \ac{srm} in the following step comprised 3599
\acp{tr} of the movie, 3524 \acp{tr} of the audio-description, and 624 \acp{tr}
of the visual localizer experiment (7747 \acp{tr} in total).
%% concatenate and z-score
The time series of all three paradigms were concatenated and $z$-scored.

\todo[inline]{well, probably I should have cut the last 75TRs of AO first, and
then z-scored the last segment but anyway...; does not make much difference}

\todo[inline]{yes, I performed a second z-scoring but across all runs/paradigms}

\begin{table*}[btp]
    \caption{
    %
    \textbf{Table heading.}
    %
    Number of remaining voxels after time series data of each paradigm
    and subject were masked with the union of individual \acp{ppa} that was
    warped from MNI space into each individual's subjects-space and the
    subject-specific field of view of audio-description.}

\label{tab:ppamaskvoxels}
\begin{tabular}{ll}
    \toprule
    \textbf{Subject} & \textbf{no. of voxels} \\
    \midrule
    sub-01 & 1665 \tabularnewline
    sub-02 & 1732 \tabularnewline
    sub-03 & 1400 \tabularnewline
    sub-04 & 1575 \tabularnewline
    sub-05 & 1664 \tabularnewline
    sub-06 & 1951 \tabularnewline
    sub-14 & 1376 \tabularnewline
    sub-09 & 1383 \tabularnewline
    sub-15 & 1683 \tabularnewline
    sub-16 & 1887 \tabularnewline
    sub-17 & 1441 \tabularnewline
    sub-18 & 1729 \tabularnewline
    sub-19 & 1369 \tabularnewline
    sub-20 & 1437 \tabularnewline
    \bottomrule
\end{tabular}
\caption*{The legend text goes here.}
\end{table*}


\paragraph{Fitting of shared response model: intro}

\todo[inline]{mention that just 'ao \& av' as input were also tested; minimal
"worse" results}

We used the probabilistic \ac{srm} algorithm that approximates the \ac{srm}
based on the Expectation Maximization (EM) algorithm proposed by
\citep{chen2015reduced}, optimized by \citet{anderson2016enabling}, and
implemented in BrainIAK v.11 \citep[Brain Imaging Analysis
Kit;][]{kumar2020brainiak, kumar2020brainiaktutorial} in order to compute the
\ac{cfs} and the transformation matrices for the training subjects.


\paragraph{Cross-validation}
%
In order to avoid data leakage, we followed a leave-one-subject-out
cross-validation procedure:
%
for every subject $n$, we used the $N-1$ training subjects' \ac{bold} \ac{fmri}
responses to the functional localizer, movie, and audio-description to let the
algorithm calculate the \ac{cfs} and $N-1$ transformation matrices.


\paragraph{number of dimensions}

% iterations:
% The number of iterations for the algorithm to minimize the error was set to 30

% features
We chose a value of $k=10$ for the number of shared responses (i.e. the number
of the CFS's dimensions) to be computed considering the temporal and spatial
resolution of our data (\ac{tr} = \unit[2]{s}; \unit[2.5 $\times$ 2.5 $\times$
2.5]{mm}), the average number of voxels per bilateral \acp{roi}, and findings
from \citet{haxby2011common}.
%
\citet{haxby2011common} first used hyperalignment to create a \ac{cfs} of 1,000
dimensions based of functional data (\ac{tr} = \unit[3]{s}) of voxels (of size
\unit[3 $\times$ 3 $\times$ 3]{mm}) located in the ventral temporal cortex.
%
Then, \citet{haxby2011common} reduced the dimensionality of the \ac{cfs} by
applying a \ac{pca} in order to determine the subspace that is sufficient to
capture the full range of response-pattern distinctions.
%
Results revealed that approximately 35 principal components (i.e. dimensions)
were sufficient to represent the information content of a one-hour movie from
which the \ac{cfs} was derived.
%
Results also showed that the cortical topographies of category-selective brain
regions was preserved in the 35-dimensional \ac{cfs} \citep{haxby2011common}.
%
In the current study, we also explored \acp{cfs} of $k=5, 20, 30, 40, 50$ but
results barely varied from a 10-dimensional \ac{cfs}.

% ``The effect of number of PCs on BSC was similar for models that were based
% only on Princeton (n = 10) or Dartmouth (n = 11) data, suggesting that this
% estimate of dimensionality is robust across differences in scanning hardware
% and scanning parameters'' \citep{haxby2011common}.
%
% ``These dimensionality estimates are a function of the spatial and temporal
% resolution of fMRI and the number and variety of response vectors used to
% derive the common space'' \citep{guntupalli2016model}.
%
% ``The true dimensionality of representation in human cortex surely involves
% vastly more distinct tuning functions. Estimates of the dimensionality of
% cortical representation, therefore, will almost certainly be much higher as
% data with higher spatial and temporal resolution for larger and more varied
% samples of response vectors are used to build new common models''
% \citep{guntupalli2016model}.


\paragraph{the math shit}
% math shit from citep{vodrahalli2018mapping}

% ``SRM learns $N$ maps $W_{i}$ with orthogonal columns such that
% $||X_{i}-W_{i}S||_{F}$ is minimized over $\left\{ W_{i}\right\} _{i=1}^{N},S$,
% where $X_{i}\in\mathbb{R}^{v\times{T}}$ is the $i^{th}$ subject's fMRI
% response ($v$ voxels by $T$ repetition times) and
% $S\in\mathbb{R}^{k\times{T}}$ is a feature time-series in a $k$-dimensional
% shared space'' \citep{vodrahalli2018mapping}.

% explain shit
During model fitting, the algorithm uses each $n^{th}$ training subject's
response time series represented as matrix $X_{n}$ ({$v$} voxel by $t$ time
points) to calculate the \ac{cfs} $C$ ($k$ shared responses by $t$ time points)
and subject-specific, orthogonal transformation matrices $T_{n}$ ($v$ voxel by
$k$ shared responses).
% what do the matrices represent?
Each transformation matrix (or \textit{weight matrix}) reflects the loadings of
voxels (i.e. a subject-specific functional topography) onto the shared
responses.
% what are they good for% what are they good for??
Each transformation matrix allows to project the responses of voxels within a
subjects' bilateral \acp{roi} from anatomical voxel space into the
$k$-dimensional \ac{cfs}, and thus functionally aligns a subject's data.
% iteratively fitted
The transformation matrices are randomly initialized and fit over iterations to
minimize the error in explaining participant data.
%
At the same time, the time course of the shared responses in the \ac{cfs} is
learned (s.
\href{https://brainiak.org/tutorials/11-SRM/}{\url{brainiak.org/tutorials/11-SRM}}).

\todo[inline]{phrasing is pretty similar to phrasing in the tutorial}


\paragraph{correlation of regressors in used in our previous studies with shared
responses}

\todo[inline]{imo, plots of correlation matrices should be presented here
already; not in results (as we did it in the PPA paper).}

\todo[inline]{what to adjust in the plots? regressors of low-level confounds
might be dropped? imo, should be kept 'cause the are correlations.}

\todo[inline]{present plots in different order?}

%
VIS regressors vs. shared responses: see Fig.~\ref{fig:corr-vis-reg-srm}.
%
AV regressors vs shared responses see: Fig.~\ref{fig:corr-av-reg-srm}.
%
AO regressors vs shared responses see: Fig.~\ref{fig:corr-ao-reg-srm}.



\begin{figure*}[tbp]
\centering
\includegraphics[width=\linewidth]{figures/corr_vis-regressors-vs-cfs_sub-01_srm-ao-av-vis_feat10-iter30_7123-7747.pdf}
    \caption{
    %
    \textbf{Correlations of regressors of visual localizer and shared responses.
    }
    %
    Lore ipsum.
}
    \label{fig:corr-vis-reg-srm}
\end{figure*}



\begin{figure*}[tbp]
\centering
    \includegraphics[width=\linewidth]{figures/corr_av-regressors-vs-cfs_sub-01_srm-ao-av-vis_feat10-iter30_3524-7123.pdf}
    \caption{
    %
    \textbf{Correlations of regressors of movie and shared responses.}
    %
    Lore ipsum.
    }
    \label{fig:corr-av-reg-srm}
\end{figure*}



\begin{figure*}[tbp]
\centering
    \includegraphics[width=\linewidth]{figures/corr_ao-regressors-vs-cfs_sub-01_srm-ao-av-vis_feat10-iter30_0-3524.pdf}
    \caption{
    %
    \textbf{Correlations of regressors of audio-description and shared
    responses.}
    %
    Pearson correlation coefficients of regressors used in the analysis of
    audio-description to model shared responses (sh. res.) correlating with
    nouns spoken by the
    narrator and features of the \ac{srm} (i.e. shared responses).
    %
    \texttt{geo\&groom} \texttt{geo\&groom\&furn} are combination of regressors
    (as used on the positive side of contrasts).
    %
    The time series of the \ac{srm} were sliced to match the TRs of the
    audio-description.
    }
    \label{fig:corr-ao-reg-srm}
\end{figure*}



\paragraph{negative control}


% shuffled data
As negative control, we randomly shuffled the order of runs of the visual
localizer, and the segments of the movie and audio-description for each training
subject independently before fitting a \ac{srm} to the time series.
% calculate Pearson's
We then calculated the correlations (Pearson's r) between the regressors modeled
in the previous studies and the shared responses in the corresponding \acp{tr}.
% results
Results revealed [minor to] no correlations between the regressors and shared
responses.


\subsection{Alignment of test subjects to a fixed \ac{cfs}}

% AO: 0-451, 0-892, 0-1330, 0-1818, 0-2280, 0-2719, 0-3261, 0-3524
% AV: 3524-3975, 3524-4416, 3524-4854, 3524-5342, 3524-5804, 3524-6243,
%     3524-6785, 3524-7123
% AO+AV: 0-7123

\todo[inline]{how is it done? srm.transform\_subject calls np.linalg.svd()}

\todo[inline]{in-code documentation says: ``Solve the Procrustes problem''}
%
In order to obtain the transformation matrix for the test subject, we then
aligned the test subject's data to the \ac{cfs} that was trained on the training
subjects' data:
%
the algorithm learns a mapping $T_{n}$ of the test subject's anatomical voxel
space into the \ac{cfs} that is kept fixed.


\subsubsection{alignment via movie}
%
In order evaluate the relationship between the length of the naturalistic
stimulus used to align the test subject to the \ac{cfs} and estimation
performance, we based the computation of the transformation matrix on different
quantity of data:
%
For each naturalistic stimulus, we used one up to eight (i.e. all; each lasting
$\sim$\unit[15]{m}) segments to let the algorithm learn the orthogonal mapping
to the corresponding \acs{tr} of the \ac{cfs}:
%
each matrix has a size of $v$ voxels by $k$ shared responses but is based on a
different quantity of data used to calculate the mapping (i.e. subject-specific
functional topographies).
%
As a result, we obtained eight different transformation matrices per subject and
per naturalistic stimulus.


\subsubsection{alignment via audio-description}


\subsubsection{alignment via localizer runs}

\todo[inline]{the following text snippets are also in the discussion}

\paragraph{from project proposal}
%
``While the functional alignment can also be applied to fMRI data from
stimulation paradigms with simplified stimuli, the transformations for
functional alignment have greatly diminished general validity
\citep{haxby2011common}, presumably because such experiments sample a sparser
range of brain states \citep{guntupalli2016model}'' [project proposal].
%
Naturalistic stimuli promise an ``increased validity of derived transformation
for functional alignment by sampling a more diverse set of mental states that
reflect (confound) statistics of the natural environment, and enable
investigation of the acquired data for a variety of research questions (e.g.
visual or auditory perception, spatial cognition; emotion; music, speech or
social perception)'' [project proposal].


\paragraph{from Haxby's group}

%
Naturalistic stimuli sample a broader range of brain states paradigms
with simplified stimuli \citep{guntupalli2016model, haxby2011common} promising
an increased validity of transformations of functional alignment and increased
generalizability to [research questions/domains/paradigms].

%
``Estimating the parameters to transform high-dimensional spaces from individual
brains into a common high-dimensional space requires a rich set of data that
samples a wide variety of cortical patterns in order to generalize to novel
stimuli or tasks.
%
For response hyperalignment, a rich variety of stimuli or conditions are
necessary to sample the response vector space.
%
For connectivity hyperalignment, the sampling of connectivity vector space is
defined by the selection of connectivity targets, but the richness and
reliability of connectivity estimates depends on the variety of brain states
over which connectivity is estimated'' \citep{haxby2020hyperalignment}.

%
``The algorithm also can be applied to simpler, controlled experimental data,
but our previous results showed that the sampling of response vectors from these
experiments is impoverished and produces a model representational space that
does not generalize well to new stimuli in other experiments (Haxby et al.
2011)'' \citep{guntupalli2016model}.

%
``Hyperalignment of data using a set of stimuli that is less diverse than the
movie is effective, but the resultant common space has validity that is limited
to a small subspace of the representational space in VT cortex''
\citep{haxby2011common}.



\subsection{Prediction of localizer}


\subsubsection{short overview}

\todo[inline]{add text about auditory PPA; we also estimate $Z$-maps from
\citet{haeusler2022processing}}

\todo[inline]{I also tested averaging data in \ac{cfs}: similar results}

\todo[inline]{what I did not do in case of anatomical alignment: averaging data
in MNI152 space}

%
We then estimated a test subject's results of the functional localizer contrast
(the empirical $Z$-map of the \ac{roi}) by projecting all training subjects'
empirical $Z$-maps from their voxel space trough the template into the test
subject's voxel space (i.e. through the \ac{cfs} in case of functional
alignment; through the MNI space in case of anatomical alignment).


\paragraph{masking; estimation via functional \& anatomical alignment}
%
First, empirical $Z$-maps were masked with the same procedure as the time series
data [by restricting the voxels to voxels within the union of the individual
\acp{ppa} \citep[s.][]{haeusler2022processing} that was warped from group space
into each subjects' space, and the subject-specific \ac{fov} of the
audio-description]
% functional alignment; into CFS (calling srm.transform(masked\_zmaps))
In the case of estimation via functional alignment, we used the transformation
matrices that were derived during training of the \ac{srm} in order to project
the masked empirical $Z$-maps from each training subjects' voxel space into the
\ac{cfs}.
% into subject
We then used the transpose of the transformation matrix that we acquired during
the alignment of the test subject, in order to project the data from the
\ac{cfs} into the test subject's anatomical voxel space.
% take the mean
The arithmetic mean of $N-1$ the projected empirical $Z$-maps served as the
predicted $Z$-map estimating the test subject's empirical $Z$-map.
% anatomical alignment; into MNI
In case of estimation via anatomical alignment, we used previously computed
transformation matrices
\citep[][\href{https://github.com/psychoinformatics-de/studyforrest-data-templatetransforms}{\url{github.com/psychoinformatics-de/studyforrest-data-templatetransforms}}]{hanke2014audiomovie}
in order to project the data via a non-linear transformation from each training
subject into the MNI space.
% from MNI into subject
We then used the transpose of the transformation matrix [did I? or was it
another matrix?] in order to project the data from MNI space into the test
subject's voxel space.
%
As in the case of functional alignment, the arithmetic mean of the projected
$Z$-maps served as a test subject's predicted $Z$-map.

\todo[inline]{was it transpose of the matrix to warp from MNI into subject, or
a separate one?}


\subsection{Prediction of auditory PPA}

\todo[inline]{add text here}


\subsection{Quantifying the performance}


\subsubsection{Pearson}
%
In order to quantify the performance of our two different estimation procedures,
we calculated the Pearson's correlation coefficients as a measure of similarity
between the empirical $Z$-maps gained from the localizer experiment and the
predicted $Z$-maps.


\subsubsection{Cronbach's alpha}

% Cronbach's Alpha
Further, we compare the prediction performance to the internal consistency (i.e.
reliability) of the visual localizer by calculating Cronbach's Alpha across the
four runs of the localizer.


``We reasoned that across-subject variability cannot be expected to be lower
than within-subject variability over time (reproducibility).
%
Hence, it can be used as a proxy for noise ceiling''
\citep{rosenke2021probabilistic}.


\subsection{backup: alternative template creation}

\todo[inline]{is this supposed to be reported? imo, it should be dropped.}

\todo[inline]{I need to take a look in the scripts (in the draft directory); I
do not understand the scripts anymore (on a Sunday evening)}

\todo[inline]{s. 'test/data\_denoise-vis.py' \& 'test/data\_srm-vis-to-ind.py'}

\todo[inline]{I think I projected all subjects' localizer time series through
model space into the test subject voxel space; then, calculated the contrast
with these data}

\todo[inline]{in general, the problem was: it gets totally messy \&
computational intensive if one wants to test the different transformation
matrices (I only did it with one; imo, based on alignment using the whole
audio-description?}

\todo[inline]{results: performance was the same if not slightly worse}



\section{Results}

\todo[inline]{we need statistical test of differences for prediction from
alignment via localizer runs; which comparisons to test?}

Unthresholded $Z$-maps [in each subject's voxel space] can be found at
\href{https://identifiers.org/neurovault.collection:12340}{\url{neurovault.org/collections/12340}}.


\subsection{Template(s) from from \citet{jiahui2020predicting}}

``We correlated the whole-cortex contrast map (faces-vs-all) based on a
participant's own data with the maps estimated from other participants' data.
%
After 1-step hyperalignment, the mean Pearson correlation values across
participants were 0.58 (N = 15, S.D. = 0.08)'' \citep{jiahui2020predicting}.

%
``Hyperalignment greatly improved the prediction performance compared with
anatomical surface alignment.
%
With surface alignment, the average Pearson correlation values across
participants were 0.40 (N = 15, S.D. = 0.08) in the studyforrest dataset (Fig.
3).
%
The difference between the hyperaligned and the surface-aligned mean correlation
values was highly significant (studyforrest: t(14) = 17.39, p < 0.001)''
\citep{jiahui2020predicting}.

%
``After 2-step hyperalignment, the mean Pearson correlation values between
face-selectivity maps based on a participant's own localizer data and the
predicted map after hyperalignment were 0.55 (N = 15, S.D. = 0.08).
%
These correlations after hyperalignment were significantly better than the
correlations after surface alignment (t(14) = 15.78, p < 0.001)''
\citep{jiahui2020predicting}.


\subsection{My text}

\todo[inline]{Cronbach's alpha could be plotted as dot per subject in the
stripplot or the average (Fisher's transformed?) could be plotted as a
horizontal line (s. Fig 2 in }

\todo[inline]{finalize 'statistics\_t-test-correlations.py' (!), and write the
text/numbers here; do not forget to mention that correlations were
Fisher-transformed (which Jiahu did not)}

%
Pearson correlation coefficients between empirical $Z$-maps (results of
localizer; y-axis) and estimated $Z$-maps (see Fig.~\ref{fig:stripplot}).



\begin{figure*}[tbp] \centering
    \includegraphics[width=\linewidth]{figures/stripplot.pdf} \caption{
    %
    \textbf{Correlations between empirical and predicted
    \textit{\textbf{Z}}-maps.}
    %
    Grey dots: A left-out subject's $Z$-map was estimated by projecting all
    other subjects ($N = 13$) $Z$-maps through the MNI152 space into the
    left-out subject space and averaging values across subject; correlations
    between empirical values from the localizer \& the predicted values using
    anatomical alignment.
    %
    Green dots: estimation from visual localizer.
    %
    Blue dots: transformation matrices computed based on an increasing number of
    segments of the audio-description; correlations between empirical values \&
    the predicted values using parts of the audio-description.
    %
    Red dots: transformation matrices computed based on an increasing number of
    segments of the movie; correlations between empirical values \& the
    predicted values using parts of the movie.
}
\label{fig:stripplot}
\end{figure*}


\subsubsection{Cronbach's Alpha by \citet{jiahui2020predicting}}

%
``We compared the correlations between maps estimated from a participant's own
data and maps estimated from other participants' data to the reliability of the
localizer.
%
We computed the reliability of the contrast maps with Cronbach's Alpha based on
variability across the four localizer runs for each set.
%
The mean Cronbach's Alpha between the four localizer runs was 0.60 (N = 15, S.D.
= 0.14)'' \citet{jiahui2020predicting}.

%
``These results mean that if we scan each participant for another 4 localizer
runs, and compute the correlation between the two maps (4 runs vs. 4 runs), the
correlation would be 0.60 on average in the studyforrest''
\citet{jiahui2020predicting}.
%
``Cronbach's alpha indicates that the predicted contrast map based on
hyperalignment is close to or as good as the real contrast map based on four
localizer runs (studyforrest: t(14) = 0.61, p = 0.55; Grand Budapest: t(20) =
3.02, p = 0.007)'' \citet{jiahui2020predicting}.
%
``The predicted contrast map based on hyperalignment was better than the
contrast map based on data from three out of four localizer runs in other
participants (t(14) = 2.36, p = 0.03) and in Grand Budapest, the predicted
contrast map was comparable to the contrast map based on three localizer runs in
other participants (t(20) = 0.48, p = 0.63)''
\citet{jiahui2020predicting}.\todo{what?}
%
``A scatterplot of the individual correlation values with hyperalignment and
with surface alignment (Fig 3), shows that predicted maps based on
hyperalignment were more accurate than those based on surface alignment in every
participant'' \citet{jiahui2020predicting}.


\subsection{plot\_bland-altman.py}

\todo[inline]{I hate that script! Hence, it should not be included ;-).}


\subsection{Plots of brain slices?}

\todo[inline]{iirc, we agreed on that it's not necessary}


\section{Discussion}


\subsection{Short Summary of...}

\subsubsection{Aim...}


``The identification of the PPA is complicated by (at least) four methodological
considerations:
%
1) the PPA definition may depend on the type of experiment, task, and stimuli
used.
%
2) The boundaries of the PPA may depend on the statistical threshold used [which
we did not not do].
%
3) The spatial extent and localization of the PPA may vary if defined within the
native brain space of an individual or based on a group analysis.
%
4) The size of the PPA may depend on data acquisition choices (e.g. large vs.
small voxels) and data analysis choices (e.g. liberal smoothing vs. no spatial
smoothing)'' \citep{weiner2018defining}.



\subsubsection{Methods...}

``Validation of the Atlas with an Independent Dataset.
%
To address whether our sample size is sufficient to achieve generalizability to
new data, we tested how well the visfAtlas predicts fROIs of 12 new subjects.
%
These data were acquired using a similar localizer in a different scanning
facility, identified by independent experiments, and have been published
previously [Stigliani et al.  2015; Weiner et al. 2017].
%
We compared their fROI definitions of mFus-faces, pFus-faces, OTS-bodies,
pOTS-characters, and CoS-places to our atlas definitions in the following
ways:
%
1) We visualized our MPMs in relation to their probability maps of
each of the fROIs (Fig. 6), and
%
2) we calculated how well our atlas predicted each of their individual
subjects' fROIs using the Dice coefficient.
%
Lastly, to address how the number of subjects affects the accuracy of our
atlas, we calculated the Dice coefficient for different iterations of the
atlas in which we incrementally increased the number of subjects from 2 to
19'' \citep{rosenke2021probabilistic}.


\subsubsection{Results...}


``Results demonstrate that
%
a) the anatomical location of place-selective voxels relative to cortical
folding is consistent across participants,
%
b) the probabilistic ROI identifies the location of the PPA in individual brains
from multiple independent datasets,
%
c) the probabilistic ROI encapsulates voxels in medial VTC exhibiting the
highest place selectivity in our data as well as shared data from other labs''
\citep{weiner2018defining}.

``Altogether, results suggest that
%
a) our pROI of place selectivity is a robust predictor of the location of high
place selectivity within medial VTC and
%
b) these predictions are generalizable across data collected with different
scanners, participants, images used for localization, data acquisition, and
analysis methods'' \citep{weiner2018defining}.


\subsection{Model space to reduce variability}

We have cross-subject, cross-experiment, and cross-scanner prediction; AV \& VIS
are from the same scanner but not from same session.

%
Results indicate that we are able to use multiple subjects to learn a
10-dimensional shared space for the fMRI data that increases performance on our
experiments.

%
``Results reveal that a probabilistic ROI (pROI) generated from one group of 12
participants accurately predicts the location and functional selectivity in
individual brains from a new group of 12 participants, despite between subject
variability in the exact location of place-selective voxels relative to the
folding of parahippocampal cortex'' \citep{weiner2018defining}.


\subsection{Alignment via Movie Runs}

%
15 min of movie watching used for functional alignment outperform prediction
using anatomical alignment;
%
30 minutes of movie watching outperform 15 minutes of movie watching;
%
more than 30 minutes do not lead to a significantly improved prediction
performance.

%
Asymptotic ``performance curve'' might be different for another brain region
(temporal receptive fields?); retinotopic mapping vs. ``higher'' cognition  vs.
executive functions (prefrontal cortex)?


\subsection{Alignment via Audio Runs}

``SRM will improve sensitivity for detecting a cognitive process of interest in
the test data if the training stimuli or trials strongly and variably engage
that process in a way that is reliable across participants''
\citep{cohen2017computational}.

%
\citep{rosenke2021probabilistic}:
``congenitally blind [
    Mahon et al. 2009;
    Bedny et al. 2011;
    Striem-Amit, Cohen, et al. 2012a;
    van den Hurk et al. 2017
] or individuals with
%
visual agnosia/prosopagnosia [
    Schiltz and Rossion 2006;
    Steeves et al. 2006;
    Sorger et al. 2007;
    Barton 2008;
    Gilaie-Dotan et al. 2009;
    Susilo et al. 2015
]'' \citep{rosenke2021probabilistic}.


\subsection{Partial alignment (why it's awesome)}

%
Reduced costs.
%
There is a benefit of shared parts of naturalistic stimuli across datasets;
shared stimulus part is easier than shared subjects (e.g.
\citep{zhang2018transfer}).


\paragraph{here: data-driven approach; PPA study: model-driven}
%
we might just have modeled the time course of "spatial responses"
insufficiently.
%
Nevertheless, current study is model free, and kind of supports findings that
responses in the PPA are different in the audio-description compared to the
movie.



\subsection{Differences to studies creating probabilistic atlases}

\subsubsection{we need fMRI not just MRI}

%
Compared to estimation procedures based on anatomical alignment (i.e.
\citep{weiner2018defining}), the current estimation procedure needs an
additional functional scan.


\subsubsection{no cherry picking of subjects}

%
We include z-maps of all subjects.
%
Compared to other studies [which studies did it?], we did not cherry picking by
excluding subjects that did not provide functional ROIS in one or both
hemisphere in the functional localizer or naturalistic stimulus paradigms.




\subsubsection{ROI creation}


\paragraph{1st: how others did it}

\todo[inline]{results in an atlas for binary classification}

``Decisions such as the statistical threshold used, the precise contrast, or the
amount of spatial smoothing influence the spatial extent of the (p)ROI on the
cortical surface'' \citep{weiner2018defining}.

%
``We identify and predict the most probable location of place-selective voxels
within medial VTC of an individual brain that is impervious to these
methodological decisions'' \citep{weiner2018defining}.



\citet{rosenke2021probabilistic} ``selected a statistical threshold of t=3 for a
whole-brain map.
%
ROIs where manually defined in individual subjects on their cortical surface
reconstruction in anatomically plausible locations.
%
In the case of an activation cluster transitioning into an adjacent one of the
same visual category, we divided those clusters into separate ROIs by following
the spatial gradient of t-values and separating the two areas at the lowest
t-value'' \citep{rosenke2021probabilistic}.

\citet{weiner2018defining} identified place-selective voxels ``within each
subject's native brain anatomy using a common threshold (t > 3, voxel-level) for
''.

\citet{weiner2018defining}'s ``pROI may not always identify the exact boundaries
of the "PPA" but it will always identify the most probable location of voxels
with the highest place selectivity within medial VTC''
\citep{weiner2018defining}.


\todo[inline]{summarize the essential identical approaches in Zhen's studies}

``Subject-specific SSRs were delineated manually in the parcel unit based on the
individual activation map from the contrast of scenes versus objects.
%
The subject-specific activation image was first thresholded (Z>2.3, uncorrected)
and partitioned into many small parcels with the watershed algorithm:
%
The algorithm partitions an image by analogous process of a landscape being
flooded by water in which water seeps in from every local minimum and the
landscape is finally divided into multiple regions separated by the watersheds
[Meyer, 1994].
%
Raters handpicked the parcels on the individual activation map to construct the
SSRs following a standard procedure:
%
1) gyri and sulci from MNI152 template were used as the anatomical landmarks to
locate the SSRs (PPA was located near the lateral lip of the collateral sulcus
and adjacent medial fusiform gyrus) coarsely [Nasr et al., 2011].
%
2) we used the SSR group labels (the functional landmark) as a spatial extent
reference (the major part of each candidate parcel overlapped with or neighbored
to the corresponding functional label).
%
3) From the parcels that showed good correspondence to both anatomical and
functional landmarks, we identified the one with the strongest scene-selective
activation as the center of a SSR, and iteratively merged the parcels which
connected with the selected parcels into the regions until no candidate parcels
met the criteria'' \citet{zhen2017quantifying}.

``Subject-specific FSR [face-selective regions] were delineated by seven raters
specializing in the ventral visual pathway.
%
Three maps were used in combination:
%
1) the subject-specific activation map
%
2) an FSR spatial reference (map of the high-probability parcels derived from
the probabilistic activation map for face recognition [Zhen et al.,
2013] providing functional landmarks for the location and extent), and
%
3) a macro-anatomical landmark reference (provided by the MNI152 T1 template
providing macro-anatomical landmarks for the location and extent)''
\citet{zhen2015quantifying}.

``A semi-automated delineation procedure was followed by seven raters:
%
1) subject-specific activation image for faces versus objects (the statistical
image derived from the three-run fixed effect analysis providing information on
subject-specific activations) was thresholded with Z=2.3 (p=0.01, right tailed,
uncorrected) and then overlaid on the FSR spatial reference map.
%
The thresholded Z=2.3 was adopted in light of empirical evidence suggesting
that a lenient statistical threshold could help to identify functional regions
in cases of relatively small amounts of data [Kawabata Duncan and Devlin, 2011].
%
2) thresholded activation image was partitioned into many small parcels using
the watershed algorithm [Meyer, 1994].
%
3) raters handpicked the small parcels from the watershed to construct the
target regions based simultaneously on the FSR spatial reference and the
macroanatomical landmark reference'' \citet{zhen2015quantifying}.

``The semi-automated delineation procedure has three major advantages over the
handpicking method.
%
1) boundaries between parcels are objectively determined with the watershed
algorithm, which avoids subjectivity in determining the boundaries of the FSRs.
%
2) guided by the spatial reference of the FSRs and macroanatomical landmark
references, the FSRs could be identified more accurately.
%
3) delineating the FSRs in parcel units rather than in voxel units as in the
traditional method can save a great deal of time'' \citet{zhen2015quantifying}.

``Seven raters took part in delineating the FSRs in four rounds to speed up the
delineation and avoid biases from individual raters, .
%
1) brains were randomly divided into seven approximately equal groups and each
rater was assigned to delineate the FSRs for a group of brains.
%
2) each rater was assigned to delineate the FSRs for another group of brains;
%
as a results, each FSR had been delineated twice with two different raters.
%
3) the delineated FSRs from the first two rounds were double checked and refined
by the two raters who had delineated them.
%
4) FSRs that failed to reach the agreement of the two raters were evaluated and
then finalized by all seven raters together'' \citet{zhen2015quantifying}.

``The choice of the threshold in localizing the FSRs inevitably influenced the
amount of variation in the location of the FSRs.
%
To avoid bias from a single arbitrary threshold, we also constructed the
probabilistic functional atlas using other commonly used Z thresholds such as
3.09 (i.e., p = 0.001, right tailed, uncorrected) and 3.71 (i.e., p = 0.0001,
right tailed, uncorrected) (Fig. S3),
%
The probabilistic atlas and maximum probability map derived at different
thresholds revealed similar patterns, indicating that our atlas is stable and
not sensitive to the choice of threshold'' \citep{zhen2015quantifying}.

``The inter-rater inconsistency in labeling is another factor potentially
contributing unwanted variability, although our procedure of requiring the
collaboration of two raters on each labeling dramatically reduced this type of
variability'' \citep{zhen2015quantifying}.

``All FSRs showed good intra-rater reliability with Dice's coefficients
approximately equal to or exceeding 0.7.
%
Raters had stable performance while delineating each FSR, and the delineations
from different raters were highly consistent.
%
No significant difference between inter- and intra-rater reliability was found
(p b 0.05, Bonferroni correction)''
\citep{zhen2015quantifying}.



``Anatomical volumes and functional time series were warped into Talairach
space.
%
The defined subject-specific landmarks were used to rotate each brain in the
AC-PC plane followed by piece-wise, linear [!] transformations to fit each brain
in the common Talairach "proportional grid" system [Talairach and Tournoux,
1980]'' \citep{frost2012measuring}.

``Thirteen functional regions of interest were defined by performing general
linear model (GLM) analysis of time course data sampled on individual cortex
meshes.
%
fMRI signal time courses were mapped from volume space to surface
space'' \citep{frost2012measuring}.

``Functional areas of interest were determined by selecting the activity cluster
on the surface falling in the region reported in previous standard localizer
studies \citep{frost2012measuring}.
%
Once the resulting cortical patches of interest (POIs) were marked and labeled
we projected the co-ordinates of a POI's vertices back [!] into the anatomical
volume in order to produce a corresponding functional volume-of-interest (VOI)
in Talairach space.
%
The overlap of POIs across subjects can thus, be compared between volume space
and (aligned) cortical surface space'' \citep{frost2012measuring}.

``In the outlined procedure the extent of identified clusters (and thus the
overlap calculation) depended on subjectively determined thresholds''
\citep{frost2012measuring}.



``We used the conventional approach of drawing are borders by hand.
%
At least 2 experimenters experienced in retinotopic mapping drew borders
independently using the same set of published criteria (detailed below) and
subsequently resolved any inconsistencies'' \citep{wang2015probabilistic}.

\paragraph{2nd: how we did it}

\todo[inline]{we uses an atlas as constrain for estimation of $z$-maps}

\todo[inline]{next time, we should use an independent one and dilate the ROI}

%
``[Julian et al., 2012] first anatomically identifies a large cortical expanse
in parahippocampal cortex, then restricts this expanse to a smaller cortical
extent that is place-selective when fMRI data are available''
\citep{weiner2018defining}.

%
We have a little leakage of test data.
%
Especially at the borders of the ROI, we miss some voxels (of some participants)
cause the binary masks is based on a ``titrated threshold''
\citep{sengupta2016extension}

%
We do not cherry pick but also estimate $z$-maps that do not have a ROI based on
Sengupta

%
Moreover, our approach is does not need "experts in the ventromedial pathway".

%
Our approach is faster.





\subsubsection{Atlas creation}

\todo[inline]{kind of corresponds to our common functional space}


``We provide a quantitative characterization of the inter-subject variability
within and across visual regions, including the likelihood that a given point
would be classified as a part of any region (full probability map) and the most
probable region for any given point (maximum probability map)''
\citep{wang2015probabilistic}.

``A full probability map (FPM) was generated for each ROI by dividing, at each
particular node, the number of times that location belonged to that ROI by the
number of subjects included for that ROI.
%
The probability values represent the likelihood that any node in the SBA would
be classified as part of a given visual area'' \citep{wang2015probabilistic}.



``The atlas accurately predicts the location [but just the location] of an
independent dataset of ventral temporal cortex ROIs and other atlases of place
selectivity, motion selectivity, and retinotopy.
%
The majority of voxel within our atlas is responding mostly to the labeled
category in a left-out subject cross-validation''
\citep{rosenke2021probabilistic}.

``The probability maps determine the probability that each vertex belongs to a
given fROI.
%
However, it is possible that a point on the brain may belong to more than one
probabilistic fROI.
%
This overlap is more likely to occur along boundaries of neighboring functional
regions'' \citep{rosenke2021probabilistic}.

``We generated a maximum probability map (MPM) of each area, once in volume
space (NVA) and once in surface space (CBA) in order to assign a unique
functional label to each vertex in the atlas.
%
A maximum probability map (MPM) was calculated for each node by comparing the
probabilities of all areas at that node and assigning the node to the area with
the highest probability.
%
Using the probabilistic fROIs, we determined which vertices were shared by more
than one probabilistic fROI and assigned these vertices to a single fROI based
on the area that showed the highest probability at that vertex''
\citep{rosenke2021probabilistic}.

``We established that a [probabilistic] group map threshold of 0.2 allows for
greatest predictability across regions by systematically varying the group map
threshold for predicting a left-out subject's fROI,
%
Using the 0.2 threshold, we generated a functional atlas of occipito-temporal
cortex by generating an MPM'' \citep{rosenke2021probabilistic}.



``Probabilistic maps were then generated by summing the 12 fROIs [from 12
participants] at each point along the cortical surface of the FS average brain
and dividing by the number of participants.
%
Each vertex within the map reflects the proportion of participants exhibiting
place selectivity at that location on the cortical surface''
\citep{weiner2018defining}.

``Probabilistic ROIs were generated using three different thresholds to test if
and how different threshold values influence the predictability of place
selectivity in a new group of participants.
%
(1) unthresholded,
%
(2) thresholded by 33\% overlapping participants, or
%
(3) thresholded by 66\% overlapping participants.
%
We used these three different thresholds to test if and how different threshold
values influence the predictability of place selectivity in a new group of
participants. Based on related work [Weiner et al., 2017], a threshold of 0.33
is sufficient for predicting functional ROIs in VTC''
\citep{weiner2018defining}.

``Probabilistic ROI identifies voxels exhibiting highest BOLD responses to
places in individual participants:
%
We extracted the response amplitudes from the pROI.
%
We used CBA to project the pROI from Study 1 to each individual brain of the
participants in Study 2.
%
We then extracted the response amplitudes from the pROI in each individual
subject.
%
We used the 33\% thresholded pROI, which produced the highest Dice coefficient
in the previous analyses.
%
The predicted ROI exhibited significantly higher mean response amplitudes to
scenes compared to cars, abstract objects, faces, and textures''
\citep{weiner2018defining}.



``Probabilistic ROI identifies the location of 'peak' place selectivity in over
500 participants:
%
The question is whether this pROI captures the 'peak' voxels exhibiting the
highest place selectivity in medial VTC.
%
We first generated a group map of place selectivity across our 24 participants
to visualize the locus of peak place selectivity across subjects and
experiments.
%
We used CBA to generate an average map of selectivity in which vertices are
assigned continuous values of place selectivity rather than a binary distinction
as in the probability maps from the prior analyses.
%
We then visualized the group place selectivity map relative to the group pROI
from all 24 participants (Fig. 4).
%
The group map illustrates place selectivity within the CoS, LG, PHG, the
parietal occipital sulcus (POS), and retrosplenial cortex (RSC).
%
Weakly place-selective voxels (t-values greater than 0 and less than 2) extend
into the medial fusiform gyrus and more posteriorly along the CoS.
%
The locus ("hotspot") of peak place selectivity is near the medial lip of the
CoS in the right hemisphere and close to the center of our pROI near the fundus
of the CoS in the left hemisphere (Fig. 4A).
%
The locus of peak place selectivity is within the group pROI (Fig. 4A-black
contour)'' \citep{weiner2018defining}.

``Examining the location of our pROI relative to the peak cluster of place
selectivity in individual participants shows that there are individual
differences in the location of voxels with highest place selectivity (Fig.
4B-C):
%
Voxels with the highest place selectivity are not always located within the same
exact location in the pROI for each participant and small clusters of high place
selectivity can also extend outside of the pROI.
%
Despite these individual differences, all participants have voxels with the
highest place selectivity within the pROI (Fig. 4B–C).
%
Crucially, our pROI not only identifies voxels exhibiting the highest
selectivity in our own data, but also identifies voxels exhibiting the highest
place selectivity in independent datasets from over 500 participants''
\citep{weiner2018defining}.

``There is a fundamental difference between the probability map of
place-selective ROIs and the selectivity maps (Figs. 4 and 5).
%
The map of pROIs across subjects is generated after assigning either a 1 (if the
ROI is present) or a 0 (if the ROI is not present) at vertices of the FS average
surface.
%
In the pROI case, there is a threshold and the group map does not reflect
selectivity and instead represents the percentage of overlap across subjects at
each vertex.
%
The selectivity map reflects a continuous metric in which a vertex is assigned a
selectivity value (a t-statistic) first for each subject and then averaged
across subjects.
%
[!] The selectivity map is not thresholded and the group map directly indicates
selectivity (each vertex reflects the average t-value across subjects)''
\citep{weiner2018defining}.




``To characterize the spatial interindividual variability (of spatial
topography) of SSRs, a probabilistic atlas (or map) was first created for each
SSR, by calculating the frequency of a respective SSR being present at a given
position across all subjects.
%
The map coded the occurrence probability of each voxel being located in the SSR,
and provides a voxel-wise description for inter-individual variability in the
location and extent'' \citep{zhen2017quantifying}.

``Because the probabilistic maps of adjacent SSRs (e.g., PPA and RSC) overlapped
to some degree in the periphery, a maximum-probability map (MPM) was then
created to obtain an integrated, non-overlapping map for all SSRs''
\citep{zhen2017quantifying}.


``Maximum-probability maps define the most likely SSR to which each voxel
belonged.
%
Each MPM integrates the multiple probabilistic SSR maps into one map, and to
characterize the spatial relations among SSRs:
%
We compared each voxel's values from each probabilistic map, and assigned that
voxel to the SSR which showed the highest probability at the voxel.
%
If one voxel showed equal probabilities for multiple SSRs, it would be assigned
to the SSR, which owned the highest average probability in the 26 immediate
neighbors of the voxel.
%
The voxels with a maximal probability below 10\% were set to 0, indicating that
they most likely did not belong to any SSR'' \citep{zhen2017quantifying}.


``The MPM revealed a clear topographical relationship between these SSRs (Fig.
3B).
%
PPA was located on the lips of the collateral sulcus and the parahippocampal
gyrus, and abutted the medial fusiform gyrus.
%
RSC occupied the retrosplenial cortex, reaching to the PPA ventrally, and
extending superiorly along the parieto-occipital sulcus.
%
TOS occupied the lateral occipital gyrus and the transverse occipital sulcus''
\citep{zhen2017quantifying}.




``A probabilistic atlas of the FSRs was created by calculating the probability
of a respective FSR being present at a given position to characterize the
interindividual variability of the locations and extents (or borders) of the
FSRs.
%
The probabilistic atlas consists of a set of 12 probability maps [12 FSR] that
reflect the interindividual variability of the respective FSRs.
%
High interindividual variability was observed for all FSRs: no one voxel showed
a 100-percent chance of being in one of the FSRs across subjects.
%
Maximum probability that a voxel belonged to an FSR ranged from 0.24 (left aFFA)
to 0.74 (right pSTS) (Table 1).
%
For each FSR, high probability was often found in the center, not the periphery.
%
Voxels in the center of the right FSRs usually had a larger probability than
those of their left counterparts, especially for pFFA, pcSTS, and pSTS (Fig.
4A), suggesting that the FSRs in the left hemisphere have a larger variability
in terms of anatomical location'' \citet{zhen2015quantifying}.

``A probabilistic atlas was constructed to characterize interindividual
variability of FSRs revealing that the FSRs were highly variable in location and
extent across subjects'' \citep{zhen2015quantifying}.

``We created a probabilistic atlas to characterize the locations and extents (or
borders) of the FSRs.
%
We quantified the variability of the FSRs on both functional (i.e., face
selectivity) and spatial (i.e., volume, location of peak activation, and
functional anatomy) features'' \citep{zhen2015quantifying}.

``For each FSR, a probabilistic map was created to characterize the likelihood
that a given voxel belonged to that FSR.
%
The subject-specific FSR delineated in individual brains were averaged in the
MNI152 space so that the value at any voxel coded the likelihood of the voxel
being located in the FSR, thus giving a measure of the variability in location
and extent of the FSR over subjects at voxel-level resolution''
\citep{zhen2015quantifying}.



``A maximum-probability map (MPM) was constructed to summarize the probabilistic
maps of all FSRs into one volume because of the interindividual variability,
probabilistic maps from adjacent FSRs often showed some overlap.
%
MPM was constructed by comparing the probabilities for each FSR (i.e., the
overlapping frequency) in each voxel and assigning that voxel to the FSR to
which it had the highest probability of belonging.
%
If two FSRs showed equal probabilities, the problematic voxel was assigned to
the FSR with the higher average probability in the 26 immediately adjacent
voxels.
%
The voxels with a maximal probability smaller than 10\% were set to 0,
indicating that they most likely did not belong to any FSR.
%
Result: MPM defined the most likely FSR to which each voxel belonged and
represented all FSRs in a continuous, but non-overlapping manner in one
volume'' \citep{zhen2015quantifying}.


``The probabilistic maps of adjacent FSRs showed overlaps to some extent in the
periphery.
%
An MPM was created to obtain a non-overlapping representation of the FSRs in
order to classify each voxel to the most likely region:
%
MPM represents multiple FSRs in one volume, which reveals a basic topographical
pattern of these FSRs (Fig. 4B)'' \citet{zhen2015quantifying}.



``We computed the overlap of functional areas across subjects for vertices
belonging to a POI of at least one subject using the formula: Nv/N.
%
Nv is the number of subjects whose functional area includes vertex v,
%
N is the number of subjects.
%
The obtained values for all relevant vertices constitute the probabilistic map
for a particular localized region-of-interest.
%
An overall probabilistic map is obtained by adding the resulting overlap vertex
values for all included POIs.
%
A threshold of N=10\% is applied to all probabilistic maps to avoid depicting
regions as "overlap" where only one subject has a functional area''
\citep{frost2012measuring}.



\subsubsection{Prediction: Dice (others) vs. $z$-maps (us)}





\paragraph{Others: Dice (explanation)}

``We estimated the predictability of the group probabilistic fROI by calculating
the Dice coefficient (DSC), a measure of similarity of two samples:
%
a Dice coefficient of zero indicates no predictability and a Dice coefficient of
1 indicates perfect predictability'' \citep{rosenke2021probabilistic}.

``The Dice coefficient is calculated with the following formula:
%
P is the surface area of the probabilistic ROI.
%
A is the surface area of the actual ROI in an individual subject.
%
Perfect alignment between the probabilistic prediction and the actual ROI in
individual subjects would result in a Dice coefficient of 1 and complete
misclassification would result in a Dice coefficient of 0''
\citep{weiner2018defining}.


``We calculated the Dice coefficient to assess the correspondence between the
probabilistic ROI and individually-defined ROIs in independent participants, and
determine how well our probabilistic ROIs from Study 1 predicted each
individually-defined ROI from Study 2 on the FS average brain''
\citep{weiner2018defining}.

``We calculated a ceiling (the empirically best dice coefficient given the noise
in the data) performance as the Dice coefficient between the probabilistic ROI
in Study 1 and each individual subject from Study 1 that contributed to
generating the probabilistic ROI from the same experiment.
%
The ceiling performance plotted in Fig. 2 (horizontal gray bar) is the average
of the two estimates'' \citep{weiner2018defining}.


\paragraph{Others: Dice (prediction performance of binary classification}

\todo[inline]{can only say if voxel belongs to PPA or not}

\todo[inline]{we predict $z$-maps that is way richer; what could you do with
that except just say that the location is here: classification etc.?}

``We calculated the Dice coefficient using different thresholds for the
probabilistic group map, ranging from a liberal unthreshold (one subject at a
given voxel/vertex is enough to assign it to the group map) map to a
conservative threshold where all N-1 subjects had to share a voxel/vertex to be
assigned to the group map'' \citet{rosenke2021probabilistic}.

``We compared Dice coefficients across the two alignment methods using a
repeated measures analysis of variance with individual regions as different
entries, alignment method (CBA vs. NVA) as within-subject factor, and hemisphere
as between-subject factor.
%
We ran this comparison on two different thresholds:
%
once on unthresholded group maps, and once on a threshold that produced - across
regions and methods - the highest predictability.
%
To determine this threshold, we averaged Dice coefficient values across
alignment methods, hemispheres, and ROIs, resulting in one Dice coefficient per
threshold level.
%
Comparison across thresholds revealed that a threshold of 0.2 produced the
highest predictability.
%
Additionally, we ran paired permutation tests within each region on Dice
coefficient results at threshold 0.2 to establish whether the specific region
showed a significant Dice coefficient for either alignment (NVA or CBA)''
\citep{rosenke2021probabilistic}.


``We used the Dice coefficient metric to quantify how well pROIs predict
functionally-defined ROIs in an independent group of participants [by
calculating] the amount of spatial correspondence between the two regions in
individual brains.
%
We measured cross-validation performance at three threshold levels (no
threshold, 33\%, and 66\%) using group pROIs from Study 1 and individual ROIs
from Study 2 and vice versa.
%
The ceiling Dice coefficient in our data is 0.73 in the left hemisphere and 0.72
the right hemisphere'' \citep{weiner2018defining}.

``Results of the cross-validation analysis reveal that our pROI predicts
individual ROIs with high accuracy across hemispheres.
%
The unthresholded ROI had the lowest predictability (about 0.56 left and right)
and the 66\% thresholded ROI had an intermediate predictability (about 0.60 left
and right); all were significantly lower than the ceiling performance (all ts >
4.5, all ps < 10-4)
%
At a threshold of 33\%, the Dice coefficient is 0.70 +/- 0.03 in the right
hemisphere and 0.65 +/- 0.03 in the left.
%
The predictability of the pROI approached (but was significantly lower than) the
ceiling performance in both hemispheres (all ts > 3.3, all ps < 0.003).
%
A more lenient pROI (an unthresholded ROI) as well as a more stringent ROI (a
thresholded ROI at 66\%) also revealed Dice coefficients larger than 0.5, with
numerical values varying with threshold'' \citep{weiner2018defining}.


``Since the 33\% thresholded pROI performed the best in the prior analyses
[Weiner, 2016?], we used this threshold for the present (and subsequent)
analyses.
%
Results revealed high predictability in the right and left hemisphere (about
0.70), which was not different than the ceiling performance (all ts < .7, all ps
> .5; Fig. 2D).
%
Cross-validated performance from 23 participants is not significantly different
than the ceiling performance.
%
This provides statistical evidence supporting that the structural-functional
predictability from our participants likely reflects a predictability that is
reflective of the general population'' \citep{weiner2018defining}.

``In summary, analyses revealed that the group pROI maintains the
structural-functional coupling observed in individual participants and it is
possible to predict the location of place-selective voxels in medial VTC from
cortical folding alone'' \citep{weiner2018defining}.


\paragraph{We: $z$-maps}

\todo[inline]{probabilistic atlas only allow binary classification of voxels}

\todo[inline]{probabilistic atlases are problematic when regions overlap}

``Future studies may generate more sophisticated atlases, which contain not only
a unique tiling of cortical regions but also allow for multiple functional
clusters to occupy overlapping areas and indicate probabilities for multiple
categories at each voxel'' \citep{rosenke2021probabilistic}.








\subsection{Prediction of auditory PPA (via movie or audio-description}

\subsubsection{We found the visual PPA in x subjects}

% usually, visual PPA works pretty well
The visual \ac{ppa} can be reliably localized using a localizer.
%
For example, \citet{zhen2017quantifying} successfully delineated the left- and
right hemispheric \acp{ppa} in 97.5\% of 202 subjects.

% localizer data
\citet{sengupta2016extension} successfully delineated the left-hemispheric
\ac{ppa} in X of X subjects and right-hemispheric \ac{ppa} in X of X subjects
based on localizer data

% movie
\citet{haeusler2022processing} successfully delineated the left-hemispheric
\ac{ppa} in X of X subjects and right-hemispheric \ac{ppa} in X of X subjects
based on movie data.
%
But the modeling was adventurously in the first place and a proof of concept.


\subsubsection{We found the auditory PPA in y subjects}

% audio-description
Using the audio-description's data, \citet{haeusler2022processing} successfully
delineated the left-hemispheric \ac{ppa} in X of X subjects and
right-hemispheric \ac{ppa} in X of X subjects.

\todo[inline]{check: in those subjects who do not have an "auditory PPA" in the
(specific) modeled contrast: is prediction performance of not just the auditory
but also visual PPA worse? Is the response pattern simply not there or just
poorly modeled poorly in \citet{haeusler2022processing}}


\subsubsection{Estimation Results}


\subsubsection{Two open questions}

%
There are two questions:
%
1) Is it a modeling problem in \citet{haeusler2022processing}, or
%
2) do the subjects simply do not give a shit about auditory spatial
information? Are they simply not attending to it? Or are they simply incapable
to process it?

%
Response in PPA during AO might be different (cf. study in chapter 3).

%
If the response pattern is simply not occurring in these subjects then
individual differences in brain patterns might reflect differences in behavior

\todo[inline]{at the moment, the text regarding individual differences in brain
activation possibly correlating with behavior is in the general discussion where
PPA is recapitulated}



\subsection{Alignment via Localizer Runs}

%
``The algorithm also can be applied to simpler, controlled experimental data,
but our previous results showed that the sampling of response vectors from these
experiments is impoverished and produces a model representational space that
does not generalize well to new stimuli in other experiments (Haxby et al.
2011)'' \citep{guntupalli2016model}.

%
``Estimating the parameters to transform high-dimensional spaces from individual
brains into a common high-dimensional space requires a rich set of data that
samples a wide variety of cortical patterns in order to generalize to novel
stimuli or tasks.
%
For response hyperalignment, a rich variety of stimuli or conditions are
necessary to sample the response vector space'' \citep{haxby2020hyperalignment}.

%
Hence, we also performed functional alignment based on a stimulation paradigm
using simplified stimuli [the localizer].

%
Our results show that is works not as good as using the transformation matrices
based on data from the naturalistic paradigms.

%
Results are in line with previous studies \citep{haxby2011common,
guntupalli2016model} that have shown that transformation matrices calculated
using data from simple paradigms have greatly diminished general validity [we
predict both localizer PPAs and naturalistic stimuli PPAs], presumably because
such experiments sample a sparser range of brain states
\citep{guntupalli2016model}''.

%
``Hyperalignment of data using a set of stimuli that is less diverse than the
movie is effective, but the resultant common space has validity that is limited
to a small subspace of the representational space in VT cortex''
\citep{haxby2011common}.

%
Hence, results provide further evidence that naturalistic stimuli promise an
increased validity of derived transformation for functional alignment
%
because they sample a more diverse set of mental states that reflect (confound)
statistics of the natural environment.
%
Hence, they enable investigation of the acquired data for a variety of research
questions (e.g.  visual or auditory perception, spatial cognition; emotion;
music, speech or social perception).

%
``The algorithm also can be applied to simpler, controlled experimental data,
but our previous results showed that the sampling of response vectors from these
experiments is impoverished and produces a model representational space that
does not generalize well to new stimuli in other experiments (Haxby et al.
2011)'' \citep{guntupalli2016model}.

%
``Naturalistic stimuli may better sample the full range of responses to faces
and other stimuli that contribute to face-selective topographies''
\citep{jiahui2020predicting}.

%
``Hyperalignment of data using a set of stimuli that is less diverse than the
movie is effective, but the resultant common space has validity that is limited
to a small subspace of the representational space in VT cortex''
\citep{haxby2011common}.

%
``Estimating the parameters to transform high-dimensional spaces from individual
brains into a common high-dimensional space requires a rich set of data that
samples a wide variety of cortical patterns in order to generalize to novel
stimuli or tasks.
%
For response hyperalignment, a rich variety of stimuli or conditions are
necessary to sample the response vector space.
%
For connectivity hyperalignment, the sampling of connectivity vector space is
defined by the selection of connectivity targets, but the richness and
reliability of connectivity estimates depends on the variety of brain states
over which connectivity is estimated'' \citep{haxby2020hyperalignment}.

%
Naturalistic stimuli promise an ``increased validity of derived transformation
for functional alignment by sampling a more diverse set of mental states  ''
[project proposal].
%
Compared to paradigms with simplified stimuli, naturalistic stimuli sample a
broad range of brain states \citep{guntupalli2016model, haxby2011common} that
reflect (confound) statistics of the natural environment, promising an increased
validity of transformations of functional alignment and enable investigation
of the acquired data for a variety of research questions to research
questions/domains/paradigms (e.g.  visual or auditory perception, spatial
cognition; emotion; music, speech or social perception).

%
``SRM will improve sensitivity for detecting a cognitive process of interest in
the test data if the training stimuli or trials strongly and variably engage
that process in a way that is reliable across participants''
\citep{cohen2017computational}.


\subsubsection{In all cases: tacit assumption that localizer = ground truth}

\todo[inline]{essentially issues of reliability \& validity}

``The tacit assumption is that the same task in the same subject will identify
essentially the same set of voxels despite various sources of physiological and
scanner noise [Aguirre et al., 1998; Handwerker et al., 2004; Kruger and Glover,
2001].
%
If there is considerable variability between runs within the same session, then
the basic idea of functional localization becomes suspect because the localized
set of voxels may not correspond well to those being tested in the main
experimental run, decreasing sensitivity and increasing both false positives and
false negatives'' \citep{duncan2009consistency}.

\todo[inline]{check \citet{jiahui2020predicting}'s supplemental materials
regarding PPA}

% take a more robust dynamic localizer using short videos
``The dynamic localizer (in Grand Budapest data set) was significantly more
reliable than the static localizer (in studyforrest data set) (t(34) = 3.76, p <
0.001) despite its shorter length (four 234s runs versus four 312s runs,
respectively)'' \citep{jiahui2020predicting}.
%
Hence, ``The introduction of dynamic videos of faces and control categories to
localize face-selective topographies provides more reliable maps and better
estimate the extent of face-selective regions than do localizers with still
image stimuli [Fox et al., 2009; Pitcher et al., 2011]''
\citep{jiahui2020predicting}.


\subsection{Future questions}


\subsubsection{Other functional alignment algorithms}

\paragraph{Volume- vs surface-based: actually, not a real problem here}

\todo[inline]{we compare volume-based anatomical alignment to volume-based
functional alignment}

\todo[inline]{might even be "better" because we are nearer on the raw data
'cause we stay in voxel-space; others project their pROIs back from surface to
voxel which might not be "perfect"?}

``The volume-based registration that was used to define the individual SSR and
generate the probabilistic atlases does not model the cortical surface,
and may fail to match the fine anatomy between subjects.
%
Residual macro-anatomical variability from volume-based registration may
confound the measured spatial variability of the SSRs''
\citep{zhen2017quantifying}.


``Our previous work quantifying the relationship between cortical folding and
face-selective regions showed that surface-based predictions out-performed
predictions based on stereotaxic coordinates [Weiner et al., 2014][used just
affine?]'' \citep{weiner2018defining}.

``The cortical surface of each of our 12 participants was aligned to the
FreeSurfer [Fischl et al., 1999] average surface (from 39 healthy adults not
used in our study) using a high-dimensional nonlinear registration algorithm.
%
We then transformed functional ROIs (fROIs) from each subject to the FreeSurfer
(FS) average brain'' \citep{weiner2018defining}.
%
``We repeated our leave-one-out cross-validation procedure across all 24
participants with an affine volume-based registration to the Talairach brain and
compared this performance to the same procedure implemented with CBA in
FreeSurfer'' \citep{weiner2018defining}.

\todo[inline]{kind of obvious and a little "for the sake of it"}

\todo[inline]{why did we use SRM?; mention some sentences in the intro, too;
nicely implemented in BrainIAK; cf. general intro on reproducibility); we do not
use the fastSRM algo, but it's probably still computationally less demanding
than hyperalignment; cf. also evaluation of algos in \citet{chen2015reduced,
bazeille2021empirical}; i.e. ``computational efficiency, as the latter is an
important consideration for scientists who may not have access to specialized
hardware''}

\todo[inline]{cf. \citet{bazeille2021template}'s dissertation}


\subsubsection{ROI vs. whole-brain (i.e. searchlight)}

\todo[inline]{searchlight SRM \citep{zhang2016searchlight}}

\todo[inline]{negative: more parameters to vary/investigate}

% Guntupalli's searchlight
\citet{guntupalli2016model} showed that hyperalignment can be extended to
predict functional organization across large proportions of the cortical
surface[, for example to predict the represented visual field coordinate in
visual cortex based on retinotopic mapping scans of other individuals].
%
How big should it be? PPA seems to vary by centimeters.


``Two main frameworks have been proposed in order to align the entire cortex
across subjects.
%
Searchlight functional alignment [give refs] and piecewise aggregation schemes
[give refs] learn local transformations and aggregate them into a single
large-scale alignment.
%
However, searchlight and piecewise differ in how they aggregate transforms.
%
The searchlight scheme [Kriegeskorte et al., 2006], popular in brain imaging
[Guntupalli et al., 2018; 2016], has been used as a way to divide the cortex
into small overlapping spheres of a field radius.
%
This method allows researchers to remain agnostic as to the location of
functional or anatomical boundaries, such as those suggested by
parcellation-based approaches.
%
A local transform can then be learned in each sphere and the full alignment is
obtained by aggregating [e.g. summing as in Guntupalli et al., 2016 or
averaging] across overlapping transforms.
%
The aggregated transformation produced is no longer guaranteed to bear the type
of regularity (e.g orthogonality, isometry, or diffeomorphicity enforced during
the local neighborhood fit)'' \citep{bazeille2021empirical}.
%
The piecewise alignment is learned within a parcellation.
%
However, the open to debate is which brain atlas should be used for piecewise
alignment.
%
Piecewise alignment [Bazeille et al., 2019], uses non-overlapping
neighborhoods either learned from the data using a parcellation method—such as
k-means—or derived from an a priori functional or anatomical atlas.
%
Local transforms are derived in each neighborhood and concatenated to yield a
single large-scale transformation.
%
This, unlike searchlight, returns a transformation matrix with the desired
regularities.
%
However, piecewise alignment might induce staircase effects or other
functionally-irrelevant discontinuities in the final transformation due to the
underlying boundaries.
%
By default, the results presented below are derived with the 300 ROI
parcellation of the Schaefer atlas unless noted otherwise.
%
In the case of searchlight Procrustes, we selected searchlight parameters to
match those used in Guntupalli et al. (2016):
%
each searchlight had 5 voxel radius, with a 3 voxel distance between searchlight
centers'' \citep{bazeille2021empirical}.


\subsubsection{Time series vs connectivity-based}

\todo[inline]{kind of a killer cause you do not need intersection of
time series}

\todo[inline]{wtf did \citep{nastase2019leveraging} do?}

``Hyperalignment projects cortical pattern vectors into a common,
high-dimensional information space [Haxby et al., 2020].
%
Derivation of this common space can be based on either neural response profiles
(e.g. data collected during tasks, such as movie viewing (Haxby et al., 2011))
or functional connectivity profiles files [Guntupalli et al., 2018]''
\citep{busch2021hybrid}.


``The number of voxels that can be considered simultaneously for functional BOLD
response time series alignment is limited by the number of timepoints in the
calibration scan (about 300-400 voxels for a 15min scan with a 2s TR,
corresponding to a local cortical neighborhood of about 1cm in diameter for a
standard resolution).
%
This limitation does not exist in this form for a functional alignment that is
based on connectivity vectors.
%
\citet{kaule2017examination} showed that congruent time-locked BOLD responses
across subjects are not required to derive a valid alignment of individuals with
a common representational space.
%
Comparable prediction performance can be achieved by using functional
connectivity patterns (correlation of a voxel's time series with reference
regions in the same brain).
%
The length of these connectivity vectors is determined by the number of
reference (or seed) regions in the brain'' [from project proposal].

% Kumar on Nastase's ugly mofo paper
``Estimating the SRM from functional connectivity data rather than
response time series circumvents the need for a single shared stimulus across
subjects; connectivity SRM allows us to derive a single shared response space
across different stimuli with a shared connectivity profile
\citep{nastase2019leveraging}'' \citep{kumar2020brainiak}.


``Both CHA and RHA increased ISCs and bsMVPC classification
accuracies significantly over anatomy-based alignment, but each algorithm
achieves better alignment for the information that it uses to derive a common
model, namely connectivity profiles and patterns of response, respectively''
\citep{guntupalli2018computational}.


``For connectivity hyperalignment, the sampling of connectivity vector space is
defined by the selection of connectivity targets, but the richness and
reliability of connectivity estimates depends on the variety of brain states
over which connectivity is estimated'' \citep{haxby2020hyperalignment}.


\subsection{studyforrest: predict other localizer t-contrasts}

\todo[inline]{especially, condition ``faces'' was hard to model in the
audio-description''; master student's project (with an additional twist?)}

%
The other algorithms could be thrown at the current (state of the) studyforrest
dataset to investigate the PPA contrasts and contrast of other object categories
available from the dataset:
%
The \ac{ffa}, and \ac{eba}  which are associated with face perception
\citep{kanwisher1997ffa, pitcher2011occipitalfacearea}, and the perception of
human bodies \citep{downing2001bodyarea}, respectively.


\subsection{New dataset: shared stimulus; generalizability}

\paragraph{sample size}

\todo[inline]{cf. variability of the \ac{cfs}}

\todo[inline]{can/should that variability somehow be quantified?}

% what is the case
The correlations of shared responses within the \acp{cfs} created from N-1
training subjects varied about the folds of the cross-validation.
% interpretation
That means, a change of 1/13 of the data for every subject's analysis.
is causing variable estimates.
% conclusion
Hence, of course, future studies, should create a \ac{cfs} based on data from
more subjects.


%
We demonstrated that 15 to 30 minutes of naturalistic stimulation is sufficient
to estimate $z$-maps of a different paradigm [the localizer]
%
Hence, a future study of a larger sample size could just sufficiently take a
just a part/intersection of the movie/audio-description.
%
The procedure of a shared stimulus across dataset has the additional benefit of
being more conveniently collected than collecting dataset providing an
intersection of study participants across datasets, i.e. letting the same
subjects participate in different experiments (of different working groups
across the world) \citep[(s.][ for an \ac{srm} based on "shared subjects acrpss
datasets]{zhang2018transfer}.


\subsection{Vision: function atlas}

\todo[inline]{mih: this chapter does not need its own outlook; outlook will
be in the general discussion anyway (and is more interesting there)}

\todo[inline]{a smallish text should be sufficient here; shift most stuff into
general discussion (and write intro in general intro}


\subsection{Functional atlas: uses cases}

\todo[inline]{cf. general introduction; SRM introduction; general discussion}

``Most of the existing atlases focus on brain anatomy, not brain functions.
%
Because of great intersubject variability in anatomy and structural-functional
correspondence, the anatomical atlases are not sufficient to allow researchers
to infer the location and extent of functional regions and their variability
across individual brains [Frost and Goebel, 2012]'' \citep{zhen2015quantifying}.

``The lack of brain atlases for functional regions renders it difficult to
accumulate accurate knowledge of specific functional regions and to understand
the structural and functional architecture of the face processing system''
\citep{zhen2015quantifying}.


``Cortical atlases have been developed, which allow localization of visual areas
in new subjects by leveraging ROI data from an independent set of typical
participants [Frost and Goebel 2012; ventral temporal cortex (VTC) category
selectivity: Julian et al. 2012; Zhen et al.  2017; Weiner et al. 2018; visual
field maps: Benson et al. 2012; Benson and Winawer 2018; Wang et al. 2015]''
\citep{rosenke2021probabilistic}.
%
``This atlas may prove especially useful for predicting an ROI when no localizer
data is available, saving scanning time and expenses, or
%
patient populations, such as patients who have a brain lesion [Schiltz and
Rossion 2006; Steeves et al. 2006; Sorger et al. 2007; Barton 2008; Gilaie-Dotan
et al. 2009; de Heering and Rossion 2015] or are blind [Mahon et al. 2009; Bedny
et al. 2011; Striem-Amit, Dakwar, et al.  2012b; van den Hurk et al. 2017]''
\citep{rosenke2021probabilistic}.

``We believe that this approach will be particularly useful in patient
populations [e.g. blind individuals; Amedi et al., 2007; He et al., 2013; Mahon
et al., 2009; Wolbers et al., 2011] and intracranial studies [e.g. Bastin et
al., 2013; Davidesco et al., 2013; Engell and McCarthy, 2010; Jacques et al.,
2015; Megevand et al., 2014; Murphey et al., 2009; Rangarajan et al., 2014] in
which it may not be possible to obtain fMRI data, but high resolution anatomical
MRI data are typically obtained'' \citep{weiner2018defining}.


``A probabilistic functional atlas could serve as a useful repository of
knowledge and facilitate the analysis of fMRI data [Evans et al., 2012;
Mazziotta et al., 2001]:
%
1) as a complement to the anatomical atlases, the atlas supplies a quantitative
spatial reference system in which information from multiple sources can be
integrated and compared.
%
By integrating with other atlases, the probabilistic functional atlas would
identify patterns of structural, connectional, and molecular variations in the
functional regions and provide a deeper understanding of the relationship
between brain structure and function.
%
2) subject-specific fROIs are always preferred over the probabilistic functional
atlas based fROIs in individual studies;
%
however, when such information is not available, the probabilistic functional
atlas can be used to provide face-selective fROIs in group analyses to explore
properties common to all human brains.
%
On the other hand, even when a functional localizer is acquired, the
probabilistic functional atlas can be used as a source of spatial constraints to
localize subject-specific fROIs.
%
Availability of well-defined fROIs could provide better sensitivity and
specificity in integrating features of the FSRs with clinical and genetic data,
and in studying the large-scale brain network for face recognition.
%
3) individual FSRs and associated features could serve as a
database of the FSRs in a healthy population.
%
All the accumulated data in the atlas could be used as a norm to quantify the
degree of deviation of the FSRs in a new subject and thereby have the potential
to detect FSR deficits in patients.
%
4) the atlas descriptions of intersubject variance in FSRs could serve as
empirical priors in the development of algorithms for automated identification
of the FSRs'' \citep{zhen2015quantifying}.

``Our probabilistic functional atlas significantly extended these atlases.
%
1) previous atlases aim to describe the likelihood of activation of each voxel
elicited by faces (characterizing the interindividual variability of activation
for the cognitive process of interest at the voxel level) and thus provide
little information on how the FSRs vary in location and extent across subjects
[but see Frost and Goebel, 2012]'' \citep{zhen2015quantifying}.


\subsection{New dataset: other localizers = t-contrasts \& brain functions}

\paragraph{functional-anatomical correspondence is different across ROIs}

%
``Analogously, the introduction of dynamic videos of faces and control
categories to localize face-selective topographies provides more reliable maps
and better estimate the extent of face-selective regions than do localizers with
still image stimuli [Fox et al., 2009; Pitcher et al., 2011]''
\citep{jiahui2020predicting}.

``We localized 13 widely studied functional areas and found a large variability
in the degree to which functional areas respect macro-anatomical boundaries
across the cortex.
%
Some areas, such as the frontal eye fields (FEF) are strongly bound to a
macro-anatomical location.
%
Fusiform face area (FFA) on the other hand, varies in its location along the
length of the fusiform gyrus even though the gyri themselves are well aligned
across subjects.
%
Language areas were found to vary greatly across subjects whilst a high degree
of overlap was observed in sensory and motor areas'' \citep{frost2012measuring}.

``data show that there is a surprising amount of variability in that not all
functional areas are tightly bound to anatomical landmarks, i.e. there is no
general rule describing how macroanatomical and functional areas correlate''
\citep{frost2012measuring}.
%
There is a strong structural-functional correspondence in some areas whilst in
others the spatial location of the functional area varies greatly across
subjects within a cortical area'' \citep{frost2012measuring}.

``Whilst it seems that CBA has brought the functional area FEF into alignment,
there is a substantial amount of variability in FFA''
\citep{frost2012measuring}.

``The percent gain in overlap differed greatly across the different functional
regions throughout the cortex.
%
A large amount of variability remained in language regions in the left
hemisphere.
%
Both the sensory and motor hand areas (bank of the central sulcus) were much
better aligned after CBA'' \citep{frost2012measuring}.

``Other motor regions exhibited smaller gains (SMA, PMA).
%
Similar to results described above, CBA is able to separate functional regions
which are more prone to blur together if individual anatomical curvature
patterns are not accounted for'' \citep{frost2012measuring}.

``The area which shows the most gain in overlap of these regions is V5 / hMT+
with 70.9\% gain in the left hemisphere and 55.6\% gain in the right
hemisphere'' \citep{frost2012measuring}.

``Area LOC also showed increased overlap after CBA with a 62.7\% gain in the
left hemisphere and 38.4\% on the right.
%
Finally PPA exhibit more gain in the right hemisphere with 27.7\% gain, than on
the left with 17.6\%'' \citep{frost2012measuring}.

``The FFA did not exhibit the same strong structural-functional correspondence
and saw more modest increases in overlap after macro-anatomical alignment with
44.1\% and 12\% gain for the left and right hemispheres.
%
There is much more inter-subject variability than in FEF.''
\citep{frost2012measuring}.


\paragraph{Hence, test ROIs of other perceptual/cognitive processes}

What is the limit of what we can estimate reliably?
%
Retinotopy, language, executive functions (from low-level perception to higher
cognition which might not even be sampled by a movie).

%
exogenous vs. endogenous representations; bottom-up vs. top-down processes.


``Results show that the computational principles underlying this common
model have broad general validity for representational spaces in occipital,
temporal, parietal, and frontal cortices'' \citep{guntupalli2016model}.

``Future studies may generate more sophisticated atlases, which contain not only
a unique tiling of cortical regions but also allow for multiple functional
clusters to occupy overlapping areas and indicate probabilities for multiple
categories at each voxel'' \citep{rosenke2021probabilistic}.

``Identifying all of the currently known topographic regions of the human visual
system requires multiple scanning sessions.
%
Given the expense and availability of fMRI, this is not always practical.
%
One way to address these limitations is to create an atlas in a standard space
that links individual points in that space with functionally defined regions.
%
Given the anatomical and functional variability across subjects, this atlas
should be "probabilistic": it defines the likelihood of a given coordinate being
associated with a given functional region. [our functional alignment procedure
can improve these atlases by providing $z$-values]
%
Such an atlas could be used to infer the topographic location in the visual
system for the results obtained from any independent dataset once transformed
into the same standard space as the atlas'' \citep{wang2015probabilistic}.

``A probabilistic atlas can be used under conditions in which collecting
the data to define maps in individual subjects is impractical or not feasible.
%
For example, time-limitations and subject-fatigue both potentially limit the
time researchers may be able to spend with patients suffering from
neurological or neuropsychological disorders, or with implanted subdural or deep
electrodes (e.g., ECoG)'' \citep{wang2015probabilistic}.


\todo[inline]{We: matrices for individual mapping; but onto shared responses}

\todo[inline]{Capturing Shared \& Individual Info \citep{turek2018capturing}}

\todo[inline]{is there a "real" paper that has been using that model already?}

``Recognizing that signal exists beyond the average or shared response of a
group, such studies exploit idiosyncratic but stable responses to account for
previously unexplained variance in brain function, behavioral performance and
clinical measures [Finn (2015). Functional fingerprinting (based on
connectivity); Rosenberg (2016). A neuromarker of sustained attention]''
\citep{cohen2017computational}.

``The flip side of focusing on shared responses is to focus on responses that
are idiosyncratic to individuals.
%
Although these responses are excluded in SRM, they are not necessarily noise and
may in fact be highly reliable within participants''
\citep{cohen2017computational}.

``SRM can be used to isolate participant-unique responses by examining the
residuals after removing shared group responses, or it can be applied
hierarchically to the residuals to identify subgroups [\citet{chen2017shared}]
'' \citep{cohen2017computational}.


``In cases where each subject's unique response is of more interest
than the shared signal, SRM can be used to factor out the shared component
thereby isolating the idiosyncratic response for each subject
[\citep{chen2015reduced}]'' \citep{kumar2020brainiak}.


\subsubsection{Functional atlas: subgroups}

``Variability was observed from a cohort of homogeneous young adults with a
narrow age range, which limits the generalization of the variability observed
here to people with different ages, especially children and the elderly.
%
Future studies are needed to characterize how the different aspects of
variability in SSRs change with ages'' \citet{zhen2017quantifying}.

``Interindividual variability of functional regions may root in
multiple sources, such as variability in the cytoarchitecture [Zilles and
Amunts, 2013], hemodynamic response functions [Aguirre et al. 1998; Handwerker
et al. 2004], connectivity patterns [Mueller et al., 2013], functional
plasticity [Polley et al., 2006; Song et al., 2010], and genetic factors
[Blokland et al., 2011; Koten et al., 2009].
%
It is notable that the spatial variability of the SSRs seen here is nearly twice
that seen for the face-selective regions [Zhen et al., 2015]. How the multiple
sources may determine different interindividual variability of the functional
regions is another direction worthy of exploration in the future [maybe, make
that point in our discussion (cf. Jiahu)]'' \citep{zhen2017quantifying}.


\subsection{We do not just provide binary decision}
%
We provide patterns that allow far more than just localization but e.g.
classification and...?


\subsection{Conclusion}

``We have cross-validated our pROI across participants and studies collected
with different voxel resolutions, stimuli, tasks, scanners, and analysis
methods.
%
Thus, our results is generalizable across many different methodological
decisions spanning experimental design, as well as fMRI data acquisition and
analyses [we do it even 'more extreme']'' \citep{weiner2018defining}.


\section{Data Availability}

\todo[inline]{all from PPA-Paper but with new GIN link leading to an empty repo}

% \href{https://gin.g-node.org/chaeusler/studyforrest-ppa-analysis}{\url{gin.g-node.org/chaeusler/studyforrest-ppa-analysis}}

% new; PPA analysis
All fMRI data and results are available as Datalad \citep{halchenko2021datalad}
datasets, published to or linked from the \emph{G-Node GIN} repository
(\href{https://gin.g-node.org/chaeusler/studyforrest-ppa-srm}{\url{gin.g-node.org/chaeusler/studyforrest-ppa-srm}}).
% original
Raw data of the audio-description, movie and visual localizer were originally
published on the \emph{OpenfMRI} portal
(\url{https://legacy.openfmri.org/dataset/ds000113}; \citep{Hanke2014ds000113},
\space \url{https://legacy.openfmri.org/dataset/ds000113d};
\citep{hanke2016ds000113d}).
% visual localizer
Results from the localization of higher visual areas are available as Datalad
datasets at \emph{GitHub}
(\href{https://github.com/psychoinformatics-de/studyforrest-data-visualrois}{\url{github.com/psychoinformatics-de/studyforrest-data-visualrois}}).
% raw data
The realigned participant-specific time series that were used in the current
analyses were derived from the raw data releases and are available as Datalad
datasets at \emph{GitHub}
(\href{https://github.com/psychoinformatics-de/studyforrest-data-aligned}{\url{github.com/psychoinformatics-de/studyforrest-data-aligned}}).
% OpenNeuro
The same data are available in a modified and merged form on OpenNeuro at
\url{https://openneuro.org/datasets/ds000113}.
% NeuroVault for z-maps of SRM
Unthresholded $Z$-maps of all contrasts can be found at
\href{https://identifiers.org/neurovault.collection:12340}{\url{neurovault.org/collections/12340}}.


\section*{Code Availability}

Scripts to generate the results as Datalad \citep{halchenko2021datalad} datasets
are available in a \emph{G-Node GIN} repository
(\href{https://gin.g-node.org/chaeusler/studyforrest-ppa-srm}{\url{gin.g-node.org/chaeusler/studyforrest-ppa-srm}}).
