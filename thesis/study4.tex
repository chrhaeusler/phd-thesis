\todo[inline]{I use "prediction" and "estimation" pretty much interchangeably}

\section{Introduction}

\todo[inline]{delete unnecessary sub(sub)sections / captions to turn them into
paragraphs}

\todo[inline]{points of the discussion are now primed in the intro; however, I
do not know where to put "what does the localizer actually do?"; could come up
for the first time in the discussion of SRM chapter; above all, it's a topic for
the general discussion: reliability of an auditory naturalistic stimulus as a
localizer; what's wrong? the modeling in \citet{haeusler2022processing}? Or does
an auditory naturalistic stimulus give too much room to participants to do not
give a shit? a.k.a. freely listening is bad?}


% higher visual areas higher visual areas
In the domain of higher-visual perception, functionally defined,
category-selective brain regions like the \ac{ppa} \citep{epstein1998ppa}, the
\ac{ffa} \citep{kanwisher1997ffa}, or \ac{eba} \citep{downing2001bodyarea}
exhibit significantly increased \ac{bold} activity correlated with a
``preferred'' \citep{debeck2008interpreting} stimulus class.
%
The topographies (i.e. the location, size and shape) of these category-selective
areas are similarly distributed across individuals but the exact topographies
vary interindividually \citep{rosenke2021probabilistic, zhen2017quantifying,
zhen2015quantifying, frost2012measuring}.


\subsection{Localizers}

% definition of localizer
In order to identify the topography of functional areas in individual persons,
block-design \textit{functional localizer} paradigms are traditionally used that
contrast modeled hemodynamic responses correlating with the corresponding
stimulus classes (i.e. landscapes, faces, or bodies).
% problem: one localizer for one domain
Functional localizers are designed to maximize detection power and thus
dedicated to map just one domain of brain functions like, for example,
retinotopic visual areas \citep{wang2015probabilistic}, category-selective
regions \citep{stigliani2015temporal}, theory of mind
\citep{spunt2014validating}, or semantic processes \citep{fedorenko2010new,
fernandez2001language}.
% which gets messy
However, if one wants to map a variety of domains, the approach ``one paradigm
for one domain of functions'' gets time-consuming and impractical.
% localizer batteries: intro
Researchers have tried to tackle that issue by creating time-efficient
multi-functional \textit{localizer batteries} \citep[e.g.,][]{barch2013function,
drobyshevsky2006rapid, pinel2007fast}.
% task based = shit
Nevertheless, the diagnostic quality of localizer paradigms relies heavily on a
participant's comprehension of the task instructions and general compliance, a
criterion that can be difficult to meet in clinical or pediatric populations
\citep{eickhoff2020towards, vanderwal2019movies}.


\subsection{Estimation from reference group}

% ppa via audio-description
% Results also suggest that a naturally engaging, purely auditory paradigm like
% an audio-description could, in principle, substitute a visual localizer as a
% diagnostic procedure to assess brain functions in visually impaired %
% individuals \citep{haeusler2022processing}.

% ppa via movie
In \citet{haeusler2022processing}, we have shown that a functionally defined
region, such as the \ac{ppa}, can be localized using a model-driven \ac{glm}
analysis that is based on the annotated temporal structure of a two-hour long
naturalistic stimulus.
% full feature film is too long
However, a two-hour long paradigm is unsuitable for a clinical application due
to practical and monetary reasons.
% hence, predict from reference
An approach to reduce time and costs is to identify a functional area in an
individual person's brain anatomy based on data collected from an independent
sample of different persons (i.e. data from a \textit{reference group}).



\subsubsection{Anatomical alignment}

% intro: estimation via common anatomical space
Previous studies estimated a functional area's most probable location from a
reference group by performing a volume-based
\citep[e.g.,][]{zhen2017quantifying, zhen2015quantifying} or surface-based
\citep[e.g.,][]{frost2012measuring, weiner2018defining,
rosenke2021probabilistic, wang2015probabilistic} \textit{anatomical alignment}.
%
First, in order to address the issue of anatomical variability across persons,
functional data of persons in the reference group are anatomically aligned to
(i.e.  projected into) a \textit{common anatomical space} (e.g., Montreal
Neurological Institute brain atlas; \citep[MNI152,][]{fonov2011unbiased}).
% project into test subject to estimate
Then, data are projected from the common anatomical space into the individual
person's brain anatomy serving as an estimation of a functional region's
location.

% volume-based alignment in one sentence
Volume-based anatomical alignment \citep[s.][for a review]{klein2009evaluation}
aligns voxels to a three-dimensional common anatomical space \citep[e.g., MNI152
atlas;][]{fonov2011unbiased}.
% surface-based alignment in one sentence
Surface-based anatomical alignment \citep{fischl1999cortical, yeo2009spherical}
aligns vertices to a two-dimensional common anatomical space \citep[e.g.,
FreeSurfer's fsaverage template;][]{fischl1999high}.
% difference in one sentence
Whereas volume-based alignment does not account for individual sulcal and gyral
folding patterns, surface-based alignment respects interindividual variability
of the cortical surface.
% surface-based estimation works better
Consequently, previous studies that compared  [linear / affine] volume-based and
[nonlinear] surface-based alignment to estimate the location of functional
regions have shown that surface-based alignment lead to reduced inter-subject
variability, and thus increased estimation performance
\citep{rosenke2021probabilistic, frost2012measuring, wang2015probabilistic,
weiner2018defining}.
% remaining variability after surface-based alignment
However, even after surface-based alignment the anatomical location of
functional regions varies anatomically across persons
\citep[e.g.,][]{coalson2018impact, benson2014correction, natu2021sulcal,
wang2015probabilistic, frost2012measuring, langers2014assessment, weiner2014mid,
rosenke2021probabilistic}.
% frost as an example
For example, \citet{frost2012measuring} localized 13 functional areas of the
high-level visual cortex and ``found a large variability in the degree to which
functional areas respect macro-anatomical boundaries''
\citep{frost2012measuring}.
% functional--anatomical correspondence
The remaining variability indicates that functional areas a not necessarily
bound to anatomical landmarks, and reflects the degree of
\textit{functional--anatomical correspondence} between a brain function and its
underlying anatomical location.


% case of PPA
% cf. also \citet{frost2012measuring, rosenke2021probabilistic}
% \citet{weiner2018defining} showed ``that cortical folding patterns and
% probabilistic predictions reliably identify place-selective voxels in medial
% VTC across individuals and experiments''.
%
% However, ``this structural-functional coupling is not always perfect and there
% is inter-subject variability as to how much the place-selective voxels extend
% within the parahippocampal gyrus, as well as the lingual gyrus and medial
% aspects of the fusiform gyrus.
%
% Despite this inter-subject variability, place-selective voxels are always
%located within the collateral sulcus across participants.''
% \citep{weiner2018defining}.


\subsubsection{Functional alignment}

\todo[inline]{I simply dropped "matrix" from "transformation matrix"; text
sounds vague to me now...}

\todo[inline]{to avoid a big treatise on functional alignment algos in the
discussion state that response-based alignment better aligns response data
compared to connectivity-based alignment}


%
Since anatomical alignment addresses the issue of anatomical variability but not
functional-anatomical variability across subjects, algorithms---like
\textit{hyperalignment} \citep{haxby2011common, guntupalli2016model} or the
\textit{shared response model} \citep{chen2015reduced, zhang2016searchlight}---
have been developed that perform a \textit{functional alignment}.
%
Whereas anatomical alignment aligns voxels (or vertices) that share the same
anatomical location to a common anatomical space, functional alignment aligns
voxels (or vertices) that share similar functional properties to a
\textit{common functional space} (CFS).
%
Functional alignment algorithms are usually used to calculate both a
high-dimensional, functional brain template (i.e. the \ac{cfs}) from study
participants' functional data as well as subject-specific transformations.
%
A subject-specific transformation allows a mapping of functional data from a
subject's three-dimensional voxel space into the \ac{cfs}, or to project data
from the \ac{cfs} into a subject's voxel space \citep{haxby2020hyperalignment,
kumar2020brainiak}.
%
The \ac{cfs} and transformations can be calculated (i.e. \textit{trained}) based
on the maximization of the inter-subject similarity of \ac{bold} response time
series correlating with a time-locked external stimulation
\citep{haxby2011common, chen2015reduced, sabuncu2010function}, or based on the
inter-subject similarity of connectivity profiles \citep{feilong2018reliable,
guntupalli2018computational, nastase2019leveraging}.
%
Whereas connectivity-based functional alignment better aligns connectivity
profiles, response-based functional alignment better aligns response
time-series \citep{guntupalli2018computational}.
%
Though functional alignment algorithms can be applied to \ac{fmri} time series
data from paradigms employing simplified stimuli, data from naturalistic stimuli
provide
%
increased generalizability of the \ac{cfs}
%
and transformation matrices\todo{non-separable claims}
%
to novel stimuli or tasks, presumably because naturalistic stimuli sample a
broader range of brain states \citep{haxby2011common, guntupalli2016model}.


\subsubsection{Estimation via functional alignment}

% \citep{guntupalli2016model}: s. Supplementary Figure S8.

\todo[inline]{Problem: \citet{feilong2022individualized} assess data quantity,
too; results suggest 30m are "good" in case of their model.}

\todo[inline]{\citet{jiahui2022cross} do connectivity-based 1-step
hyperalignment across different movie datasets; i.e no \ac{cfs} but one
transformation matrix for every pair of subjects (which is a shitty procedure to
scale, imo)}

\todo[inline]{point for the discussion: what makes the prediction of the
localizer based on localizer runs for alignment so "bad"? Is it our model or the
transformation matrices? In any case, our results are different from
\citep{haxby2011common}: CFS \& transformation matrices based on controlled
paradigms were as good as / better than CFS \& transformation matrices based on
movie when estimating results of controlled paradigm; if it is discussed
(lengthy), then add findings from \citet{haxby2011common} that compare (within-
and across-paradigm) prediction based on naturalistic \ac{cfs} (more TRs) to
non-naturalistic \ac{cfs} (less TRs) here; if this is mentioned here, then also
define  within- and across-paradigm prediction here already and not below in
"Here, we..."}


%
Hence, a more recent procedure \citep[e.g., ][]{jiahui2020predicting,
guntupalli2016model, haxby2011common} to estimate the most probable location of
a functional area in a person's anatomy from a reference group performs an
functional alignment.
% solve functional-anatomical variability
First, functional data from persons in the reference group are anatomical
aligned to a common anatomical space.
%
Second, in order to address the issue of functional-anatomical variability
across persons, functional data are functionally aligned (i.e. projected into) a
\ac{cfs}.
%
Then, data are projected from the \ac{cfs} into the individual person's brain
anatomy serving as an estimation of a functional region's location.

% Jiahui
For example, \citet{jiahui2020predicting} used surface-based hyperalignment to
calculated both a \ac{cfs} and transformations based on data from
%
a) the whole movie ``Grand Budapest Hotel'' ($\approx$\unit[50]{m};
\ac{tr}=\unit[1]{s}), and
%
b) the whole movie ``Forrest Gump'' ($\approx$\unit[120]{m};
\ac{tr}=\unit[2]{s})
%
in order to estimate $Z$-maps resulting from a visual localizer's
mass-univariate \ac{glm} $t$-contrast that aimed at localizing the \ac{ffa}.
% summary of results
\citet{jiahui2020predicting}'s results revealed that mass-univariate GLM
contrast maps correlate more highly with contrast maps that were estimated via
hyperalignment from the reference group than contrast maps that were estimated
via surface-based anatomical alignment \citep{jiahui2020predicting}.




\subsection{Here, we...}

\todo[inline]{mentioned here that number of voxels is constrained to a ROI?}

\todo[inline]{I "defined" the terms \textit{criterion} and \textit{predictors}
to make the discussion easier; however, it's not predictor / criterion in a
strict sense as usually used (which refer to single variables)}

% focus: ppa
Here again, we focus on the \ac{ppa} \citep[e.g.,][for
reviews]{epstein2014neural, aminoff2013role}, and explore whether we can
estimate the results of $t$-contrasts (i.e. $Z$-maps) that were created to
localize the \ac{ppa} using functional data of three different paradigms serving
as the to predicted \textit{criteria}:
%
a) a classic visual localizer \citep{sengupta2016extension} as the assumed
``gold standard'',
%
b) a movie \citep{haeusler2022processing}, and
%
c) an auditory narrative \citep{haeusler2022processing}.


\subsubsection{SRM}

\todo[inline]{check via code if $W_{n}^{T}W_{n}=I_{k}$ == True}

% math stuff from citep{vodrahalli2018mapping}
% ``SRM learns $N$ maps $W_{i}$ with orthogonal columns such that
% $||X_{i}-W_{i}S||_{F}$ is minimized over $\left\{ W_{i}\right\} _{i=1}^{N},S$,
% where $X_{i}\in\mathbb{R}^{v\times{T}}$ is the $i^{th}$ subject's fMRI
% response ($v$ voxels by $T$ repetition times) and
% $S\in\mathbb{R}^{k\times{T}}$ is a feature time-series in a $k$-dimensional
% shared space'' \citep{vodrahalli2018mapping}.

% Inverse vs. transpose of a matrix:
% for orthogonal transformations (like we should have here, i.e. only rotation,
% expansion) these two are one and the same thing:
% https://www.quora.com/When-is-the-inverse-of-a-matrix-equal-to-its-transpose

% why SRM
Our volume-based functional alignment approach utilizes the \ac{srm} algorithm
\citep{chen2015reduced, richard2019fast} as implemented in the open-source
software package BrainIAK \citep[Brain Imaging Analysis Kit;
\href{https://brainiak.org}{\url{brainiak.org}};][]{kumar2020brainiak,
kumar2020brainiaktutorial}.
% general overview of SRM
The \ac{srm} is an unsupervised probabilistic latent-factor model that
decomposes \ac{bold} \ac{fmri} response time series of participants experiencing
the same stimulus into a \ac{cfs} of \textit{shared features} \citep[also called
``\textit{shared feature space}'';][]{chen2015reduced} and subject-specific
linear transformations matrices.
% math stuff
More specifically, the \ac{srm} algorithm uses each $n^{th}$ subject's response
time series represented as matrix $X_{n}$ ({$v$} voxels by $t$ time points) to
calculate the \ac{cfs} $C$ ($k$ shared responses by $t$ time points) and
subject-specific, transformation matrices $W_{n}$ ($v$ voxels by $k$ shared
responses) with orthonormal columns ($W_{n}^{T}W_{n}=I_{k}$).
% iteratively fitted
The algorithm randomly initializes and fits the transformation matrices over
iterations to minimize the error in explaining the participants' data, while
also learning the time course of the shared responses (s.
\href{https://brainiak.org/tutorials/11-SRM/}{\url{brainiak.org/tutorials/11-SRM}}).
% number of dimensions
In contrast to hyperalignment, the number of dimensions of the \ac{cfs} is not
set by the number of voxels but is set by the researcher to a number lower than
the number of voxels, a procedure that also filters out noise and reduces
overfitting \citep{chen2015reduced}.
% phrase math in words
Hence, each shared feature can be understood as a weighted sum of many voxels
across subjects \citep{kumar2020brainiak}.
% result = alignment
A subject-specific transformation matrix can be understood as the weight of each
voxel in a subject's voxel space on each shared feature, and allows to
functionally align a subject's functional data to the \ac{cfs} by projecting
responses within the voxels into the $k$-dimensional \ac{cfs}.



\subsubsection{Three aspects under investigation (actually 4)}

\todo[inline]{Shift comparison to anatomical alignment upwards and label it as
aspect 1}

\todo[inline]{biggest issue for discussion: how to separate validity /
generalizability of CFS from validity / generalizability of transformation
matrices? If I do not find a solution merge those two points to one?
Four aspects are too much anyway}

\todo[inline]{it might be too "method-detailed" here but number of respective
\acp{tr} are now stated here \& in the plot's (as in the review of
\citet{jiahui2020predicting} above)}

% multi-paradigm model
Contrary to previous studies \citep[e.g.][]{jiahui2020predicting,
guntupalli2016model, haxby2011common} that created a \ac{cfs} based on data from
a single paradigm, we calculate a \textit{multi-paradigm \ac{cfs}}:
% cross-validation
following an exhaustive leave-one-subject-out cross-validation (N$=$14
subjects), we train a shared feature space (i.e. the \ac{cfs}) based on
concatenated response time series of
%
the movie ``Forrest Gump'' ($\approx$\unit[120]{m}; \ac{tr}=\unit[2]{s}),
%
the movie's audio-description  that was produced for a visually impaired
audience ($\approx$\unit[120]{m}; \ac{tr}=\unit[2]{s}), and
%
a visual localizer ($\approx$\unit[21]{m}; \ac{tr}=\unit[2]{s})
%
from $N-1$ \textit{training subjects} as seen in
Fig.~\ref{fig:multi-stimulus-cfs}.



\begin{figure*}[tbp]
\centering
\includegraphics[width=\linewidth]{figures/multi-stimulus-cfs.pdf}
\caption{
%
\textbf{Overview of the shared response model.
}
    %
    For each fold of the leave-one-subject-out cross-validation, each training
    subject's response time series of
    %
    a movie ($\approx$\unit[120]{m}; \ac{tr}=\unit[2]{s}),
    %
    the movie's audio-description ($\approx$\unit[120]{m}; \ac{tr}=\unit[2]{s}),
    %
    and the visual localizer ($\approx$\unit[21]{m}; \ac{tr}=\unit[2]{s})
    %
    were concatenated to serve as input for the \ac{srm} algorithm.
%
From these response time series represented as matrix $X_{n}$ ({$v$} voxels
by $t$ time points), the algorithm calculates the common functional
space (CFS) $C$ ($k$ shared features by $t$ time points) and
subject-specific, transformation matrices $W_{n}$
($v$ voxels by $k$ shared features) with orthonormal columns
($W_{n}^{T}W_{n}=I_{k}$).
} \label{fig:multi-stimulus-cfs} \end{figure*}

\todo[inline]{Problem: validity of \ac{cfs} is not really separable from
validity of transformations in the current approach (if at all separable):}

\todo[inline]{maybe, speak of "cross-subject-within-paradigm prediction" instead
of just "within-paradigm prediction" (as done in the short overview at the
beginning of the discussion)}


% three aspects to explore
In the present study, we aim to investigate three [or four?] aspects:

% benchmark: anatomical alignment
zero, we compare the performance of your volume-based, functional alignment
procedures to the performance of a volume-based, anatomical alignment approach
that serves as a benchmark.
% the criteria
first, in order to explore the validity and generalizability of our
multi-paradigm \ac{cfs}, we predict a left-out \textit{test subject}'s results
from the analysis of
%
a) the localizer \citep{sengupta2016extension} as the assumed ``gold standard'',
%
b) the movie \citep{haeusler2022processing}, and
%
c) the auditory narrative \citep{haeusler2022processing}
%
serving as the criteria.
% three predictors
Second, in order to explore the validity and generalizability of a test
subject's transformation matrices based on data from different paradigms, we use
a test subject's response time series from each of the three paradigms
separately serving as the \textit{predictors} in order to align the test
subject with the corresponding \acp{tr} within the \ac{cfs} (i.e. one
\textit{within-paradigm prediction}, and two \textit{cross-paradigm predictions}
per predictor).
% partial alignment
Third, since using a complete naturalistic stimulus to align a test subject is
impractical in a clinical setting, we also explore the relationship between the
estimation performance of the $t$-contrast's results and the quantity of data
from each of the three paradigms used to functionally align the subject with the
multi-paradigm \ac{cfs}.


\textbf{Questions as discussed during our meeting}:
\begin{itemize}

\item What is causing the difference?

\item a) is it the quality of the alignment? or:

\item b) is the model doing a better prediction
    (i.e. is the criterion doing an individual overfit?)

\item SRM could just do a denoising, too?

\item How do I know if the "localizer PPA" is the "gold standard"?

\item "tacit assumption"; what does the localizer actually do?
    does it what we want it to do?

\end{itemize}


\subsection{Summary of results}

\todo[inline]{2-3 sentences here}

%
Our results provide evidence that transformation matrices calculated based
on data from naturalistic stimuli promise an increased validity of derived
transformation for functional alignment over transformation matrices based on
data (of the same!) paradigm based on simplified stimuli.



\subsection{Conclusion \& Vision}

\todo[inline]{2-3 sentences here; better write more in discussion}

Our results suggest that it is possible to ``scan once, estimate many''...



\section{Methods}

% we get the data from the naturalistic PPA paper (its subdataset)
% datalad get -n inputs/studyforrest-ppa-analysis/inputs/studyforrest-data-aligned
% datalad get inputs/studyforrest-ppa-analysis/inputs/studyforrest-data-aligned/sub-??/in\_bold3Tp2/sub-??\_task-a?movie\_run-?\_bold*.*

% reference to PPA-Paper
For the current study, we used the same subset of the studyforrest dataset that
was used in \citet{haeusler2022processing}:
%
the same subjects ($N=14$) were
% VIS
a) participating in a dedicated six-category block-design visual localizer
\citep{sengupta2016extension},
% AV
b) watching the audio-visual movie ``Forrest Gump''
\citep{hanke2016simultaneous}, and
% AD
c) listening to the movie's audio-description \citep{hanke2014audiomovie}.
% see corresponding papers for details
An exhaustive description of participants, stimulus creation, procedure,
stimulation setup, and fMRI acquisition can be found in the corresponding
publications, whereas a summary is provided in \citet{haeusler2022processing}.



\subsection{Preprocessing}

% data sources
The current analyses were carried out on the same preprocessed fMRI data (s.
\href{https://github.com/psychoinformatics-de/studyforrest-data-aligned
}{\url{github.com/psychoinformatics-de/studyforrest-data-aligned}}) that were
used for
%
a) the technical validation of the dataset \citep{hanke2016simultaneous},
%
b) the localization of higher-visual areas \citep{sengupta2016extension}, and
%
c) the investigation of responses of the \ac{ppa} correlating with naturalistic
spatial information in study 2 \citep{haeusler2022processing}.

%
We reran the preprocessing and analyses steps performed in
\citet{sengupta2016extension} and \citet{haeusler2022processing} using FEAT
v6.00 \citep[FMRI Expert Analysis Tool;][]{woolrich2001autocorr} as shipped with
FSL v5.0.9 \citep[\href{https://www.fmrib.ox.ac.uk/fsl}{FMRIB's Software
Library;}][]{smith2004fsl} in order to reproduce both the time series that
served as final input for the statistical analyses in the two previous studies
as well as their results (i.e. the statistical $Z$-maps).
% temporal filtering
In summary, high-pass temporal filtering was applied using a Gaussian-weighted
least-squares straight line to every run of the visual localizer (cutoff period
of \unit[100]{s}; sigma= \unit[100]{s}/2)\todo{???}, and every segment of the
movie and audio-description (\unit[150]{s}; sigma=\unit[75.0]{s}).
% brain extraction
Brains were extracted from surrounding tissues using BET \citep{smith2002bet}.
% spatial smoothing
As in the previous studies, data from all three paradigms were spatially
smoothed (Gaussian kernel with full width at half maximum of \unit[4.0]{mm}).
% grand mean normalization
A grand-mean intensity normalization was applied to each run of the functional
localizer and each segment of the naturalistic stimuli.

%
Further analyses on theses reproduced times series were performed via Python
scripts that relied on
%
NiBabel v3.2.1 (\href{https://nipy.org}{\url{nipy.org}}),
%
NumPy v1.20.2 (\href{https://numpy.org}{\url{numpy.org}}),
%
Pandas v1.2.3 (\href{https://pandas.pydata.org}{\url{pandas.pydata.org}}),
%
Scipy v1.6.2 (\href{https://scipy.org}{\url{scipy.org}}),
%
scikit-learn v1.0 (\href{https://scikit-learn.org}{\url{scikit-learn.org}}),
%
BrainIAK v0.11
\citep[\href{https://brainiak.org}{\url{brainiak.org}}][]{kumar2020brainiak,
kumar2020brainiaktutorial},
%
Matplotlib v3.4.0 (\href{https://matplotlib.org}{\url{matplotlib.org}}),
%
seaborn v0.11.2 (\href{https://seaborn.pydata.org}{\url{seaborn.pydata.org}}),
%
and calling command line functions of FSL.

%\paragraph{Fixing FSL output}

% grand_mean_for_4d.py (formerly: data_normalize_4d.py):
% is not necessary anymore: FSL has applied grand mean scaling to
% 'filtered_func_data.nii.gz'

% input: 'sub-*/run-?.feat/filtered_func_data.nii.gz' (of VIS, AO & AV)
% output: saved to 'sub-??_task-*_run-?_bold_filtered.nii.gz'

% FSL adds back the mean value for each voxel's time course at the end of the
% preprocessing;
% hence, the script substracts that mean again but multiplies it by 10000
% (like FSL does it, too)

% definition of grand mean scaling for 4d data:
% voxel values in every image are divided by the average global mean
% intensity of the whole session. This effectively removes any mean global
% differences in intensity between sessions.

% FSL User Guide:
% filtered_func_data will normally have been temporally high-pass filtered,
% it is not zero mean; the mean value for each voxel's time course has been
% added back in for various practical reasons.
% When FILM begins the linear modeling, it starts by removing this mean.


\subsubsection{Region of Interest}

\todo[inline]{two minor issues: grpPPA contains N=14 subject, not N-1 subjects;
warped back into subject-space, some voxels of individual masks is outside of
the back-projected union of masks}

% masks-from-mni-to-bold3Tp2.py:
% - merges unilateral ROIs overlaps (already in MNI) to bilateral ROI
% - output: 'masks/in_mni/PPA_overlap_prob.nii.gz'
% - warps union of ROIs from MNI into each subjects space
% output: 'sub-*/masks/in_bold3Tp2/grp_PPA_bin.nii.gz' + audio_fov.nii.gz dilate
% the ROI masks by 1 voxel; output: 'grp_PPA_bin_dil.nii.gz'

% masks-from-mni-to-bold3Tp2.py:
% warp MNI masks into individual bold3Tp2 spaces

% masks-from-t1w-to-bold3Tp2.py:
% transforms 'inputs/tnt/sub-*/t1w/brain_seg*.nii.gz'
% into individual's bold3Tp2
% output: 'sub-*/masks/in_bold3Tp2/brain_seg*.nii.gz'

% mask-builder-voxel-counter.py:
% builds different individual masks by dilating, merging other masks
% creates a FoV of AO stimulus for every subject from 4d time-series of AO run
% output: sub-*/masks/in_bold3Tp2/audio_fov.nii.gz'
% counts the voxels
% long story short: we cannot used all gyri that contain PPA to some degree
% even if the mask by FoV of AO stimulus and individual gray matter mask

% data_mask_concat_runs.py:
% masks are not dilated and not masked with subject-specific gray matter mask
% outputs:
% 'sub-*_task_aomovie-avmovie_run-1-8_bold-filtered.npy
% 'sub-*_task_visloc_run-1-4_bold-filtered.npy'

% reason why we do it
The \ac{srm} requires that the number of samples (i.e. the number of \acp{tr})
exceed the number of features (the number of voxels).
%
In order to restrict the number of voxels, we created bilateral \acp{roi} for
each subject by warping the union of individual \acp{ppa}
\citep[s.][]{haeusler2022processing} from MNI space into each subjects' voxel
space using previously computed subject-specific, non-linear transformation
matrices
\citep[][\href{https://github.com/psychoinformatics-de/studyforrest-data-templatetransforms
}{\url{github.com/psychoinformatics-de/studyforrest-data-templatetransforms}}]{hanke2014audiomovie}.
% applying masks
Each subject's time series data were then masked by the union of individual
\acp{ppa} and the subject-specific \ac{fov} of the audio-description.
% voxels = [1665, 1732, 1400, 1575, 1664, 1951, 1376, 1383, 1683, 1887, 1441,
% 1729, 1369, 1437]
% median = 1619.5
The number of remaining voxels per subject (range 1369--1951,
$\overline{X}=1592$, $SD=188$) can be seen in Fig.~\ref{fig:plot_voxel-counts}.


\begin{figure*}[tbp]
\centering
\includegraphics[width=\linewidth]{figures/plot_voxel-counts.pdf}
\caption{
%
\textbf{Number of voxels in the bilateral regions of interest (ROIs)
of each subject.}
%
In order to reduce the number of voxels, we warped the union of
individual \acp{ppa} \citep[cf. Fig. 1 in][]{haeusler2022processing} from
MNI152 space into each subject's native voxel space.
%
The remaining voxels of each subject were further constrained to those
voxels that are included in the respective subject's \ac{fov} of the
audio-description \citep[cf.][]{hanke2014audiomovie}.
}
\label{fig:plot_voxel-counts}
\end{figure*}


\begin{comment}

The number of remaining voxels per subject can be seen in Table
\ref{tab:ppamaskvoxels} (range 1369--1951, $\overline{X}=1592$, $SD=188$).


\begin{table*}[btp]
\caption{
%
\textbf{Table heading.}
%
Number of remaining voxels after time series data of each paradigm
and subject were masked with the union of individual \acp{ppa} that was
warped from MNI space into each individual's subjects-space and the
subject-specific field of view of audio-description.}

\label{tab:ppamaskvoxels}
\begin{tabular}{ll}
\toprule
\textbf{Subject} & \textbf{no. of voxels} \\
\midrule
sub-01 & 1665 \tabularnewline
sub-02 & 1732 \tabularnewline
sub-03 & 1400 \tabularnewline
sub-04 & 1575 \tabularnewline
sub-05 & 1664 \tabularnewline
sub-06 & 1951 \tabularnewline
sub-14 & 1376 \tabularnewline
sub-09 & 1383 \tabularnewline
sub-15 & 1683 \tabularnewline
sub-16 & 1887 \tabularnewline
sub-17 & 1441 \tabularnewline
sub-18 & 1729 \tabularnewline
sub-19 & 1369 \tabularnewline
sub-20 & 1437 \tabularnewline
\bottomrule
\end{tabular}
\caption*{The legend text goes here.}
\end{table*}

\end{comment}



\subsubsection{Concatenation of time series}

\todo[inline]{this is actually the first step of the functional
alignment procedure; but: it kind of belongs to preprocessing}

\todo[inline]{I z-scored the runs runwise, concatenated them, and z-scored
again}

% normalization
Data of every run were separately normalized ($z$-scored) to a mean of zero
($\overline{X}=0$) and a standard deviation of one ($SD=1$).
%
The last 75 \acp{tr} of the audio-description were missing in subject 04 due to
an image reconstruction problem \citep[s.][]{hanke2014audiomovie}.
%
The \ac{srm} allows the number of voxels to be different across subjects but the
number of \acp{tr} must be the same.
%
Hence, we removed the last 75 \acp{tr} of the audio-description from the all
other subjects' time series.
% summary; AO + AV = 7123 TRs (not 7198 TRs anymore); localizer has 4 x 156 TRs
As a result, the data to fit the \ac{srm} in the following step comprised 3599
\acp{tr} of the movie, 3524 \acp{tr} of the audio-description, and 624 \acp{tr}
of the visual localizer experiment (7747 \acp{tr} in total).
%% concatenate and z-score
The time series of all three paradigms were concatenated and $z$-scored.



\subsection{Estimation via anatomical alignment}

\todo[inline]{strictly speaking data were not projected through the MNI152 atlas
but through a study-specific brain template co-registered to the MNI152 atlas,
wasn't it?}

%
As a baseline, we used anatomical alignment procedure to estimate the results of
the three $t$-contrast (i.e. the \textit{empirical $Z$-maps}).
%
A test subject's empirical $Z$-map from the analysis of
%
the visual localizer \citep{sengupta2016extension},
%
the movie \citep{haeusler2022processing}, and
%
the audio-description \citep{haeusler2022processing}
%
were predicted based on the training subjects' results of the same paradigm
(hence, a within-paradigm prediction).
%
First, the empirical $Z$-maps of the training subjects' localizer contrasts were
masked with the same subject-specific masks as the time series data that were
used to generate the \ac{cfs}.
% anatomical alignment; into MNI
We used previously computed transformation matrices
\citep[][\href{https://github.com/psychoinformatics-de/studyforrest-data-templatetransforms}{\url{github.com/psychoinformatics-de/studyforrest-data-templatetransforms}}]{hanke2014audiomovie}
to project the respective $t$-contrast's masked $Z$-map via a non-linear
transformation from each training subject's voxel space into the MNI space.
% from MNI into subject
Then, we  used the transpose of the transformation matrix [or was it another
matrix?]\todo{check that} to project the data from MNI space into the test
subject's voxel space.
% take the mean
For each of the three $t$-contrast, the arithmetic mean of the respective
projected $Z$-maps served as an estimation (hence, a \textit{predicted $Z$-map})
of the test subject's empirical $Z$-map.



\subsection{Estimation via functional alignment}
%
Our procedure to estimate the $t$-contrasts' empirical $Z$-maps via functional
alignment is conducted in four steps:
% create CFS and training subjects' matrices
first, for every fold of a leave-one-out cross-validation (N$=$14 subjects), we
trained a \ac{srm} on $N-1$ training subjects' response time series
of the movie, the audio-description, and the visual localizer.
% results in...
This first step provides both a \ac{cfs} for every fold of the cross-validation
and an orthonormal transformation matrix for each training subject.
% align test subject
Second, we used time series data from the visual localizer, the movie, or the
audio-description to align the test subject to the corresponding \acp{tr} within
the \ac{cfs}.
%
Therefore, this second step provides different transformation matrices for the
test subject based on data from different paradigms.
% quantity vs. performance
Moreover, we wanted to investigate the relationship between the estimation
performance [of the $t$-contrast's results] and the quantity of data used to
acquire a transformation matrix.
% therefore
Therefore, we also varied the number of runs of the paradigms used to align the
test subject resulting in different transformation matrices based on different
amounts of \ac{tr}.
% project into CFS
In the third step, we used the training subjects' transformations matrices to
perform a mapping of the training subjects' empirical $Z$-maps from their
respective voxel space into the \ac{cfs}.
% project from CFS into test subject
In the fourth step, we used the transpose of the test subject's transformation
matrix to project the training subjects' $Z$-maps from the \ac{cfs} into the
test subject's voxel space.
% actual prediction
For each of the three $t$-contrasts, the arithmetic mean of the projected
$Z$-maps serves as a test subject's predicted $Z$-map.



\subsubsection{Fitting the shared response model (SRM)}

%
In order to acquire the \ac{cfs} and the training subjects' transformation
matrices, we used the probabilistic \ac{srm} algorithm that is implemented in
BrainIAK v.11 \citep[Brain Imaging Analysis Kit;][]{kumar2020brainiak,
kumar2020brainiaktutorial}, and approximates the \ac{srm} based on the
Expectation Maximization (EM) algorithm as proposed by \citet{chen2015reduced}
and optimized by \citet{anderson2016enabling}.


\paragraph{Number of dimensions}

\todo[inline]{is description of \citet{haxby2011common} too long?}

% ``The effect of number of PCs on BSC was similar for models that were based
% only on Princeton (n = 10) or Dartmouth (n = 11) data, suggesting that this
% estimate of dimensionality is robust across differences in scanning hardware
% and scanning parameters'' \citep{haxby2011common}.
%
% ``These dimensionality estimates are a function of the spatial and temporal
% resolution of fMRI and the number and variety of response vectors used to
% derive the common space'' \citep{guntupalli2016model}.
%
% ``The true dimensionality of representation in human cortex surely involves
% vastly more distinct tuning functions. Estimates of the dimensionality of
% cortical representation, therefore, will almost certainly be much higher as
% data with higher spatial and temporal resolution for larger and more varied
% samples of response vectors are used to build new common models''
% \citep{guntupalli2016model}.

% features
For the number of shared features (i.e. the number of the \ac{cfs}'s
dimensions), we chose a value of $k=10$ considering a) the temporal and spatial
resolution of our data (\ac{tr} = \unit[2]{s}; \unit[2.5 $\times$ 2.5 $\times$
2.5]{mm}), b) the average number of voxels per \acp{roi}, b) and findings from
\citet{haxby2011common}.
%
\citet{haxby2011common} first used hyperalignment to create a \ac{cfs} of 1,000
dimensions based of functional data (\ac{tr} = \unit[3]{s}) of voxels (of size
\unit[3 $\times$ 3 $\times$ 3]{mm}) located in the ventral temporal cortex.
%
Then, \citet{haxby2011common} reduced the dimensionality of the \ac{cfs} by
applying a \ac{pca} in order to determine the subspace that is sufficient to
capture the full range of response-pattern distinctions.
%
Results revealed that approximately 35 principal components (i.e. dimensions)
were sufficient to represent the information content of a one-hour movie from
which the \ac{cfs} was derived.
%
Results also showed that the cortical topographies of category-selective brain
regions was preserved in the 35-dimensional \ac{cfs} \citep{haxby2011common}.
%
In the current study, we also computed \acp{cfs} of $k=5, 20, 30, 40, 50$ but
[---judged by awesome eyeballing---]prediction performance based on these
\acp{cfs} barely varied from a 10-dimensional \ac{cfs}.
% iterations:
The number of iterations for the algorithm to minimize the error was set to 30.



\paragraph{Correlation of regressors used in our previous studies with shared
responses}

\todo[inline]{sounds better now but is still complex}

\todo[inline]{make colors of non-used TRs more transparent (alpha=15)}

\todo[inline]{which could be added: plots of the distance matrix showing the
similarity of \acp{tr} in the model (across shared features); imo, reader does
not gain a lot of information it; it's more "for the sake of showing it"}

\todo[inline]{One sentence about what the plots "suggest"?}

% Intro
In order to visualize characteristics of the \ac{cfs}, we calculated the Pearson
correlation coefficients between the shared responses and the regressors that
were previously modeled \citep[cf.][]{sengupta2016extension,
haeusler2022processing} to investigate hemodynamic responses during the three
paradigms.
%
As an example, we chose the \ac{cfs} that was created in the first fold of the
cross-validation from $N-1$ subjects to estimate $Z$-maps of subject 01.
%
The time series of the shared features were trimmed to match the corresponding
\acp{tr} of the respective paradigms.
%
The correlations between regressors that were created to model hemodynamic
responses during the visual localizer and shared responses (trimmed to \acp{tr}
that match the \acp{tr} of the visual localizer) can be seen in
Fig.~\ref{fig:corr-vis-reg-srm}.
%
The correlations between regressors that were created to model hemodynamic
responses during the movie \citep[cf. Table 3 in][]{haeusler2022processing} and
shared responses can be seen in Fig.~\ref{fig:corr-av-reg-srm}.
%
The correlations between regressors that were created to model hemodynamic
responses during the audio-description \citep[cf. Table 3
in][]{haeusler2022processing} and shared responses can be seen in
Fig.~\ref{fig:corr-ao-reg-srm}.

\todo[inline]{point for the discussion: the highest correlations between
individual regressors \& shared responses can be seen in the visual localizer
(the model cannot be that bad for the localizer TRs?) => but prediction via
localizer runs suck when localizer results are predicted: interpretation?}


\begin{figure*}[tbp]
\centering
\includegraphics[width=\linewidth]{figures/corr_vis-regressors-vs-cfs_sub-01_srm-ao-av-vis_feat10-iter30_7123-7747.pdf}
\caption{
%
\textbf{Pearson correlation coefficients between regressors of the visual
localizer and shared features.}
%
The time series of the shared features within the multi-paradigm \ac{cfs}
%
(as calculated for subject 01 in the first fold of the cross-validation)
%
were trimmed to match the corresponding \acp{tr} of the visual localizer
paradigm \citep{sengupta2016extension}.
%
The six regressors of the visual localizer model hemodynamic responses to
the six categories of pictures that were presented in blocks.
}
\label{fig:corr-vis-reg-srm}
\end{figure*}


\begin{figure*}[tbp]
\centering
\includegraphics[width=\linewidth]{figures/corr_av-regressors-vs-cfs_sub-01_srm-ao-av-vis_feat10-iter30_3524-7123.pdf}
\caption{
%
\textbf{Pearson correlation coefficients between regressors of the movie
and shared features}
%
The time series of the shared features within the multi-paradigm \ac{cfs}
%
(as calculated for subject 01 in the first fold of the cross-validation)
%
were trimmed to match the corresponding \acp{tr} of the movie
\citep{hanke2016simultaneous}.
%
The regressors \texttt{vse\_new} to \texttt{vno\_cut} are based on
annotations movie frames, whereas the regressors
\texttt{fg\_av\_ger\_lr} to \texttt{fg\_av\_ger\_ud} represent low-level
visual or auditory confounds
\citep[cf. Table 3 in][]{haeusler2022processing}.
}
\label{fig:corr-av-reg-srm}
\end{figure*}


\begin{figure*}[tbp]
\centering
\includegraphics[width=\linewidth]{figures/corr_ao-regressors-vs-cfs_sub-01_srm-ao-av-vis_feat10-iter30_0-3524.pdf}
\caption{
%
\textbf{Pearson correlation coefficients between regressors of the
audio-description and shared features.}
%
The time series of the shared features within the multi-paradigm \ac{cfs}
%
(as calculated for subject 01 in the first fold of the cross-validation)
%
were trimmed to match the corresponding \acp{tr} of the
audio-description \citep{hanke2014audiomovie}.
%
The regressors \texttt{body} to \texttt{sex\_m} are based on
annotations of nouns spoken by the audio-description's narrator,
whereas the regressors \texttt{fg\_ad\_ger\_lrdiff} and
\texttt{fg\_ad\_ger\_rms} represent low-level auditory confounds
\citep[cf. Table 3 in][]{haeusler2022processing}.
%
\texttt{geo\&groom} is a combination of
regressors as used on the positive side of the primary contrasts aimed to
localize the \ac{ppa} \citep[cf. Table 5 in][]{haeusler2022processing}.
}
\label{fig:corr-ao-reg-srm}
\end{figure*}


\paragraph{Negative control}

\todo[inline]{Add representation of the model to the plots (which are located in
the appendix)}

% shuffle runs
As a negative control, we randomly shuffled the order of runs of the visual
localizer and the segments of the naturalistic stimuli for each paradigm and
training subject separately before concatenating the time series, fitting the
\ac{srm}, and calculating the Pearson correlation coefficients.
%
We assumed that the \ac{srm} algorithm would not be able to fit "meaningful"
\todo{??} shared responses to randomly shuffled training data.
%
As hypothesized, results based on shuffled time series revealed no or minor
correlations between the shared responses and regressors as can be seen in
Fig.~\ref{fig:corr-vis-reg-srm-shuffled},
Fig.~\ref{fig:corr-av-reg-srm-shuffled}, and
Fig.~\ref{fig:corr-ao-reg-srm-shuffled}.

\todo[inline]{Results suggest that... ???, "no meaningful shared responses";
"random shared response"; "shared responses that do not represent hemodynamic
responses that have anything to do with XY"?}



\subsubsection{Alignment of the test subject}

% AO: 0-451, 0-892, 0-1330, 0-1818, 0-2280, 0-2719, 0-3261, 0-3524
% AV: 3524-3975, 3524-4416, 3524-4854, 3524-5342, 3524-5804, 3524-6243,
%     3524-6785, 3524-7123
% AO+AV: 0-7123

\todo[inline]{How is it done? in-code documentation says:
    \# Solve the Procrustes problem;
    A = subjectFMRIdata.dot(SharedResponses.T);
    U, \_, V = np.linalg.svd(A, full\_matrices = False);
    return U.dot(V)
    }

\todo[inline]{cf.
\href{https://github.com/brainiak/brainiak/blob/master/brainiak/funcalign/srm.py}{\url{https://github.com/brainiak/brainiak/blob/master/brainiak/funcalign/srm.py}}}

\todo[inline]{imo, the paragraph needs to be rephrased starting at "we let the
algorithm learn an orthonormal mapping..."}

\todo[inline]{confirm that matrices are in fact orthonormal}

%
We then used the test subject's response time series of the visual localizer,
the movie, or the audio-description to align the test subject \textbf{via
singular value decomposition / via Procrustes transformation} of the test
subject's data matrix to the corresponding \acp{tr} within the \ac{cfs}.
%
For the time series of each paradigm separately, we let the algorithm learn
an orthonormal transformation matrix $W_{n}$ that performs a mapping of the
responses in the test subject's voxel space during the respective paradigm into
the \ac{cfs}.
%
In order to investigate the relationship between the estimation performance [of
the t-contrast's results] and the quantity of data used to acquire a
transformation matrix, we also varied the number of runs per paradigm.
%
In case of \acp{tr} of the visual localizer, we used one up to four runs (each
lasting \unit[5.2]{m}) to align the test subject to the corresponding \acp{tr}
within the \ac{cfs}.
%
In case of the naturalistic stimuli, we used one up to eight segments (each
lasting $\approx$\unit[15]{m}) to align the test subject to the corresponding
\acp{tr} within the \ac{cfs}.
%
Consequently, for every test subject, we obtained four matrices from data of the
visual localizer, and eight different matrices per naturalistic stimulus.
%
Each of transformation matrices has a size of $v$ voxels by $k$ shared responses
but is based on an increasing quantity of data used to calculate the mapping.


\subsubsection{Estimation of $t$-contrasts' results}

% overview
We then estimated the empirical $Z$-maps of the test subject by projecting all
training subjects' empirical $Z$-maps from their voxel space trough the \ac{cfs}
into the test subject's voxel space:
% functional alignment; into CFS (calling srm.transform(masked\_zmaps))
first, we used the training subjects' transformation matrices that were derived
during training of the \ac{cfs} to perform a mapping of the masked empirical
$Z$-maps from each training subject's voxel space into the \ac{cfs}.
% into subject
Then, we used the transpose of a transformation matrix that was acquired from
the alignment of the test subject in order to project the $Z$-maps from the
\ac{cfs} into the test subject's voxel space.
% take the mean
The arithmetic mean of $N-1$ the projected empirical $Z$-maps served as the
predicted $Z$-map that estimates the test subject's empirical $Z$-map.



\subsection{Cronbach's alpha}

\todo[inline]{I rephrased this paragraph a couple of times; still sounds meh...}

\todo[inline]{point for the discussion:  when audio PPA is estimated, the lowest
correlation between predicted and empirical $Z$-maps are in those participants
that have a poor Cronbach's (and no significant cluster for the audio PPA in
\citet{haeusler2022processing}); imo, they not "reliable outliers" being
different from the norm but probably just noisy asses; hence, SRM should
probably do more of a denoising as opposed to modeling (reliable) outliers
incorrectly}

\todo[inline]{A negative Cronbach's alpha is really shitty; should be discussed
(in the general discussion on naturalistic stimuli + \ac{glm}, too)}

\todo[inline]{similar point for the general discussion: even if the \ac{glm}
model is freaking awesome, does a naturalistic stimulus provide enough events
sufficiently balanced across segments when $t$-contrasts for whatever functions
are supposed to be modeled? Naturalistic stimuli are not the panacea...!}

\todo[inline]{I have no clue where \citet{jiahui2020predicting, jiahui2022cross}
have the statement about what Cronbach's expresses from...}

%
We calculated Cronbach's $\alpha$ for the empirical $Z$-maps of each paradigm
and subject as a measure of the empirical $Z$-maps reliability and amount of
measurement error \citep{cronbach1951coefficient, cortina1993coefficient}.
%
Cronbach's $\alpha$ expresses the expected correlation between the currently
used empirical $Z$-maps and an additional set of empirical $Z$-maps calculated
based on data of a hypothetical independent dataset collected from the same
paradigm and subjects \citep{jiahui2020predicting, jiahui2022cross}.
%
These expected correlations represented by Cronbach's $\alpha$ were calculated
based on values of the first-level \ac{glm} $Z$-maps (four in case of the visual
localizer; eight in case of the naturalistic stimuli) that were averaged in the
second-level \ac{glm} analyses of their respective study
\citep{sengupta2016extension, haeusler2022processing} yielding the to be
estimated empirical $Z$-maps of the present study.


\todo[inline]{Strictly, the following is about results, not methods}

\todo[inline]{Shift results stated here to actual result section in case a
t-test involving Cronbach's is calculated (which is not the case so far)}

\todo[inline]{movie's outlier: sub-06 (0.28); but when movie PPA is predicted he
/ she is not the outlier}

\todo[inline]{audio-description's outlier: sub-05 (-0.5), sub-02 (0.0), sub-20
(0.27); when audio PPA is predicted (based on max of functional data per
paradigm), these three subjects are the outliers}

%
Cronbach's $\alpha$ of empirical $Z$-maps for each subject and paradigm can be
seen in Fig.~\ref{fig:cronbachs}, descriptive statistics across subjects for
each paradigm can be seen in Table~\ref{fig:cronbachs}.

\todo[inline]{Provide (some of the) descriptive statistics in the text? just
some numbers without real text:}

Visual localizer:  mean=0.899990, std=0.087051, min=0.658523,
25\%=0.906643, 50\%=0.934019, 75\%=0.947906, max=0.963065.
%
Movie: mean=0.611332, std=0.137878, min=0.284440,
25\%=0.555529, 50\%=0.627240, 75\%=0.676353, max=0.800254.
%
Audio-description: mean=0.476194, std=0.358019, min=-0.526626,
25\%=0.428975, 50\%=0.627799, 75\%=0.679987, max=0.823584.


\begin{table*}[btp]
\centering
    \caption{
    %
    \textbf{Descriptive statistics of Cronbach's $\alpha$ across subjects.}
    %
    Imo, not super necessary to provide these numbers. Stripplot + boxplots
    could be sufficient. If table is supposed to be kept, round the numbers,
    write a good description for the table.}
\label{tab:cronbachs}
\begin{tabular}{llll}
    \toprule
    \textbf{statistic} & \textbf{localizer} & \textbf{movie} & \textbf{audio-description} \\
    \midrule
    mean & 0.89999 & 0.611332 & 0.476194 \tabularnewline
    std & 0.087051 & 0.137878 & 0.358019 \tabularnewline
    min & 0.658523 & 0.28444 & -0.526626 \tabularnewline
    25\% & 0.906643 & 0.555529 & 0.428975 \tabularnewline
    50\% & 0.934019 & 0.62724 & 0.627799 \tabularnewline
    75\% & 0.947906 & 0.676353 & 0.679987 \tabularnewline
    max & 0.963065 & 0.800254 & 0.823584 \tabularnewline
    \bottomrule
\end{tabular}
\caption*{The legend text goes here.}
\end{table*}


\begin{figure*}[tbp] \centering
    \includegraphics[width=\linewidth]{figures/plot_cronbachs.pdf}
    \caption{\textbf{Cronbach's $\alpha$ of the empirical $Z$-maps for each
    paradigm and subject.}
    %
    Cronbach's $\alpha$ was calculated based on the $Z$-maps yielded by the
    first-level \ac{glm} analyses of the visual localizer
    \citep{sengupta2016extension} (four runs) and naturalistic stimuli
    \citep{haeusler2022processing} (eight segments each) respectively.
    %
    The second-level \ac{glm} analyses across runs / segments yielded the
    empirical $Z$-maps that were estimated in the present study.
    }
    \label{fig:cronbachs}
\end{figure*}



\pagebreak



\section{Results}

\todo[inline]{I tested all samples of Fisher transformed correlations for
normality via Shapiro-Wilk test (imo, the most appropriate test here)}

\todo[inline]{of all samples that were part of one of the eleven t-test, just
one sample (4 runs of localizer in order to estimate localizer) accepted H1
(i.e. the sample is NOT drawn from normal distribution); compute Wilcoxon
signed-rank test (which is for dependent samples)?}

\todo[inline]{t-test assumes equal variances which is probably often not
fulfilled}

\todo[inline]{alternative: two sample permutation test?}

\todo[inline]{So far, I did not test any difference(s) to Cronbach's; s.
comments below where it would makes sense to test difference to Cronbach's}


\subsection{Overview}

\todo[inline]{Since the order is "Intro, Methods, Results" and not "Intro,
Results, Discussion, Method", so the overview here ---if at all necessary---
can be pretty short}

\todo[inline]{to avoid two pretty similar texts to adjust constantly; I deleted
the overview here and stuck to just one at the beginning of the discussion}

%
Unthresholded empirical and predicted $Z$-maps in their respective subject's
voxel space can be found at
\href{https://identifiers.org/neurovault.collection:12340}{\url{neurovault.org/collections/12340}}\todo{still
empty}.
%
In order to quantify the estimation performance of the alignment procedures, we
correlated each individual's empirical $Z$-maps gained from previous analyses
\citep{sengupta2016extension, haeusler2022processing} with their respective
predicted $Z$-map (cf. Fig.~\ref{fig:stripplot}).


\subsection{Cronbach's alpha}

\todo[inline]{shift the Table and Plot of Cronbach's to here? Should
definitively be done when values are actually used for a t-test}



%
``Because the localizer task comprises several scanning runs, we calculated the
reliability of the localizer across runs with Cronbach's alpha to provide an
estimate of the noise ceiling for these correlations'' \citep{jiahui2022cross}.


\begin{figure*}[tbp] \centering
    \includegraphics[width=\linewidth]{figures/plot_corr-emp-vs-estimation.pdf}
    \caption{
    %
    \textbf{Correlations between empirical and predicted
    \textit{\textbf{Z}}-maps for each paradigm and subject.}
    %
    Functional alignment was performed based on an increasing amount of
    functional data used to align a test subject to the common functional space
    (CFS): runs of the visual localizer paradigm lasted \unit[5.2]{m}; segments
    of the naturalistic stimuli lasted $\approx$\unit[15]{m}.
    %
    Solid horizontal lines:
    %
    median of Cronbach's $\alpha$ across subjects for empirical $Z$-maps of the
    respectively estimated paradigm (cf. Fig.~\ref{fig:cronbachs}).
    %
    Dotted horizontal lines:
    %
    mean of Cronbach's $\alpha$ across subjects for empirical $Z$-maps of the
    respectively estimated paradigm (cf. Fig.~\ref{fig:cronbachs});
    %
    Grey dots:
    %
    correlations between empirical $Z$-maps and an estimation using anatomical
    alignment.
    %
    A left-out subject's $Z$-map was estimated by projecting the training
    subjects' $Z$-maps ($N=13$) from their respective voxel space through the
    MNI152 space into the test subject's voxel space, then averaging the values
    across projected $Z$-maps;
    %
    Green dots:
    %
    correlations between empirical $Z$-map and an estimation using functional
    alignment based an transformation matrices calculated from one up to four
    runs (each lasting \unit[5.2]{m}) of the visual localizer.
    %
    Red dots:
    %
    correlations between empirical $Z$-map and an estimation using functional
    alignment based an transformation matrices calculated from one up to eight
    segments (each lasting $\approx$\unit[15]{m}) of the movie.
    %
    Blue dots:
    %
    correlations between empirical $Z$-map and an estimation using functional
    alignment based an transformation matrices calculated from one up to eight
    segments (each lasting $\approx$\unit[15]{m}) of the audio-description.
    %
    \textbf{to do: keep both, median and mean of Cronbach's? Improve legend
    somehow without crowding figure / subplots?}
   }
    \label{fig:stripplot}
\end{figure*}


\subsection{General trend(s) in the results}

\todo[inline]{Somehow communicate general trends according to the "three aspects
under investigation" in the intro?}

\todo[inline]{Jeez! Phrasing!}

%
The [mean] Pearson correlation coefficients vary depending on the criterion to
be estimated (i.e. $Z$-maps of the visual localizer, movie, or
audio-description), the functional alignment procedure (anatomical vs.
functional alignment), and---in case of functional alignment---on the quantity
of a paradigm's data used to align a test subject to the \ac{cfs} as can be seen
in Fig.~\ref{fig:stripplot}.
%
However, the functional alignment procedure across criteria and predictors
reveals a monotonically increasing estimation performance the more functional
data are used to align the test subjects.


\subsection{The exemplary tests}

\todo[inline]{Maybe an ordering according to the "three aspects under
investigation" given in the intro is somehow possible}

\todo[inline]{correct the alpha-level (as it is done now) or the p-values?}

\todo[inline]{Report descriptive statistics of the correlations?}

\todo[inline]{Round the values or just write $p$<.0001}

\todo[inline]{Give one-sentence interpretations / summary à la "Results
suggest..."? imo, that's not good}



\todo[inline]{correlations based on 4 localizer runs fail the test for
normality; I ran \& report t-test here anyway}

%
In order to investigate the differences between some conditions [well, if you
know what I mean] more closely, we post-hoc\todo{?? it's not post-hoc an ANOVA}
performed eleven [final number?] paired t-tests on Fisher z-transformed
correlation values, and set the $\alpha$-level to a Bonferroni corrected
$\alpha$ of $0.05 / 11 = 0.00\overline{45}$.

%
\textbf{For example, in case of estimating the $Z$-maps of the visual
localizer}, the correlations between empirical $Z$-maps and $Z$-maps predicted
via the first movie segment were significantly higher than the correlations
between empirical $Z$-maps and $Z$-maps predicted via anatomical alignment
(Fisher z-transformed, t(14)= 6.3525802, $p$=0.0000253).
%
However, functional alignment based on data of the visual localizer
(within-paradigm prediction) and audio-description was lower than anatomical
alignment independent of the number of runs / segments used to align the test
subjects to the \ac{cfs} [not tested; clear from eyeballing].
%
Whereas the functional alignment via the first movie segment (451 \acp{tr})
yielded significantly higher correlations than a functional alignment via four
localizer runs lasting 624 \acp{tr} (Fisher z-transformed, t(14)=5.8905545,
$p$<0.0000532),
%
the functional alignment via the first segment of the audio-description yielded
significantly lower correlations than the functional alignment via four
localizer runs (Fisher z-transformed, t(14)=-2.3000009, $p$<0.0386588, will not
be significant when Bonferroni corrected).
%
The mean correlation between empirical $Z$-maps and $Z$-maps predicted via the
first movie segment were significantly lower than the correlations between
empirical $Z$-maps and $Z$-maps predicted via the first two movie segments
(Fisher z-transformed, t(14)= -5.4946197, $p$=0.0001031, Bonferroni corrected).
%
The difference between mean correlations based on two movie segments in
comparison the mean correlation based on three movie segments was not
significant (Fisher z-transformed, t(14)= -0.1293547, $p$=0.8990569).

\todo[inline]{difference to Cronbach's: maybe, run test that compares 8 movie
segments to Cronbach's alpha? independent of segments movie functional alignment
will stay lower than Cronbach's (i.e. also not "as good as").}

\todo[inline]{Following phrasing is just copied from above; values are adjusted
accordingly}
\textbf{In case of estimating the $Z$-maps of movie}, the correlations between
empirical $Z$-maps and predicted $Z$-maps via the first movie segment were
significantly higher than the correlations between empirical $Z$-maps and
predicted $Z$-maps via anatomical alignment (Fisher z-transformed, t(14)=
5.7754451, $p$=0.0000643).
%
Whereas the functional alignment via the first movie segment (451 \acp{tr})
yielded significantly higher correlation than a functional alignment via four
localizer runs lasting 624 \acp{tr} (Fisher z-transformed, t(14)=6.8532349,
$p$<0.0000116),
%
the functional alignment via the first segment of the audio-description yielded
no significantly different correlations than the functional alignment via four
localizer runs (Fisher z-transformed, t(14)=-1.8674144, $p$<0.084551).
%
The mean correlation between empirical $Z$-maps and predicted $Z$-maps via the
first segment alone of the movie were significantly lower than the correlations
between empirical $Z$-maps and predicted $Z$-maps predicted $Z$-maps via the
first two movie segments (Fisher z-transformed, t(14)= -3.7454592,
$p$=0.0024485).
%
The prediction performance based on data of the audio-description
increases with more data. However, the "relevant" test compared "similar" amount
of \acp{tr}:
%
The difference between mean correlations based on two movie segments in
comparison the mean correlation based on three movie segments was [not]
different [when adjusted alpha-level used] (Fisher z-transformed, t(14)=
-2.5759899, $p$=0.02303).


\todo[inline]{difference to Cronbach's: run test(s) that compare x movie
segments for alignment and Cronbach's alpha of movie's empirical $Z$-map?
functional alignment will sooner or later significantly higher than Cronbach. }


\todo[inline]{The following part is the ugly one}

\todo[inline]{All samples will probably reject H0 assuming normality}

\todo[inline]{The three outliers are always sub-02, sub-05, sub-20 (always=
anatomical alignment, 4 runs of localizer, 8 runs of movie, 8 runs of
audio-description)}

\textbf{In case of estimating the $Z$-maps of the audio-description} it's a
fucking mess, which is why we tested only one difference [so far]:
%
Correlations between empirical $Z$-maps and predicted $Z$-maps via the first
movie segment were significantly lower than the correlations between empirical
$Z$-maps and predicted $Z$-maps via anatomical alignment (Fisher z-transformed,
t(14)= -4.2004329, $p$=0.0010387, Bonferroni corrected).

\todo[inline]{any differences compared to Cronbach's alpha to test?
audio-description for alignment will sooner or later be significantly higher
than Cronbach's; might be a cue that SRM does a denoising of the map (and does
not estimate "reliable patterns deviant from the norm" wrongly)}


\subsection{Template of reporting descriptive statistics of (untransformed
correlations)}

\todo[inline]{Report descriptive statistics of the correlations?}

\todo[inline]{imo, it gets pretty crowded, they can somewhat be inferred from
the stripplot; additionally, there is now the table; correlations get Fisher
transformed anyway before testing}

%
The mean Pearson correlation values [not yet Fisher transformed] between
empirical $Z$-maps and predicted $Z$-maps via anatomical alignment were 0.xx
($N=14$, $\overline{X}=0.xx$, $SD=0.xx$, range [?], median
[9], 25\%, 50\%, 75\%).
%


\subsection{Template of procedure in case normality tests are relevant to
perform and report}

\todo[inline]{Correlations (Z-transformed) of estimation of visual localizer via
4 runs of visual localizer are not normally distributed according to
Shapiro-Wilk test -> use non parametric test instead? Wilcoxon signed-rank
should be the correct one?}

%
In case a test failed to reject the H0 that assumes a normal distribution for
both samples, we perform a t-test [which assumes equal variances, too, but
whatever]; in case H1 (non-normality) is accepted for one of the two to be
compared samples, we calculated a Wilcoxon signed-rank test.


\subsection{plot\_bland-altman.py}

\todo[inline]{I hate that script! Hence, it should not be included ;-).}

\todo[inline]{mih: but results?}

\subsection{Plots of brain slices?}

\todo[inline]{iirc, we agreed on that it's not necessary}



\pagebreak


\section{Discussion}

\todo[inline]{generally, I think it's best to abstract A LOT from the specific
results (other studies discuss general stuff only, too; fucking joke, anyway)}

\todo[inline]{Therefor, two possibilities: a) write general shit \& give some
concrete examples (a.k.a. statistical tests) to make the point, or b) discuss
concrete examples (a.k.a. the tests) \& draw the general inference (how it is
done now)}

\subsection{Short summary of aims \& hypotheses, method, results}

\todo[inline]{if necessary, write similar text at the beginning of results}

\todo[inline]{Try to start with aims \& goals before reviewing methods}

\todo[inline]{in general, it's probably (still) too long}


\paragraph{Methods}
%
In the present study, we estimated the results of $t$-contrasts that were
created in previous studies \citep{sengupta2016extension,
haeusler2022processing} in order to localize the \ac{ppa}.
%
Following a leave-one-subject-out cross-validation, the $t$-contrasts' empirical
$Z$-maps serving as the to be predicted criteria were estimated using an
anatomical alignment approach \citep[cf.][]{zhen2015quantifying,
zhen2017quantifying} and a functional alignment approach.
%
In case of the functional alignment approach, we fit a shared response model
\citep{chen2015reduced} to the training subjects' concatenated response time
series of three different paradigms in order to acquire a \ac{cfs} and the
training subjects' subject-specific transformation matrices.
%
In order to acquire the test subject's transformation matrix, we used an
increasing amount of the test subject's response time series of the three
paradigms separately as predictors by functionally aligning the test subject
to the corresponding \acp{tr} of the shared feature space (i.e. \ac{cfs}).
%
The empirical $Z$-maps of each training subject were projected from their
respective voxel space through the \ac{cfs} into the test subject's voxel space
to yield the test subject's predicted $Z$-maps.
%
In case of the anatomical alignment approach, the training subjects' $Z$-maps
were projected via nonlinear, volume-based transformation through the MNI152
brain atlas \todo{kind of incorrect} into the test subject's voxel space.

%
The estimation performance of the alignment approaches was quantified by
correlating the empirical $Z$-maps with their respective predicted $Z$-map
(which can be found at
\href{https://identifiers.org/neurovault.collection:12340}{\url{neurovault.org/collections/12340}}).


\paragraph{Three aspects under investigation}

\todo[inline]{strictly speaking, with comparison to anatomical alignment it's
four aspects}

\todo[inline]{biggest issue for discussion: how to separate validity /
generalizability of \ac{cfs} from validity / generalizability of transformation
matrices? If I do not find a solution for that the intro should be adjusted (to,
maybe, just two points?)}

\todo[inline]{the highest correlations between individual regressors and shared
responses can be seen in the visual localizer => but the matrices based on
localizer runs suck the most for estimation; interpretation?}

\todo[inline]{in any case: within-paradigm prediction is more about validity of
model (and matrices?); cross-paradigm prediction is more generalizability of
(model and) matrices}

%
Zero, we compare anatomical to functional alignment.
%
First, we assess the \textbf{validity of the \ac{cfs}} by estimating $Z$-maps
from the analysis (i.e. $t$-contrast) of same paradigms used to align the test
subject (cross-subject-\textbf{within-paradigm} prediction).
%
Second, we assess the \textbf{generalizability of the transformation matrices}
by estimating $Z$-maps from the analysis (i.e. $t$-contrast) of paradigms that
were not used to align the test subject (cross-subject-\textbf{cross-paradigm}
prediction).
%
Third, we explore the relationship between the estimation performance of the
t-contrast's results and the quantity of data from each of the three paradigms
used to functionally align the subject with the \ac{cfs}.


\paragraph{Results}

\todo[inline]{if a high-level summary of results is supposed to be written here
(at the moment I think it does not make a lot of sense), the summary heavily
depends on the "final" phrasing of the discussion}

\todo[inline]{Maybe, it's simply better to write a stupid high-level discussion
and ---to make the point--- cherry-pick some statistical tests as concrete
examples}

Results suggest that "partial alignment" works for multi-paradigm \ac{cfs}
derived from concatenated time series of multiple paradigms.
%
For all criteria, the prediction performance improves continuously with more
data of the paradigms used as predictors.


\subsection{Functional alignment vs. anatomical alignment}
%
When the visual paradigms (i.e. localizer and movie PPA) are estimated, the
movie beats anatomical alignment with just one movie segment (which is
statistically tested).
%
Movie gets (sooner or later) as good as anatomical alignment when audio PPA is
estimated (but is worse when just one movie segment is used).

%
Localizer runs suck compared to anatomical alignment when used to estimate
localizer $Z$-map (within-paradigm prediction) and movie PPA (cross-paradigm
prediction).
%
Surprisingly, when audio PPA is estimated (cross-paradigm), localizer for
alignment gets close to anatomical alignment; i.e. audio PPA is the criterion
the visual localizer can estimate the best.\todo{but why???}

%
Audio-description sucks---compared to anatomical alignment---for cross-paradigm
estimation.
%
Though, it gets better when more / all segments are used for alignment:
%
For within-paradigm prediction, audio-description eventually outperforms
anatomical alignment (when more than 4-6 segments are used)
%
Generally, the more data of the audio-description is used, the narrower the gap
between movie and audio-description gets.

%
For every alignment procedure (functional as well as anatomical), the outliers
of Cronbach's of audio PPA are difficult to estimate.


\paragraph{Summary \& conclusion on anatomical vs. functional alignment}

\todo[inline]{Results suggest...? Interpretation}

Anatomical alignment is actually not that bad.
%
It's pretty consistent across paradigms, too (always a median correlation of
about 0.6).



\subsection{Functional alignment across predictors \& criteria}

\todo[inline]{at the moment, I think it's best to order by paradigm used as
predictor; for every predictor, discuss each criterion and the quantity of
data}

\todo[inline]{problem: \citet{feilong2022individualized} already assessed data
quantity; results suggest 30m are "good" in case of their model; plus, they
model individual component.}

The multi-paradigm model is kind of valid, I guess...






\subsubsection{Validity \& generalizability of movie}

\todo[inline]{a.k.a. movie is best (which is in line with previous research)}

\paragraph{criterion: localizer}

\paragraph{criterion: movie PPA}

%
It's not a dynamic localizer employing video snippets of landscapes and faces
but a "real naturalistic localizer".

\paragraph{criterion: audio PPA}

\paragraph{general stuff about quantity of data}

%
Results indicate that $\approx$15 minutes (\ac{tr}=2s) of movie watching used
for functional alignment outperform prediction using anatomical alignment.
%
Prediction performance further increases when $\approx$30 minutes of movie data
submitted to the algorithm to calculate the subject-specific transformation
matrices.
%
However, three segments ($\approx$45 minutes) do not lead to an significantly
increased estimation performance suggesting a decreasing benefit of longer
scanning time than 30 minutes during audio-visual naturalistic stimulation.



\subsubsection{Validity \& generalizability of visual localizer}

\todo[inline]{a.k.a. why is it so "bad"???}

\todo[inline]{We have way more \acp{tr} in case of naturalistic stimuli; however
not necessarily equal number of events; and events are not "clean" events but
heavily confounded by other stuff}

\paragraph{criterion: localizer}

%
Surprisingly, prediction of localizer via localizer runs (i.e. within-experiment
prediction) is worse than anatomical alignment.
%
How does that come? At least the correlations of regressors with shared
responses show the "clearest" correlations of a regressor with a shared response
(which is not the case in the \acp{tr} of the movie or audio-description).
%
Maybe, our naturalistic stimuli (about 7000 TRs) fucked up our shared responses
in the 450 \acp{tr} within the SRM (imo, Fig.~\ref{fig:corr-vis-reg-srm}
suggests otherwise)


\paragraph{criterion: movie PPA}

\paragraph{criterion: audio PPA}

\paragraph{general stuff about quantity of data}




\subsubsection{Validity \& generalizability of audio-description}

\todo[inline]{a.k.a. it's totally different from different previous studies!}

\todo[inline]{here it is about audio-description as predictor!}

\todo[inline]{below is a paragraph about "audio PPA as criterion"; separate
points more clearly or somehow merge if the two topics have too many
intersecting points}


\paragraph{criterion: localizer}
%
A daring cross-modal prediction.
%
Kind of works, but you need a shit ton of data, i.e. it gets better, the more
data are available


\paragraph{criterion: movie PPA}
%
Another daring cross-modal prediction (similar pattern, eventually outperforms
visual localizer for alignment and gap between audio-description and movie for
alignment gets narrower the more data are used).

\paragraph{criterion: audio PPA}
%
Even with just one segment it outperforms (slightly) the functional alignment
using localizer runs or movie segments (not statistically tested though).
%
Eventually, audio-description based functional alignment outperforms anatomical
alignment.


\paragraph{general stuff about quantity of data}
%
It does not look like "garbage in, garbage out".

\paragraph{Inference}

\todo[inline]{yeah, what does that mean?}

%
Audio-description is lacking visual stimulation, audio-description is
"not-as-rich" as the movie.
%
Additionally, maybe, the auditory response in PPA is (too) different from visual
response?
%
Still, results show that data collected during listening to an audio-description
[which is richer than a mere narrative] could, in principle, be used to estimate
a visual category-selective area [but impractical amounts of data with current
approach].

%
Interesting would be estimation of results from a controlled speech
paradigm, i.e. another same-modality criterion.


\subsection{Interim summary: Validity \& generalizability of multi-stimulus
model and validity \& generalizability of matrices}

\todo[inline]{how to separate validity \& generalizability of \ac{cfs} from
validity \& generalizability of matrices?}

%
Our results provide evidence that transformation matrices calculated based on
data from naturalistic stimuli promise an increased validity and [or?]
generalizability for functional alignment over transformation matrices based on
data of a controlled paradigm based on simplified stimuli.



%
This is the case for both within-paradigm prediction (e.g., audio-description
for alignment to estimate $Z$-maps from the audio-description's $t$-contrast)
and cross-paradigm prediction (e.g. audio-description for alignment to estimate
$Z$-maps from the visual localizer's $t$-contrast).

%
One segment of audio-description is not statistically different from four
runs of localizer to estimate the localizer (after Bonferroni correction;
p=0.03).

\todo[inline]{probably discuss \citet{haxby2011common} here; cf. templates at
the very end of this chapter}

\todo[inline]{if \citet{haxby2011common} is discussed (a lot), it should
probably be primed before "Here, we..." in the introduction}

%
Our results are different results of \citet{haxby2011common} who found that the
prediction of $Z$-maps from a controlled paradigm via matrices \& \ac{cfs} based
on the same controlled paradigm was as good or better than prediction (of the
same) $Z$-maps via matrices \& \ac{cfs} based on movie data.

%
In summary, the multi-stimulus \ac{cfs} generalizes over paradigms
to be estimated but performance depends on the paradigm used to align the test
subject to the \ac{cfs}.



\subsection{More specific stuff}

\subsubsection{Localizer as criterion: "ground truth"?}

\todo[inline]{imo, this is more a topic for the general discussion}

\todo[inline]{cf. general discussion: pros \& cons of naturalistic stimuli}


\subsubsection{audio PPA as criterion: the issue of reliability}

\todo[inline]{also true (but less severe) in case of movie PPA}

\todo[inline]{topic for general discussion, too: pros \& cons of naturalistic}

\todo[inline]{the "deviant" participants in the audio-description are the ones
that have a poor Cronbach's; imo, they're not "reliable outliers" being
different from the norm but noisy asses}

\todo[inline]{but potential point to make: results could be interpreted as the
SRM doing some denoising (as opposed to modeling reliable outliers incorrectly)}

Results of \citet{haeusler2022processing} could be influenced by paradigm and
methodological choices.


\paragraph{Issues of methodological decisions}

\todo[inline]{a.k.a. operationalization \& construct validity}

%
\citet{haeusler2022processing} might have modeled the time course of "spatial
responses" "insufficiently".
%
However, the primary audio-description contrast in
\citet{haeusler2022processing} ``yielded bilateral clusters in nine participants
that are within or overlapping with the block-design localizer results.
%
For another participant (sub-09) the analysis yielded one cluster in the
left-hemispheric PPA'' \citep{haeusler2022processing}.
%
Hence, we should have modeled the construct "correctly" for >50\% of subjects in
\citet{haeusler2022processing}.


Is the number of events across segments too unbalanced?
%
Here again, the dead-end argument is that it worked for most subjects.


\paragraph{Issues of the paradigm}
%
Results of \citet{haeusler2022processing} suggest that the audio-description
samples the responses to (at least auditory) spatial information.
%
Moreover, the plots of shared responses \& AO regressors (cf.
Fig.~\ref{fig:corr-ao-reg-srm}) suggest that the shared responses during
\acp{tr} of the audio-description are not total garbage.
%
Shared responses are correlated with specific regressors of the visual
localizer, whereas shared features during \acp{tr} of the movie /
audio-description seem to be more abstract (or the regressors simply suck).
%
Finally, the within-paradigm prediction of audio PPA works for most subjects;
%
which also suggests that the Cronbach's-a-outliers do not degrade the model too
much.


\todo[inline]{No, I am not eager to look at the outliers' first-level
$Z$-maps in detail...}

%
So, maybe, the audio-description gave too much room for variations across the
time of the paradigm in some subjects?
%
It might be an issue auditory domain in general (whatever that is supposed to
mean)?
%
It might be the case that a 2h-long auditory stimulation is not as immersive /
engaging as the flashy multi-modal audio-visual movie?
%
Problem to control: the attentional focus is harder to judge / control (compared
to eye-tracking during movie);
%
alertness (via EEG) should be possible though.



\paragraph{Inference}

\todo[inline]{add text about SRM as denoising technique?}
%
But still, current results suggest that the audio-description engages the
process of auditory spatial information reliable across subjects in such a way
that the ``SRM will improve sensitivity for detecting a cognitive process of
interest in the test data'' \citep{cohen2017computational}.


\todo[inline]{Following are pretty bold statements}
%
Present results support evidence that results in \citet{haeusler2022processing}
that are restricted to the anterior part of the localizer PPA are not based on
the methodological decisions.
%
Further, present results add evidence that that the responses to auditory
spatial information lead to different activity patterns than visual stimulation
[cf. interpretation in \citet{haeusler2022processing}].
%
We need---of course!---further studies that use controlled paradigms to
investigate auditory spatial information, and studies that use auditory
narratives and employ more sophisticated models of event structure in order to
assess the suitability of naturalistic stimuli as "true naturalistic localizer".



\subsubsection{audio PPA as criterion: the issue of individual differences}

\todo[inline]{imo, this topic can be dropped because the differences are not
reliable}



\subsection{Vision: calibration scan + database}

% A shared calibration scan across datasets could be used to transfer data
% between datasets, a procedure that is easier to accomplish than shared
% subjects across datasets \citep[cf.][an extension of the \ac{srm} for shared
% subjects across datasets]{zhang2018transfer}.

\todo[inline]{text needs to match with general discussion's "Vision"}

\todo[inline]{focus on what we did in the present study}

\todo[inline]{especially, "clinical context" belongs to general discussion}



\subsubsection{Intro}

\todo[inline]{repeat phrasing from introduction (cf. localizer paradigms)}

\todo[inline]{do not open pandora's box of "functional atlas"!!}

%
Our results suggest that additional 15 minutes functional scanning using an
engaging naturalistic stimuli could provide sufficient data for a
\textit{calibration scan}.
%
A standardized calibration scan could be used to align a new subject to a
\ac{cfs} that was derived from extensive scans of a reference group.
%
Extensive scans of the reference group would comprise data collected from
naturalistic paradigms but also controlled paradigms considered to be the ``gold
standard'' to probe low-level auditory or visual perception, or higher cognition
like theory of mind \citep{spunt2014validating} or semantic processes
\citep{fedorenko2010new, fernandez2001language}.
%
The diagnostic run would be based on a multifaceted naturalistic stimulus that
samples a broad range of brain states in order to allow a valid alignment to the
reference \ac{cfs}.
%
Compared to a diagnostic run based on a controlled paradigm, a naturalistic
stimulus would have the additional benefit of higher engagement and better
compliance \citep{vanderwal2015inscapes, eickhoff2020towards} in, e.g., children
or a patients.


\paragraph{application: estimate \& quantifiy regular vs. deviant pattern}

%
Once a new subjects is aligned to the \ac{cfs}, functional patterns collected in
a reference group could be mapped through the \ac{cfs} into the new subject's
voxel space.
%
On the one hand, this would allow to estimate regular patterns in a new subject
when additional functional scans are not possible to scanner availability,
time-limiations or monetary constrains, or compliance issues.
%
On the other hand, this would allow to quantify the similarity (or difference)
of a new subject's actual pattern (i.e. a empirical $Z$-map) to a pattern
estimated from a healthy or clinical reference group.


\paragraph{Examples for clinical populations}

%
Patient populations, such as patients who
%
are blind
%
[Mahon et al. 2009; Bedny et al. 2011; Striem-Amit, Dakwar, et al.  2012b; van
den Hurk et al. 2017] \citep{rosenke2021probabilistic},
%
[Amedi et al., 2007; He et al., 2013; Mahon et al., 2009; Wolbers et al., 2011]
\citep{weiner2018defining}, or
%
have a brain lesion
%
[Schiltz and Rossion 2006; Steeves et al. 2006; Sorger et al. 2007; Barton 2008;
Gilaie-Dotan et al.  2009; de Heering and Rossion 2015]
\citep{rosenke2021probabilistic},
%
individuals with visual agnosia/prosopagnosia
%
[Schiltz and Rossion 2006; Steeves et al. 2006; Sorger et al. 2007; Barton 2008;
Gilaie-Dotan et al. 2009; Susilo et al. 2015] \citep{rosenke2021probabilistic}.



\paragraph{Example: \citet{yates2021emergence} quantify difference}

\todo[inline]{This is essentially the abstract of \citet{yates2021emergence}}

For example, \citet{yates2021emergence} tested ``the presence and localization
of adult functions in children using shared response modeling.
%
The feature space was learned from fMRI activity of adults watched a movie.
%
The shared features were then translated into the anatomical brain space of
children 3--12 years old.
%
The found reliable correlations between reconstructed activity and children's
actual fMRI activity as they watched the same movie.
%
The strength of the correlation in the precuneus, inferior frontal gyrus, and
lateral occipital cortex predicted chronological age''
\citep{yates2021emergence}.



\subsection{Shortcomings \& future questions}

\subsubsection{Bigger sample size}
%
The SRM is computationally less demanding [= "computationally more efficient"?]
than hyperalignment, an advantage for scientists who want to replicate our
results but do not have access to a high-performance computer cluster [or
"high-throughput computer cluster" or simply "specialized hardware"?].
%
Moreover, it should scale pretty well (compared that to
\citep{jiahui2020predicting, jiahui2022cross} 1-step alignment that needs
pair-wise matrices for subjects 'cause no \ac{cfs}; similarly,
\citep{busch2021hybrid})\todo{check Busch}.


\paragraph{...because our sample size was small with following effect}

\todo[inline]{topic might be dropped}

% what is the case
The correlations of shared responses within the \acp{cfs} created from $N-1$
training subjects varied across the folds of the cross-validation.
% interpretation
That means, a change of 1/13 of the data for every subject's analysis is causing
the estimates to vary [how much?].
% conclusion
Future studies, should create a \ac{cfs} based on data from more subjects and
investigate the relationship between number of participants, variability of
parameters, and estimation performance.



\subsubsection{Other ROIs than category-selective areas}

\todo[inline]{a.k.a. explore estimation performance in case of other functions}

\todo[inline]{don't write too much here}

\todo[inline]{more a topic for general discussion: naturalistic stimuli
might be limited (e.g., might not sample executive functions sufficiently)}

%
In the present study, we focused on the \ac{ppa} as an classic example of a
higher-visual area.

%
However, studies have shown a varying degree of \textit{functional--anatomical
correspondence} between a brain function and its underlying anatomical location.
%
Previous studies that used either volume-based \citep{zhen2017quantifying,
zhen2015quantifying} or surface-based alignment \citep{rosenke2021probabilistic,
frost2012measuring} in order to estimate the most probable location of
functional area in a test subject have ``found a large variability in the degree
to which functional areas respect macro-anatomical boundaries across the
cortex'' \citep{frost2012measuring}.
%
``There is a strong structural-functional correspondence in some areas whilst in
others the spatial location of the functional area varies greatly across
subjects within a cortical area'' \citep{frost2012measuring}.

%
For example, in domain of category-selective areas, the interindividual
variability varies across functional areas \citep{zhen2017quantifying,
zhen2015quantifying, frost2012measuring}:
%
``Scene-selective regions showed larger interindividual variability [after
nonlinear volume-based alignment] than the face-selective regions in spatial
topography'' \citep{zhen2017quantifying}.




\paragraph{Studyforrest dataset's other localizer t-contrasts}

\todo[inline]{does it make sense to discuss that? it's such a low-hanging fruit}

\todo[inline]{if it makes sense to be discussed, discuss it here and not in the
general discussion}

%
The studyforrest dataset's visual localizer \citep{sengupta2016extension} offers
other contrasts (and thus \acp{roi} masks) aimed at localizing the \ac{ffa} and
\ac{ofa} that are associated with face perception \citep{kanwisher1997ffa,
pitcher2011occipitalfacearea}, the \ac{eba} that is associated with the
perception of human bodies \citep{downing2001bodyarea}, and the \ac{loc} that is
associated with the perception of (small) objects (like tools or toys)
\citep{malach1995loc}.

Quickly shifted from general discussion to here:
%
The visual localizer performed by \citet{sengupta2016extension} employed images
from six categories (houses, landscapes, faces, bodies without heads, small
objects, and scrambled images).
%
As a results, the corresponding dataset provides subject-specific \acp{roi}
masks for higher visual areas besides the \ac{ppa}:
%
the fusiform face area (FFA) \citep{kanwisher1997ffa} and the occipital face
area (OFA) \citep{pitcher2011occipitalfacearea},
%
the extrastriate body area (EBA) \citep{downing2001bodyarea},
%
and the lateral occipital complex (LOC) \citep{malach1995loc}.
%
Future studies (e.g., a master's thesis or part of a PhD project) could adjust
our extension of the annotation of speech created in study 2 and the
corresponding analysis pipeline in order to explore hemodynamic responses
correlating with auditory information related to faces, body parts or small
objects.


\paragraph{Examples}

\todo[inline]{shorten this part}

% Zhen 2017
``Scene-selective regions showed larger interindividual variability [after
nonlinear volume-based alignment] than the face-selective regions in both
spatial topography and functional selectivity'' \citet{zhen2017quantifying}.


% category-specific areas
Similar to nonlinear volume-based alignment, similarity across person is higher
after surface-based alignment ``for retinotopically defined regions, with
character-selective regions showing the lowest consistency for both alignments,
closely followed by mFus- and IOG-faces'' \citep{rosenke2021probabilistic}.

%
``We localized 13 widely studied functional areas and found a large variability
in the degree to which functional areas respect macro-anatomical boundaries
across the cortex'' \citep{frost2012measuring}.
%
``The percent gain in overlap [after surface-based alignment] differed greatly
across the different functional regions throughout the cortex''
\citep{frost2012measuring}.
%
``There is a strong structural-functional correspondence in some areas whilst in
others the spatial location of the functional area is not tightly bound to
anatomical landmarks and varies greatly across subjects within a cortical area''
\citep{frost2012measuring}.
%
``Language areas were found to vary greatly across subjects whilst a high degree
of overlap was observed in sensory and motor areas'' \citep{frost2012measuring}.

%
``Area LOC also showed increased overlap after CBA with a 62.7\% gain in the
left hemisphere and 38.4\% on the right'' \citep{frost2012measuring}.
%
Finally PPA exhibit more gain in the right hemisphere with 27.7\% gain, than on
the left with 17.6\%'' \citep{frost2012measuring}.
%
``FFA varies in its location along the length of the fusiform gyrus even though
the gyri themselves are well aligned across subjects''
%
The FFA did not exhibit the same strong structural-functional correspondence and
saw more modest increases in overlap after macro-anatomical alignment with
44.1\% and 12\% gain for the left and right hemispheres''
\citep{frost2012measuring}.



\paragraph{More severe: language areas}

\todo[inline]{If this is mentioned at all, write half a sentence about speech
lateralization; problem: even in case whole-cortex alignment algo is used,
speech areas probably vary too much (search lights a probably too small in case
of atypical speech topographies)}

%
More severe: ``Language areas were found to vary greatly across subjects whilst
a high degree of overlap was observed in sensory and motor areas''
\citep{frost2012measuring}.


\paragraph{Inference}

\todo[inline]{Mention whole-cortex (searchlight) alignment here already?
...instead of writing a dedicated treatise about "Other functional alignment
algorithms" below?}

\todo[inline]{maybe, cite studies that did research on other brain functions
using naturalistic stimuli; or just cite a good review}

%
Hence, a new dataset should not just have more subjects but also more
localizers.
%
Then, future studies can investigate other functional domains.
%
Naturalistic stimuli have been used to investigate a variety of domains like XYZ
[check for, e.g., visual or auditory perception, spatial cognition; emotion;
music, speech or social perception], and possibly allow valid mapping of
functional data in all of these domains.


\subsubsection{\ac{srm} is problematic in case of atypical organization /
outliers}

\todo[inline]{Check \citet{feilong2022individualized, jiahui2022cross,
turek2018capturing}; at least two of them have modeled an individual component.}

%
A drawback of the SRM algorithm we employed here is that it models responses
that are common across persons without an individual component.
%
A person's idiosyncratic responses are excluded from the shared response model
but ``are not necessarily noise and may in fact be highly reliable within
participants'' \citep{cohen2017computational}.
%
In order to predict a reliably atypical pattern (and not just quantify the
deviation from norm), you need matching \ac{cfs} or---probably better---model
has to have a shared response + individual component + noise.
%
Assess performance of whatever functional alignment algo to [directly] estimate
[reliable] outliers in a sample.
%
Do you estimate the regular pattern (which allows you do quantify the difference
of a deviant actual pattern to a norm) or do you directly estimate the deviant
pattern from regular reference? Hm...

%
Imo, present results suggest estimation of regular pattern (a.k.a. denoising).


\paragraph{Templates that can probably be dropped}
%
``SRM can be used to isolate participant-unique responses by examining the
residuals after removing shared group responses, or it can be applied
hierarchically to the residuals to identify subgroups [\citet{chen2017shared}]
'' \citep{cohen2017computational}.
%
``In cases where each subject's unique response is of more interest than the
shared signal, SRM can be used to factor out the shared component thereby
isolating the idiosyncratic response for each subject
[\citep{chen2015reduced}]'' \citep{kumar2020brainiak}.
%
``Recognizing that signal exists beyond the average or shared response of a
group, such studies exploit idiosyncratic but stable responses to account for
previously unexplained variance in brain function, behavioral performance and
clinical measures [e.g., Finn (2015). Functional fingerprinting (based on
connectivity)]'' \citep{cohen2017computational}.




\subsubsection{Other functional alignment algorithms}

\todo[inline]{In general, a big treatise of pro \& cons of different algorithms
needs to be avoided}

\todo[inline]{Evtl. kann man alle Punkte dieser Subsubsection bereits oben
verwursten; hier kommen die Punkte eh so ein wenig zusammenhangslos als
Selbstzweck und ziehen allgemeine ganz am Ende die Ergebnisse runter (``a.k.a.
"warum hast du nicht gleich Algo XY benutzt?'')}


\subsubsection{Volume- vs surface-based}

\todo[inline]{An "issue" for reviewers but, imo, it's not a real issues}

\todo[inline]{$Z$-maps of localizer were calculated in voxel space; cf. general
discussion (opportunity costs)}

\todo[inline]{we are nearer on the raw data (less error accumulation?) 'cause we
work with voxels (surface vertices need additional mapping of $Z$-maps
calculated from smoothed data), and both our anatomical and functional alignment
use voxels as input (i.e. it is not mixed)}

We compare volume-based [nonlinear] anatomical alignment to volume-based
functional alignment.
%
Future work could compare surface-based alignment that respects cortical folding
structure -- that out-performs predictions based on [affine] volume-based
anatomical alignment \citep{weiner2018defining} -- to surface-based functional
alignment.


\subsubsection{ROI vs. whole-brain (i.e. searchlight)}

\todo[inline]{Maybe, shift that to "future studies on other ROIs"}

\todo[inline]{searchlight SRM \citep{zhang2016searchlight}}

\todo[inline]{negative: more parameters to vary and to assess}

``Searchlight functional alignment [\citep{zhang2016searchlight,
guntupalli2016model}] learns local transformations and aggregates them into a
single large-scale alignment.
%
The searchlight scheme [Kriegeskorte, 2006, Information-based functional brain
mapping], popular in brain imaging [Guntupalli et al., 2018; 2016], has been
used as a way to divide the cortex into small overlapping spheres of a field
radius.
%
This method allows researchers to remain agnostic as to the location of
functional or anatomical boundaries, such as those suggested by
parcellation-based approaches.
%
A local transform can then be learned in each sphere and the full alignment is
obtained by aggregating (e.g. summing as in \citep{guntupalli2016model} or
averaging) across overlapping transforms.
%
The aggregated transformation produced is no longer guaranteed to bear the type
of regularity (e.g orthogonality, isometry, or diffeomorphicity enforced during
the local neighborhood fit)'' \citep{bazeille2021empirical}.
%
``In the case of searchlight Procrustes, we selected searchlight parameters to
match those used in Guntupalli et al. (2016):
%
each searchlight had 5 voxel radius, with a 3 voxel distance between searchlight
centers'' \citep{bazeille2021empirical}.


\subsubsection{Time series vs connectivity-based}

\todo[inline]{kind of a killer cause you do not need intersection of
time series}

\todo[inline]{avoid a treatise by stating in the intro that response based
aligns responses better, connectivity-based aligns connectivity better}

\todo[inline]{what did \citep{nastase2019leveraging} do? As of now, I do not
care anymore tbh}

\todo[inline]{\citet{jiahui2022cross} do connectivity-based 1-step
hyperalignment across different movie datasets which is as good as response
hyperalignment; however, it's a shitty procedure to scale because the need a
transformation matrix for every pair of subjects (i.e no \ac{cfs})}


%
``Response-based hyperalignment (RHA) mapped data from the anatomical space to a
common information space based on time-point response patterns across cortical
vertices.
%
Connectivity-based hyperalignment (CHA) mapped data from the
anatomical space to a separate common information space based on functional
connectivity patterns derived from the movie response data''
\citep{busch2021hybrid}.

``Response-based hyperalignment was shown to align response-based data better
than connectivity-based hyperalignment, whereas connectivity-based
hyperalignment was shown to better align connectivity-based data than
response-based hyperalignment [Guntupalli et al., 2018]''
\citep{busch2021hybrid}.
%
``Response-based common spaces better align held-out response data, whereas
connectivity-based common spaces better align held-out connectivity data.
[Guntupalli et al., 2018] \citep{busch2021hybrid}.

%
``Both connectivity hyperalignment and response hyperalignment increased ISCs
and bsMVPC classification accuracies significantly over anatomy-based alignment,
but each algorithm achieves better alignment for the information that it uses to
derive a common model, namely connectivity profiles and patterns of response,
respectively'' \citep{guntupalli2018computational}.


%
In other words, RHA outperforms CHA on response-based metrics of alignment,
whereas CHA outperforms RHA on connectivity-based metrics'' ``Hyperalignment
projects cortical pattern vectors into a common, high-dimensional information
space \citep{haxby2020hyperalignment}.

%
Derivation of this common space can be based on either neural response profiles
(e.g. data collected during tasks, such as movie viewing (Haxby et al., 2011))
or functional connectivity profiles files \citep{guntupalli2018computational}''
\citep{busch2021hybrid}.

%
``The number of voxels that can be considered simultaneously for functional BOLD
response time series alignment is limited by the number of timepoints in the
calibration scan (about 300-400 voxels for a 15min scan with a 2s TR,
corresponding to a local cortical neighborhood of about 1cm in diameter for a
standard resolution).
%
This limitation does not exist in this form for a functional alignment that is
based on connectivity vectors.
%
The length of these connectivity vectors is determined by the number of
reference (or seed) regions in the brain'' [project proposal].

% Kumar on Nastase's ugly mofo paper
``Estimating the SRM from functional connectivity data rather than response time
series circumvents the need for a single shared stimulus across subjects.
%
Connectivity SRM allows us to derive a single shared response space across
different stimuli with a shared connectivity profile
\citep{nastase2019leveraging}'' \citep{kumar2020brainiak}.
%
``The sampling of connectivity vector space is defined by the selection of
connectivity targets, but the richness and reliability of connectivity estimates
depends on the variety of brain states over which connectivity is estimated''
\citep{haxby2020hyperalignment}.




\subsection{Conclusion}

God bless America!


\section{Data Availability}

\todo[inline]{all from PPA-Paper but with new GIN link leading to an empty repo}

% \href{https://gin.g-node.org/chaeusler/studyforrest-ppa-analysis}{\url{gin.g-node.org/chaeusler/studyforrest-ppa-analysis}}

% new; PPA analysis
All fMRI data and results are available as Datalad \citep{halchenko2021datalad}
datasets, published to or linked from the \emph{G-Node GIN} repository
(\href{https://gin.g-node.org/chaeusler/studyforrest-ppa-srm}{\url{gin.g-node.org/chaeusler/studyforrest-ppa-srm}}).
% original
Raw data of the audio-description, movie and visual localizer were originally
published on the \emph{OpenfMRI} portal
(\url{https://legacy.openfmri.org/dataset/ds000113}; \citep{Hanke2014ds000113},
\space \url{https://legacy.openfmri.org/dataset/ds000113d};
\citep{hanke2016ds000113d}).
% visual localizer
Results from the localization of higher visual areas are available as Datalad
datasets at \emph{GitHub}
(\href{https://github.com/psychoinformatics-de/studyforrest-data-visualrois}{\url{github.com/psychoinformatics-de/studyforrest-data-visualrois}}).
% raw data
The realigned participant-specific time series that were used in the current
analyses were derived from the raw data releases and are available as Datalad
datasets at \emph{GitHub}
(\href{https://github.com/psychoinformatics-de/studyforrest-data-aligned}{\url{github.com/psychoinformatics-de/studyforrest-data-aligned}}).
% OpenNeuro
The same data are available in a modified and merged form on OpenNeuro at
\url{https://openneuro.org/datasets/ds000113}.
% NeuroVault for z-maps of SRM
Unthresholded $Z$-maps of all contrasts can be found at
\href{https://identifiers.org/neurovault.collection:12340}{\url{neurovault.org/collections/12340}}.


\section*{Code Availability}

Scripts to generate the results as Datalad \citep{halchenko2021datalad} datasets
are available in a \emph{G-Node GIN} repository
(\href{https://gin.g-node.org/chaeusler/studyforrest-ppa-srm}{\url{gin.g-node.org/chaeusler/studyforrest-ppa-srm}}).






\pagebreak


\section{Backup of texts regarding "done but not mentioned"}

\subsection{Calculate $Z$-maps mean in the common space already}
%
I also tested averaging $Z$-maps in the \ac{cfs} (i.e.: not in the test
subject's voxel space): similar results
%
In case of anatomical alignment, I did
not test averaging data in MNI152 space.

\subsection{Calculate $Z$-map from training subjects' TRs in FEAT}

iirc, I projected all subjects' localizer time series through
model space into test subject voxel space; then, calculated the contrast
with these data (s. scripts 'test/data\_denoise-vis.py' \&
'test/data\_srm-vis-to-ind.py').
%

The problem was: if one wants to test the different transformation matrices (I
only did it with one; imo, based on alignment using the whole audio-description)
it gets totally messy \& computational intensive.
%
Results were similar to the original procedure if not slightly worse.



\subsection{Shortcoming: leakage of test data in union of individual
\acp{ppa}}

\todo[inline]{yeah, whatever}
%
We used the union of individual \acp{ppa} as spatial constrain for $Z$-maps.
%
But we have a leakage of test data (test subject is in data for the mask).
%
We might miss some voxels (of some participants) at the borders of the \ac{roi},
because the subject-specific, binary masks are based on a ("titrated")
threshold.  \citep{sengupta2016extension}
%
In the future, an independent probabilistic atlas should be used, the \ac{roi}
dilated, [and a separate model calculated for each hemisphere].



\pagebreak

\section{Templates for Cronbach's $\alpha$}


\paragraph{\citet{jiahui2020predicting}}
%
``We compared the correlations between maps estimated from a participant's own
data and maps estimated from other participants' data to the reliability of the
localizer (across the four localizer runs).
%
The mean Cronbach's Alpha between the four localizer runs was 0.60 (N = 15, S.D.
= 0.14)'' \citep{jiahui2020predicting}.
%
``Results mean that if we scan each participant for another 4 localizer
runs, and compute the correlation between the two maps (4 runs vs. 4 runs), the
correlation would be 0.60 on average in the studyforrest''
\citep{jiahui2020predicting}.
%
``Cronbach's alpha indicates that the predicted contrast map based on
hyperalignment is close to or as good as the real contrast map based on four
localizer runs (studyforrest: t(14) = 0.61, p = 0.55; Grand Budapest: t(20) =
3.02, p = 0.007)'' \citep{jiahui2020predicting}.
%
``The predicted contrast map based on hyperalignment was better than the
contrast map based on data from three out of four localizer runs in other
participants (t(14) = 2.36, p = 0.03) and in Grand Budapest, the predicted
contrast map was comparable to the contrast map based on three localizer runs in
other participants (t(20) = 0.48, p = 0.63)''
\citep{jiahui2020predicting}.


\paragraph{\citet{jiahui2022cross}}
%
``Because the localizer task comprises several scanning runs, we calculated the
reliability of the localizer across runs with Cronbach's alpha to provide an
estimate of the noise ceiling for these correlations'' \citep{jiahui2022cross}.
%
``We compared the correlations between topographies estimated from a
participant's own localizer data and those from other participants' data to the
reliability of the localizer, calculated with Cronbach's alpha.
%
Predictions made with hyperalignment were close to and sometimes even exceeded
the reliability values (Figure 1B).
%
This indicates that the predicted category-selective topographies from other
participants' data using hyperalignment were as precise and sometimes even
better than the topographies estimated with their own localizer data''
\citep{jiahui2022cross}.


\paragraph{\citet{feilong2022individualized}}
%
``Due to the presence of noise in localizer data, the estimated face-selectivity
map is a combination of a “true” face-selectivity map of the participant and
some noise.
%
The component from the “true” map is supposed to be shared by all localizer
runs, and thus the data quality and the level of noise can be estimated based on
the similarity between the 4 maps (i.e., one from each run).
%
We used Cronbach's alpha to estimate the reliability  of the average map of the
4 runs.
%
We estimated the reliability of the localizer-based map using Cronbach's alpha,
which is the expected correlation between two average maps, each based on 4 runs
of independent data.
%
If we were to collect another 4 localizer runs from the participant and get a
new average map based on the 4 new runs (i.e., independent data), then the
expected correlation between the two average maps would be Cronbach's alpha.
%
In other words, if the correlation between the model-predicted map and the
localizer-based (average) map is higher than Cronbach's alpha, then the
model-predicted map is more accurate than the average map based on 4 runs''
\citep{feilong2022individualized}.

%
``Based on the similarity between these four maps, we computed the Cronbach's
alpha coefficient for each participant, which estimates the reliability of the
average map.
%
That is, if we were to scan the participant for another four localizer runs and
correlate the new average map with the current average map, the expected
correlation would be Cronbach's alpha.'' \citep{feilong2022individualized}

%
``For both datasets, the localizer-based and model-predicted face-selectivity
maps were highly correlated (Forrest: r = 0.618 ± 0.089 [mean ± SD], Raiders: r
= 0.716 ± 0.074), and the correlations were higher than our previous
state-of-the-art model using the same dataset and hyperalignment (Jiahui et al.,
2020).
%
Across all participants, the average Cronbach's alpha was 0.606 ± 0.126 for
Forrest, and 0.764 ± 0.089 for Raiders.
%
For approximately a third of the participants (Forrest: 6 out of 15, 40\%;
Raiders: 6 out of 20, 30\%), the correlation exceeded the Cronbach's alpha of
localizer-based maps.
%
In other words, for these participants, the predicted map based on our model can
be more accurate than the map based on a typical localizer scanning session
comprising four runs'' \citep{feilong2022individualized}.

%
``Note that for the Forrest dataset, the similarity sometimes exceeded
Cronbach's alpha, which means the model-predicted map is more accurate than a
map based on 4 localizer runs (21 minutes).
%
The quality of localizer-based maps increases with more localizer data, which
can be estimated using the Spearman–Brown prediction formula [Brown, 1910;
Spearman, 1910].
%
Based on the Spearman–Brown prediction formula, we can estimate how Cronbach's
alpha changes with the amount of data (i.e., the number of localizer runs), and
correspondingly, how much localizer data is needed to achieve the quality of the
model-predicted map
%
For the Forrest dataset, the maps predicted by 15, 30, 60, and 120 minutes of
movie data were as accurate as 17, 22, 26, and 30 minutes of localizer data,
respectively.
%
For the Raiders dataset, the maps predicted by 15, 28, and 56 minutes of movie
data were as accurate as 10, 11 and 12 minutes of localizer data, respectively''
\citep{feilong2022individualized}


\pagebreak

\section{Templates from \citet{haxby2011common} on validity \& generalizability}

%
``The algorithm also can be applied to simpler, controlled experimental data but
our previous results showed that the sampling of response vectors from these
experiments is impoverished and produces a model representational space that
does not generalize well to new stimuli in other experiments
\citep{haxby2020hyperalignment}'' \citep{guntupalli2016model}.

%
``We base the derivation of the transformation matrices and the common space on
responses to the movie---a complex, naturalistic, dynamic stimulus.
%
Although the algorithm also can be applied to fMRI data from more controlled
experiments, we found that a common model based on such data has greatly
diminished general validity \citep{haxby2011common}, presumably because,
relative to a rich and dynamic naturalistic stimulus, such experiments sample an
impoverished range of brain states \citep{guntupalli2016model}''.

``The general validity of the model across the varied stimulus sets that we
tested could be achieved only when hyperalignment was based on responses to the
movie.
%
Common models based on responses to smaller, more controlled stimulus
sets---still images of a limited number of categories---were valid only for
restricted stimulus domains, indicating that these models captured only a
subspace of the substantially larger representational space in VT cortex''
\citep{haxby2011common}.

``We used a complex and dynamic natural stimulus---a full-length action
movie---to sample a diverse variety of representational states.
%
The results show that hyperalignment based on responses to this stimulus affords
a single model of VT cortex with general validity across a broad range of
stimuli, whereas hyperalignment based on responses to still images in more
controlled, conventional experiments does not.
%
Thus, by virtue of the rich diversity of a complex, natural stimulus, our model
of the representational space in VT cortex also has general validity across
stimuli''\citep{haxby2011common}.

``We also derived common models based on responses to the face and object
categories in ten subjects and on responses to the pictures of animals in 11
subjects.
%
These alternative common models afforded high levels of accuracy for BSC of the
stimulus categories used to derive the common space but did not generalize to
BSC for the movie time segments.
%
Thus, models based on hyperalignment of responses to a limited number of
stimulus categories align only a small subspace within the representational
space in VT cortex and are, therefore, inadequate as general models of that
space.
%
On the positive side, these results also show that hyperalignment can be used
for BSC of an fMRI experiment without data from movie viewing''
\citep{haxby2011common}.

%
``Further analyses revealed other desirable properties of the movie as a
stimulus for model derivation.
%
The movie evoked responses in VT cortex that were more distinctive than were
responses to the still images in the category perception experiments.
%
Moreover, the general validity of the model based on the responses to the movie
is not dependent on responses to stimuli that are in both the movie and the
category perception experiments but, rather, appears to rest on stimulus
properties that are more abstract and of more general utility''
\citep{haxby2011common}.

%
``We investigated whether hyperalignment of the face and object data and
hyperalignment of the animal species data would afford high levels of BSC
accuracy using only the data from those experiments.
%
In each experiment, we derived a common space based on all runs but one. We
transformed the data from all runs, including the left-out run, into this common
space.
%
We trained the classifier on those runs used for hyperalignment in all subjects
but one and tested the classifier on the data from the left-out run in the
left-out subject (i.e. the test data for determining classifier accuracy played
no role either in hyperalignment or in classifier training [Kriegeskorte et al.,
2009]).
%
BSC of face and object categories after hyperalignment based on data from that
experiment was equivalent to BSC after movie-based hyperalignment (62.9\% ±
2.9\% versus 63.9\% ± 2.2\%, respectively; Figure 4).
%
BSC of the animal species after hyperalignment based on data from that
experiment was significantly better than BSC after movie-based hyperalignment
(76.2\% ± 3.7\% versus 68.0\% ± 2.8\%, respectively; p < 0.05; Figure 4).
%
Result suggests that the validity for a model of a specific subspace may be
enhanced by designing a stimulus paradigm that samples the brain states in that
subspace more extensively'' \citep{haxby2011common}.

%
``We next asked whether hyperalignment based on these simpler stimulus sets was
sufficient to derive a common space with general validity across a wider array
of complex stimuli.
%
We applied the hyperalignment parameters derived from the face and object data
to the movie data in the ten Princeton subjects and the hyperalignment
parameters derived from the animal species data to the movie data in the 11
Dartmouth subjects.
%
BSC of 18s movie time segments after hyperalignment based on category perception
experiment data was markedly worse than BSC after hyperalignment based on movie
data (17.6\% ± 1.3\% versus 65.8\% ± 2.7\% for Princeton subjects; 28.3\% ±
2.8\% versus 74.9\% ± 4.1\% for Dartmouth subjects; p < 0.001 in both cases;
Figure 4).
%
Thus, hyperalignment of data using a set of stimuli that is less diverse than
the movie is effective, but the resultant common space has validity that is
limited to a small subspace of the representational space in VT cortex''
\citep{haxby2011common}.

%
``We also tested whether the general validity of the model space reflects
responses to stimuli that are in both the movie and the category perception
experiments or reflects stimulus properties that are not specific to these
stimuli.
%
We recomputed the common model after removing all movie time points in which a
monkey, a dog, an insect, or a bird appeared. We also removed time points for
the 30 s that followed such episodes to factor out effects of delayed
hemodynamic responses.
%
BSC of the face and object and animal species categories, including distinctions
among monkeys, dogs, insects, and birds, was not affected by removing these time
points from the movie data [65.0\% ± 1.9\% versus 64.8\% ± 2.3\% for faces and
objects; 67.1\% ± 3.0\% versus 67.6\% ± 3.1\% for animal species; Figure S4B].
%
This result suggests that the movie-based hyperalignment parameters that afford
generalization to these stimuli are not stimulus specific but, rather, reflect
stimulus properties that are more abstract and of more general utility for
object representations'' \citep{haxby2011common}.
