This part of the dissertation has been published:

\bigbreak

\noindent
%
Häusler, C. O.,
%
Eickhoff, S. B.,
%
\& Hanke, M.
%
(2022).
%
Processing of visual and non-visual naturalistic spatial information in the
``parahippocampal place area''.
%
Scientific Data, 9(1).
%
doi: \href{https://doi.org/10.1038/s41597-022-01250-4}
{\url{10.1038/s41597-022-01250-4}}.

% \noindent\vspace{0.5in}

\begin{chapterabstract}
% < 170 words new analysis of existing data of interest to a broad section of
% our audience highlighting innovative examples of data reuse may be used to
% present compelling new findings & conclusions derived from published.
The ``parahippocampal place area'' (PPA) in the human ventral visual stream
exhibits increased hemodynamic activity correlated with the perception of
landscape photos compared to faces or objects.
% AV, AD: operationalization
Here, we investigate the perception of scene-related, spatial information
embedded in two naturalistic stimuli.
%
The same 14 participants were watching a Hollywood movie and listening to its
audio-description as part of the open-data resource
\href{https://www.studyforrest.org}{\url{studyforrest.org}}.
% auditory localizer?
%in order to explore if a naturalistic auditory narrative could be used to
%localize a ``visual area''.
We model hemodynamic activity based on annotations of selected stimulus
features,
% VIS
and compare results to a block-design visual localizer.
% no refs in abstract (probably)
%\citep{sengupta2016extension}.  results: group AV
On a group level, increased activation correlating with visual spatial
information occurring in the movie is overlapping with a traditionally
localized PPA.
% results: group AD
Activation correlating with semantic spatial information occurring in the
audio-description is more restricted to the anterior PPA.
% results: individual AD
On an individual level, we find significant bilateral activity in the PPA of
nine individuals and unilateral activity in one individual.
% conclusion: generalizability
Results suggest that activation in the PPA generalizes to spatial information
embedded in a movie and an auditory narrative, and may
% conclusion: PPA has subregions
call for considering a functional subdivision of the PPA.
%
\end{chapterabstract}
