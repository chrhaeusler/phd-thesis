This part of the dissertation has been published:

\bigbreak

\noindent
%
Halchenko, Y. O.,
Meyer, K.,
Poldrack, B.,
Solanky, D. S.,
Wagner, A. S.,
Gors, J.,
MacFarlane, D.,
Pustina, D.,
Sochat, V.,
Ghosh, S. S.,
Mönch, C.,
Markiewicz, C. J.,
Waite, L.,
Shlyakhter, I.,
de la Vega, A.,
Hayashi, S.,
Häusler, C. O.,
Poline, J.-P.,
Kadelka, T.,
Skytén, K.,
Jarecka, D.,
Kennedy, D.,
Strauss, T.,
Cieslak, M.,
Vavra, P.,
Ioanas, H.-I.,
Schneider, R.,
Pflüger, M.,
Haxby, J. V.,
Eickhoff, S. B.,
\& Hanke, M.
%
(2021).
%
DataLad: distributed system for joint management of code, data, and their
relationship.
%
Journal of Open Source Software, 6(63), 3262.
%
doi: \href{https://doi.org/10.21105/joss.03262}{\url{10.21105/joss.03262}}.

% Yaroslav O. Halchenko, Kyle Meyer, Benjamin Poldrack, Debanjum Singh Solanky,
% Adina S. Wagner, Jason Gors, Dave MacFarlane, Dorian Pustina, Vanessa Sochat,
% Satrajit S. Ghosh, Christian Mönch, Christopher J. Markiewicz, Laura Waite,
% Ilya Shlyakhter, Alejandro de la Vega, Soichi Hayashi, Christian Olaf
% Häusler, Jean-Baptiste Poline, Tobias Kadelka, Kusti Skytén, Dorota Jarecka,
% David Kennedy, Ted Strauss, Matt Cieslak, Peter Vavra, Horea-Ioan Ioanas,
% Robin Schneider, Mika Pflüger, James V. Haxby, Simon B. Eickhoff, and Michael
% Hanke.  DataLad: distributed system for joint management of code, data, and
% their relationship. Journal of Open Source Software, 6(63):3262, 2021. doi:
% 10.21105/joss.03262.

\begin{chapterabstract}
%
DataLad is a Python-based tool for the joint management of code, data, and
their relationship, built on top of a versatile system for data logistics
(\href{https://git-annex.branchable.com}{\url{git-annex}}) and the most popular
distributed version control system (\href{https://git-scm.com}{\url{Git}}).
%
It adapts principles of open-source software development and distribution to
address the technical challenges of data management, data sharing, and digital
provenance collection across the life cycle of digital objects.
%
DataLad aims to make data management as easy as managing code. It streamlines
procedures to consume, publish, and update data, for data of any size or type,
and to link them as precisely versioned, lightweight dependencies.
%
DataLad helps to make science more reproducible and FAIR
\citep{wilkinson2016fair}.
%
It can capture complete and actionable process provenance of data
transformations to enable automatic re-computation.
%
The DataLad project (\href{http://datalad.org}{\url{datalad.org}}) delivers a
completely open, pioneering platform for flexible decentralized research data
management (RDM) \citep{hanke2021defense}.
%
It features a Python and a command-line interface, an extensible architecture,
and does not depend on any centralized services but facilitates
interoperability with a plurality of existing tools and services.
%
In order to maximize its utility and target audience, DataLad is available for
all major operating systems, and can be integrated into established workflows
and environments with minimal friction.
%
\end{chapterabstract}
