% \singlespacing

% scientific background
Functional localizers are fMRI experiments that aim to characterize the
functional neuroanatomy on the level of individuals.
%
However, these paradigms employ selectively sampled, tightly controlled
stimuli, rely on an individual's compliance, and can typically map only one
domain of brain functions.
% current state of research
In contrast, naturalistic stimuli, such as movies or auditory narratives,
provide an engaging, task-free paradigm that more closely resembles the
complexity and richness of real-life experiences, and sample a wide range of
brain functions.
% research question and objectives
This dissertation focuses on the ``parahippocampal place'' (PPA), a high-level
visual area, that exhibits increased hemodynamic activity when participants view
images of landscapes or landmarks, as opposed to other stimuli, such as faces
or tools.
% stimuli
Following the principles of open, transparent, and reproducible science, the
thesis explores whether a movie and an auditory narrative could replace a
visual localizer in two methodological ways.

% methodology speech-anno: haeusler2021speechanno; ppa-paper:
% haeusler2022processing
As the first approach, we performed a model-driven analysis of hemodynamic
activity during the movie ``Forrest Gump'' and its audio-description, which are
the core stimuli of the publicly accessible studyforrest dataset
(\href{www.studyforrest.org}{\url{studyforrest.org}}).
%
An exhaustive annotation of speech occurring in the movie and audio-description
was created to establish the foundation for modeling hemodynamic responses and
to extend the studyforrest project as an open science resource.
% but technically similar to traditional
Subsequently, we performed a general linear model (GLM) analysis to localize
the PPA, which had previously been identified in the same group of participants
using a visual localizer.
% results
Results suggest that a model-driven analysis based on annotations of a movie or
an exclusively auditory naturalistic stimulus can be used to localize a visual
area on an individual level.

% SRM
As the second approach, we explored a novel functional alignment procedure that
allows estimating the location of the PPA in an individual by leveraging data
collected from of a reference group.
%
Using a shared response model (SRM), we created a common functional space (CFS)
and subject-specific transformations to project functional data from the
reference through the CFS into an individual's brain space.
%
Additionally, we investigated the relationship between the quantity of data
used for functional alignment and subsequent estimation performance.
%
Results suggest that an auditory narrative can, in principle, be used to
estimate a visual area such as the PPA.
%
Moreover, 15 minutes of functional scanning during movie watching can generate
a sufficient amount of data to estimate brain patterns more accurately than a
procedure based on anatomical alignment.

% discussion
The thesis also highlights obstacles in the pursuit of developing a
multi-functional naturalistic localizer.
%
Applying a model-driven analysis to naturalistic stimuli is challenging, as
these stimuli stress physiological and statistical assumptions.
%
Moreover, traditional localizers aim to minimize interindividual variability
and reliably localize functional areas in all healthy individuals, whereas
naturalistic stimuli allow for more variability.
% conclusions
Therefore, the potential of a naturalistic stimulus to replace one or multiple
traditional localizers relies on further developments that address the
statistical and methodological challenges.
%
Nevertheless, a data-driven approach based on functional alignment using a
naturalistic stimulus of similar duration to that of one traditional localizer
could potentially estimate the results of many functional localizers.
