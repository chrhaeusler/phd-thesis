% scientific background
\todo[inline]{name studyforrest dataset}

\paragraph{Scientific background:}
% clinical application
Functional localizers that aim to characterize the topography of functional
areas on an individual level are a promising tool to advance human brain
mapping towards a clinical application.

% current state of research
\paragraph{Current state of research:}
% contra localizers
However, traditional localizer paradigms employ selectively sampled, tightly
controlled stimuli, rely heavily on a participant's compliance, and can
typically map only one domain of brain functions.
% naturalistic stimuli could replace
Contrary, naturalistic stimuli like movies or auditory narratives provide an
engaging, task-free paradigm that more closely resembles the richness of
real-life experiences.
%
Naturalistic stimuli elicit a variety of different brain functions ranging from
low-level perception to high-level cognition.

% research question and objectives
\paragraph{Research question and objectives:}
% focus: ppa
As a proof of concept, this dissertation focuses on the \ac{ppa} that exhibits
increased hemodynamic activity when participants view images of landscapes or
landmarks, as opposed to, for example, images of faces or tools.
%
While adhering to the principles of open, transparent, and reproducible
science, the thesis explores whether a movie and the movie's audio-description
that was produced for a visually impaired audience could, in principle, replace
a visual localizer in two ways.

% methodology
\paragraph{Methodology, modeling:}

\todo[inline]{analysis as usually done but adapted}

% 1. modeling hemodynamic responses
In the first approach, we modeled hemodynamic activity based on annotations of
the movie and the audio-description.
% but technically similar to traditional localizers
We then created \ac{glm} $t$-contrasts using the modeled hemodynamic time
courses (i.e. regressors) to locate the \ac{ppa}, which was previously
identified in the same group of participants by a conventional visual localizer
\citep{sengupta2016extension}.

% results
\paragraph{Results, modeling:}
% previous studies: anatomical alignment

\paragraph{Methodology, estimation}

\todo[inline]{fucking new}
%
However, conducting a two-hour long \ac{fmri} scan session may not be desirable
or feasible due to potential compliance issues or constraints on time and
resources.
%
The second approach to localize the \ac{ppa} in an individual leverages data
collected from an independent sample of other individuals (i.e.  data from a
\textit{reference group}).
% intro to functional alignment
To address the issue of functional-anatomical variability across persons, we
performed a functional alignment with a common functional space, and investigated
whether we can estimate the results of $t$-contrasts (i.e. statistical
$Z$-maps) that we created in the previously.
%
Considering that functional alignment relies on functional scans, we also
evaluated the relationship between the quantity of functional data used for
functional alignment and its subsequent estimation performance.

\paragraph{Results estimation:}

% discussion
\paragraph{Discussion:}

% conclusions
\paragraph{Conclusions:}
