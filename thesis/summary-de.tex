% wissenschaftlicher Hintergrund
\textit{Functional localizers} sind fMRT-Experimente, die die funktionelle
Neuroanatomie von Individuen chrakterisieren sollen.
% aktueller Forschungsstand
Allerdings verwenden diese Paradigmen selektiv ausgewählte, experimentell
streng kontrollierte Reize, stützen sich auf die Folgebereitschaft des
jeweiligen Individuums, und üblicherweise nur eine Domäne von Gehirnfunktionen
abbilden.
%
Im Gegensatz dazu bieten naturalistische Reize wie Filme oder auditive
Erzählungen ein vereinnehmendes, aufgabenfreies Paradigma, das der Komplexität
und Vielfalt realer Erfahrungen näher kommt und eine viele Gehirnfunktionen
abdecken könnte.
% Fragestellung und Ziele
Der Schwerpunkt dieser Dissertation richtet sich auf das "Parahippocampal
Place" (PPA), ein funktionalles Areal höherer visueller Wahrnemung, das erhöhte
hämodynamische Aktivität aufweist, wenn Studienteilnehmer Bilder von
Landschaften oder Wahrzeichen betrachten, im Gegensatz zu anderen Reizen wie
Gesichtern oder Werkzeugen.
%
Unter Berücksichtigung der Prinzipien offener, transparenter und
reproduzierbarer Wissenschaft untersucht die Arbeit mit zwei methodischen
Ansätzen, ob ein Film und eine auditorische Erzählung einen visuellen Localizer
ersetzen könnten.

% Methodik 1
Als erster Ansatz führten wir eine modellgesteuerte Analyse der hämodynamischen
Aktivität während des Films "Forrest Gump" und seiner Audiodeskription durch,
die die Hauptstimuli des öffentlich zugänglichen "StudyForrest"-Datensatzes
(\href{www.studyforrest.org}{\url{studyforrest.org}}) sind.
%
Zunächste wurde eine umfassende Annotation der im Film und in der
Audiodeskription vorkommenden gesprochenen Sprache erstellt, um die Grundlage
für die Modellierung hämodynamischer Reaktionen zu schaffen und das Projekt
"StudyForrest" als offene Wissenschaftsressource zu erweitern.
%
Anschließend führten wir eine massenunivariate Analyse mit dem allgemeinen
linearen Modell (GLM) durch, um die PPA in Personen zu lokalisieren, die zuvor
bereits an einem Localizer-Experiment teilgenommen hatten.
% Ergebnisse 1
Die Ergebnisse legen nahe, dass eine modellgetriebene auf der Grundlage von
Annotationen eines Films oder eines ausschließlich auditorischen
naturalistischen Stimulus verwendet werden kann, um eine visuelles Areal in
Individuen lokalisieren zu können.


% Methodik 2
Als zweiten Ansatz untersuchte die Arbeit ein neuartiges, datengetriebenes Verfahren für die
funktionelle Lokalisierung, das es ermöglicht mittels \textit{functional
alignments}, die Lage des PPA in einem Individuum zu schätzen, indem es sich
Daten von anderen Individuum zu Nutze macht.
%
Unter Verwendung des \textit{shared response models} (SRM) haben wir einen
\textbf{common functional space} (CFS) und individuelle Transformationen
erstellt, um funktionalle Daten andere Individuen durch den CFS in den
Gehirnraum des zu untersuchenden Individuums zu projizieren.
%
Darüber hinaus untersuchten wir die Beziehung zwischen der Menge funktioneller
benutzt wurde, und der sich daraus ergebenden Schätzleistung.
% Ergebnisse 2
Die Ergebnisse legen nahe, dass eine auditorische Erzählung grundsätzlich dazu
verwendet werden kann, um die neuroanatomischen Position eines visuelles Areal
wie der PPA zu schätzen.
%
Darüber hinaus können Daten eines 15-minütigen Scans, während dem ein
Individuum einen Film schaut, hinreichend sein, um Gehirnmuster genauer zu
schätzen als ein Verfahren, das auf anatomischen Alignment beruht.

% Diskussion
Die Arbeit zeigt jedoch auch Hindernisse in der Entwicklung eines
multifunktionalen naturalistischen Localizers auf.
%
Daten naturalistischer Stimuli stellen eine Herausforderung für
modellgetriebene Analysen insofern da, weil sie physiologische und statistische
Modelannahmen strapazieren.
%
Außerdem zielen traditionelle Localizer darauf ab, die interindividuelle
Variabilität zu minimieren und funktionelle Areale in allen gesunden Personen
reliabel zu lokalisieren, hingegen naturalistische Stimuli jedoch höhere
Variabilität zulassen.

% Schlussfolgerungen
Daher hängt das Potenzial eines naturalistischen Stimulus, einen oder mehrere
traditionelle Localizer zu ersetzen, von weiteren Entwicklungen ab, die die
statistischen und methodologischen Herausforderungen angehen.
%
Dennoch könnte ein datengetriebener Ansatz auf Grundlage eines functional Alignments,
 der sich Daten eines natürlichen
basierend auf funktioneller Ausrichtung unter
Verwendung eines naturalistischen Stimulus ähnlicher Dauer wie bei einem
traditionellen Lokalisator potenziell die Ergebnisse vieler funktioneller
Lokalisatoren abschätzen.

Jedoch zeigt die Arbeit, dass ein datengetriebener Ansatz, der auf functional
alignment unter Verwendung eines Stimulus von ähnlicher Dauer wie der eines
traditionellen localizers beruht, das Potential hat, die Ergebnisse
funktioneller Localizer zu schätzen.
