% wissenschaftlicher Hintergrund
Funktionale Lokalisatoren sind fMRT-Experimente, die das funktionale Neu-
ronatom auf individueller Ebene charakterisieren sollen.
% aktueller Forschungsstand
Allerdings verwenden diese Paradigmen selektiv ausgewählte, streng
kontrollierte Reize, stützen sich auf die Kooperation des Einzelnen und können
typischerweise nur einen Bereich der Gehirnfunktionen abbilden. Im Gegensatz
dazu bieten naturalistische Reize wie Filme oder auditive Erzählungen ein
fesselndes, aufgabenfreies Paradigma, das der Komplexität und Vielfalt realer
Erfahrungen näher kommt und eine breite Palette von Gehirnfunktionen abdeckt.
% Fragestellung und Ziele
Diese Dissertation konzentriert sich auf den "Parahippocampal Place" (PPA),
eine hochvisuelle Region, die eine erhöhte hämodynamische Aktivität aufweist,
wenn Teilnehmer Bilder von Landschaften oder Wahrzeichen betrachten, im
Gegensatz zu anderen Reizen wie Gesichtern oder Werkzeugen. Unter
Berücksichtigung der Prinzipien von offener, transparenter und reproduzierbarer
Wissenschaft untersucht die Arbeit, ob ein Film und eine auditive Erzählung
einen visuellen Lokalisator auf zwei Arten ersetzen könnten.

% Methodik 1
Als erster Ansatz führten wir eine modellgesteuerte Analyse der hämodynamischen
Aktivität während des Films "Forrest Gump" und seiner Audio-Beschreibung durch,
die die Kernstimuli des öffentlich zugänglichen Studien-Datensatzes
"StudyForrest" (studyforrest.org) sind. Eine umfassende Annotation der im Film
und in der Audio-Beschreibung vorkommenden Sprache wurde erstellt, um die
Grundlage für die Modellierung hämodynamischer Reaktionen zu schaffen und das
Projekt "StudyForrest" als offene Wissenschaftsressource zu erweitern.
Anschließend führten wir eine massenunivariate Analyse mit dem allgemeinen
linearen Modell (GLM) durch, um den PPA zu lokalisieren, der zuvor in derselben
Gruppe von Teilnehmern mithilfe eines visuellen Lokalisators identifiziert
worden war.
% Ergebnisse 1
Die Ergebnisse legen nahe, dass eine modellgesteuerte Analyse auf der Grundlage
von Annotationen eines Films oder eines ausschließlich auditiven
naturalistischen Stimulus verwendet werden kann, um eine visuelle Region auf
individueller Ebene zu lokalisieren.

% Methodik 2
Als zweiter Ansatz haben wir ein neuartiges Verfahren zur funktionellen
Ausrichtung untersucht, das es ermöglicht, die Lage des PPA in einem Individuum
zu schätzen, indem Daten von einer Referenzgruppe genutzt werden. Unter
Verwendung eines gemeinsamen Antwortmodells (SRM) haben wir einen gemeinsamen
funktionalen Raum (CFS) und individuelle Transformationen erstellt, um
funktionale Daten von der Referenz durch den CFS in den Gehirnraum eines
Individuums zu projizieren. Darüber hinaus untersuchten wir die Beziehung
zwischen der Menge der für die funktionelle Ausrichtung verwendeten Daten und
der anschließenden Schätzleistung.
% Ergebnisse 2
Die Ergebnisse legen nahe, dass eine auditive Erzählung grundsätzlich dazu
verwendet werden kann, eine visuelle Region wie den PPA abzuschätzen. Darüber
hinaus kann eine 30-minütige funktionelle Bildgebung während des Filmsehens
eine ausreichende Datenmenge generieren, um Gehirnmuster genauer zu schätzen
als ein Verfahren, das auf anatomischer Ausrichtung basiert.
% Diskussion
Die Arbeit hebt auch Hindernisse bei der Entwicklung eines multifunktionalen
naturalistischen Lokalisators hervor. Die Anwendung einer modellgesteuerten
Analyse auf naturalistische Reize ist herausfordernd, da diese Reize
physiologische und statistische Annahmen belasten. Außerdem zielen
traditionelle Lokalisatoren darauf ab, die zwischenindividuelle Variabilität zu
minimieren und funktionale Bereiche bei allen gesunden Personen zuverlässig zu
lokalisieren, während naturalistische Reize mehr Variabilität zulassen.

% Schlussfolgerungen
Daher hängt das Potenzial eines naturalistischen Stimulus, einen oder mehrere
traditionelle Lokalisatoren zu ersetzen, von weiteren Entwicklungen ab, die die
statistischen und methodologischen Herausforderungen angehen. Dennoch könnte
ein datengetriebener Ansatz basierend auf funktioneller Ausrichtung unter
Verwendung eines naturalistischen Stimulus ähnlicher Dauer wie bei einem
traditionellen Lokalisator potenziell die Ergebnisse vieler funktioneller
Lokalisatoren abschätzen.
