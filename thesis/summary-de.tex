% wissenschaftlicher Hintergrund
\textit{Functional localizers} sind fMRT-Experimente, die die funktionelle
Neuroanatomie von Individuen chrakterisieren sollen.
% aktueller Forschungsstand
Allerdings verwenden diese Paradigmen selektiv ausgewählte, experimentell
streng kontrollierte Reize, stützen sich auf die kooperative Mitarbeit des
jeweiligen Individuums, und können üblicherweise nur eine Domäne von
Gehirnfunktionen abbilden.
%
Im Gegensatz dazu bieten naturalistische Stimuli wie Filme oder auditorische
Erzählungen ein fesselndes, aufgabenfreies Paradigma, das der Komplexität und
Vielfalt des alltäglichen Erlebens näher kommt, und dadurch eine Vielzahl von
Gehirnfunktionen abbilden könnte.
% Fragestellung und Ziele
Der Schwerpunkt dieser Dissertation richtet sich auf das ``Parahippocampal
Place'' (PPA), ein funktionelles Areal höherer visueller Wahrnemung, das
erhöhte hämodynamische Aktivität aufweist, wenn Studienteilnehmer Bilder von
Landschaften oder Orientierungspunkten, im Gegensatz zu anderen Reizen wie
Gesichtern oder Werkzeugen, betrachten.
%
Unter Berücksichtigung der Prinzipien offener, transparenter und
reproduzierbarer Wissenschaft untersucht die Arbeit mit zwei methodischen
Ansätzen, ob ein Film und eine auditorische Erzählung einen visuellen Localizer
ersetzen könnten.

% Methodik 1
Als ersten Ansatz führten wir eine modellgetriebene Analyse hämodynamischer
Aktivität während des Films ``Forrest Gump'' und seiner Audiodeskription durch,
die die Hauptstimuli des öffentlich zugänglichen StudyForrest-Datensatzes sind
(\href{www.studyforrest.org}{\url{studyforrest.org}}).
%
Zunächste wurde eine umfassende Annotation der im Film und in der
Audiodeskription vorkommenden gesprochenen Sprache erstellt, um die Grundlage
für die Modellierung hämodynamischer Antworten zu schaffen und das
StudyForrest-Projekt als öffentlich zugängliche Wissenschaftsressource zu
erweitern.
%
Anschließend führten wir eine massenunivariate Analyse mit dem allgemeinen
linearen Modell (GLM) durch, um die PPA in Personen zu lokalisieren, die zuvor
bereits an einem Localizer-Experiment teilgenommen hatten.
% Ergebnisse 1
Die Ergebnisse legen nahe, dass eine modellgetriebene Analyse auf der Grundlage
von Annotationen eines Films oder eines ausschließlich auditorischen
naturalistischen Stimulus verwendet werden kann, um eine visuelles Areal in
Individuen lokalisieren.


% Methodik 2
Als zweiten Ansatz untersuchte die Arbeit ein neuartiges, datengetriebenes
Verfahren für eine funktionelle Lokalisierung, das es ermöglicht die Lage der
PPA in einem Individuum zu schätzen, indem es sich Daten mittels
\textit{functional alignments} von anderen Individuen zu Nutze macht.
%
Unter Verwendung des \textit{shared response models} (SRM) erstellten  wir
einen \textit{common functional space} (CFS) und individuelle Transformationen,
um funktionalle Daten von Individuen in einer Referenzgruppe durch den CFS in
den Gehirnraum des zu untersuchenden Individuums zu projizieren.
%
Darüber hinaus untersuchten wir die Beziehung zwischen der Menge funktioneller
Daten, die für das Alignments eines Individuums mit dem CFS genutzt wurden und
der sich anschließenden Schätzleistung.
% Ergebnisse 2
Die Ergebnisse legen nahe, dass eine auditorische Erzählung grundsätzlich dazu
verwendet werden kann, um die neuroanatomische Position eines visuelles Areal
wie der PPA zu schätzen.
%
Darüber hinaus zeigen die Ergebnisse, dass Daten eines 15-minütigen Scans,
während dem ein Individuum einen Film schaut, hinreichend sind, um Gehirnmuster
genauer zu schätzen als ein Verfahren, das auf einem anatomischen Alignment beruht.

% Diskussion
Die Arbeit zeigt jedoch auch Hindernisse in der Entwicklung eines
multifunktionalen naturalistischen Localizers auf.
%
Daten naturalistischer Stimuli stellen eine Herausforderung für
modellgetriebene Analysen insofern da, weil sie physiologische und statistische
Modelannahmen strapazieren.
%
Außerdem zielen traditionelle Localizer darauf ab, die interindividuelle
Variabilität zu minimieren und funktionelle Areale in allen gesunden Personen
reliabel zu lokalisieren, wohingegen naturalistische Stimuli höhere
inderidividuelle Variabilität zulassen.
% Schlussfolgerungen
Daher hängt das Potenzial eines naturalistischen Stimulus, einen oder mehrere
traditionelle Localizer zu ersetzen, von weiteren Entwicklungen ab, die die
statistischen und methodologischen Herausforderungen angehen.
%
Jedoch könnte ein datengetriebener Ansatz, der auf functional
alignment beruht, möglicherweise einen naturalistischen Stimulus von ähnlicher 
Dauer eines traditionellen Localizers verwenden, um die Ergebnisse vieler 
funktioneller Localizer schätzen.
